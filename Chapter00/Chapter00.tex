% !TEX root = ../Quantum.tex
\thispagestyle{empty}
\begin{center}
{\Huge\bf Quantum Mechanics}\\[1ex]
~\\
~\\
{\Large\em  Richard Fitzpatrick}\\[1.5ex]~\\[1.5ex]
{\Large\sf  Professor  of Physics}\\[1.5ex]
{\Large\sf  The University of Texas at Austin}
\end{center}

\section*{Major Sources}
The textbooks that I have consulted most frequently while developing
course material are:
\begin{description}
\item [\em The Principles of Quantum Mechanics.] P.A.M.~Dirac, 4th Edition (revised)
(Oxford University Press, Oxford UK, 1958).
\item [{\em The Feynman Lectures on Physics}.] R.P.~Feynman, R.B.~Leighton,
and M.~Sands, Volumes~I--III (Addison-Wesley, Reading MA, 1965).
\item [{\em Quantum Mechanics}.] E.~Merzbacher, 2nd Edition (John Wiley \& Sons,
New York NY, 1970).
\item [{\em Quantum Physics}.] S.~Gasiorowicz, 2nd Edition (John Wiley \& Sons, New York NY, 1996).
\item [{\em Modern Quantum Mechanics}.] J.J.~Sakurai, and J.~Napolitano, 2nd Edition (Addison-Wesley, Boston MA, 2011).
\end{description}

\section*{Purpose of Course}
Quantum mechanics was developed during the first few decades of the twentieth century via a series of inspired guesses made
by various physicists,  including Planck, Einstein, Bohr, Schr\"{o}dinger, Heisenberg, Pauli, and Dirac.  All of these scientists were
trying to construct a self-consistent theory of microscopic dynamics that was compatible with experimental observations. 
The purpose of this course is to present quantum mechanics in a systematic fashion, starting from the
fundamental postulates, and developing the theory in as logical a manner as possible. 

