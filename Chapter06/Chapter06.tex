\chapter{Addition of Angular Momentum}\label{c6}
% !TEX root = ../Quantum.tex

Consider a hydrogen atom whose orbiting electron is in an $l=1$ state. The electron, consequently, 
possesses orbital angular momentum of magnitude $\hbar$, and spin angular
momentum of magnitude $\hbar/2$. So, what is the electron's total angular momentum? 

In order to answer this question, we need to learn how to add
angular momentum operators. Consider the most general case. Suppose
that we have two sets of angular momentum operators, ${\bf J}_1$ and ${\bf J}_2$.
By definition, these operators are Hermitian, and obey the fundamental commutation
relations
\begin{align}
{\bf J}_1\times {\bf J}_1& = {\rm i}\,\hbar\,{\bf J}_1,\\[0.5ex]
{\bf J}_2\times {\bf J}_2& = {\rm i}\,\hbar\,{\bf J}_2.
\end{align}
Let us assume that the two groups of operators  correspond to different degrees
of freedom of the system, so that 
\begin{equation}
[J_{1\,i}, J_{2\,j}] = 0,
\end{equation}
where $i, j$ stand for either $x$, $y$, or $z$. 
For instance, ${\bf J}_1$ could be an orbital angular momentum operator, and ${\bf J}_2$
a spin angular momentum operator. Alternatively, ${\bf J}_1$ and ${\bf J}_2$ could
be the orbital angular momentum operators
 of two different particles in a multi-particle
system. We know, from the general
properties of angular momentum, that the eigenvalues of $J_1^{\,2}$ and $J_2^{\,2}$ 
can be
written $j_1\,(j_1+1)\,\hbar^2$ and $j_2\,(j_2+1)\, \hbar^2$, respectively, where
$j_1$ and $j_2$ are either integers, or half-integers. We also know that the
eigenvalues of $J_{1\,z}$ and $J_{2\,z}$ take the form $m_1\,\hbar$ and
$m_2\,\hbar$, respectively, where $m_1$ and $m_2$ are numbers
lying  in the ranges $j_1, j_1-1,\cdots,
-j_1+1, -j_1$ and  $j_2, j_2-1,\cdots,
-j_2+1, -j_2$, respectively.

Let us define the total angular momentum operator
\begin{equation}
{\bf J} = {\bf J}_1 + {\bf J}_2.
\end{equation}
Now,  ${\bf J}$ is an Hermitian operator, because it is the sum of Hermitian operators. 
Moreover, according to Equations~(\ref{e5.5}) and (\ref{e5.8}), ${\bf J}$ satisfies the fundamental commutation
relation
\begin{equation}
{\bf J} \times {\bf J} = {\rm i}\,\hbar\, {\bf J}.
\end{equation}
Thus, ${\bf J}$ possesses all of the expected properties of an
angular momentum operator. It follows that the eigenvalue of $J^{\,2}$ can be
written $j\,(j+1)\,\hbar^2$, where $j$ is an integer, or a half-integer. Moreover, the eigenvalue
of $J_z$ takes the form $m\,\hbar$, where $m$ lies in the range $j, j-1,\cdots,
-j+1, -j$. At this stage, however, we do not know the relationship between the quantum
numbers of the total angular momentum, $j$ and $m$, and those of the
individual angular momenta, $j_1$, $j_2$, $m_1$, and $m_2$. 

Now,
\begin{equation}\label{e5.206}
J^{\,2} = J_1^{\,2} + J_2^{\,2} + 2\,{\bf J}_1  \cdot {\bf J}_2.
\end{equation}
Furthermore, we know that
\begin{align}
[J_1^{\,2}, J_{1\,i} ] &= 0,\\[0.5ex]
[J_2^{\,2}, J_{2\,i} ] &= 0,
\end{align}
and also that all of the $J_{1\,i}$,  $J_1^{\,2}$ operators commute with the $J_{2\,i}$, $J_2^{\,2}$ operators. 
It follows from Equation~(\ref{e5.206}) that
\begin{equation}
[J^{\,2}, J_1^{\,2}] = [J^{\,2}, J_2^{\,2}] = 0.
\end{equation}
This implies  that the quantum numbers $j_1$, $j_2$, and $j$ can all be measured
simultaneously. In other words, we can know the magnitude of the total
angular momentum  together with the magnitudes of the component
angular momenta. However, it is apparent from Equation~(\ref{e5.206})
that
\begin{align}
[J^{\,2}, J_{1\,z}] &\neq 0,\\[0.5ex]
[J^{\,2}, J_{2\,z}] &\neq  0.
\end{align}
This suggests  that it is not possible to measure the quantum numbers $m_1$ and $m_2$
simultaneously with the quantum number $j$. Thus, we cannot determine
the projections of the individual angular momenta along the $z$-axis
at the same time as the magnitude of the total angular momentum.

It is clear, from the preceding discussion, that we can form two alternate groups
of mutually commuting operators. The first group
is $J_1^{\,2}, J_2^{\,2}, J_{1\,z}$, and
$J_{2\,z}$. The second group is $J_1^{\,2}, J_2^{\,2}, J^{\,2},$ and $J_z$. These two
groups of operators are incompatible with one another.  We can define simultaneous 
eigenkets of each operator group. The simultaneous eigenkets of
$J_1^{\,2}, J_2^{\,2}, J_{1z}$, and
$J_{2z}$ are denoted $|j_1,j_2; m_1,m_2\rangle$, where
\begin{align}
J_1^{\,2}\, |j_1,j_2; m_1,m_2\rangle &= j_1\,(j_1+1)\,\hbar^2\,|j_1,j_2; m_1,m_2\rangle,
\\[0.5ex]
J_2^{\,2} \,|j_1,j_2; m_1,m_2\rangle &= j_2\,(j_2+1)\,\hbar^2\,|j_1,j_2; m_1,m_2\rangle,
\\[0.5ex]
J_{1z}\, |j_1,j_2; m_1,m_2\rangle &= m_1\,\hbar\,|j_1,j_2; m_1,m_2\rangle,
\\[0.5ex]
J_{2z}\, |j_1,j_2; m_1,m_2\rangle &= m_2\,\hbar\,|j_1,j_2; m_1,m_2\rangle.
\end{align}
The simultaneous eigenkets of
$J_1^{\,2}, J_2^{\,2}, J^{\,2}$ and $J_z$ are denoted 
$|j_1, j_2; j, m\rangle$, where 
\begin{align}
J_1^{\,2}\, |j_1,j_2; j,m\rangle &= j_1\,(j_1+1)\,\hbar^2\,|j_1,j_2; j,m\rangle,
\\[0.5ex]
J_2^{\,2} \,|j_1,j_2; j,m\rangle &= j_2\,(j_2+1)\,\hbar^2\,|j_1,j_2; j,m\rangle,
\\[0.5ex]
J^{\,2} \,|j_1,j_2; j,m\rangle &=j\,(j+1)\,\hbar^2\,|j_1,j_2; j,m\rangle,
\\[0.5ex]
J_{z}\, |j_1,j_2; j,m\rangle &= m\,\hbar\,|j_1,j_2; j,m\rangle.
\end{align}
Each set of eigenkets are complete, mutually orthogonal (for eigenkets corresponding
to different sets of eigenvalues), and have unit norms. Since the operators
$J_1^{\,2}$ and $J_2^{\,2}$ are common to both  operator groups, we can  assume
that the quantum numbers $j_1$ and $j_2$ are known. In other words, we 
can always determine
 the magnitudes of the individual angular momenta. In addition, we can either
know the quantum numbers $m_1$ and $m_2$, or the quantum numbers $j$ and
$m$, but we cannot know both pairs of quantum numbers at the same time. 
We can write a conventional completeness relation for both sets of
eigenkets:
\begin{align}\label{e5.212a}
\sum_{m_1}\sum_{m_2 }|j_1,j_2; m_1, m_2\rangle \langle j_1,j_2; m_1, m_2|& =1,\\[0.5ex]
\sum_{j}\sum_{m} |j_1,j_2; j, m\rangle \langle j_1,j_2; j, m|& =1,
\end{align}
where the right-hand sides denote  the identity operator in the ket space corresponding
to states of given $j_1$ and $j_2$. The summation is over all allowed values
of $m_1$, $m_2$, $j$, and $m$.

As we have seen, the operator group $J_1^{\,2}$, $J_2^{\,2}$, $J^{\,2}$, and $J_z$
is incompatible with the group $J_1^{\,2}$, $J_2^{\,2}$, $J_{1\,z}$, and $J_{2\,z}$.
This means that if the system is in a simultaneous eigenstate of the former group
then, in general, it is not in an eigenstate of the latter. In other words,
if the quantum numbers $j_1$, $j_2$, $j$, and $m$ are known with
certainty then a measurement of the quantum numbers $m_1$ and $m_2$ will
give a range of possible values. We can use the completeness relation
(\ref{e5.212a}) to write
\begin{equation}
|j_1,j_2;j,m\rangle = \sum_{m_1}\sum_{m_2} \langle j_1,j_2;m_1,m_2|j_1,j_2;j,m\rangle
|j_1,j_2;m_1,m_2\rangle.
\end{equation}
Thus, we can write the eigenkets of the first group of operators
as a weighted sum of the eigenkets of the second set. The weights,
$\langle j_1,j_2;m_1,m_2|j_1,j_2;j,m\rangle$, are called the {\em Clebsch-Gordon
coefficients}. If the system is in a state where a measurement of
$J_1^{\,2}, J_2^{\,2}, J^{\,2}$, and $J_z$ is bound to give the results
$j_1\,(j_1+1)\,\hbar^2, j_2\,(j_2+1)\,\hbar^2, j\,(j+1)\,\hbar^2$,
and $j_z\,\hbar$, respectively, then a measurement of $J_{1\,z}$ and $J_{2\,z}$
will give the results $m_1\,\hbar$ and $m_2\,\hbar$, respectively,  with
probability $|\langle j_1,j_2;m_1,m_2|j_1,j_2;j,m\rangle|^{\,2}$. 

The Clebsch-Gordon coefficients possess a number of very important properties.
First, the coefficients are zero unless
\begin{equation}\label{e5.214}
m = m_1 + m_2.
\end{equation}
To prove this, we note that
\begin{equation}
(J_z - J_{1\,z} - J_{2\,z})\, |j_1,j_2; j, m\rangle =0.
\end{equation}
Forming the inner product with $\langle j_1, j_2; m_1, m_2|$, we obtain
\begin{equation}
(m-m_1-m_2) \,\langle j_1, j_2; m_1, m_2|j_1,j_2; j, m\rangle=0,
\end{equation}
which proves the assertion. Thus, the $z$-components of different angular momenta
add algebraically. So, an electron in an $l=1$ state, with orbital
angular momentum $\hbar$, and spin angular momentum $\hbar/2$, projected along the
$z$-axis,  constitutes a state whose total angular momentum projected
along the $z$-axis is $3\,\hbar/2$. What is uncertain is the magnitude of the
total angular momentum. 

Second, the coefficients vanish unless
\begin{equation}\label{e5.217}
|j_1-j_2| \leq j \leq j_1+j_2.
\end{equation}
We can assume, without loss of generality, that $j_1\geq j_2$. We know,
from Equation~(\ref{e5.214}), that for given
$j_1$ and $j_2$ the largest possible value of $m$ is $j_1+j_2$ (because 
$j_1$ is the largest possible value of $m_1$, {\rm etc}.). This implies that
the largest possible value of $j$ is $j_1+j_2$ (since, by definition,
the largest value of $m$ is equal to $j$).
Now, there are $(2\,j_1+1)$ allowable values of $m_1$ and $(2\,j_2+1)$ allowable
values of $m_2$. Thus, there are $(2\,j_1+1)\,(2\,j_2+1)$ independent
eigenkets, $|j_1, j_2; m_1, m_2\rangle$, needed to span the ket space
corresponding to fixed $j_1$ and $j_2$. Because the eigenkets
$|j_1, j_2; j, m\rangle$ span the same space, they must also form
a set of  $(2\,j_1+1)\,(2\,j_2+1)$ independent kets. In other words, there
can only be $(2\,j_1+1)\,(2\,j_2+1)$ distinct  allowable values of the quantum numbers
$j$ and $m$. For each allowed value of $j$, there are $2\,j+1$ allowed values
of $m$. We have already seen that the maximum allowed value of $j$ is
$j_1+j_2$. It is easily seen that if the minimum allowed value of
$j$ is $j_1-j_2$ then the total number of allowed values of $j$ and $m$
is  $(2\,j_1+1)\,(2\,j_2+1)$: {\rm i.e.},
\begin{equation}
\sum_{j=j_1-j_2,j_1+j_2} (2\,j+1) \equiv (2\,j_1+1)\,(2\,j_2+1).
\end{equation}
This proves our assertion. 

Third, the sum of the modulus squared of all of the Clebsch-Gordon coefficients
is unity: {\rm i.e.},
\begin{equation}\label{e5.219}
\sum_{m_1}\sum_{m_2} |\langle j_1,j_2;m_1,m_2|j_1,j_2;j,m\rangle|^{\,2} =1.
\end{equation}
This assertion is proved as follows:
\begin{align}
\langle j_1, j_2; j, m| j_1, j_2; j, m\rangle &=\sum_{m_1}\sum_{m_2} \langle j_1, j_2; j, m|j_1, j_2; m_1, m_2\rangle
\langle j_1, j_2; m_1, m_2|j_1, j_2; j, m\rangle\nonumber\\[0.5ex]
&=\sum_{m_1}\sum_{m_2}  |\langle j_1,j_2;m_1,m_2|j_1,j_2;j,m\rangle|^{\,2} =1,
\end{align}
where use has been made of the completeness relation (\ref{e5.212a}).

Finally, the Clebsch-Gordon coefficients obey two recursion relations. 
To obtain these relations, we start from
\begin{equation}
J^{\pm} |j_1,j_2;j,m\rangle = (J_1^\pm + J_2^\pm )\sum_{m_1'}\sum_{m_2'} \langle j_1, j_2; m_1', m_2'|j_1, j_2; j, m\rangle
|j_1, j_2; m_1', m_2'\rangle.
\end{equation}
Making use of the well-known properties of the shift operators,
which are specified  by Equations~(\ref{e5.44a})--(\ref{e5.44b}), we obtain
\begin{align}
\sqrt{j\,(j+1)- m\,(m\pm 1)}\, |j_1,j_2;j,m\pm 1\rangle &=
\sum_{m_1'}\sum_{m_2'} \left( \sqrt{j_1\,(j_1+1)- m_1'\,(m_1'\pm 1)}\,
|j_1, j_2; m_1'\pm 1, m_2'\rangle\right.\nonumber\\[0.5ex]
& \left.+\sqrt{ j_2\,(j_2+1)- m_2'\,(m_2'\pm 1)}\,
|j_1, j_2; m_1', m_2'\pm 1\rangle\right) \nonumber\\[0.5ex]
 &\times\langle j_1, j_2; m_1', m_2'|j_1, j_2; j, m\rangle.
\end{align}
Taking the inner product with $\langle j_1, j_2; m_1, m_2|$, and making
use of the orthonormality property of the basis eigenkets, we obtain 
the desired recursion relations:
\begin{align}\label{e5.223}
\sqrt{j\,(j+1)- m\,(m\pm 1)}\,\langle j_1, j_2; m_1, m_2|j_1,j_2;j, m\pm 1\rangle=
&\sqrt{j_1\,(j_1+1) - m_1\,(m_1\mp 1)}\nonumber\\[0.5ex]
&\times\langle j_1, j_2; m_1\mp 1, m_2|j_1,j_2;j, m\rangle\nonumber \\[0.5ex]
&+ \sqrt{j_2\,(j_2+1) - m_2\,(m_2\mp 1)}\nonumber\\[0.5ex]
&\times\langle j_1, j_2; m_1, m_2\mp 1|j_1,j_2;j, m\rangle.
\end{align}
It is clear, from the absence of complex coupling coefficients in the above relations,
that we can always choose the Clebsch-Gordon coefficients to be real numbers.
This is  convenient, because it ensures that the inverse Clebsch-Gordon
coefficients, $\langle j_1, j_2; j, m|j_1, j_2; m_1, m_2\rangle$, are
identical to the Clebsch-Gordon coefficients. In other words,
\begin{equation}
\langle j_1, j_2; j, m|j_1, j_2; m_1, m_2\rangle = 
\langle j_1, j_2; m_1, m_2|j_1, j_2; j, m\rangle.
\end{equation}
The inverse Clebsch-Gordon coefficients are the weights in the expansion
of the $|j_1,j_2; m_1, m_2\rangle$ in terms of the $|j_1, j_2; j,m\rangle$:
\begin{equation}
|j_1,j_2; m_1,m_2\rangle = \sum_{j}\sum_m \langle j_1, j_2; j, m|j_1, j_2; m_1, m_2\rangle |j_1,j_2; j,m\rangle.
\end{equation}

It turns out that the recursion relations (\ref{e5.223}), together with the normalization
condition (\ref{e5.219}), are sufficient to completely determine the Clebsch-Gordon
coefficients to within an arbitrary sign (multiplied into
all of the coefficients). This sign is fixed by convention. The easiest
way of demonstrating this assertion is by considering a specific example. 
 
Let us add the angular momentum of two spin one-half systems: {\rm e.g.}, two
electrons at rest. So, $j_1=j_2=1/2$. We know, from general principles,
that $|m_1| \leq 1/2$ and $|m_2|\leq 1/2$. We also know, from Equation~(\ref{e5.217}),
that $0\leq j\leq 1$, where the allowed values of $j$ differ by integer amounts.
It follows that either $j=0$ or $j=1$. Thus, two spin one-half systems can
be combined  to form either a spin zero  system or a spin one system. 
It is helpful to arrange  all of the possibly non-zero Clebsch-Gordon coefficients
in a table:
$$
\begin{tabular}{||c|c||c|c|c||c|}\hline
$m_1$    & $m_2$      &     &    &      &     \\ \hline\hline
1/2      &  1/2       & ?   & ?  &  ?   & ?   \\ \hline
1/2      & -1/2       & ?   & ?  &  ?   & ?   \\ \hline
-1/2     &  1/2       & ?   & ?  &  ?   & ?   \\ \hline
-1/2     & -1/2       & ?   & ?  &  ?   & ?   \\ \hline\hline
 $\scriptstyle j_1=1/2$        &  $j$       & 1   & 1  &  1   & 0   \\  \hline
  $\scriptstyle j_2=1/2$       &  $m$       & 1   & 0  & -1   & 0   \\ \hline
\end{tabular}
$$
The box in this table corresponding to $m_1=1/2, m_2=1/2, j=1, m=1$ gives
the Clebsch-Gordon coefficient $\langle 1/2, 1/2; 1/2, 1/2
| 1/2, 1/2; 1, 1\rangle$,
or the inverse Clebsch-Gordon coefficient $\langle  1/2, 1/2; 1, 1|
1/2, 1/2; 1/2, 1/2
\rangle$. All the boxes contain question marks  because, at this stage,  we do
not know the values of any  Clebsch-Gordon coefficients.

A Clebsch-Gordon coefficient is automatically zero unless
$m_1+m_2=m$. In other words, the $z$-components of angular momentum have to
add algebraically. Many of the boxes in the above table correspond to
$m_1+m_2\neq m$. We immediately conclude that these boxes must contain zeroes:
{\rm  i.e.},
$$
\begin{tabular}{||c|c||c|c|c||c|}\hline
$m_1$    & $m_2$      &     &    &      &     \\ \hline\hline
1/2      &  1/2       & ?   & 0  &  0   & 0   \\ \hline
1/2      & -1/2       & 0   & ?  &  0   & ?   \\ \hline
-1/2     &  1/2       & 0   & ?  &  0   & ?   \\ \hline
-1/2     & -1/2       & 0   & 0  &  ?   & 0   \\ \hline\hline
 $\scriptstyle j_1=1/2$        &  $j$       & 1   & 1  &  1   & 0   \\  \hline
  $\scriptstyle j_2=1/2$       &  $m$       & 1   & 0  & -1   & 0   \\ \hline
\end{tabular}
$$

The normalization condition (\ref{e5.219}) implies  that the sum of the squares
of all the rows and columns of the above table must be unity. There are two
rows and two columns that only contain a single non-zero entry. We conclude that
these entries must be $\pm 1$, but we have no way of determining the 
signs at present. Thus,
$$
\begin{tabular}{||c|c||c|c|c||c|}\hline
$m_1$    & $m_2$      &     &    &      &     \\ \hline\hline
1/2      &  1/2       & $\pm 1 $  & 0  &  0   & 0   \\ \hline
1/2      & -1/2       & 0   & ?  &  0   & ?   \\ \hline
-1/2     &  1/2       & 0   & ?  &  0   & ?   \\ \hline
-1/2     & -1/2       & 0   & 0  &  $\pm 1$   & 0   \\ \hline\hline
 $\scriptstyle j_1=1/2$        &  $j$       & 1   & 1  &  1   & 0   \\  \hline
  $\scriptstyle j_2=1/2$       &  $m$       & 1   & 0  & -1   & 0   \\ \hline
\end{tabular}
$$

Let us evaluate the  recursion relation (\ref{e5.223})  for $j_1=j_2=1/2$, with
$j=1$, $m=0$, $m_1=m_2=\pm 1/2$, taking the upper/lower sign. We
find that
\begin{equation}
\langle 1/2, -1/2|1, 0\rangle
+ \langle -1/2, 1/2|1, 0\rangle =\sqrt{2}\,\langle 1/2, 1/2|1,1\rangle=\pm\sqrt{2},
\end{equation}
and 
\begin{equation}
 \langle 1/2, -1/2|1, 0\rangle
+ \langle -1/2, 1/2|1, 0\rangle =\sqrt{2}\, \langle -1/2, -1/2|1,-1\rangle
=\pm\sqrt{2}.
\end{equation}
Here, the $j_1$ and $j_2$ labels have been suppressed for ease of notation.
We also know that 
\begin{equation}
\langle 1/2, -1/2|1, 0\rangle^{\,2} + 
\langle -1/2, 1/2|1, 0\rangle^{\,2} = 1,
\end{equation}
from the normalization condition. The only real solutions to the above set
of equations are
\begin{align}
\sqrt{2} \,\langle 1/2, -1/2|1, 0\rangle &= \sqrt{2} \,\langle -1/2, 1/2|1, 0\rangle
\nonumber\\[0.5ex]
&= \langle 1/2,1/2|1,1\rangle = \langle 1/2,1/2|1,-1\rangle = \pm 1.
\end{align}
The choice of sign is arbitrary---the conventional choice is a positive
sign. Thus, our table now reads
$$
\begin{tabular}{||c|c||c|c|c||c|}\hline
$m_1$    & $m_2$      &     &    &      &     \\ \hline\hline
1/2      &  1/2       & 1   & 0  &  0   & 0   \\ \hline
1/2      & -1/2       & 0   & $1/\sqrt{2}$  &  0   & ?   \\ \hline
-1/2     &  1/2       & 0   & $1/\sqrt{2}$ &  0   & ?   \\ \hline
-1/2     & -1/2       & 0   & 0  &  1   & 0   \\ \hline\hline
 $\scriptstyle j_1=1/2$        &  $j$       & 1   & 1  &  1   & 0   \\  \hline
  $\scriptstyle j_2=1/2$       &  $m$       & 1   & 0  & -1   & 0   \\ \hline
\end{tabular}
$$

We could fill in the remaining unknown entries of our table by using the recursion
relation again. However, an easier method is to observe that the rows and columns
of the table must all be mutually orthogonal. That is, the dot product
of a row with any other row must be zero. Likewise, for the dot product of
a column with any other column. This follows because the entries in the
table give the expansion coefficients of one of our alternative sets of eigenkets
in terms of the other set, and each set of eigenkets contains
mutually orthogonal vectors with unit norms. The normalization
condition tells us that the dot product of a row or column with itself must
be unity. The only way that the dot product of the fourth column with
the second column can be zero is if the unknown entries are equal and opposite. 
The requirement that the dot product of the fourth column with itself is
unity tells us that the magnitudes of the unknown entries have to be $1/\sqrt{2}$. 
The unknown entries are undetermined to an arbitrary sign multiplied into them both.
Thus, the final form of our table (with the conventional choice of arbitrary
signs) is
$$
\begin{tabular}{||c|c||c|c|c||c|}\hline
$m_1$    & $m_2$      &     &    &      &     \\ \hline\hline
1/2      &  1/2       & 1   & 0  &  0   & 0   \\ \hline
1/2      & -1/2       & 0   & $1/\sqrt{2}$  &  0   & $1/\sqrt{2}$   \\ \hline
-1/2     &  1/2       & 0   & $1/\sqrt{2}$ &  0   & -$1/\sqrt{2}$   \\ \hline
-1/2     & -1/2       & 0   & 0  &  1   & 0   \\ \hline\hline
 $\scriptstyle j_1=1/2$        &  $j$       & 1   & 1  &  1   & 0   \\  \hline
  $\scriptstyle j_2=1/2$       &  $m$       & 1   & 0  & -1   & 0   \\ \hline
\end{tabular}
$$

The table can be read in one of two ways. The columns give the expansions
of the eigenstates of overall  angular momentum in terms of the eigenstates
 of the individual
angular momenta of the two component systems. Thus, the second column
tells us that
\begin{equation}
|1,0\rangle = \frac{1}{\sqrt{2}} \left(|1/2,-1/2\rangle  + |-1/2,1/2\rangle\right).
\end{equation}
The ket on the left-hand side is a $|j,m\rangle$ ket, whereas those on the
right-hand side are $|m_1, m_2\rangle$ kets. The rows give the expansions
of the eigenstates of individual angular momentum in terms of those of overall
angular momentum. Thus, the second row tells us that
\begin{equation}
|1/2,-1/2\rangle = \frac{1}{\sqrt{2}} \left(|1,0\rangle + |0,0\rangle\right).
\end{equation}
Here, the ket on the left-hand side is a  $|m_1, m_2\rangle$ ket, whereas those
on the right-hand side are $|j, m\rangle$ kets.

Note that our table is really a combination of two sub-tables, one involving
$j=0$ states, and one involving $j=1$ states. The Clebsch-Gordon coefficients
corresponding to two different choices of $j$ are completely independent:
{\rm i.e.}, there is no recursion relation linking Clebsch-Gordon coefficients
corresponding to different values of $j$. Thus, for every choice of $j_1$, $j_2$,
and $j$ we can construct a table of Clebsch-Gordon coefficients corresponding
to the different allowed values of $m_1$, $m_2$, and $m$ (subject to the
constraint that $m_1+m_2=m$). A complete knowledge of angular momentum addition
is equivalent to a knowing all possible tables of Clebsch-Gordon coefficients.
These tables are listed (for moderate values of
$j_1, j_2$ and $j$) in many standard reference books. 

\subsection*{Exercises}
\begin{enumerate}[label=\thechapter.\arabic*,leftmargin=*,widest=9.20]

\item Calculate the Clebsch-Gordon coefficients for adding spin one-half
to spin one. 

\item Calculate the Clebsch-Gordon coefficients for adding spin one 
to spin one. 

\item An electron in a hydrogen atom occupies the combined spin
and position state whose wavefunction is 
$$
\psi = R_{2\,1}(r)\,\left[\sqrt{1/3}\,Y_{1\,0}(\theta,\varphi)\,\chi_+ + \sqrt{2/3}\,Y_{1\,1}(\theta,\varphi)\,\chi_-\right].
$$
\begin{enumerate}
\item What values would a measurement of $L^2$ yield, and with
what probabilities?
\item Same for $L_z$.
\item Same for $S^2$.
\item Same for $S_z$.
\item Same for $J^{\,2}$.
\item Same for $J_z$.
\item What is the probability density for finding the electron at
$r$, $\theta$, $\varphi$?
\item What is the probability density for finding the electron in the
spin up state (with respect to the $z$-axis) at radius $r$?
\end{enumerate}

\item In a low energy neutron-proton system (with zero orbital angular
momentum) the potential energy is given by
$$
V({\bf x}) = V_1(r) + V_2(r)\left[3\,\frac{(\bsigma_n\cdot{\bf x})\,(\bsigma_p\cdot
{\bf x})}{r^2} -\bsigma_n\cdot\bsigma_p\right] + V_3(r)\,\bsigma_n\cdot\bsigma_p,
$$
where $r=|{\bf x}|$, $\bsigma_n$ denotes the vector of the Pauli matrices of the neutron,
and $\bsigma_p$ denotes the vector of the Pauli matrices of the proton. Calculate
the potential energy for the neutron-proton system:
\begin{enumerate}
\item In the spin singlet (i.e., spin zero) state.
\item In the spin triplet (i.e., spin one) state.
\end{enumerate}
[Hint: Calculate the expectation value of $V({\bf x})$ with respect to the overall spin state.]

\item Consider two electrons in a spin singlet (i.e., spin zero) state.
\begin{enumerate}
\item If a measurement of the spin of one of the electrons shows that it
is in the state with $S_z=\hbar/2$, what is the probability that a
measurement of the $z$-component of the spin of the
other electron yields $S_z=\hbar/2$?
\item If a measurement of the spin of one of the electrons shows
that it is in the state with $S_y=\hbar/2$, what is the probability that a
measurement of the $x$-component of the spin of the
other electron yields $S_x=-\hbar/2$?
\item Finally, if electron 1 is in a spin state described by $\cos\alpha_1\,\chi_+
+ \sin\alpha_1\,{\rm e}^{\,{\rm i}\,\beta_1}\,\chi_-$, and
electron 2 is in a spin state described by  $\cos\alpha_2\,\chi_+
+ \sin\alpha_2\,{\rm e}^{\,{\rm i}\,\beta_2}\,\chi_-$, what is
the probability that the two-electron spin state is a triplet (i.e., spin one) state?
\end{enumerate}
\end{enumerate}