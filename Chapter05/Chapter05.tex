\chapter{Angular Momentum}\label{s5}
\section{Orbital Angular Momentum}\label{s5.1}
Consider a particle described by the Cartesian coordinates 
$(x, y, z)\equiv {\bf r}$
 and their conjugate momenta $(p_x, p_y, p_z)\equiv {\bf p}$. The classical
definition of the orbital angular momentum of such a particle about the
origin is ${\bf L} = {\bf r}\times{\bf p}$, giving
\begin{eqnarray}\label{e5.1a}
L_x &=& y\, p_z - z\, p_y,\\[0.5ex]
L_y &=& z\, p_x - x\, p_z,\\[0.5ex]
L_z &=& x\,p_y - y \,p_x.\label{e5.1c}
\end{eqnarray}
Let us assume that the operators $(L_x, L_y, L_z)\equiv {\bf L}$ which
represent the components of
orbital angular momentum in quantum mechanics can be defined in
an analogous manner to the corresponding components of
classical angular momentum. In other words, we are 
going to assume that the above equations specify  the angular momentum operators
in terms of the position and linear momentum operators. Note that $L_x$, $L_y$,
and $L_z$ are Hermitian, so they represent things which can, in principle,
be measured. Note, also, that there is no ambiguity regarding the order
in which operators appear in products on the right-hand sides of Eqs.~(\ref{e5.1a})--(\ref{e5.1c}),
since all of the products consist of operators which commute. 

The fundamental commutation relations satisfied by the
position  and linear momentum operators are [see Eqs.~(\ref{e3.14a})--(\ref{e3.14c})]
\begin{eqnarray}\label{e5.2a}
[x_i, x_j] &=& 0,\\[0.5ex]
[p_i, p_j] &=&0,\\[0.5ex]
[x_i, p_j] &=& {\rm i}\,\hbar \,\delta_{ij},\label{e5.2c}
\end{eqnarray}
where $i$ and $j$ stand for either $x$, $y$, or $z$. 
Consider the commutator of the operators $L_x$ and $L_z$:
\begin{eqnarray}
[L_x, L_y] &= &[(y\,p_z-z\,p_y), (z\,p_x-x \,p_z)]
= y\,[p_z, z]\,p_x + x\,p_y\,[z, p_z] \nonumber\\[0.5ex]
&=& {\rm i}\,\hbar\,(-y \,p_x+ x\,p_y) = {\rm i}\,\hbar\, L_z.
\end{eqnarray}
The cyclic permutations of the above result yield
the fundamental commutation relations satisfied 
by the   components of an angular momentum:
\begin{eqnarray}\label{e5.4a}
[L_x, L_y] &= &{\rm i}\,\hbar\, L_z,\\[0.5ex]
[L_y, L_z] &= &{\rm i}\,\hbar\, L_x,\\[0.5ex]
[L_z, L_x] &= &{\rm i}\,\hbar\, L_y.\label{e5.4c}
\end{eqnarray}
These can be summed up more succinctly by writing
\begin{equation}\label{e5.5}
{\bf L}\times {\bf L} = {\rm i}\,\hbar \,{\bf L}.
\end{equation}
The three commutation relations (\ref{e5.4a})--(\ref{e5.4c})  are the foundation for the whole
theory of angular momentum in quantum mechanics. Whenever we encounter 
three operators having these commutation relations, we know that the
dynamical variables  which they represent have identical properties
to those of the components of an
 angular momentum (which we are about to derive). In fact,
we shall  assume that {\em any three operators which satisfy the commutation
relations (\ref{e5.4a})--(\ref{e5.4c}) represent the components of an angular momentum}. 

Suppose that there are $N$ particles in the system, with
angular momentum vectors ${\bf L}_i$ (where $i$ runs from 1 to $N$). 
Each of these vectors satisfies Eq.~(\ref{e5.5}), so that
\begin{equation}\label{e5.6}
{\bf L}_i\times {\bf L}_i = {\rm i}\,\hbar \,{\bf L}_i.
\end{equation}
However, we expect  the angular momentum operators
 belonging to  different particles to commute, since they represent different
degrees of freedom of the system. So,
we can write
\begin{equation}\label{e5.7}
{\bf L}_i\times {\bf L}_j + {\bf L}_j\times {\bf L}_i =0,
\end{equation}
for $i\neq j$. Consider the total angular momentum of the system, 
${\bf L} = \sum_{i=1}^N {\bf L}_i$. It is clear from Eqs.~(\ref{e5.6}) and (\ref{e5.7})
 that
\begin{eqnarray}
{\bf L} \times {\bf L}& =& \sum_{i=1}^N {\bf L}_i\times
\sum_{j=1}^N {\bf L}_j  = \sum_{i=1}^N {\bf L}_i \times
{\bf L}_i +\frac{1}{2}\!\sum_{i,j = 1}^N (  {\bf L}_i\times {\bf L}_j + {\bf L}_j\times {\bf L}_i) \nonumber\\[0.5ex]
&=& {\rm i}\,\hbar\,\sum_{i=1}^N {\bf L}_i = {\rm i}\,\hbar \,{\bf L}.\label{e5.8}
\end{eqnarray}
Thus, the sum of two or more angular momentum vectors satisfies the
same commutation relation as a primitive  angular momentum vector.
In particular, the total angular momentum of the system satisfies the
commutation relation (\ref{e5.5}).

The immediate conclusion which can be drawn from the commutation relations
(\ref{e5.4a})--(\ref{e5.4c}) is that the three components of an angular momentum vector cannot
be specified (or measured) simultaneously. In fact,  once we have specified one
component, the values of other two components become uncertain.  It is
conventional to specify the $z$-component, $L_z$. 

Consider the magnitude squared of the angular momentum vector, $L^2 \equiv
L_x^{~2} + L_y^{~2}+L_z^{~2}$. The commutator of $L^2$ and $L_z$ is
written
\begin{equation}
[L^2, L_z] = [L_x^{~2}, L_z] + [L_y^{~2}, L_z] + [L_z^{~2}, L_z].
\end{equation}
It is easily demonstrated that
\begin{eqnarray}
[L_x^{~2}, L_z] &=& -{\rm i}\,\hbar\,(L_x\, L_y + L_y \,L_x),\\[0.5ex]
[L_y^{~2}, L_z] &=& +{\rm i}\,\hbar\,(L_x\,L_y + L_y \,L_x),\\[0.5ex]
[L_z^{~2}, L_z] &=& 0,
\end{eqnarray}
so 
\begin{equation}\label{e5.11}
[L^2, L_z] = 0.
\end{equation}
Since there is nothing special about the $z$-axis, we conclude that $L^2$ also
commutes with $L_x$ and $L_y$. It is clear from Eqs.~(\ref{e5.4a})--(\ref{e5.4c}) and
(\ref{e5.11}) that the best we
can do in quantum mechanics is to specify the 
magnitude of an angular momentum vector 
along with {\em one} of its components (by convention, the $z$-component).

It is convenient to define the {\em shift operators} $L^+$ and $L^-$:
\begin{eqnarray}
L^+ &=& L_x + {\rm i}\, L_y,\\[0.5ex]
L^- &=& L_x -{\rm i} \,L_y.
\end{eqnarray}
Note that
\begin{eqnarray}\label{e5.13a}
[L^+, L_z ] &=& -\hbar\,L^+,\\[0.5ex]
[L^-, L_z] &=& +\hbar\,L^-,\\[0.5ex]
[L^+, L^-] &=& 2\,\hbar\,L_z.
\end{eqnarray}
Note, also, that both  shift operators commute with $L^2$. 

\section{Eigenvalues of Angular Momentum}\label{s5.2}
Suppose that the simultaneous eigenkets of $L^2$ and $L_z$ are completely
specified by two quantum numbers, $l$ and $m$. These  kets are denoted
$|l, m\rangle$. The quantum number $m$ is defined by
\begin{equation}
L_z \,|l, m\rangle = m\,\hbar |l, m\rangle.
\end{equation}
Thus, $m$ is the eigenvalue of $L_z$ divided by $\hbar$. It is possible
to write such an equation because $\hbar$ has the dimensions of angular momentum.
Note that $m$ is a real number, since  $L_z$ is an Hermitian operator. 

We can write
\begin{equation}\label{e5.15}
L^2 |l, m\rangle = f(l,m)\, \hbar^2\,|l, m\rangle,
\end{equation}
without loss of generality, 
where $f(l,m)$ is some real dimensionless function of $l$ and $m$. Later on,
we will show that $f(l,m) = l\,(l+1)$. 
Now, 
\begin{equation}\label{e5.16}
\langle l, m | L^2 - L_z^{~2} |l, m\rangle =\langle l, m |
f(l, m) \,\hbar^2 - m^2\, \hbar^2 |l, m\rangle =[f(l,m) - m^2] \hbar^2,
\end{equation}
assuming that the $|l, m\rangle$ have unit norms. However,
\begin{eqnarray}
\langle l, m | L^2 - L_z^{~2}|l, m\rangle &=&\langle l, m |
L_x^{~2} + L_y^{~2} |l, m\rangle\nonumber 
\\[0.5ex]&=& \langle l, m |L_x^{~2}|l, m\rangle+
\langle l, m|L_y^{~2}|l, m\rangle.
\end{eqnarray}
It is easily demonstrated that
\begin{equation}\label{e5.18}
\langle A|\xi^2|A\rangle\geq 0,
\end{equation}
where $|A\rangle$ is a general  ket, and $\xi$ is an Hermitian operator. 
The proof follows from the observation that 
\begin{equation}
\langle A|\xi^2|A\rangle
= \langle A|\xi^{\dag}\, \xi |A\rangle = \langle B| B\rangle, 
\end{equation}
where $|B\rangle = \xi |A\rangle$, plus the fact that $\langle B|B\rangle\geq 0$
for a general  ket $|B\rangle$ [see Eq.~(\ref{e2.21})]. It follows from
Eqs.~(\ref{e5.16})--(\ref{e5.18}) that
\begin{equation}\label{e5.20}
m^2 \leq f(l,m).
\end{equation}

Consider the effect of the shift operator $L^+$ on the eigenket $|l, m\rangle$.
It is easily demonstrated that
\begin{equation}
L^2 (L^+ |l, m\rangle) = \hbar^2\, f(l,m)\, (L^+ |l,m\rangle),
\end{equation}
where use has been made of Eq.~(\ref{e5.15}), plus
the fact that $L^2$ and $L_z$ commute. 
It follows that the ket $L^+ |l,m\rangle$ has the same
eigenvalue of $L^2$ as the ket $|l,m\rangle$. Thus, the shift operator
$L^+$ does not affect the magnitude of the angular momentum of 
any eigenket it acts upon. Note that
\begin{eqnarray}
L_z \,L^+ |l, m\rangle = (L^+ L_z + [L_z, L^+])|l,m\rangle
&=& (L^+ L_z + \hbar\, L^+) |l,m\rangle\nonumber\\[0.5ex]
&=& (m+1)\,\hbar \,L^+|l, m\rangle,
\end{eqnarray}
where use has been made of Eq.~(\ref{e5.13a}). The above equation implies
that $L^+ |l,m\rangle$ is proportional to $|l, m+1\rangle$. We can
write
\begin{equation}\label{e5.23}
L^+ |l ,m\rangle = c^+_{l, m}\, \hbar\,|l, m+1\rangle,
\end{equation}
where $c^+_{l, m}$ is a number. It is clear that when the operator $L^+$ 
acts on a simultaneous  eigenstate of $L^2$ and $L_z$, 
the eigenvalue of $L^2$ remains unchanged, but the  eigenvalue
of $L_z$ is increased by $\hbar$. For this reason, $L^+$  is called
a {\em raising operator}. 

Using similar arguments to those given above, it is possible
to demonstrate that
\begin{equation}\label{e5.24}
L^- |l ,m\rangle = c^-_{l, m}\,\hbar\, |l, m-1\rangle.
\end{equation}
Hence, $L^-$ is called a {\em lowering operator}. 

The shift operators step the value of $m$ up and down by unity
each time they operate on one of the simultaneous eigenkets of 
$L^2$ and $L_z$. It would appear, at first sight, that any value of
$m$ can be obtained by applying the shift operators a sufficient
number of times. However, according to Eq.~(\ref{e5.20}), there is
a definite upper bound to the values that $m^2$ can take. This 
bound is determined by  the eigenvalue of $L^2$
[see Eq.~(\ref{e5.15})]. It follows that there is a maximum and a minimum
possible
value which  $m$ can take. 
Suppose that we attempt to raise the value
of $m$ above its maximum value $m_{\rm max}$. Since there is  no
state with $m> m_{\rm max}$, we must have
\begin{equation}
L^+ |l, m_{\rm max}\rangle = |0\rangle.
\end{equation}
This implies that 
\begin{equation}\label{e5.26}
L^-\, L^+ |l, m_{\rm max}\rangle = |0\rangle.
\end{equation}
However,
\begin{equation}
L^-\, L^+ = L_x^{~2} + L_y^{~2} + {\rm i}\,[L_x,  L_y]
= L^2 - L_z^{~2} - \hbar \,L_z,
\end{equation}
so Eq.~(\ref{e5.26}) yields
\begin{equation}
(L^2 - L_z^{~2} - \hbar \,L_z) |l, m_{\rm max}\rangle = |0\rangle.
\end{equation}
The above equation can be rearranged to give
\begin{equation}
L^2 |l, m_{\rm max}\rangle = (L_z^{~2} + \hbar \,L_z) |l, m_{\rm max}\rangle 
= m_{\rm max}(m_{\rm max} + 1) \,\hbar^2\, |l, m_{\rm max}\rangle.
\end{equation}
Comparison of this equation with Eq.~(\ref{e5.15}) yields the result
\begin{equation}
f(l, m_{\rm max}) = m_{\rm max} (m_{\rm max} + 1).
\end{equation}
But, when $L^-$ operates on $|n, m_{\rm max}\rangle$ it generates
$|n, m_{\rm max}-1\rangle$, $|n, m_{\rm max}-2\rangle$, {\em etc}. Since
the lowering operator does not change the eigenvalue of $L^2$, all of these states
must correspond to the same value of $f$, namely $m_{\rm max}(m_{\rm max} + 1)$.
Thus,
\begin{equation}
L^2 |l, m\rangle = m_{\rm max}(m_{\rm max} + 1)\,\hbar^2 |l, m\rangle.
\end{equation}
At this stage, we can give the unknown quantum number $l$ the value $m_{\rm max}$,
without loss of generality. 
We can also write the above equation in the form
\begin{equation}
L^2 |l, m\rangle = l\,(l+1)\, \hbar^2 |l, m\rangle.
\end{equation}

It is easily seen that
\begin{equation}
L^- \,L^+ |l, m\rangle = (L^2 - L_z^{~2}-\hbar\, L_z)|l, m \rangle
= \hbar^2 [l\,(l+1) - m\,(m+1)]|l,m\rangle.
\end{equation}
Thus,
\begin{equation}
\langle l,m| L^- \,L^+| l,m\rangle =\hbar^2 [l\,(l+1) - m\,(m+1)].
\end{equation}
However, we also know that
\begin{equation}
\langle l,m| L^- \,L^+ |l,m\rangle = \langle l, m| L^-\, \hbar\, c^+_{l,m}
|l,m+1\rangle = \hbar^2\,  c^+_{l,m} \,c^{-}_{l, m+1},
\end{equation}
where use has been made of Eqs.~(\ref{e5.23}) and (\ref{e5.24}). 
It follows that
\begin{equation}\label{e5.36}
 c^+_{l,m}\, c^{-}_{l, m+1} = [l\,(l+1) - m\,(m+1)].
\end{equation}

Consider the following:
\begin{eqnarray}
\langle l, m| L^- |l, m+1\rangle &=&
\langle l, m| L_x|l,m+1 \rangle - {\rm i}\, \langle l, m| L_y|l,m+1 \rangle \nonumber
\\[0.5ex]
&=& \langle l, m+1| L_x|l,m \rangle^\ast - {\rm i}\, \langle l, m+1| L_y|l,m \rangle^\ast\nonumber\\[0.5ex]
&=&( \langle l, m+1| L_x|l,m \rangle + {\rm i}\, 
\langle l, m+1| L_y|l,m \rangle)^\ast\nonumber\\[0.5ex]
&=& \langle l, m+1|L^+|l, m\rangle^\ast,
\end{eqnarray}
where use has been made of the fact that $L_x$ and $L_y$ are Hermitian.
The above equation reduces to
\begin{equation}\label{e5.38}
c^-_{l, m+1} = (c_{l,m}^+)^\ast
\end{equation}
with the aid of Eqs.~(\ref{e5.23}) and (\ref{e5.24}). 

Equations (\ref{e5.36}) and (\ref{e5.38}) can be combined to give
\begin{equation}\label{e5.39}
|c^+_{l,m}|^2 = [l\,(l+1) - m \,(m+1)].
\end{equation}
The solution of the above equation is
\begin{equation}
c_{l, m}^+ = \sqrt{l\,(l+1)- m \,(m+1)}.
\end{equation}
Note that $c_{l,m}^+$ is undetermined to an arbitrary phase-factor
[{\em i.e.}, we can replace $c_{l, m}^+$, given above, by $c_{l, m}^+\exp({\rm i}\,\gamma)$,
where $\gamma$ is real, and we still satisfy Eq.~(\ref{e5.39})]. We have made the arbitrary, but convenient, choice that $c_{l,m}^+$ is real and positive. This is equivalent
to choosing  the relative phases of the eigenkets $|l, m\rangle$. 
According to Eq.~(\ref{e5.38}),
\begin{equation}\label{e5.41}
c_{l, m}^- = (c_{l, m-1}^+)^\ast =  \sqrt{l\,(l+1)- m\, (m-1)}.
\end{equation}

We have already seen that the inequality (\ref{e5.20}) implies that there is a
maximum and a minimum possible value of $m$. The maximum value of $m$
is denoted $l$. What is the minimum value? Suppose that we try
to lower the value of $m$ below its minimum value $m_{\rm min}$. Since
there is no state with $m<m_{\rm min}$, we must have
\begin{equation}
L^- |l, m_{\rm min}\rangle = 0.
\end{equation}
According to Eq.~(\ref{e5.24}), this implies that
\begin{equation}
c_{l,  m_{\rm min}}^- = 0.
\end{equation}
It can be seen from Eq.~(\ref{e5.41})  that $m_{\rm min} = -l$. 
We conclude that $m$ can take a ``ladder'' of discrete values, each rung differing
from its immediate neighbours by unity. The top rung is $l$, and the
bottom rung is  $-l$. There are only two possible choices for $l$.
Either it is an integer ({\em e.g.}, $l=2$, which  allows $m$ to take the values
$-2, -1, 0, 1, 2$), or it is a half-integer ({\em e.g.}, $l=3/2$, which  allows
$m$ to take the values $-3/2, -1/2, 1/2, 3/2$). We will prove in the next
section that an orbital angular momentum can only take integer values
of $l$. 

In summary, using  just  the fundamental commutation relations (\ref{e5.4a})--(\ref{e5.4c}),
plus the fact that $L_x$, $L_y$, and $L_z$ are Hermitian operators, we have
shown that the eigenvalues of $L^2\equiv L_x^{~2} + L_y^{~2}+L_z^{~2}$
can be written $l\,(l+1)\,\hbar^2$, where $l$ is an integer, or a half-integer. 
We have also demonstrated that the eigenvalues of $L_z$ can only
take the values $m\,\hbar$, where $m$ lies in the range $-l, -l+1,\cdots
l-1, l$. Let $|l, m\rangle$ denote a properly normalized simultaneous eigenket
of $L^2$ and $L_z$, belonging to the eigenvalues $l\,(l+1)\,\hbar^2$
and $m\,\hbar$, respectively.
We have shown that 
\begin{eqnarray}\label{e5.44a}
L^+ |l, m\rangle &=& \sqrt{l\,(l+1)-m\,(m+1)}\,\hbar\,|l, m+1\rangle\\[0.5ex]
L^- |l,m \rangle &=& \sqrt{l\,(l+1)-m\,(m-1)}\,\hbar\,|l, m-1\rangle,\label{e5.44b}
\end{eqnarray}
where $L^\pm = L_x \pm {\rm i} \,L_y$ are the so-called shift operators.

\section{Rotation Operators}\label{s5.3}
Consider a particle described by the spherical polar coordinates
$(r, \theta, \varphi)$. The classical momentum conjugate to the azimuthal
angle $\varphi$ is the $z$-component of angular momentum, $L_z$. 
According to Sect.~\ref{s3.5}, in quantum mechanics we can always adopt Schr\"{o}dinger's
representation, for  which ket space is spanned by the simultaneous eigenkets
of the position operators $r$, $\theta$, and $\phi$, and $L_z$ takes the
form
\begin{equation}\label{e5.45}
L_z = -{\rm i}\,\hbar\, \frac{\partial}{\partial \varphi}.
\end{equation}
We can do this because there is nothing in Sect.~\ref{s3.5} which specifies that
we have to use  Cartesian coordinates---the representation (\ref{e3.63}) works for 
any well-defined set of coordinates. 

Consider an operator $R(\Delta\varphi)$ which rotates the system an angle
$\Delta\varphi$ about the $z$-axis. This operator is very similar to the
operator $D(\Delta x)$, introduced in Sect.~\ref{s3.8}, which translates the system
a distance $\Delta x$ along the $x$ axis. 
We were able to demonstrate in Sect.~\ref{s3.8}
that
\begin{equation}
p_x = {\rm i}\,\hbar\, \lim_{\delta x\rightarrow 0}\frac{D(\delta x)-1}
{\delta x},
\end{equation}
where $p_x$ is the linear momentum conjugate to $x$. There is nothing
in our derivation of this result which specifies that $x$ has to be a Cartesian
coordinate. Thus, the result should apply just as well to an angular
coordinate. We conclude that
\begin{equation}\label{e5.47}
L_z = {\rm i}\,\hbar\, \lim_{\delta \varphi\rightarrow 0}\frac{R(\delta \varphi)-1}
{\delta \varphi}.
\end{equation}

According to Eq.~(\ref{e5.47}), we can write
\begin{equation}
R(\delta \varphi) = 1 -{\rm i}\,L_z\,\delta\varphi/\hbar
\end{equation}
in the limit $\delta\varphi\rightarrow 0$. In other words, the angular momentum 
operator $L_z$
can be used to rotate the system about the $z$-axis by an infinitesimal amount.
We say that $L_z$ is the {\em generator} of rotations about the $z$-axis. 
The above equation implies that 
\begin{equation}
R(\Delta\varphi) = \lim_{N\rightarrow\infty} \left(1-{\rm i} \,\frac{\Delta
\varphi}{N} \frac{L_z}{\hbar}\right)^N,
\end{equation}
which reduces to
\begin{equation}\label{e5.50}
R(\Delta\varphi) = \exp(-{\rm i}\,L_z \,\Delta\varphi/\hbar).
\end{equation}
Note that $R(\Delta\varphi)$ has all of the properties we would expect of
a rotation operator
\begin{eqnarray}
R(0) &=& 1,\\[0.5ex]
R(\Delta\varphi)\,R(-\Delta\varphi) &=& 1,\\[0.5ex]
R(\Delta\varphi_1)\,R(\Delta\varphi_2)& =& R(\Delta\varphi_1+ \Delta\varphi_2).
\end{eqnarray}

Suppose that the system is in a simultaneous eigenstate of $L^2$ and $L_z$. 
As before, this state is represented by the eigenket $|l,m\rangle$, where
the eigenvalue of $L^2$ is $l\,(l+1)\,\hbar^2$, and
the eigenvalue of $L_z$ is $m\,\hbar$.  We expect the wave-function
 to remain unaltered  if we rotate
the system $2\pi$ degrees about the $z$-axis. Thus,
\begin{equation}
R(2\pi)|l,m\rangle = \exp(-{\rm i}\,L_z \,2\pi/\hbar)|l,m\rangle 
= \exp(-{\rm i}\, 2\,\pi\,m) |l,m\rangle = |l,m\rangle.
\end{equation}
We conclude that $m$ must be an integer. This implies, from the previous
section, that $l$ must also be an integer. Thus, {\em orbital} angular momentum
can only take on {\em integer} values of the quantum numbers $l$ and $m$. 

Consider the action of the rotation operator $R(\Delta\varphi)$ on an eigenstate
possessing zero angular momentum about the $z$-axis ({\em i.e.}, an $m=0$
state). We have
\begin{equation}
R(\Delta\varphi)|l, 0\rangle =  \exp(0)| l, 0\rangle = |l, 0\rangle.
\end{equation}
Thus, the eigenstate is invariant to rotations about the $z$-axis. Clearly,
its wavefunction must be symmetric about the $z$-axis. 

There is nothing special about the $z$-axis, so we can write
\begin{eqnarray}\label{e5.54a}
R_x(\Delta\varphi_x) &=& \exp(-{\rm i}\,L_x \,\Delta\varphi_x/\hbar),\\[0.5ex]
R_y(\Delta\varphi_y) &=& \exp(-{\rm i}\,L_y \,\Delta\varphi_y/\hbar),\\[0.5ex]\label{e5.54c}
R_z(\Delta\varphi_y) &=& \exp(-{\rm i}\,L_z\, \Delta\varphi_z/\hbar),
\end{eqnarray}
by analogy with Eq.~(\ref{e5.50}). Here, $R_x(\Delta\varphi_x)$ denotes an operator
which rotates the system by an angle $\Delta\varphi_x$ about the $x$-axis, {\em etc}.
Suppose that the system is in an eigenstate of zero overall orbital angular momentum
({\em i.e.}, an $l=0$ state).
We know that the system is also in an eigenstate of zero orbital angular momentum 
about any particular axis. This follows because $l=0$ implies $m=0$, according
to  the
previous section, and we can choose the $z$-axis to point in any direction. 
Thus,
\begin{eqnarray}
R_x(\Delta \varphi_x) |0,0\rangle &=&\exp(0)|0,0\rangle = |0,0\rangle,\\[0.5ex]
R_y(\Delta \varphi_y) |0,0\rangle &=&\exp(0)|0,0\rangle = |0,0\rangle,\\[0.5ex]
R_z(\Delta \varphi_z) |0,0\rangle &=&\exp(0)|0,0\rangle = |0,0\rangle.
\end{eqnarray}
Clearly, a zero angular momentum state is invariant to rotations about {\em any}\/
axis.
Such a state  must possess a spherically symmetric wave-function. 

Note that a rotation about the $x$-axis does not commute with a rotation
about the $y$-axis. In other words, if the system is rotated an angle
$\Delta\varphi_x$ about the $x$-axis, and then $\Delta\varphi_y$ about the
$y$-axis, it ends up in a different state to that obtained by rotating
an angle $\Delta\varphi_y$ about the
$y$-axis, and then $\Delta\varphi_x$ about the $x$-axis. In quantum
mechanics, this implies that $R_y(\Delta\varphi_y)\,R_x(\Delta\varphi_x)
\neq R_x(\Delta\varphi_x)\,R_y(\Delta\varphi_y)$, or 
$L_y \,L_x \neq L_x\, L_y$, [see  Eqs.~(\ref{e5.54a})--(\ref{e5.54c})]. Thus, the noncommuting
nature of the angular momentum operators is a direct consequence
of  the fact that
rotations do not commute. 

\section{Eigenfunctions of Orbital Angular Momentum}\label{s5.4}
In Cartesian coordinates, the three components of orbital angular
momentum can be written 
\begin{eqnarray}
L_x &=& -{\rm i}\,\hbar\left(y\,\frac{\partial}{\partial z} - z\,\frac{\partial}
{\partial y}\right),\\[0.5ex]
L_y &=& -{\rm i}\,\hbar\left(z\,\frac{\partial}{\partial x} - x\,\frac{\partial}
{\partial z}\right),\\[0.5ex]
L_z &=& -{\rm i}\,\hbar\left(x\,\frac{\partial}{\partial y} - y\,\frac{\partial}
{\partial x}\right),
\end{eqnarray}
using the Schr\"{o}dinger representation. Transforming to standard
spherical polar coordinates,
\begin{eqnarray}
x &=& r \,\sin\theta\, \cos\varphi,\\[0.5ex]
y &=& r\, \sin\theta\, \sin\varphi,\\[0.5ex]
z &=& r\,\cos\theta,
\end{eqnarray}
we obtain
\begin{eqnarray}\label{e5.58a}
L_x &=& {\rm i}\,\hbar\,\left(\sin\varphi\, \frac{\partial}{\partial \theta}
+ \cot\theta \cos\varphi\,\frac{\partial}{\partial \varphi}\right)
\\[0.5ex]
L_y &=& -{\rm i} \,\hbar\,\left(\cos\varphi\, \frac{\partial}{\partial\theta}
-\cot\theta \sin\varphi \,\frac{\partial}{\partial \varphi}\right)\\[0.5ex]
L_z&=& -{\rm i}\,\hbar\,\frac{\partial}{\partial\varphi}.\label{e5.58c}
\end{eqnarray}
Note that Eq.~(\ref{e5.58c}) accords with Eq.~(\ref{e5.45}). The shift
operators $L^\pm = L_x \pm {\rm i}\,L_y$ become
\begin{equation}\label{e5.59}
L^\pm = \pm \hbar\,\exp(\pm{\rm i}\,\varphi)\left(\frac{\partial}{\partial\theta}
\pm{\rm i} \,\cot\theta\,\frac{\partial}{\partial\varphi}\right).
\end{equation}
Now,
\begin{equation}
L^2 = L_x^{~2}+L_y^{~2}+L_z^{~2} = L_z^{~2} + (L^+\, L^- + L^- \,L^+) /2,
\end{equation}
so
\begin{equation}\label{e5.61}
L^2 = - \hbar^2\left( \frac{1}{\sin\theta}\frac{\partial}{\partial\theta}
\sin\theta\frac{\partial}{\partial\theta} + \frac{1}{\sin^2\theta}\frac{\partial^2}
{\partial\varphi^2}\right).
\end{equation}

 The eigenvalue problem for
$L^2$ takes the form
\begin{equation}\label{e5.62}
L^2 \,\psi = \lambda \,\hbar^2\, \psi,
\end{equation}
where $\psi(r, \theta, \varphi)$ is the wave-function, and $\lambda$ is a number. 
Let us write
\begin{equation}
\psi(r, \theta, \varphi) = R(r) \,Y(\theta, \varphi).
\end{equation}
Equation~(\ref{e5.62}) reduces to
\begin{equation}
\left( \frac{1}{\sin\theta}\frac{\partial}{\partial\theta}
\sin\theta\frac{\partial}{\partial\theta} + \frac{1}{\sin^2\theta}\frac{\partial^2}
{\partial\varphi^2}\right)Y + \lambda \,Y = 0,
\end{equation}
where use has been made of Eq.~(\ref{e5.61}). As is well-known,
square integrable solutions to this
equation only exist when $\lambda$ takes the values $l\,(l+1)$, where $l$ is
an integer. These solutions are known as {\em spherical harmonics}, and
can be written
\begin{equation}\label{e5.65}
Y_l^m(\theta, \varphi) = \sqrt{ \frac{2\,l+1}{4\pi} \frac{(l-m)!}{(l+m)!}}
\,(-1)^m\, {\rm e}^{\,{\rm i} \,m\,\varphi}\, P_l^m(\cos\varphi),
\end{equation}
where $m$ is a positive  integer lying in the range $0\leq  m\leq l$. Here, 
$P_l^m(\xi)$ is an {\em associated Legendre function} satisfying the
equation
\begin{equation}
\frac{d}{d\xi}\! \left[ (1-\xi^2)\frac{dP_l^m}{d\xi}\right]
- \frac{m^2}{1-\xi^2} P_l^m + l\,(l+1)\,P_l^m = 0.
\end{equation}
We define 
\begin{equation}
Y_l^{-m} = (-1)^m\, (Y^{m}_l)^\ast,
\end{equation}
which allows $m$ to take the negative values $-l\leq m< 0$.
The spherical harmonics are {\em orthogonal} functions, and are 
properly normalized with respect to integration over
the entire solid angle:
\begin{equation}
\int_0^\pi \int_0^{2\pi} Y_l^{m\ast} (\theta,\varphi)\,
Y_{l'}^{m'}(\theta, \varphi) \,\sin\theta\,d\theta\,d\varphi = \delta_{l l'} \,\delta_{m m'}.
\end{equation}
The spherical harmonics also form a complete set for representing general functions
of $\theta$ and $\varphi$. 

By definition,
\begin{equation}\label{e5.69}
L^2 \,Y_l^m = l\,(l+1)\,\hbar^2\,Y_l^m,
\end{equation}
where $l$ is an integer.
It follows from Eqs.~(\ref{e5.58c}) and (\ref{e5.65}) that
\begin{equation}
L_z \,Y^m_l = m\,\hbar\,Y_l^m,
\end{equation}
where $m$ is an integer lying in the range $-l\leq m \leq l$. Thus, the
wave-function $\psi(r, \theta, \varphi) = R(r) \,Y_l^m(\theta, \phi)$, where
$R$ is a general function,  has
all of the expected features of the wave-function 
of a simultaneous eigenstate of $L^2$ and $L_z$
belonging to the quantum numbers $l$ and $m$. The well-known formula
\begin{eqnarray}
\frac{d P_l^m}{d\xi} &=& \frac{1}{\sqrt{1-\xi^2}}P_l^{m+1}
- \frac{m\,\xi}{1-\xi^2} P_l^m \nonumber\\[0.5ex]
&=& - \frac{(l+m)(l-m+1)}{\sqrt{1-\xi^2}}P_l^{m-1} + \frac{m\,\xi}
{1-\xi^2} P_l^m
\end{eqnarray}
can be combined with Eqs.~(\ref{e5.59}) and (\ref{e5.65}) to give
\begin{eqnarray}
L^+ Y_l^m = \sqrt{l\,(l+1)- m\,(m+1)}\,\hbar\,Y_l^{m+1},\\[0.5ex]
L^- Y_l^m = \sqrt{l\,(l+1) - m \,(m-1)} \,\hbar \,Y_l^{m-1}.
\end{eqnarray}
These equations are equivalent to Eqs.~(\ref{e5.44a})--(\ref{e5.44b}). Note that a spherical
harmonic  wave-function
is symmetric about the $z$-axis ({\em i.e.}, independent of $\varphi$) whenever
$m=0$, and is spherically symmetric whenever $l=0$ (since
$Y_0^0 = 1/\sqrt{4\pi}$). 

In summary, by solving directly
for the 
eigenfunctions of $L^2$ and $L_z$  in Schr\"{o}d\-inger's representation, we have been able to reproduce
all of the results of Sect.~\ref{s5.2}. Nevertheless, the results of Sect.~\ref{s5.2}
are more general than those obtained in this section, because they still apply
when the quantum number $l$ takes on half-integer values. 

\section{Motion in a Central Field}\label{s5.5}
Consider a particle of mass $M$ moving in a spherically symmetric potential.
The Hamiltonian takes the form
\begin{equation}
H = \frac{{\bf p}^2}{2\,M} + V(r).
\end{equation}
Adopting Schr\"{o}dinger's representation, we can write ${\bf p} = -({\rm i}/\hbar)
\nabla$. Hence,
\begin{equation}
H = -\frac{\hbar^2}{2\,M} \nabla^2 + V(r).
\end{equation}
When written in spherical polar coordinates, the above equation becomes
\begin{equation}
H= -\frac{\hbar^2}{2\,M}\left[ \frac{1}{r^2}\frac{\partial}{\partial r}
r^2\frac{\partial}{\partial r}  + \frac{1}{r^2\sin\theta}
\frac{\partial}{\partial \theta} 
\sin\theta \frac{\partial}{\partial\theta} 
+ \frac{1}{r^2\sin^2\theta} \frac{\partial^2}{\partial\varphi^2}\right]
+ V(r).
\end{equation}
Comparing this equation with Eq.~(\ref{e5.61}), we find that
\begin{equation}\label{e5.76}
H= \frac{\hbar^2}{2\,M}\left[- \frac{1}{r^2}\frac{\partial}{\partial r}
r^2\frac{\partial}{\partial r} + \frac{L^2}{\hbar^2 r^2}\right] +
V(r).
\end{equation}

Now, we know that the three components of angular momentum commute with $L^2$ (see Sect.~\ref{s5.1}). We also know, from Eqs.~(\ref{e5.58a})--(\ref{e5.58c}), that $L_x$, $L_y$, and $L_z$ take the
form of partial derivative operators of the {\em angular} coordinates,
when written in terms of spherical polar coordinates using  Schr\"{o}dinger's representation. It follows from Eq.~(\ref{e5.76}) that all three components of the angular
momentum commute with the Hamiltonian:
\begin{equation}
[{\bf L}, H] = 0.
\end{equation}
It is also easily seen that $L^2$ commutes with the Hamiltonian:
\begin{equation}
[L^2, H] = 0.
\end{equation}
According to Sect.~\ref{s4.2},  the previous two equations
ensure that the angular momentum ${\bf L}$ and its magnitude squared $L^2$
are both constants of the motion. This is as expected for a spherically
symmetric potential. 

Consider the energy eigenvalue problem
\begin{equation}
H\,\psi = E\,\psi,
\end{equation}
where $E$ is a number. Since $L^2$ and $L_z$ commute with each other and
the Hamiltonian, it is always possible to represent the state of the
system in terms of the simultaneous eigenstates of $L^2$, $L_z$, and $H$.
But, we already know that the most general form for the wave-function of
a simultaneous
eigenstate of $L^2$ and $L_z$ is (see previous section)
\begin{equation}\label{e5.80}
\psi(r, \theta, \varphi) = R(r) \,Y_l^m(\theta, \varphi).
\end{equation}
Substituting Eq.~(\ref{e5.80}) into Eq.~(\ref{e5.76}), and making use of Eq.~(\ref{e5.69}), we
obtain
\begin{equation}\label{e5.81}
\left[\frac{\hbar^2}{2\,M} \left(-\frac{1}{r^2} \frac{d}{dr}r^2\frac{d}{dr}
+\frac{l\,(l+1)}{r^2}\right) + V(r) - E\right] R = 0.
\end{equation}
This is a Sturm-Liouville equation for the function $R(r)$. We know,
from the general properties of this type of equation, 
that if $R(r)$ is required to be well-behaved  at $r=0$ and as $r\rightarrow
\infty$ then solutions only exist for a discrete set of values of $E$. These
are the energy eigenvalues. In general, the energy eigenvalues depend
on the quantum number $l$, but are independent of the quantum number $m$. 

\section{Energy Levels of the Hydrogen Atom}\label{s5.6}
Consider a hydrogen atom, for which the potential takes the specific form
\begin{equation}
V(r) = -\frac{e^2}{4\pi\epsilon_0\,r}.
\end{equation}
The radial eigenfunction $R(r)$ satisfies Eq.~(\ref{e5.81}), which can be written
\begin{equation}
\left[\frac{\hbar^2}{2\,\mu} \left(-\frac{1}{r^2} \frac{d}{dr}r^2\frac{d}{dr}
+\frac{l\,(l+1)}{r^2}\right)   -\frac{e^2}{4\pi\epsilon_0\,r}- E\right] R = 0.
\end{equation}
Here, $\mu = m_e \,m_p/(m_e+ m_p)$ is the {\em reduced mass}, which takes into
account the fact that the electron (of mass $m_e$) and the proton (of mass $m_p$)
both rotate about a common centre, which is equivalent to a particle of
mass $\mu$ rotating about a fixed point. Let us write the product $r\, R(r)$
as the function $P(r)$. The above equation transforms to
\begin{equation}\label{e5.84}
\frac{d^2 P}{d r^2} - \frac{2\,\mu}{\hbar^2}\left(
\frac{l\,(l+1)\hbar^2}{2\,\mu \,r^2} - \frac{e^2}{4\pi \epsilon_0 \,r}-E\right) P =0,
\end{equation}
which is the one-dimensional Schr\"{o}dinger equation for a particle of
mass $\mu$ moving in the {\em effective potential}
\begin{equation}
V_{\rm eff}(r) = -\frac{e^2}{4\pi \epsilon_0 \,r} + \frac{l\,(l+1)\,\hbar^2}{2\,\mu\, r^2}.
\end{equation}
The effective potential has a simple physical interpretation. The first part is the
attractive Coulomb potential, and the second part corresponds
to the repulsive centrifugal force.

Let 
\begin{equation}\label{e5.86}
a= \sqrt{\frac{-\hbar^2}{2\,\mu \,E}},
\end{equation}
and $y=r/a$, with
  \begin{equation}\label{e5.87}
P(r) = f(y) \exp(-y).
\end{equation}
Here, it is assumed that the energy eigenvalue $E$ is negative.
Equation~(\ref{e5.84}) transforms to 
\begin{equation}\label{e5.88}
\left(\frac{d^2}{dy^2} -2\,\frac{d}{dy} -\frac{l\,(l+1)}{y^2}
+ \frac{2\,\mu\, e^2\, a}{4\pi \epsilon_0\, \hbar^2\,y}\right) f= 0.
\end{equation}
Let us look for a power-law solution of the form
\begin{equation}\label{e5.89}
f(y) = \sum_{n} c_n\, y^n.
\end{equation}
Substituting this solution into Eq.~(\ref{e5.88}), we obtain
\begin{equation}
\sum_n c_n \left\{ n\,(n-1)\,y^{n-2} - 2\,n\, y^{n-1} - l\,(l+1) \,y^{n-2} +
\frac{2\,\mu\, e^2 \,a}{4\pi \epsilon_0 \,\hbar^2}\, y^{n-1} \right\} = 0.
\end{equation}
Equating the coefficients of $y^{n-2}$ gives
\begin{equation}\label{e5.91}
c_n[ n\,(n-1) - l\,(l+1) ] = c_{n-1} \left [2\,(n-1) - \frac{2\,\mu\, e^2\, a}{4\pi \epsilon_0\, \hbar^2}\right].
\end{equation}
Now, the power law series (\ref{e5.89}) must terminate at small $n$, at some positive
value of $n$, otherwise $f(y)$ behaves unphysically as $y\rightarrow 0$. This
is only possible if $[n_{\rm min} (n_{\rm min} -1) - l\,(l+1) ] =0$, where
the first term in the series is $c_{n_{\rm min}}\,y^{n_{\rm min}}$. There
are two possibilities: $n_{\rm min} = -l$ or $n_{\rm min} = l+1$. The
former predicts  unphysical behaviour of the wave-function at $y=0$.
Thus, we conclude that $n_{\rm min} = l+1$. Note that for an $l=0$ state
there is a finite probability of finding the electron at the nucleus,
whereas for an $l>0$ state there is zero  probability of finding 
the electron at the nucleus ({\em i.e.}, $|\psi|^2 =0$ at $r=0$, except when
$l=0$). Note, also,  that it is only possible to obtain sensible behaviour of the
wave-function as $r\rightarrow 0$ if $l$ is an integer. 

For large values of $y$, the ratio of successive terms in the series
(\ref{e5.89}) is
\begin{equation}
\frac{c_n \,y}{c_{n-1}} = \frac{2\, y}{n},
\end{equation}
according to Eq.~(\ref{e5.91}). This is the same as the ratio of
successive terms in the series
\begin{equation}
\sum_n \frac{(2\,y)^n}{n!},
\end{equation}
which converges to $\exp(2\,y)$. We conclude that $f(y)\rightarrow \exp(2\,y)$
as $y\rightarrow \infty$. It follows from Eq.~(\ref{e5.87})  that $R(r) \rightarrow 
\exp(r/a) /r $ as $r\rightarrow \infty$. This does not correspond to
physically acceptable behaviour of the wave-function, since $\int |\psi|^2\,dV$
must be finite. The only way in which we can avoid this unphysical
behaviour is if the series (\ref{e5.89}) terminates at some maximum value of $n$.
According to the recursion relation (\ref{e5.91}), this is only possible
if
\begin{equation}
\frac{\mu\, e^2 \,a}{4\pi \epsilon_0\, \hbar^2} = n,
\end{equation}
where the last term in the series is $c_n\, y^n$. It follows from Eq.~(\ref{e5.86})
that the energy eigenvalues are {\em quantized}, and can only take the values
\begin{equation}\label{e5.95}
E = - \frac{\mu\, e^4}{32\pi^2\epsilon_0^{~2}\, \hbar^2\, n^2}.
\end{equation}
Here, $n$ is a positive integer which must exceed the quantum number $l$,
otherwise there would be no terms in the series (\ref{e5.89}).

It is clear that the wave-function for a hydrogen atom can be written
\begin{equation}
\psi(r, \theta, \varphi) = R(r/a)\, Y_l^m (\theta, \varphi),
\end{equation}
where
\begin{equation}
a =\frac{n\,4\pi \epsilon_0\,\hbar^2}{\mu \,e^2} = 5.3\times 10^{-11}\,n\,\,\,{\rm meters},
\end{equation}
and $R(x)$ is a well-behaved solution of the differential equation
\begin{equation}\label{e5.98}
\left(\frac{1}{x^2} \frac{d}{dx} x^2 \frac{d}{dx}-\frac{l\,(l+1)}{x^2}
+ \frac{2\,n}{x} - 1\right) R = 0.
\end{equation}
Finally, the $Y_l^m$ are spherical harmonics. The restrictions on the quantum numbers
are  $|m| \leq l< n$. Here, $n$ is a positive integer, $l$ is
a non-negative integer, and $m$ is an integer. 

The ground state of hydrogen corresponds to $n=1$. The only permissible values
of the other quantum numbers are $l=0$ and $m=0$. Thus, the ground state is
a spherically symmetric,  zero angular momentum state. The energy of the
ground state is
\begin{equation}
E_0 = - \frac{\mu \,e^4}{32\pi^2\,\epsilon_0^{~2}\, \hbar^2} = -13.6\,\,\,{\rm electron~volts}.
\end{equation}
The next energy level corresponds to $n=2$. The other quantum numbers are
allowed to take the values $l=0$, $m=0$ or $l=1$, $m=-1, 0, 1$. Thus, there are
$n=2$ states with non-zero angular momentum. Note that the energy levels given
in Eq.~(\ref{e5.95}) are independent of the quantum number $l$, despite the fact that
$l$ appears in the radial eigenfunction equation (\ref{e5.98}). This is a special
property of a $1/r$ Coulomb potential. 

In addition to the quantized negative energy state of the
hydrogen atom, which we have just found, there
is  also a continuum of unbound positive energy states. 

\section{Spin Angular Momentum}
Up to now, we have tacitly assumed that the state of a particle in quantum
mechanics can be completely specified by giving the wave-function $\psi$ 
as a function of the spatial coordinates $x$, $y$, and $z$. Unfortunately,
there is a wealth of experimental evidence which suggests that this simplistic
approach is incomplete. 

Consider an isolated system at rest, and let the eigenvalue of its total
angular momentum be $j\,(j+1)\,\hbar^2$. According to the theory of orbital
angular momentum outlined in Sects.~\ref{s5.4} and \ref{s5.5}, there are two possibilities.
For a system consisting of a single particle, $j=0$. For a system consisting
of two (or more) particles, $j$ is a non-negative integer. 
However, this does not
agree with observations, because we often find systems which appear to
be structureless, and yet have $j\neq 0$. Even worse, systems where $j$
has half-integer values abound in nature. 
In order to  explain this apparent discrepancy
between theory and experiments, Gouldsmit and Uhlenbeck (in 1925)
introduced the concept of an internal, purely quantum mechanical, angular momentum
called {\em spin}. For a particle with spin, the total angular momentum in the
rest frame is non-vanishing. 

Let us denote the three components of the spin angular momentum of a
particle by the Hermitian operators
 $(S_x, S_y, S_z)\equiv {\bf S}$. We assume that these 
operators obey the fundamental commutation relations (\ref{e5.4a})--(\ref{e5.4c}) for the components
of an angular momentum. Thus, we can write
\begin{equation}\label{e5.100}
{\bf S} \times {\bf S} = {\rm i}\,\hbar \, {\bf S}.
\end{equation}
We can also define the operator
\begin{equation}
S^2 = S_x^{~2}+S_y^{~2} + S_z^{~2}.
\end{equation}
According to the quite general analysis of Sect.~\ref{s5.1},
\begin{equation}
[{\bf S}, S^2] = 0.
\end{equation}
Thus, it is possible to find simultaneous eigenstates of $S^2$ and $S_z$. 
These are denoted $|s, s_z\rangle$, where
\begin{eqnarray}
S_z |s, s_z\rangle &=& s_z \,\hbar \,|s, s_z\rangle,\\[0.5ex]
S^2 |s, s_z\rangle &=& s\,(s+1)\,\hbar^2 |s, s_z\rangle.
\end{eqnarray}
According to the equally general
analysis of Sect.~\ref{s5.2}, the quantum number $s$ can, in principle, 
take integer or half-integer values,
and the quantum number $s_z$ can only take the values $s, s-1 \cdots -s+1, -s$. 

Spin angular momentum clearly has many properties in common with
orbital angular momentum. However, there is one vitally important difference. 
Spin angular momentum operators {\em cannot} be expressed  in terms of
position and momentum operators, like  in Eqs.~(\ref{e5.1a})--(\ref{e5.1c}), since this 
identification depends on an analogy with classical mechanics, and the concept
of spin is purely quantum mechanical: {\em i.e.}, it has no analogy in classical physics. 
Consequently, the restriction that the quantum number of the overall angular
momentum must take {\em integer} values is lifted for spin angular momentum,
since this restriction (found in Sects.~\ref{s5.3} and \ref{s5.4}) depends on Eqs.~(\ref{e5.1a})--(\ref{e5.1c}).
In other words, the quantum number $s$ is allowed to take {\em half-integer} values.

Consider a spin one-half particle, for which
\begin{eqnarray}\label{e5.104a}
S_z |\pm \rangle &=&\pm \frac{\hbar}{2} \,|\pm \rangle,\\[0.5ex]
S^2 |\pm\rangle &=& \frac{3 \,\hbar^2}{4}\,|\pm \rangle.\label{e5.104b}
\end{eqnarray}
Here, the $|\pm \rangle$ denote eigenkets of the $S_z$ operator corresponding to
the eigenvalues $\pm \hbar/2$. These kets are orthonormal (since $S_z$ is
an Hermitian operator), so
\begin{equation}
\langle +| -\rangle = 0.
\end{equation}
They  are also properly normalized and complete, so that
\begin{equation}
\langle +| + \rangle=\langle -| - \rangle = 1,
\end{equation}
and
\begin{equation}
|+\rangle \langle +| + |-\rangle \langle -| = 1.
\end{equation}

It is easily verified that the Hermitian operators defined by
\begin{eqnarray}\label{e5.108a}
S_x &= &\frac{\hbar}{2} \left(\, |+\rangle \langle -| + |-\rangle \langle +|\,
\right),
\\[0.5ex]
S_y &=& \frac{{\rm i}\,\hbar}{2}\left(\, -\,|+\rangle \langle -| +
 |-\rangle \langle +|\,\right),
\\[0.5ex]
S_z &=& \frac{\hbar}{2}\left(\, |+\rangle \langle +| - |-\rangle \langle -|\,\right),\label{e5.108c}
\end{eqnarray}
satisfy the commutation relations (\ref{e5.4a})--(\ref{e5.4c}) (with the $L_j$ replaced by the $S_j$).
The operator $S^2$ takes the form 
\begin{equation}\label{e5.109}
S^2 = \frac{3\,\hbar^2}{4}.
\end{equation}
It is also easily demonstrated that $S^2$ and $S_z$,
defined in this manner,  satisfy the eigenvalue
relations (\ref{e5.104a})--(\ref{e5.104b}). Equations (\ref{e5.108a})--(\ref{e5.109}) constitute a realization
of the spin operators ${\bf S}$ and $S^2$ (for a spin one-half particle)
in {\em spin space} ({\em i.e.}, that Hilbert sub-space consisting of kets which
correspond to the different spin states of the particle). 

\section{Wavefunction of a Spin One-Half Particle}\label{s5.8}
The state of a spin one-half particle is represented as a vector in ket space.
Let us suppose that this space is spanned by the basis kets
$|x', y', z', \pm\rangle$. Here, $|x',y',z', \pm\rangle$ denotes a
simultaneous eigenstate of the position operators $x$, $y$, $z$, and
the spin operator $S_z$, corresponding to the eigenvalues $x'$, $y'$, $z'$,
and $\pm \hbar/2$, respectively. The basis kets are assumed to
satisfy the completeness relation
\begin{equation}
\int\!\int\!\int \left(\,
|x',y',z',+\rangle\langle x', y', z',+|+|x',y',z',-\rangle\langle x', y', z',-|
\,\right) \,dx'dy'dz' = 1.
\end{equation}

It is helpful to think of the ket $|x', y', z', +\rangle$ as the product
of two kets---a position space ket $|x', y', z'\rangle$, and
a spin space ket $|+\rangle$. We assume that such a product obeys
the commutative and distributive axioms of multiplication:
\begin{eqnarray}
|x', y', z'\rangle |+\rangle &=& |+\rangle |x', y', z'\rangle,\\[0.5ex]
\left(c' |x', y', z'\rangle + c''| x'', y'', z''\rangle\right)
|+\rangle &=& c' |x', y', z'\rangle |+\rangle \nonumber\\[0.5ex]
&&+ c'' |x'', y'', z''\rangle |+\rangle
\\[0.5ex]
|x', y', z'\rangle\left(c_+ |+\rangle + c_- |-\rangle\right)&=& c_+ 
|x', y', z'\rangle|+\rangle\nonumber\\[0.5ex]
&& + c_-|x', y', z'\rangle|-\rangle,
\end{eqnarray}
where the $c$'s are numbers. We can give meaning to any
position space operator (such as $L_z$)  acting on the product $|x', y', z'\rangle
|+\rangle$ by assuming that it operates only on the $|x', y', z'\rangle$
factor, and commutes with the $|+\rangle$ factor. 
Similarly, we can give a meaning to any spin operator  (such as $S_z$) acting
on $|x', y', z'\rangle
|+\rangle$ by assuming that it operates only on $|+\rangle$, and
commutes with $|x', y', z'\rangle$. This implies that every position
space operator
commutes with every spin operator. In this manner, we can give
meaning to the equation
\begin{equation}\label{e5.112}
 |x', y', z', \pm\rangle = |x', y', z'\rangle| \pm\rangle = | \pm\rangle
|x', y', z'\rangle.
\end{equation}

The multiplication in the above equation is of quite a different type to
any which we have encountered previously. The ket vectors $|x',y', z'\rangle$ and
$|\pm\rangle$ are in two quite separate vector spaces, and their product
$|x',y', z'\rangle|\pm\rangle$ is in a third vector space. 
In mathematics, the latter space
is termed the {\em product space} of the former spaces, which are
termed {\em factor spaces}.  The number of
dimensions of a product space is equal to the product of the number of dimensions
of each of the factor spaces. A general ket of the product space is not
of the form (\ref{e5.112}), but is instead a sum or integral of kets of this form. 

A general state $A$ of a spin one-half particle is represented as a ket
$||A\rangle\rangle$ in the product  of the spin and position spaces. 
This state can be completely specified by {\em two} wavefunctions:
\begin{eqnarray}
\psi_+(x', y', z') &=& \langle x', y', z' |\langle +||A\rangle\rangle,\\[0.5ex]
\psi_-(x', y', z') &=& \langle x', y', z' |\langle -||A\rangle\rangle.
\end{eqnarray}
The probability of observing the particle in the region $x'$ to $x'+dx'$,
$y'$ to $y'+dy'$, and $z'$ to $z'+dz'$, with $s_z = +1/2$ is
$|\psi_+ (x', y', z')|^2\,dx' dy' dz'$. Likewise, 
the probability of observing the particle in the region $x'$ to $x'+dx'$,
$y'$ to $y'+dy'$, and $z'$ to $z'+dz'$, with $s_z = -1/2$ is
$|\psi_- (x', y', z')|^2\,dx' dy' dz'$.
The normalization condition for the wavefunctions is
\begin{equation}
\int\!\int\!\int \left(\, |\psi_+|^2 + |\psi_-|^2\,\right)\, dx'dy'dz' = 1.
\end{equation}

\section{Rotation Operators in Spin Space}\label{s5.9}
Let us, for the moment, forget about the spatial position of the particle,
and concentrate on its spin state. A general
spin state $A$ is represented by the ket
\begin{equation}\label{e5.115}
|A\rangle = \langle +|A\rangle |+\rangle + \langle -|A\rangle |-\rangle
\end{equation}
in spin space.
In Sect.~\ref{s5.3}, we were able  to construct an operator $R_z(\Delta\varphi)$ which
rotates the system by an angle $\Delta\varphi$ about the $z$-axis in position
space. Can we also construct an operator $T_z(\Delta\varphi)$ which rotates the
system by an angle $\Delta\varphi$ about the $z$-axis in spin space? By analogy
with Eq.~(\ref{e5.50}), we would expect such an operator to take the form
\begin{equation}\label{e5.116}
T_z(\Delta\varphi) = \exp(-{\rm i} \,S_z\, \Delta\varphi/\hbar).
\end{equation}
Thus, after rotation, the ket $|A\rangle$ becomes
\begin{equation}
|A_R\rangle = T_z(\Delta\varphi) |A\rangle.
\end{equation}

To demonstrate that the operator (\ref{e5.116})  really does rotate the spin of the system,
let us consider its effect on $\langle S_x\rangle$. Under rotation, this
expectation value changes as follows:
\begin{equation}
\langle S_x\rangle \rightarrow \langle A_R| S_x |A_R \rangle
= \langle A| T_z^{\dag}\, S_x \,T_z |A\rangle.
\end{equation}
Thus, we need to compute
\begin{equation}\label{e5.119}
\exp(\,{\rm i}\,S_z\, \Delta\varphi/\hbar)\, S_x \,
\exp(-{\rm i}\,S_z \,\Delta\varphi/\hbar).
\end{equation}
This can be achieved in two different ways. 

First, we can use the explicit formula for $S_x$ given in Eq.~(\ref{e5.108a}). We find
that Eq.~(\ref{e5.119}) becomes
\begin{equation}
\frac{\hbar}{2} \exp(\,{\rm i}\,S_z\, \Delta\varphi/\hbar)\,
(\,|+\rangle \langle -| + |-\rangle \langle +|\,)\,
\exp(-{\rm i}\,S_z \,\Delta\varphi/\hbar),
\end{equation}
or 
\begin{equation}
\frac{\hbar}{2} 
\left( {\rm e}^{\,{\rm i}\,\Delta\varphi/2}\,|+\rangle \langle -|\,
{\rm e}^{\,{\rm i}\,\Delta\varphi/2} + {\rm e}^{\,-{\rm i}\,\Delta\varphi/2}
\,|-\rangle \langle +|\,{\rm e}^{\,-{\rm i}\,\Delta\varphi/2}\,\right),
\end{equation}
which reduces to
\begin{equation}
S_x\,\cos\Delta\varphi - S_y\,\sin\Delta\varphi,
\end{equation}
where use has been made of Eqs.~(\ref{e5.108a})--(\ref{e5.108c}).

A second approach is to use the so called {\em Baker-Hausdorff lemma}. This
takes the form
\begin{eqnarray}
\exp(\,{\rm i}\, G\,\lambda)\,A\, \exp(-{\rm i} \,G \,\lambda)&=& A + {\rm i} \,\lambda
[G,A] + \left(\frac{{\rm i}^2 \lambda^2}{2!}\right) [G, [G,A]]+\\[0.5ex]
&&\cdots +   
\left(\frac{{\rm i}^n\lambda^n}{n!}\right)[G, [G, [G, \cdots [G,A]]]\cdots]
\nonumber,\label{e5.123}
\end{eqnarray}
where $G$ is a Hermitian operator, and $\lambda$ is a real parameter. The proof
of this lemma is left as an exercise. Applying the Baker-Hausdorff lemma
to Eq.~(\ref{e5.119}), we obtain 
\begin{equation}
S_x + \left(\frac{{\rm i}\,\Delta\varphi}{\hbar}\right) [S_z, S_x]
+ \left(\frac{1}{2!}\right) \left(\frac{{\rm i}\,\Delta\varphi}{\hbar}\right)^2
[S_z, [S_z, S_x]] + \cdots,
\end{equation}
which reduces to 
\begin{equation}
S_x\left[ 1- \frac{\Delta\varphi^2}{2!} + \cdots\right] - S_y \left[\varphi - 
\frac{\Delta\varphi^3}{3!} +\cdots\right],
\end{equation}
or
\begin{equation}
S_x\,\cos\Delta\varphi - S_y\,\sin\Delta\varphi,
\end{equation}
where  use has been made of Eq.~(\ref{e5.100}). The second
proof is more general than the first, since it only uses the fundamental
 commutation relation (\ref{e5.100}), and is, therefore, valid for systems with spin
angular momentum higher than one-half.

For a  spin one-half system, both methods imply that
\begin{equation}\label{e5.127}
\langle S_x \rangle \rightarrow \langle S_x\rangle \,\cos\Delta\varphi
- \langle S_y\rangle\,\sin\Delta\varphi
\end{equation}
under the action of the rotation operator (\ref{e5.116}). It is straight-forward to
show that 
\begin{equation}
\langle S_y \rangle \rightarrow \langle S_y\rangle \,\cos\Delta\varphi
+ \langle S_x\rangle\,\sin\Delta\varphi.
\end{equation}
Furthermore, 
\begin{equation}\label{e5.129}
\langle S_z \rangle \rightarrow\langle S_z\rangle,
\end{equation}
since $S_z$ commutes with the rotation operator. Equations (\ref{e5.127})--(\ref{e5.129})
 demonstrate that
the operator (\ref{e5.116}) rotates the expectation value of ${\bf S}$ by an
angle $\Delta \varphi$ about the $z$-axis. In fact, the expectation value
of the spin operator behaves like a classical  vector under rotation:
\begin{equation}
\langle S_k \rangle \rightarrow \sum_l R_{kl} \langle S_l\rangle,
\end{equation}
where the $R_{kl}$ are the elements of the  conventional rotation matrix 
for the rotation in question. It is clear, from our  second derivation of
the result (\ref{e5.127}), that this property is not restricted to the spin operators of
a spin one-half system. In fact, we have effectively demonstrated that
\begin{equation}
\langle J_k \rangle \rightarrow \sum_l R_{kl} \langle J_l\rangle,
\end{equation}
where the $J_k$ are the generators of rotation, satisfying the fundamental
commutation relation ${\bf J}\times {\bf J} = {\rm i}\,\hbar\, {\bf J}$,
and the rotation operator about the $k$th axis is written
$R_k (\Delta\varphi) = \exp(-{\rm i}\,J_k\, \Delta\varphi/\hbar)$. 

Consider the effect of the rotation operator (\ref{e5.116}) on the state ket (\ref{e5.115}).
It is easily seen that
\begin{equation}\label{e5.132}
T_z(\Delta\varphi)|A\rangle = {\rm e}^{-{\rm i}\,\Delta\varphi/2}
\langle +|A\rangle |+\rangle +  {\rm e}^{\,{\rm i}\,\Delta\varphi/2}
\langle -|A\rangle |-\rangle.
\end{equation}
Consider a rotation by $2\pi$ radians. We find that
\begin{equation}\label{e5.133}
|A\rangle \rightarrow T_z(2\pi)|A\rangle = -|A\rangle.
\end{equation}
Note that a ket rotated by $2\pi$ radians differs from the original ket by a
{\em minus} sign. In fact, a rotation by $4\pi$ radians is needed to transform a ket
into itself. The minus sign does not affect the expectation value of
${\bf S}$, since ${\bf S}$ is sandwiched between $\langle A|$ and $| A\rangle$,
both of which change sign. Nevertheless, the minus sign does give rise to
observable consequences, as we shall see  presently. 

\section{Magnetic Moments}
Consider a particle of charge $q$ and velocity $v$
performing  a  circular orbit of radius $r$ 
in the $x$-$y$ plane. The charge is equivalent to a current loop of radius $r$
in the $x$-$y$ plane carrying  current $I=q\,v/2\pi\, r$. The magnetic moment
$\bmu$  of
the loop is of magnitude $\pi\,  r^2\, I$ and is directed along the $z$-axis.
Thus, we can write
\begin{equation}
\bmu = \frac{q}{2}\, {\bf r} \times {\bf v},
\end{equation}
where ${\bf r}$ and ${\bf v}$ are the vector position and velocity of the particle,
respectively. However, we know that ${\bf p} = {\bf v} /m$, where ${\bf p}$
is the vector momentum of the particle, and $m$ is its mass. We also know that
${\bf L} = {\bf r}\times {\bf p}$, where ${\bf L}$ is the orbital angular momentum.
It follows that
\begin{equation}
\bmu = \frac{q}{2\,m} \,{\bf L}.
\end{equation}
Using the  usual analogy between classical and quantum mechanics, we 
expect the above relation to also hold between the quantum mechanical operators,
$\bmu$ and ${\bf L}$, which represent magnetic moment and orbital angular momentum,
respectively.
This is indeed found to the the case. 

Does spin angular momentum also give rise to a contribution to the magnetic
moment of a charged particle? The answer is ``yes''. In fact, relativistic quantum
mechanics actually predicts that a charged particle possessing spin should also
possess a magnetic moment (this was first demonstrated by Dirac). We can write
\begin{equation}
\bmu  = \frac{q}{2\,m} \left({\bf L} + g \,{\bf S}\right),
\end{equation}
where $g$ is called the {\em gyromagnetic ratio}. For an electron this ratio
is found to be
\begin{equation}
g_e = 2\left( 1 + \frac{1}{2\pi} \frac{e^2}{4\pi \epsilon_0\,\hbar \,c} \right).
\end{equation}
The factor 2 is correctly predicted by Dirac's relativistic theory of the electron.
The small correction $1/(2\pi\, 137)$, derived  originally by Schwinger, is due to
quantum field effects. We shall ignore this correction in the following,
so
\begin{equation}\label{e5.138}
\bmu  \simeq - \frac{e}{2\,m_e} \left({\bf L} + 2 \,{\bf S}\right)
\end{equation}
for an electron (here, $e>0$).

\section{Spin Precession}
The Hamiltonian for an electron at rest in a $z$-directed  magnetic field, ${\bf B}=
B\,\hat{ {\bf z}}$,
is
\begin{equation}\label{e5.139}
H = - \bmu\! \cdot \!{\bf B} = \left(\frac{e}{m_e}\right) {\bf S}\! \cdot\! {\bf B}
= \omega\, S_z,
\end{equation}
where
\begin{equation}\label{e5.140}
\omega = \frac{e\,B}{m_e}.
\end{equation}
According to Eq.~(\ref{e4.28}), the time evolution operator for this system is
\begin{equation}
T(t,0) = \exp(-{\rm i} \,H t/\hbar) = \exp(-{\rm i} \,S_z\, \omega\, t/\hbar).
\end{equation}
It can be seen, by comparison with Eq.~(\ref{e5.116}), that the time evolution operator
is precisely the same as the rotation operator for spin, with $\Delta\varphi$ set
equal to $\omega \,t$. It is immediately clear that the Hamiltonian (\ref{e5.139})
 causes the electron 
spin to precess about the $z$-axis with angular frequency $\omega$. In fact,
Eqs.~(\ref{e5.127})--(\ref{e5.129}) imply  that
\begin{eqnarray}
\langle S_x\rangle_t &=& \langle S_x\rangle_{t=0} \cos\omega t - 
\langle S_y\rangle_{t=0} \sin\omega t,\\[0.5ex]
\langle S_y\rangle_t &=& \langle S_y\rangle_{t=0} \cos\omega t +
\langle S_x\rangle_{t=0} \sin\omega t,\\[0.5ex]
\langle S_z\rangle_t&=& \langle S_z\rangle_{t=0}.
\end{eqnarray}
The time evolution of the state ket is given by analogy with Eq.~(\ref{e5.132}):
\begin{equation}
|A, t\rangle = {\rm e}^{-{\rm i}\,\omega \,t/2}
\langle +|A, 0\rangle |+\rangle +  {\rm e}^{\,{\rm i}\,\omega \,t/2}
\langle -|A, 0\rangle |-\rangle.
\end{equation}
Note that it takes time $t= 4\pi/\omega$ for the state ket to return  to its
original state. 
By contrast, it only takes times $t=2\pi/\omega$ for the spin vector to point
in its original direction. 

We now describe an experiment to detect the minus sign in Eq.~(\ref{e5.133}). An almost
 monoenergetic beam of neutrons is split in two, sent along two different
paths, $A$ and $B$, and then recombined. Path $A$ goes through a magnetic field
free region. However, path $B$ enters a small  region where a static magnetic
field is present. As a result, a neutron state ket going along path
$B$ acquires a phase-shift $\exp(\mp{\rm i}\, \omega \,T/2)$ (the $\mp$
signs correspond to $s_z = \pm 1/2$ states). Here, $T$ is the
time spent in the magnetic field, and $\omega$ is the spin precession frequency
\begin{equation}
\omega = \frac{g_n\, e\,B}{m_p}.
\end{equation}
This frequency is defined in an analogous manner to Eq.~(\ref{e5.140}). The gyromagnetic
ratio for a neutron is found experimentally to be $g_n = -1.91$. 
(The magnetic moment of a neutron is entirely a quantum field effect).
When neutrons from path $A$ and path $B$ meet they undergo interference. We
expect the observed  neutron intensity in the interference region to
exhibit a $\cos( \pm \omega\, T/2 + \delta)$ variation, 
where $\delta$ is the phase difference 
between paths $A$ and $B$ in the absence of a magnetic field. In experiments,
the time of flight $T$ through the magnetic field region is kept constant, while
the field-strength $B$ is varied. It follows that the change in magnetic
field required to produce successive maxima is 
\begin{equation}
\Delta B = \frac{4\pi \,\hbar}{e\, g_n\, \lambdabar\, l},
\end{equation}
where $l$ is the path-length through the magnetic field region, and $\lambdabar$
is the de Broglie wavelength over $2\pi$ of the neutrons. The above prediction has been verified
experimentally to within a fraction of a percent. This prediction depends crucially
on the fact that it takes a $4\pi$ rotation to return a state ket to its
original state. If it only took a $2\pi$ rotation then $\Delta B$ would be half
of the value given above, which does not agree with the experimental data. 

\section{Pauli Two-Component Formalism}
We have seen, in Sect.~\ref{s5.4}, that the eigenstates of orbital angular momentum
can be conveniently represented as spherical harmonics. In this
representation, the orbital  angular momentum 
operators take the form of  differential operators involving only
angular coordinates. It is conventional to represent the eigenstates of spin
angular momentum as column (or row) matrices. In this representation,
the spin angular momentum operators take the form of matrices. 

The matrix representation of a  spin one-half system was introduced by Pauli in 1926. 
Recall, from Sect.~\ref{s5.9}, that a general spin ket can be expressed as
a linear combination of the two eigenkets of $S_z$ belonging to the
eigenvalues $\pm \hbar/2$. These are denoted  $|\pm\rangle$. Let us
represent these basis eigenkets as column matrices:
\begin{eqnarray}
|+\rangle&\rightarrow &\left(\!\begin{array}{c}1\\0\end{array}\!\right) \equiv \chi_+,\\
[0.5ex]
|-\rangle &\rightarrow& \left(\!\begin{array}{c}0\\1\end{array}\!\right) \equiv \chi_-.
\end{eqnarray}
The corresponding eigenbras are represented as row matrices:
\begin{eqnarray}
\langle +| &\rightarrow& (1, 0) \equiv \chi_+^{\dag}, \\[0.5ex]
\langle - |&\rightarrow& (0, 1) \equiv \chi_-^{\dag}.
\end{eqnarray}
In this scheme, a general  bra takes the form 
\begin{equation}\label{e5.148}
|A\rangle = \langle +|A\rangle |+\rangle + \langle -|A\rangle |-\rangle
\rightarrow
\left(\!\begin{array}{c}\langle +|A\rangle\\
\langle -|A\rangle\end{array}\!\right),
\end{equation}
and a general ket becomes
\begin{equation}\label{e5.149}
\langle A| =\langle A|+\rangle \langle +| + \langle A|-\rangle \langle -|
\rightarrow (\langle A|+\rangle, \langle A|-\rangle).
\end{equation}
The column matrix (\ref{e5.148})  is called a two-component {\em spinor}, and can be written
\begin{equation}
\chi \equiv \left(\!\begin{array}{c}\langle +|A\rangle\\
\langle -|A\rangle\end{array}\!\right) =\left(\!\begin{array}{c}
c_+\\ c_- \end{array}\!\right) = c_+\, \chi_+ + c_- \,\chi_-,
\end{equation}
where the $c_\pm$ are complex numbers. The row matrix (\ref{e5.149}) becomes
\begin{equation}
\chi^{\dag} \equiv (\langle A|+\rangle, \langle A|-\rangle) =(c_+^{~\ast}, c_-^{~\ast}) =c_+^{~\ast}\, \chi_+^{\dag} + c_-^{~\ast}\,\chi_-^{\dag}.
\end{equation}

Consider the ket obtained by the action of a  spin operator on 
ket $A$:
\begin{equation}\label{e5.152}
|A'\rangle = S_k |A\rangle.
\end{equation}
This ket is represented as
\begin{equation}
|A'\rangle 
\rightarrow
\left(\!\begin{array}{c}\langle +|A'\rangle\\
\langle -|A'\rangle\end{array}\!\right)\equiv \chi'.
\end{equation}
However,
\begin{eqnarray}
\langle + |A'\rangle &=& \langle + |S_k| +\rangle \langle +|A\rangle
+ \langle +|S_k |-\rangle \langle -|A\rangle,\\[0.5ex]
\langle - |A'\rangle &=& \langle - |S_k| +\rangle \langle +|A\rangle
+ \langle -|S_k |-\rangle \langle -|A\rangle,
\end{eqnarray}
or
\begin{equation}
\left(\!\begin{array}{c}\langle +|A'\rangle\\[0.5ex]
\langle -|A'\rangle\end{array}\!\right) = 
\left(\!\begin{array}{cc}
\langle + |S_k| +\rangle&\langle +|S_k |-\rangle\\[0.5ex]
\langle - |S_k| +\rangle& \langle -|S_k |-\rangle\end{array}\!\right)
\left(\!\begin{array}{c}\langle +|A\rangle\\[0.5ex]
\langle -|A\rangle\end{array}\!\right).\label{e5.155}
\end{equation}
It follows that we can represent the operator/ket relation
(\ref{e5.152})  as the matrix relation
\begin{equation}
\chi' =\left( \frac{\hbar}{2}\right)\sigma_k \,\chi,
\end{equation}
where the $\sigma_k$ are the matrices of the $\langle \pm |S_k|\pm \rangle$
values divided by $\hbar/2$. These matrices, which are called the
{\em Pauli matrices}, can easily be evaluated using  the explicit forms for the
spin operators given in Eqs.~(\ref{e5.108a})--(\ref{e5.108c}). We find that
\begin{eqnarray}\label{e5.157a}
\sigma_1 &=& \left(\!\begin{array}{cc} 0 &1\\1&0\end{array}\!\right),\\[0.5ex]
\sigma_2 &=& \left(\!\begin{array}{cc} 0 &-{\rm i}\\{\rm i}&0\end{array}\!\right),\\[0.5ex]
\sigma_3 &=& \left(\!\begin{array}{cc} 1 &0\\0&-1\end{array}\!\right).\label{e5.157c}
\end{eqnarray}
Here, 1, 2, and 3 refer to $x$, $y$, and $z$, respectively. Note that, in this
scheme, we are effectively representing the spin operators in terms
of the Pauli matrices:
\begin{equation}\label{e5.158}
S_k \rightarrow \left(\frac{\hbar}{2} \right)\sigma_k.
\end{equation}
The expectation value of $S_k$ can be written in terms of spinors
and the Pauli matrices:
\begin{equation}\label{e5.159}
\langle S_k \rangle = \langle A|S_k |A\rangle = \sum_\pm
\langle A|\pm \rangle \langle \pm |S_k|\pm \rangle \langle \pm |A\rangle
= \left(\frac{\hbar}{2}\right) \chi^{\dag}\, \sigma_k\, \chi.
\end{equation}

The fundamental commutation relation for angular momentum, Eq.~(\ref{e5.100}), can
be combined with (\ref{e5.158}) to give the following commutation relation
for the Pauli matrices:
\begin{equation}\label{e5.160}
\bsigma\times \bsigma = 2\,{\rm i}\,\bsigma.
\end{equation}
It is easily seen that the matrices (\ref{e5.157a})--(\ref{e5.157c}) actually satisfy these relations
({\em i.e.},  $\sigma_1\, \sigma_2 - \sigma_2\,\sigma_1 = 2\,{\rm i} \,\sigma_3$, plus
all cyclic permutations). It is also easily seen that the Pauli matrices
satisfy the anti-commutation relations
\begin{equation}\label{e5.161}
\{ \sigma_i, \sigma_j \} = 2 \,\delta_{ij}.
\end{equation}

Let us examine how the Pauli scheme can be extended to take into account the
position of a spin one-half particle. Recall, from Sect.~\ref{s5.8},
that we can represent a general basis ket  as the product
of basis kets in position space and  spin space:
\begin{equation}
|x', y', z', \pm\rangle = |x',y',z'\rangle |\pm \rangle = |\pm \rangle|
x',y',z'\rangle .
\end{equation}
The ket corresponding to state $A$ is denoted $||A\rangle\rangle$, and resides
in the product space of the position and spin ket spaces. State $A$ is completely
specified by the two wave-functions
\begin{eqnarray}
\psi_+(x', y', z') &=& \langle x',y', z'|\langle +||A\rangle\rangle,\\[0.5ex]
\psi_-(x', y', z') &=& \langle x',y',z'|\langle -||A\rangle\rangle.
\end{eqnarray}
Consider the operator relation
\begin{equation}\label{e5.164}
||A'\rangle\rangle = S_k ||A\rangle\rangle.
\end{equation}
It is easily seen that 
\begin{eqnarray}
\langle x', y', z'| \langle +|A'\rangle\rangle &=&
\langle + |S_k |+\rangle \langle x',y',z'|\langle +||A\rangle\rangle\nonumber\\[0.5ex]
&&+\langle + |S_k |-\rangle \langle x',y',z'|\langle -||A\rangle\rangle,\\[0.5ex]
\langle x', y', z'| \langle -|A'\rangle\rangle &=&
\langle - |S_k |+\rangle \langle x',y',z'|\langle +||A\rangle\rangle\nonumber\\[0.5ex]
&&+\langle -
|S_k |-\rangle \langle x',y',z'|\langle -||A\rangle\rangle,
\end{eqnarray}
where use has been made of the fact that the spin operator $S_k$  commutes with the
eigenbras $\langle x', y', z'|$. 
It is fairly obvious  that we can represent the operator relation (\ref{e5.164}) as a matrix relation
if we generalize our definition of a spinor by writing
\begin{equation}\label{e5.166}
||A\rangle\rangle \rightarrow \left(\! \begin{array}{c}\psi_+({\bf r}') \\
\psi_-({\bf r}')\end{array}\!\right)\equiv \chi,
\end{equation}
and so on. The components of a spinor are now wave-functions, instead of 
complex numbers. In this scheme, the operator equation (\ref{e5.164})  becomes simply
\begin{equation}
\chi' = \left(\frac{\hbar}{2}\right) \sigma_k \,\chi.
\end{equation}

Consider the operator relation
\begin{equation}\label{e5.168}
||A'\rangle\rangle = p_k ||A\rangle\rangle.
\end{equation}
In the Schr\"{o}dinger representation, we have
\begin{eqnarray}
\langle x', y', z'|\langle + |A'\rangle\rangle& =&
\langle x', y', z'|p_k \langle +||A\rangle\rangle \nonumber\\[0.5ex]
&=& -{\rm i}\,\hbar\frac{\partial}
{\partial x_k'} \langle x', y', z'| \langle +||A\rangle\rangle,\\[0.5ex]
\langle x', y', z'|\langle - |A'\rangle\rangle& =&
\langle x', y', z'|p_k \langle -||A\rangle\rangle\nonumber\\[0.5ex]
& =& -{\rm i}\,\hbar\frac{\partial}
{\partial x_k'} \langle x', y', z'| \langle -||A\rangle\rangle,
\end{eqnarray}
where use has been made of Eq.~(\ref{e3.67}). The above equation reduces to
\begin{equation}
\left(\! \begin{array}{c} \psi_+'({\bf r}')\\
\psi_-' ({\bf r}') \end{array}\!\right) = 
\left(\begin{array}{c} -{\rm i}\,\hbar \,\partial \psi_+({\bf r}') /\partial x_k'\\
 -{\rm i}\,\hbar \,\partial \psi_-({\bf r}')/\partial x_k'\end{array}\!\right).
\end{equation}
Thus, the operator equation (\ref{e5.168})
can be written
\begin{equation}
\chi' = p_k\, \chi,
\end{equation}
where 
\begin{equation}\label{e5.172}
p_k \rightarrow -{\rm i}\,\hbar\frac{\partial}{\partial x_k'}\, {\bf I}.
\end{equation}
Here, ${\bf I}$ is the $2\times 2$ unit matrix. In fact, any position operator
({\em e.g.}, $p_k$ or $L_k$) is represented in the Pauli scheme as some differential
operator of the position eigenvalues multiplied by the $2\times2$ unit matrix. 

What about combinations of position and spin operators? The most
commonly occurring combination  is a dot product: {\em e.g.}, 
${\bf S}\!\cdot\! {\bf L} = (\hbar/2)\,\bsigma\! \cdot\! {\bf L}$. 
Consider the hybrid
operator  $\bsigma \!\cdot \!{\bf a}$, where ${\bf a} \equiv (a_x, a_y, a_z)$ is
some  vector position  operator. This quantity is represented as 
a $2\times 2$ matrix:
\begin{equation}\label{e5.173}
\bsigma\! \cdot\! {\bf a} \equiv \sum_k a_k \sigma_k = 
\left(\!\begin{array}{cc} +a_3 & a_1 -{\rm i}\,a_2\\[0.5ex]
a_1 + {\rm i}\,a_2 & -a_3 \end{array}\!\right).
\end{equation}
Since, in the Schr\"{o}dinger representation, a general position operator takes
the form of a differential operator in $x'$, $y'$, or $z'$, it is clear that
the above quantity must be regarded as a matrix differential operator which
acts on spinors of the general form (\ref{e5.166}).
The important identity
\begin{equation}\label{e5.174}
(\bsigma \!\cdot \!{\bf a} ) \,(\bsigma \!\cdot\! {\bf b}) = {\bf a}\! \cdot\! {\bf b}
+{\rm i}\,\bsigma\!\cdot \!({\bf a} \times {\bf b} )
\end{equation}
follows from the commutation and anti-commutation relations (\ref{e5.160}) and (\ref{e5.161}). Thus,
\begin{eqnarray}
\sum_j \sigma_j \,a_j \sum_k \sigma_k \,b_k &=& \sum_j \sum_k \left(\frac{1}{2}\,
\{\sigma_j, \sigma_k\} + \frac{1}{2} [\sigma_j, \sigma_k]\right) a_j \,b_k\nonumber\\[0.5ex]
&=& \sum_j \sum_k (\sigma_{jk} + {\rm i}\,\epsilon_{jkl} \,\sigma_l)\,a_j \,b_k\nonumber\\[0.5ex]
&=& {\bf a} \!\cdot\! {\bf b}
+{\rm i}\,\bsigma\!\cdot\! ({\bf a} \times {\bf b} ).
\end{eqnarray}

A general rotation operator in spin space is written
\begin{equation}
T (\Delta\phi) = \exp\left(-{\rm i} \,{\bf S}\!\cdot\!{\bf n}\,\Delta\varphi/\hbar\right),
\end{equation}
by analogy with Eq.~(\ref{e5.116}), where ${\bf n}$ is a unit vector pointing along
the axis of rotation, and $\Delta\varphi$ is the angle of rotation.
Here, ${\bf n}$ can be regarded as a trivial position operator.  The
rotation operator is represented 
\begin{equation}
\exp\left(-{\rm i} \,{\bf S}\!\cdot\!{\bf n}\,\Delta\varphi/\hbar\right)\rightarrow
\exp\left(-{\rm i} \,\bsigma\!\cdot\!{\bf n}\,\Delta\varphi/2\right)
\end{equation}
in the Pauli scheme. 
The term on the right-hand side of the above  expression is the exponential
of a matrix. This can easily be evaluated using the Taylor series for an exponential,
plus the rules
\begin{eqnarray}
(\bsigma\!\cdot \!{\bf n})^n &=& 1 \mbox{\hspace{1.79cm}for $n$ even},\\[0.5ex]
(\bsigma\!\cdot \!{\bf n})^n &=&(\bsigma\!\cdot \!{\bf n})
 \mbox{\hspace{1cm}for $n$ odd}.
\end{eqnarray}
These rules follow trivially from the identity (\ref{e5.174}).  Thus, we can write
\begin{eqnarray}
\exp\left(-{\rm i} \,\bsigma\!\cdot\!{\bf n}\,\Delta\varphi/2\right)&=& 
\left[ 1 - \frac{(\bsigma \! \cdot \! {\bf n})^2}{2!} \left(\frac{\Delta\varphi}{2}
\right)^2 + \frac{(\bsigma \! \cdot \! {\bf n})^4}{4!} \left(\frac{\Delta\varphi}{2}
\right)^4 + \cdots\right]\nonumber\\[0.5ex]
&& - {\rm i} \left[ (\bsigma \! \cdot {\bf n} )\left( \frac{\Delta\varphi}{2}\right)-
\frac{(\bsigma \! \cdot \! {\bf n})^3}{3!} \left(\frac{\Delta\varphi}{2}\right)^3
+ \cdots\right]\nonumber\\[0.5ex]
&=& \cos(\Delta\varphi/2)\,{\bf I}- {\rm i}\,\sin (\Delta\varphi/2)\,\bsigma\!\cdot\! {\bf n}.
\end{eqnarray}
The explicit $2\times 2$ form of this matrix is
\begin{equation}
\left(\begin{array}{cc} 
\cos(\Delta\varphi/2) - {\rm i}\,n_z \sin(\Delta\varphi/2)&
(-{\rm i}\,n_x - n_y) \sin(\Delta\varphi/2) \\[0.5ex]
(-{\rm i}\,n_x + n_y) \sin(\Delta\varphi/2) &
\cos(\Delta\varphi/2) + {\rm i}\, n_z \sin(\Delta\varphi/2)
\end{array} \right).
\end{equation}
Rotation matrices act on spinors in much the same manner as the corresponding
rotation operators act on state kets. Thus,
\begin{equation}
\chi' = \exp\left(-{\rm i} \,\bsigma\!\cdot\!{\bf n}\,\Delta\varphi/2\right) \chi,
\end{equation}
where $\chi'$ denotes the  spinor  obtained after rotating the spinor
$\chi$ an angle $\Delta\varphi$ about the ${\bf n}$-axis. 
The Pauli matrices remain unchanged under rotations. 
However, the quantity $\chi^\dagger \,\sigma_k \,\chi$ is proportional to the expectation
value of $S_k$ [see Eq.~(\ref{e5.159})], so we would expect it to transform like a
vector under rotation (see Sect.~\ref{s5.9}). In fact, we
require
\begin{equation}
(\chi^\dagger \,\sigma_k \,\chi)' \equiv (\chi^\dagger)' \sigma_k\, \chi' = \sum_l R_{kl}\,
(\chi^\dagger \sigma_l\, \chi),
\end{equation}
where the $R_{kl}$ are the elements of a conventional rotation matrix. This
is easily demonstrated, since
\begin{equation}
\exp\left(\frac{\,{\rm i}\,\sigma_3 \,\Delta\varphi}{2}\right) \sigma_1 \exp\left(
\frac{-{\rm i}\,\sigma_3 \,\Delta\varphi}{2}\right) = \sigma_1 \cos\Delta\varphi
-\sigma_2 \sin\Delta\varphi
\end{equation}
plus all cyclic permutations. The above expression is the $2\times 2$ matrix analogue
of (see Sect.~\ref{s5.9})
\begin{equation}
\exp\left(\frac{\,{\rm i}\,S_z \,\Delta\varphi}{\hbar}\right) S_x \exp\left(
\frac{-{\rm i}\,S_z \,\Delta\varphi}{\hbar}\right) = S_x\,  \cos\Delta\varphi
-S_y\, \sin\Delta\varphi.
\end{equation}
The previous two formulae can both be validated using the Baker-Hausdorff lemma,
(\ref{e5.123}), which holds for Hermitian matrices, in addition to Hermitian operators. 

\section{Spin Greater Than One-Half Systems}
In the absence of spin, the Hamiltonian can be written as some function
of the position and momentum operators. Using the Schr\"{o}dinger representation,
in which ${\bf p} \rightarrow -{\rm i}\,\hbar\,\nabla$, the energy eigenvalue
problem,
\begin{equation}
H\,|E\rangle = E\,|E\rangle,
\end{equation}
can be transformed into a partial differential equation for the wave-function
$\psi({\bf r}') \equiv \langle {\bf r'}|E\rangle$. This function specifies the
probability density for observing the particle at a given position, ${\bf r}'$. 
In general, we find
\begin{equation}
H \,\psi = E\, \psi,
\end{equation}
where $H$ is now a partial differential operator.
 The boundary conditions (for a bound state) are obtained
 from the normalization constraint
\begin{equation}
\int |\psi|^2\, dV = 1.
\end{equation}

This is all very familiar. However, we now know how to generalize this scheme
to  deal with  a spin one-half particle. Instead of representing the 
state of the particle by a single wave-function, we use {\em two} wave-functions.
The first, $\psi_+({\bf r'})$, specifies the probability density of
observing the particle at position ${\bf r}'$
with spin angular momentum $+\hbar/2$ 
in the  $z$-direction. The second, $\psi_-({\bf r'})$, specifies the
 probability density of
observing the particle at position ${\bf r}'$  with spin angular momentum $-\hbar/2$ 
in the $z$-direction. In the Pauli scheme, these wave-functions
are combined into a {\em spinor}, $\chi$, which is simply the row vector of $\psi_+$ and $\psi_-$. 
In general, the Hamiltonian is a function of the position, momentum, and spin
operators. Adopting the Schr\"{o}dinger representation, and the Pauli scheme,
the energy eigenvalue problem reduces to 
\begin{equation}\label{e5.188}
H \,\chi = E \,\chi,
\end{equation}
where $\chi$ is a spinor ({\em i.e.}, a $1\times 2$
 matrix of wave-functions)
 and $H$ is a $2\times 2$ matrix partial differential operator [see Eq.~(\ref{e5.173})]. 
The above spinor equation can always be  written out explicitly  as {\em two
coupled partial differential equations} for $\psi_+$ and $\psi_-$. 

Suppose that the Hamiltonian has no dependence on the spin operators. In this
case, the Hamiltonian is represented as {\em diagonal}\/ $2\times 2$ matrix partial
differential operator in the Schr\"{o}dinger/Pauli scheme [see Eq.~(\ref{e5.172})].
In other words, the partial differential equation for $\psi_+$ decouples
from that for $\psi_-$. In fact, both equations have the same form, so there
is only really one differential equation.  In this
situation, the most general  solution to Eq.~(\ref{e5.188})  can be written
\begin{equation}\label{e5.189}
\chi = \psi({\bf r}') \left(\!\begin{array}{c} c_+\\ c_-\end{array}\!\right).
\end{equation}
Here, $\psi({\bf r}')$ is determined by the solution of the differential equation,
and the $c_\pm$ are arbitrary complex numbers. The physical significance of
the above expression is clear. The Hamiltonian determines the relative probabilities
of finding the particle at various different positions, but the direction
of its spin angular momentum remains undetermined.

Suppose that the Hamiltonian depends only on the spin operators. In this
case, the Hamiltonian is represented as a $2\times 2$ matrix of complex numbers 
in the Schr\"{o}dinger/Pauli scheme [see Eq.~(\ref{e5.158})], and the spinor eigenvalue
equation (\ref{e5.188})  reduces to a straight-forward matrix eigenvalue problem. 
The most general solution can again be written
\begin{equation}\label{e5.190}
\chi = \psi({\bf r}') \left(\!\begin{array}{c} c_+\\ c_-\end{array}\!\right).
\end{equation}
Here, the ratio $c_+/c_-$ is determined by the matrix eigenvalue problem,
and the wave-function $\psi({\bf r}')$ is arbitrary. Clearly, the Hamiltonian
determines the direction of the particle's spin angular momentum, but leaves
its position undetermined. 

In general, of course, the Hamiltonian is a function of both position and
spin operators. In this case, it is not possible to decompose the 
spinor as in Eqs.~(\ref{e5.189})  and (\ref{e5.190}).
 In other words, a general Hamiltonian causes the
direction of the particle's spin angular momentum to vary with position in
some specified manner. This can only be represented as a spinor involving 
different wave-functions, $\psi_+$ and $\psi_-$. 

But, what happens if we have a spin one  or a spin three-halves particle? 
It turns out that we can generalize the Pauli two-component scheme in a fairly
straight-forward manner. Consider a spin-$s$ particle: {\em i.e.}, a particle for which
the eigenvalue of $S^2$ is $s\,(s+1)\,\hbar^2$. Here, $s$ is either an integer, or a half-integer. The eigenvalues of $S_z$ are written $s_z\,\hbar$, where
$s_z$ is allowed to take the values $s, s-1, \cdots, -s+1, -s$. In fact,
there are $2\,s+1$ distinct allowed values of $s_z$. Not surprisingly, we can represent
the state of the particle by $2\,s+1$ different wave-functions, denoted $\psi_{s_z}
({\bf r}')$. Here, $\psi_{s_z}({\bf r}')$ specifies the probability density 
for observing the particle at position ${\bf r'}$ with spin angular
momentum $s_z\,\hbar$ in the $z$-direction. More exactly,
\begin{equation}
\psi_{s_z}({\bf r}') = \langle {\bf r'}|\langle s, s_z| |A\rangle\rangle,
\end{equation}
where $||A\rangle\rangle$ denotes a state ket in the product space of the position
and spin operators. The state of the particle can be represented more
succinctly by a spinor, $\chi$, which is simply the $2\,s+1$ component row
vector  of the $\psi_{s_z}({\bf r}')$.
Thus, a spin one-half particle is represented by a two-component spinor,
a spin one particle by a three-component spinor, a spin three-halves particle
by a four-component spinor, and so on. 

In this extended Schr\"{o}dinger/Pauli
scheme, position space operators take the form of {\em diagonal} $(2\,s+1) \times (2\,s+1)$
matrix differential operators. Thus, we can represent the momentum operators
as [see
Eq.~(\ref{e5.172})]
\begin{equation}
p_k \rightarrow -{\rm i}\,\hbar \,\frac{\partial}{\partial x_k'}\, {\bf I},
\end{equation}
where ${\bf I}$ is the $(2\,s+1)\times (2\,s+1)$ unit matrix.
 We represent the spin
operators as
\begin{equation}
S_k \rightarrow s\,\hbar \,\sigma_k,
\end{equation}
where the $(2\,s+1)\times (2\,s+1)$ extended Pauli matrix $\sigma_k$ has elements
\begin{equation}\label{e5.194}
(\sigma_k)_{jl} = \frac{ \langle s, j|S_k | s, l\rangle}{s\,\hbar}.
\end{equation}
Here, $j, l$ are integers, or half-integers, lying in the range $-s$ to $+s$. 
But, how can we evaluate the brackets $\langle s, j|S_k | s, l\rangle$
and, thereby, construct the extended Pauli matrices? In fact, it is trivial
to construct the $\sigma_z$ matrix. By definition,
\begin{equation}
S_z | s, j\rangle = j\,\hbar\, | s, j\rangle.
\end{equation}
Hence, 
\begin{equation}\label{e5.196}
(\sigma_z)_{jl} = \frac{\langle s, j|S_z | s, l\rangle}{s\,\hbar}
= \frac{j}{s}\, \delta_{jl},
\end{equation}
where use has been made of the orthonormality property of the $| s, j\rangle$.
Thus, $\sigma_z$ is the suitably normalized diagonal matrix of the eigenvalues
of $S_z$. The matrix elements of $\sigma_x$ and $\sigma_y$ are most easily
obtained by considering the shift operators,
\begin{equation}\label{e5.197}
S^\pm = S_x \pm {\rm i}\, S_y.
\end{equation}
We know, from Eqs.~(\ref{e5.44a})--(\ref{e5.44b}), that
\begin{eqnarray}\label{e5.198a}
S^+ |s, j\rangle &=& \sqrt{s\,(s+1) - j \,(j+1)} \,\hbar\, |s, j+1\rangle,\\[0.5ex]
S^- |s, j\rangle &=& \sqrt{s\,(s+1) - j \,(j-1)}\, \hbar \,|s, j-1\rangle.\label{e5.198b}
\end{eqnarray} 
It follows from Eqs.~(\ref{e5.194}), and (\ref{e5.197})--(\ref{e5.198b}), that
\begin{eqnarray}\label{e5.199a}
(\sigma_x)_{jl} &=& \frac{\sqrt{ s\,(s+1) - j\,(j-1)} \,\delta_{j, l+1}}{2\,s}\nonumber\\[0.5ex]
&&+ \frac{\sqrt{s\,(s+1) - j\,(j+1)}\,\delta_{j, l-1} }{2\,s},\\[0.5ex]
(\sigma_y)_{jl} &=& \frac{\sqrt{ s\,(s+1) - j\,(j-1)} \,\delta_{j, l+1}}{2\,{\rm i}\,s}\nonumber\\[0.5ex]
&&- \frac{\sqrt{s\,(s+1) - j\,(j+1)}\,\delta_{j, l-1} }{2\,{\rm i}\,s}.\label{e5.199b}
\end{eqnarray}
According to Eqs.~(\ref{e5.196}) and (\ref{e5.199a})--(\ref{e5.199b}), the Pauli matrices for a spin one-half 
($s=1/2$)
particle are 
\begin{eqnarray}
\sigma_x &=& \left(\!\begin{array}{cc} 0 &1\\1&0\end{array}\!\right),\\[0.5ex]
\sigma_y &=& \left(\!\begin{array}{cc} 0 &-{\rm i}\\{\rm i}&0\end{array}\!\right),\\[0.5ex]
\sigma_z &=& \left(\!\begin{array}{cc} 1 &0\\0&-1\end{array}\!\right),
\end{eqnarray}
as we have seen previously. For a spin one ($s=1$) particle, we find that
\begin{eqnarray}
\sigma_x &=&\frac{1}{\sqrt{2}}\left(\!
\begin{array}{ccc} 0 &1&0\\1&0&1\\0&1&0\end{array}\!\right),\\[0.5ex]
\sigma_y &=& \frac{1}{\sqrt{2}}
\left(\!\begin{array}{ccc} 0 &-{\rm i}&0\\{\rm i}&0&{-\rm i}\\
0&{\rm i}& 0\end{array}\!\right),\\[0.5ex]
\sigma_z &=& \left(\!\begin{array}{ccc} 1 &0&0\\0&0&0\\0&0&-1\end{array}\!\right).
\end{eqnarray}
In fact, we can now construct the Pauli matrices for a spin anything particle. 
This means that we can convert the general energy eigenvalue problem for a spin-$s$ particle, where the Hamiltonian is some function of position and spin operators,
into $2\,s+1$ coupled partial differential equations involving the 
$2\,s+1$ wave-functions
$\psi_{s_z}({\bf r'})$. Unfortunately, such a system
of equations is generally too complicated
to solve exactly.

\section{Addition of Angular Momentum}
Consider a hydrogen atom in an $l=1$ state. The electron
possesses orbital angular momentum of magnitude $\hbar$, and spin angular
momentum of magnitude $\hbar/2$. So, what is the total angular momentum of
the system? 

In order to answer this question, we are going to have to learn how to add
angular momentum operators. Let us consider the most general case. Suppose
that we have two sets of angular momentum operators, ${\bf J}_1$ and ${\bf J}_2$.
By definition, these operators are Hermitian, and obey the fundamental commutation
relations
\begin{eqnarray}
{\bf J}_1\times {\bf J}_1& =& {\rm i}\,\hbar\,{\bf J}_1,\\[0.5ex]
{\bf J}_2\times {\bf J}_2& =& {\rm i}\,\hbar\,{\bf J}_2.
\end{eqnarray}
We assume that the two groups of operators  correspond to different degrees
of freedom of the system, so that 
\begin{equation}
[J_{1i}, J_{2j}] = 0,
\end{equation}
where $i, j$ stand for either $x$, $y$, or $z$. 
For instance, ${\bf J}_1$ could be an orbital angular momentum operator, and ${\bf J}_2$
a spin angular momentum operator. Alternatively, ${\bf J}_1$ and ${\bf J}_2$ could
be the orbital angular momentum operators
 of two different particles in a multi-particle
system. We know, from the general
properties of angular momentum, that the eigenvalues of $J_1^{~2}$ and $J_2^{~2}$ 
can be
written $j_1\,(j_1+1)\,\hbar^2$ and $j_2\,(j_2+1)\, \hbar^2$, respectively, where
$j_1$ and $j_2$ are either integers, or half-integers. We also know that the
eigenvalues of $J_{1z}$ and $J_{2z}$ take the form $m_1\,\hbar$ and
$m_2\,\hbar$, respectively, where $m_1$ and $m_2$ are numbers
lying  in the ranges $j_1, j_1-1,\cdots,
-j_1+1, -j_1$ and  $j_2, j_2-1,\cdots,
-j_2+1, -j_2$, respectively.

Let us define the total angular momentum operator
\begin{equation}
{\bf J} = {\bf J}_1 + {\bf J}_2.
\end{equation}
Now  ${\bf J}$ is an Hermitian operator, since it is the sum of Hermitian operators. 
According to Eqs.~(\ref{e5.5}) and (\ref{e5.8}), ${\bf J}$ satisfies the fundamental commutation
relation
\begin{equation}
{\bf J} \times {\bf J} = {\rm i}\,\hbar\, {\bf J}.
\end{equation}
Thus, ${\bf J}$ possesses all of the expected properties of an
angular momentum operator. It follows that the eigenvalue of $J^{\,2}$ can be
written $j\,(j+1)\,\hbar^2$, where $j$ is an integer, or a half-integer. The eigenvalue
of $J_z$ takes the form $m\,\hbar$, where $m$ lies in the range $j, j-1,\cdots,
-j+1, -j$. At this stage, we do not know the relationship between the quantum
numbers of the total angular momentum, $j$ and $m$, and those of the
individual angular momenta, $j_1$, $j_2$, $m_1$, and $m_2$. 

Now
\begin{equation}\label{e5.206}
J^{\,2} = J_1^{~2} + J_2^{~2} + 2\,{\bf J}_1 \!\cdot\! {\bf J}_2.
\end{equation}
We know that
\begin{eqnarray}
[J_1^{~2}, J_{1i} ] &=& 0,\\[0.5ex]
[J_2^{~2}, J_{2i} ] &=& 0,
\end{eqnarray}
and also that all of the $J_{1i}$ operators commute with the $J_{2i}$ operators. 
It follows from Eq.~(\ref{e5.206}) that
\begin{equation}
[J^{\,2}, J_1^{~2}] = [J^{\,2}, J_2^{~2}] = 0.
\end{equation}
This implies  that the quantum numbers $j_1$, $j_2$, and $j$ can all be measured
simultaneously. In other words, we can know the magnitude of the total
angular momentum  together with the magnitudes of the component
angular momenta. However, it is clear from Eq.~(\ref{e5.206})
that
\begin{eqnarray}
[J^2, J_{1z}] &\neq& 0,\\[0.5ex]
[J^2, J_{2z}] &\neq& 0.
\end{eqnarray}
This suggests  that it is not possible to measure the quantum numbers $m_1$ and $m_2$
simultaneously with the quantum number $j$. Thus, we cannot determine
the projections of the individual angular momenta along the $z$-axis
at the same time as the magnitude of the total angular momentum.

It is clear, from the preceding discussion, that we can form two alternate groups
of mutually commuting operators. The first group
is $J_1^{~2}, J_2^{~2}, J_{1z}$, and
$J_{2z}$. The second group is $J_1^{~2}, J_2^{~2}, J^{\,2},$ and $J_z$. These two
groups of operators are incompatible with one another.  We can define simultaneous 
eigenkets of each operator group. The simultaneous eigenkets of
$J_1^{~2}, J_2^{~2}, J_{1z}$, and
$J_{2z}$ are denoted $|j_1,j_2; m_1,m_2\rangle$, where
\begin{eqnarray}
J_1^{~2} |j_1,j_2; m_1,m_2\rangle &=& j_1\,(j_1+1)\,\hbar^2\,|j_1,j_2; m_1,m_2\rangle,
\\[0.5ex]
J_2^{~2} |j_1,j_2; m_1,m_2\rangle &=& j_2\,(j_2+1)\,\hbar^2\,|j_1,j_2; m_1,m_2\rangle,
\\[0.5ex]
J_{1z} |j_1,j_2; m_1,m_2\rangle &=& m_1\,\hbar\,|j_1,j_2; m_1,m_2\rangle,
\\[0.5ex]
J_{2z} |j_1,j_2; m_1,m_2\rangle &=& m_2\,\hbar\,|j_1,j_2; m_1,m_2\rangle.
\end{eqnarray}
The simultaneous eigenkets of
$J_1^{~2}, J_2^{~2}, J^{\,2}$ and $J_z$ are denoted 
$|j_1, j_2; j, m\rangle$, where 
\begin{eqnarray}
J_1^{~2} |j_1,j_2; j,m\rangle &=& j_1\,(j_1+1)\,\hbar^2\,|j_1,j_2; j,m\rangle,
\\[0.5ex]
J_2^{~2} |j_1,j_2; j,m\rangle &=& j_2\,(j_2+1)\,\hbar^2\,|j_1,j_2; j,m\rangle,
\\[0.5ex]
J^2 |j_1,j_2; j,m\rangle &=& j\,(j+1)\,\hbar^2\,|j_1,j_2; j,m\rangle,
\\[0.5ex]
J_{z} |j_1,j_2; j,m\rangle &=& m\,\hbar\,|j_1,j_2; j,m\rangle.
\end{eqnarray}
Each set of eigenkets are complete, mutually orthogonal (for eigenkets corresponding
to different sets of eigenvalues), and have unit norms. Since the operators
$J_1^{~2}$ and $J_2^{~2}$ are common to both  operator groups, we can  assume
that the quantum numbers $j_1$ and $j_2$ are known. In other words, we 
can always determine
 the magnitudes of the individual angular momenta. In addition, we can either
know the quantum numbers $m_1$ and $m_2$, or the quantum numbers $j$ and
$m$, but we cannot know both pairs of quantum numbers at the same time. 
We can write a conventional completeness relation for both sets of
eigenkets:
\begin{eqnarray}\label{e5.212a}
\sum_{m_1}\sum_{m_2 }|j_1,j_2; m_1, m_2\rangle \langle j_1,j_2; m_1, m_2|& =&1,\\
[0.5ex]
\sum_{j}\sum_{m} |j_1,j_2; j, m\rangle \langle j_1,j_2; j, m|& =&1,
\end{eqnarray}
where the right-hand sides denote  the identity operator in the ket space corresponding
to states of given $j_1$ and $j_2$. The summation is over all allowed values
of $m_1$, $m_2$, $j$, and $m$.

The operator group $J_1^{~2}$, $J_2^{~2}$, $J^{\,2}$, and $J_z$
is incompatible with the group $J_1^{~2}$, $J_2^{~2}$, $J_{1z}$, and $J_{2z}$.
This means that if the system is in a simultaneous eigenstate of the former group
then, in general, it is not in an eigenstate of the latter. In other words,
if the quantum numbers $j_1$, $j_2$, $j$, and $m$ are known with
certainty, then a measurement of the quantum numbers $m_1$ and $m_2$ will
give a range of possible values. We can use the completeness relation
(\ref{e5.212a}) to write
\begin{equation}
|j_1,j_2;j,m\rangle = \sum_{m_1}\sum_{m_2} \langle j_1,j_2;m_1,m_2|j_1,j_2;j,m\rangle
|j_1,j_2;m_1,m_2\rangle.
\end{equation}
Thus, we can write the eigenkets of the first group of operators
as a weighted sum of the eigenkets of the second set. The weights,
$\langle j_1,j_2;m_1,m_2|j_1,j_2;j,m\rangle$, are called the {\em Clebsch-Gordon
coefficients}. If the system is in a state where a measurement of
$J_1^{~2}, J_2^{~2}, J^{\,2}$, and $J_z$ is bound to give the results
$j_1\,(j_1+1)\,\hbar^2, j_2\,(j_2+1)\,\hbar^2, j\,(j+1)\,\hbar^2$,
and $j_z\,\hbar$, respectively, then a measurement of $J_{1z}$ and $J_{2z}$
will give the results $m_1\,\hbar$ and $m_2\,\hbar$ with
probability $|\langle j_1,j_2;m_1,m_2|j_1,j_2;j,m\rangle|^2$. 

The Clebsch-Gordon coefficients possess a number of very important properties.
First, the coefficients are zero unless
\begin{equation}\label{e5.214}
m = m_1 + m_2.
\end{equation}
To prove this, we note that
\begin{equation}
(J_z - J_{1z} - J_{2z}) |j_1,j_2; j, m\rangle =0.
\end{equation}
Forming the inner product with $\langle j_1, j_2; m_1, m_2|$, we obtain
\begin{equation}
(m-m_1-m_2) \langle j_1, j_2; m_1, m_2|j_1,j_2; j, m\rangle=0,
\end{equation}
which proves the assertion. Thus, the $z$-components of different angular momenta
add algebraically. So, an electron in an $l=1$ state, with orbital
angular momentum $\hbar$, and spin angular momentum $\hbar/2$, projected along the
$z$-axis,  constitutes a state whose total angular momentum projected
along the $z$-axis is $3\hbar/2$. What is uncertain is the magnitude of the
total angular momentum. 

Second, the coefficients vanish unless
\begin{equation}\label{e5.217}
|j_1-j_2| \leq j \leq j_1+j_2.
\end{equation}
We can assume, without loss of generality, that $j_1\geq j_2$. We know,
from Eq.~(\ref{e5.214}), that for given
$j_1$ and $j_2$ the largest possible value of $m$ is $j_1+j_2$ (since
$j_1$ is the largest possible value of $m_1$, {\em etc}.). This implies that
the largest possible value of $j$ is $j_1+j_2$ (since, by definition,
the largest value of $m$ is equal to $j$).
Now, there are $(2\,j_1+1)$ allowable values of $m_1$ and $(2\,j_2+1)$ allowable
values of $m_2$. Thus, there are $(2\,j_1+1)\,(2\,j_2+1)$ independent
eigenkets, $|j_1, j_2; m_1, m_2\rangle$, needed to span the ket space
corresponding to fixed $j_1$ and $j_2$. Since the eigenkets
$|j_1, j_2; j, m\rangle$ span the same space, they must also form
a set of  $(2\,j_1+1)\,(2\,j_2+1)$ independent kets. In other words, there
can only be $(2\,j_1+1)\,(2\,j_2+1)$ distinct  allowable values of the quantum numbers
$j$ and $m$. For each allowed value of $j$, there are $2\,j+1$ allowed values
of $m$. We have already seen that the maximum allowed value of $j$ is
$j_1+j_2$. It is easily seen that if the minimum allowed value of
$j$ is $j_1-j_2$ then the total number of allowed values of $j$ and $m$
is  $(2\,j_1+1)\,(2\,j_2+1)$: {\em i.e.},
\begin{equation}
\sum_{j=j_1-j_2}^{j_1+j_2} (2\,j+1) \equiv (2\,j_1+1)\,(2\,j_2+1).
\end{equation}
This proves our assertion. 

Third, the sum of the modulus squared of all of the Clebsch-Gordon coefficients
is unity: {\em i.e.},
\begin{equation}\label{e5.219}
\sum_{m_1}\sum_{m_2} |\langle j_1,j_2;m_1,m_2|j_1,j_2;j,m\rangle|^2 =1.
\end{equation}
This assertion is proved as follows:
\begin{eqnarray}
&&\langle j_1, j_2; j, m| j_1, j_2; j, m\rangle =\mbox{\hspace{7cm}}\nonumber\\[0.5ex] 
&&\sum_{m_1}\sum_{m_2} \langle j_1, j_2; j, m|j_1, j_2; m_1, m_2\rangle
\langle j_1, j_2; m_1, m_2|j_1, j_2; j, m\rangle\nonumber\\[0.5ex]
&&~~~=\sum_{m_1}\sum_{m_2}  |\langle j_1,j_2;m_1,m_2|j_1,j_2;j,m\rangle|^2 =1,
\end{eqnarray}
where use has been made of the completeness relation (\ref{e5.212a}).

Finally, the Clebsch-Gordon coefficients obey two recursion relations. 
To obtain these relations we start from
\begin{eqnarray}
J^{\pm} |j_1,j_2;j,m\rangle& =& (J_1^\pm + J_2^\pm )\\[0.5ex]
&&\times\sum_{m_1'}\sum_{m_2'} \langle j_1, j_2; m_1', m_2'|j_1, j_2; j, m\rangle
|j_1, j_2; m_1', m_2'\rangle.\nonumber
\end{eqnarray}
Making use of the well-known properties of the shift operators,
which are specified  by Eqs.~(\ref{e5.44a})--(\ref{e5.44b}), we obtain
\begin{eqnarray}
\sqrt{j\,(j+1)- m\,(m\pm 1)}\, |j_1,j_2;j,m\pm 1\rangle =&&\nonumber\\[0.5ex]
\sum_{m_1'}\sum_{m_2'} \left( \sqrt{j_1\,(j_1+1)- m_1'\,(m_1'\pm 1)}\,
|j_1, j_2; m_1'\pm 1, m_2'\rangle\right.&&\nonumber\\[0.5ex]
 \left.+\sqrt{ j_2\,(j_2+1)- m_2'\,(m_2'\pm 1)}\,
|j_1, j_2; m_1', m_2'\pm 1\rangle\right) \nonumber\\[0.5ex]
 \times\langle j_1, j_2; m_1', m_2'|j_1, j_2; j, m\rangle.&&
\end{eqnarray}
Taking the inner product with $\langle j_1, j_2; m_1, m_2|$, and making
use of the orthonormality property of the basis eigenkets, we obtain 
the desired recursion relations:
\begin{eqnarray}\label{e5.223}
\sqrt{j\,(j+1)- m\,(m\pm 1)} \,\langle j_1, j_2; m_1, m_2|j_1,j_2;j, m\pm 1\rangle=
&&\nonumber \\[0.5ex]
\sqrt{j_1\,(j_1+1) - m_1\,(m_1\mp 1)} \,
\langle j_1, j_2; m_1\mp 1, m_2|j_1,j_2;j, m\rangle&&\nonumber \\[0.5ex]
+ \sqrt{j_2\,(j_2+1) - m_2\,(m_2\mp 1)} \,
\langle j_1, j_2; m_1, m_2\mp 1|j_1,j_2;j, m\rangle.
\end{eqnarray}
It is clear, from the absence of complex coupling coefficients in the above relations,
that we can always choose the Clebsch-Gordon coefficients to be real numbers.
This is a convenient choice, since it ensures that the inverse Clebsch-Gordon
coefficients, $\langle j_1, j_2; j, m|j_1, j_2; m_1, m_2\rangle$, are
identical to the Clebsch-Gordon coefficients. In other words,
\begin{equation}
\langle j_1, j_2; j, m|j_1, j_2; m_1, m_2\rangle = 
\langle j_1, j_2; m_1, m_2|j_1, j_2; j, m\rangle.
\end{equation}
The inverse Clebsch-Gordon coefficients are the weights in the expansion
of the $|j_1,j_2; m_1, m_2\rangle$ in terms of the $|j_1, j_2; j,m\rangle$:
\begin{equation}
|j_1,j_2; m_1,m_2\rangle = \sum_{j}\sum_m \langle j_1, j_2; j, m|j_1, j_2; m_1, m_2\rangle |j_1,j_2; j,m\rangle.
\end{equation}

It turns out that the recursion relations (\ref{e5.223}), together with the normalization
condition (\ref{e5.219}), are sufficient to completely determine the Clebsch-Gordon
coefficients to within an arbitrary sign (multiplied into
all of the coefficients). This sign is fixed by convention. The easiest
way of demonstrating this assertion is by considering some specific examples. 
 
Let us add the angular momentum of two spin one-half systems: {\em e.g.}, two
electrons at rest. So, $j_1=j_2=1/2$. We know, from general principles,
that $|m_1| \leq 1/2$ and $|m_2|\leq 1/2$. We also know, from Eq.~(\ref{e5.217}),
that $0\leq j\leq 1$, where the allowed values of $j$ differ by integer amounts.
It follows that either $j=0$ or $j=1$. Thus, two spin one-half systems can
be combined  to form either a spin zero  system or a spin one system. 
It is helpful to arrange  all of the possibly non-zero Clebsch-Gordon coefficients
in a table:
$$
\begin{tabular}{||c|c||c|c|c||c|}\hline
$m_1$    & $m_2$      &     &    &      &     \\ \hline\hline
1/2      &  1/2       & ?   & ?  &  ?   & ?   \\ \hline
1/2      & -1/2       & ?   & ?  &  ?   & ?   \\ \hline
-1/2     &  1/2       & ?   & ?  &  ?   & ?   \\ \hline
-1/2     & -1/2       & ?   & ?  &  ?   & ?   \\ \hline\hline
 $\scriptstyle j_1=1/2$        &  $j$       & 1   & 1  &  1   & 0   \\  \hline
  $\scriptstyle j_2=1/2$       &  $m$       & 1   & 0  & -1   & 0   \\ \hline
\end{tabular}
$$
The box in this table corresponding to $m_1=1/2, m_2=1/2, j=1, m=1$ gives
the Clebsch-Gordon coefficient $\langle 1/2, 1/2; 1/2, 1/2
| 1/2, 1/2; 1, 1\rangle$,
or the inverse Clebsch-Gordon coefficient $\langle  1/2, 1/2; 1, 1|
1/2, 1/2; 1/2, 1/2
\rangle$. All the boxes contain question marks  because  we do
not know any  Clebsch-Gordon coefficients at the moment.

A Clebsch-Gordon coefficient is automatically zero unless
$m_1+m_2=m$. In other words, the $z$-components of angular momentum have to
add algebraically. Many of the boxes in the above table correspond to
$m_1+m_2\neq m$. We immediately conclude that these boxes must contain zeroes:
{\em  i.e.},
$$
\begin{tabular}{||c|c||c|c|c||c|}\hline
$m_1$    & $m_2$      &     &    &      &     \\ \hline\hline
1/2      &  1/2       & ?   & 0  &  0   & 0   \\ \hline
1/2      & -1/2       & 0   & ?  &  0   & ?   \\ \hline
-1/2     &  1/2       & 0   & ?  &  0   & ?   \\ \hline
-1/2     & -1/2       & 0   & 0  &  ?   & 0   \\ \hline\hline
 $\scriptstyle j_1=1/2$        &  $j$       & 1   & 1  &  1   & 0   \\  \hline
  $\scriptstyle j_2=1/2$       &  $m$       & 1   & 0  & -1   & 0   \\ \hline
\end{tabular}
$$

The normalization condition (\ref{e5.219}) implies  that the sum of the squares
of all the rows and columns of the above table must be unity. There are two
rows and two columns which only contain a single non-zero entry. We conclude that
these entries must be $\pm 1$, but we have no way of determining the 
signs at present. Thus,
$$
\begin{tabular}{||c|c||c|c|c||c|}\hline
$m_1$    & $m_2$      &     &    &      &     \\ \hline\hline
1/2      &  1/2       & $\pm 1 $  & 0  &  0   & 0   \\ \hline
1/2      & -1/2       & 0   & ?  &  0   & ?   \\ \hline
-1/2     &  1/2       & 0   & ?  &  0   & ?   \\ \hline
-1/2     & -1/2       & 0   & 0  &  $\pm 1$   & 0   \\ \hline\hline
 $\scriptstyle j_1=1/2$        &  $j$       & 1   & 1  &  1   & 0   \\  \hline
  $\scriptstyle j_2=1/2$       &  $m$       & 1   & 0  & -1   & 0   \\ \hline
\end{tabular}
$$

Let us evaluate the  recursion relation (\ref{e5.223})  for $j_1=j_2=1/2$, with
$j=1$, $m=0$, $m_1=m_2=\pm 1/2$, taking the upper/lower sign. We
find that
\begin{equation}
\langle 1/2, -1/2|1, 0\rangle
+ \langle -1/2, 1/2|1, 0\rangle =\sqrt{2}\,\langle 1/2, 1/2|1,1\rangle=\pm\sqrt{2},
\end{equation}
and 
\begin{equation}
 \langle 1/2, -1/2|1, 0\rangle
+ \langle -1/2, 1/2|1, 0\rangle =\sqrt{2}\, \langle -1/2, -1/2|1,-1\rangle
=\pm\sqrt{2}.
\end{equation}
Here, the $j_1$ and $j_2$ labels have been suppressed for ease of notation.
We also know that 
\begin{equation}
\langle 1/2, -1/2|1, 0\rangle^2 + 
\langle -1/2, 1/2|1, 0\rangle^2 = 1,
\end{equation}
from the normalization condition. The only real solutions to the above set
of equations are
\begin{eqnarray}
\sqrt{2} \,\langle 1/2, -1/2|1, 0\rangle &=& \sqrt{2} \,\langle -1/2, 1/2|1, 0\rangle
\nonumber\\[0.5ex]
&= &\langle 1/2,1/2|1,1\rangle = \langle 1/2,1/2|1,-1\rangle = \pm 1.
\end{eqnarray}
The choice of sign is arbitrary---the conventional choice is a positive
sign. Thus, our table now reads
$$
\begin{tabular}{||c|c||c|c|c||c|}\hline
$m_1$    & $m_2$      &     &    &      &     \\ \hline\hline
1/2      &  1/2       & 1   & 0  &  0   & 0   \\ \hline
1/2      & -1/2       & 0   & $1/\sqrt{2}$  &  0   & ?   \\ \hline
-1/2     &  1/2       & 0   & $1/\sqrt{2}$ &  0   & ?   \\ \hline
-1/2     & -1/2       & 0   & 0  &  1   & 0   \\ \hline\hline
 $\scriptstyle j_1=1/2$        &  $j$       & 1   & 1  &  1   & 0   \\  \hline
  $\scriptstyle j_2=1/2$       &  $m$       & 1   & 0  & -1   & 0   \\ \hline
\end{tabular}
$$

We could fill in the remaining unknown entries of our table by using the recursion
relation again. However, an easier method is to observe that the rows and columns
of the table must all be mutually orthogonal. That is, the dot product
of a row with any other row must be zero. Likewise, for the dot product of
a column with any other column. This follows because the entries in the
table give the expansion coefficients of one of our alternative sets of eigenkets
in terms of the other set, and each set of eigenkets contains
mutually orthogonal vectors with unit norms. The normalization
condition tells us that the dot product of a row or column with itself must
be unity. The only way that the dot product of the fourth column with
the second column can be zero is if the unknown entries are equal and opposite. 
The requirement that the dot product of the fourth column with itself is
unity tells us that the magnitudes of the unknown entries have to be $1/\sqrt{2}$. 
The unknown entries are undetermined to an arbitrary sign multiplied into them both.
Thus, the final form of our table (with the conventional choice of arbitrary
signs) is
$$
\begin{tabular}{||c|c||c|c|c||c|}\hline
$m_1$    & $m_2$      &     &    &      &     \\ \hline\hline
1/2      &  1/2       & 1   & 0  &  0   & 0   \\ \hline
1/2      & -1/2       & 0   & $1/\sqrt{2}$  &  0   & $1/\sqrt{2}$   \\ \hline
-1/2     &  1/2       & 0   & $1/\sqrt{2}$ &  0   & -$1/\sqrt{2}$   \\ \hline
-1/2     & -1/2       & 0   & 0  &  1   & 0   \\ \hline\hline
 $\scriptstyle j_1=1/2$        &  $j$       & 1   & 1  &  1   & 0   \\  \hline
  $\scriptstyle j_2=1/2$       &  $m$       & 1   & 0  & -1   & 0   \\ \hline
\end{tabular}
$$

The table can be read in one of two ways. The columns give the expansions
of the eigenstates of overall  angular momentum in terms of the eigenstates
 of the individual
angular momenta of the two component systems. Thus, the second column
tells us that
\begin{equation}
|1,0\rangle = \frac{1}{\sqrt{2}} \left(\,|1/2,-1/2\rangle  + |-1/2,1/2\rangle\,\right).
\end{equation}
The ket on the left-hand side is a $|j,m\rangle$ ket, whereas those on the
right-hand side are $|m_1, m_2\rangle$ kets. The rows give the expansions
of the eigenstates of individual angular momentum in terms of those of overall
angular momentum. Thus, the second row tells us that
\begin{equation}
|1/2,-1/2\rangle = \frac{1}{\sqrt{2}} \left(\,|1,0\rangle + |0,0\rangle\,\right).
\end{equation}
Here, the ket on the left-hand side is a  $|m_1, m_2\rangle$ ket, whereas those
on the right-hand side are $|j, m\rangle$ kets.

Note that our table is really a combination of two sub-tables, one involving
$j=0$ states, and one involving $j=1$ states. The Clebsch-Gordon coefficients
corresponding to two different choices of $j$ are completely independent:
{\em i.e.}, there is no recursion relation linking Clebsch-Gordon coefficients
corresponding to different values of $j$. Thus, for every choice of $j_1$, $j_2$,
and $j$ we can construct a table of Clebsch-Gordon coefficients corresponding
to the different allowed values of $m_1$, $m_2$, and $m$ (subject to the
constraint that $m_1+m_2=m$). A complete knowledge of angular momentum addition
is equivalent to a knowing all possible tables of Clebsch-Gordon coefficients.
These tables are listed (for moderate values of
$j_1, j_2$ and $j$) in many standard reference books. 
