\chapter{Spin Angular Momentum}\label{s5}
% !TEX root = ../Quantum.tex

\section{Introduction}
Up to now, we have tacitly assumed that the state of a particle in quantum
mechanics can be completely specified by giving the wavefunction $\psi$ 
as a function of the spatial coordinates $x$, $y$, and $z$. Unfortunately,
there is a wealth of experimental evidence that suggests that this simplistic
approach is incomplete. 

Consider an isolated system at rest, and let the eigenvalue of its total
angular momentum be $j\,(j+1)\,\hbar^2$. According to the theory of orbital
angular momentum outlined in Sections~\ref{s5.4} and \ref{s5.5}, there are two possibilities.
For a system consisting of a single particle, $j=0$. For a system consisting
of two (or more) particles, $j$ is a non-negative integer. 
However, this does not
agree with observations, because we often encounter systems that appear to
be structureless, and yet have $j\neq 0$. Even worse, systems where $j$
has half-integer values abound in nature. 
In order to  explain this apparent discrepancy
between theory and experiments, Gouldsmit and Uhlenbeck (in 1925)
introduced the concept of an internal, purely quantum mechanical, angular momentum
called {\em spin}. For a particle with spin, the total angular momentum in the
rest frame is non-vanishing. 

\section{Properties of Spin Angular Momentum}
Let us denote the three components of the spin angular momentum of a
particle by the Hermitian operators
 $(S_x, S_y, S_z)\equiv {\bf S}$. We assume that these 
operators obey the fundamental commutation relations (\ref{e5.4a})--(\ref{e5.4c}) for the components
of an angular momentum. Thus, we can write
\begin{equation}\label{e5.100}
{\bf S} \times {\bf S} = {\rm i}\,\hbar \, {\bf S}.
\end{equation}
We can also define the operator
\begin{equation}
S^2 = S_x^{\,2}+S_y^{\,2} + S_z^{\,2}.
\end{equation}
According to the quite general analysis of Section~\ref{s5.1},
\begin{equation}
[{\bf S}, S^2] = 0.
\end{equation}
Thus, it is possible to find simultaneous eigenstates of $S^2$ and $S_z$. 
These are denoted $|s, s_z\rangle$, where
\begin{align}
S_z \,|s, s_z\rangle &= s_z \,\hbar \,|s, s_z\rangle,\\[0.5ex]
S^2 \,|s, s_z\rangle &= s\,(s+1)\,\hbar^2\, |s, s_z\rangle.
\end{align}
According to the equally general
analysis of Section~\ref{s5.2}, the quantum number $s$ can, in principle, 
take integer or half-integer values,
and the quantum number $s_z$ can only take the values $s, s-1 \cdots -s+1, -s$. 

Spin angular momentum clearly has many properties in common with
orbital angular momentum. However, there is one vitally important difference. 
Spin angular momentum operators {\em cannot}\/ be expressed  in terms of
position and momentum operators, like  in Equations~(\ref{e5.1a})--(\ref{e5.1c}), because this 
identification depends on an analogy with classical mechanics, and the concept
of spin is purely quantum mechanical: {\rm i.e.}, it has no analogy in classical physics. 
Consequently, the restriction that the quantum number of the overall angular
momentum must take {\em integer}\/ values is lifted for spin angular momentum,
since this restriction (found in Sections~\ref{s5.3} and \ref{s5.4}) depends on Equations~(\ref{e5.1a})--(\ref{e5.1c}).
In other words, the spin quantum number $s$ is allowed to take {\em half-integer}\/ values.

Consider a spin one-half particle, for which
\begin{align}\label{e5.104a}
S_z \,|\pm \rangle &=\pm \frac{\hbar}{2} \,|\pm \rangle,\\[0.5ex]
S^2\, |\pm\rangle &= \frac{3 \,\hbar^2}{4}\,|\pm \rangle.\label{e5.104b}
\end{align}
Here, the $|\pm \rangle$ denote eigenkets of the $S_z$ operator corresponding to
the eigenvalues $\pm \hbar/2$. These kets are mutually orthogonal (since $S_z$ is
an Hermitian operator), so
\begin{equation}
\langle +| -\rangle = 0.
\end{equation}
They  are also properly normalized and complete, so that
\begin{equation}
\langle +| + \rangle=\langle -| - \rangle = 1,
\end{equation}
and
\begin{equation}
|+\rangle \langle +| + |-\rangle \langle -| = 1.
\end{equation}

It is easily verified that the Hermitian operators defined by
\begin{align}\label{e5.108a}
S_x &= \frac{\hbar}{2} \left(\, |+\rangle \langle -| + |-\rangle \langle +|\,
\right),
\\[0.5ex]
S_y &= \frac{{\rm i}\,\hbar}{2}\left(\, -\,|+\rangle \langle -| +
 |-\rangle \langle +|\,\right),
\\[0.5ex]
S_z &= \frac{\hbar}{2}\left(\, |+\rangle \langle +| - |-\rangle \langle -|\,\right),\label{e5.108c}
\end{align}
satisfy the commutation relations (\ref{e5.4a})--(\ref{e5.4c}) (with the $L_j$ replaced by the $S_j$).
The operator $S^2$ takes the form 
\begin{equation}\label{e5.109}
S^2 = \frac{3\,\hbar^2}{4}.
\end{equation}
It is also easily demonstrated that $S^2$ and $S_z$,
defined in this manner,  satisfy the eigenvalue
relations (\ref{e5.104a})--(\ref{e5.104b}). Equations~(\ref{e5.108a})--(\ref{e5.109}) constitute a realization
of the spin operators ${\bf S}$ and $S^2$ (for a spin one-half particle)
in {\em spin space}\/ ({\rm i.e.}, the Hilbert sub-space consisting of kets which 
correspond to the different spin states of the particle). 

\section{Wavefunction of  Spin One-Half Particle}\label{s5.8}
The state of a spin one-half particle is represented as a vector in ket space.
Let us suppose that this space is spanned by the basis kets
$|x', y', z', \pm\rangle$. Here, $|x',y',z', \pm\rangle$ denotes a
simultaneous eigenstate of the position operators $x$, $y$, $z$, and
the spin operator $S_z$, corresponding to the eigenvalues $x'$, $y'$, $z'$,
and $\pm \hbar/2$, respectively. The basis kets are assumed to
satisfy the completeness relation
\begin{equation}
\int\!\int\!\int  dx'dy'dz' \left(\,
|x',y',z',+\rangle\langle x', y', z',+|+|x',y',z',-\rangle\langle x', y', z',-|
\,\right)= 1.
\end{equation}

It is helpful to think of the ket $|x', y', z', +\rangle$ as the product
of two kets---a position space ket $|x', y', z'\rangle$, and
a spin space ket $|+\rangle$. We assume that such a product obeys
the commutative and distributive axioms of multiplication:
\begin{align}
|x', y', z'\rangle |+\rangle &= |+\rangle |x', y', z'\rangle,\\[0.5ex]
\left(c'\, |x', y', z'\rangle + c''\,| x'', y'', z''\rangle\right)\,
|+\rangle &= c'\, |x', y', z'\rangle |+\rangle+ c'' \,|x'', y'', z''\rangle |+\rangle,
\\[0.5ex]
|x', y', z'\rangle\left(c_+ \,|+\rangle + c_-\, |-\rangle\right)&= c_+ \,
|x', y', z'\rangle|+\rangle+ c_-\,|x', y', z'\rangle|-\rangle,
\end{align}
where the $c$'s are numbers. We can give meaning to any
position space operator (such as $L_z$)  acting on the product $|x', y', z'\rangle
|+\rangle$ by assuming that it operates only on the $|x', y', z'\rangle$
factor, and commutes with the $|+\rangle$ factor. 
Similarly, we can give a meaning to any spin operator  (such as $S_z$) acting
on $|x', y', z'\rangle
|+\rangle$ by assuming that it operates only on $|+\rangle$, and
commutes with $|x', y', z'\rangle$. This implies that every position
space operator
commutes with every spin operator. In this manner, we can give
meaning to the equation
\begin{equation}\label{e5.112}
 |x', y', z', \pm\rangle = |x', y', z'\rangle| \pm\rangle = | \pm\rangle
|x', y', z'\rangle.
\end{equation}

The multiplication in the above equation is of  a quite different type to
any that we have encountered previously. The ket vectors $|x',y', z'\rangle$ and
$|\pm\rangle$ lie in two completely separate vector spaces, and their product
$|x',y', z'\rangle|\pm\rangle$ lies in a third vector space. 
In mathematics, the latter space
is termed the {\em product space}\/ of the former spaces, which are
termed {\em factor spaces}.  The number of
dimensions of a product space is equal to the product of the number of dimensions
of each of the factor spaces. A general ket of the product space is not
of the form (\ref{e5.112}), but is instead a sum or integral of kets of this form. 

A general state $A$ of a spin one-half particle is represented as a ket
$||A\rangle\rangle$ in the product  of the spin and position spaces. 
This state can be completely specified by {\em two}\/ wavefunctions:
\begin{align}
\psi_+(x', y', z') &= \langle x', y', z' |\langle +||A\rangle\rangle,\\[0.5ex]
\psi_-(x', y', z') &= \langle x', y', z' |\langle -||A\rangle\rangle.
\end{align}
The probability of observing the particle in the region $x'$ to $x'+dx'$,
$y'$ to $y'+dy'$, and $z'$ to $z'+dz'$, with $s_z = +1/2$ is
$|\psi_+ (x', y', z')|^{\,2}\,dx' dy' dz'$. Likewise, 
the probability of observing the particle in the region $x'$ to $x'+dx'$,
$y'$ to $y'+dy'$, and $z'$ to $z'+dz'$, with $s_z = -1/2$ is
$|\psi_- (x', y', z')|^{\,2}\,dx' dy' dz'$.
The normalization condition for the wavefunctions is
\begin{equation}
\int\!\int\!\int dx'dy'dz' \left(|\psi_+|^{\,2} + |\psi_-|^{\,2}\right)= 1.
\end{equation}

\section{Rotation Operators in Spin Space}\label{s5.9}
Let us, for the moment, forget about the spatial position of the particle,
and concentrate on its spin state. A general
spin state $A$ is represented by the ket
\begin{equation}\label{e5.115}
|A\rangle = \langle +|A\rangle |+\rangle + \langle -|A\rangle |-\rangle
\end{equation}
in spin space.
In Section~\ref{s5.3}, we were able  to construct an operator $R_z({\mit\Delta}\varphi)$ that 
rotates the system through an angle ${\mit\Delta}\varphi$ about the $z$-axis in position
space. Can we also construct an operator $T_z({\mit\Delta}\varphi)$ that rotates the
system through an angle ${\mit\Delta}\varphi$ about the $z$-axis in spin space? By analogy
with Equation~(\ref{e5.50}), we would expect such an operator to take the form
\begin{equation}\label{e5.116}
T_z({\mit\Delta}\varphi) = \exp(-{\rm i} \,S_z\, {\mit\Delta}\varphi/\hbar).
\end{equation}
Thus, after rotation, the ket $|A\rangle$ becomes
\begin{equation}
|A_R\rangle = T_z({\mit\Delta}\varphi)\, |A\rangle.
\end{equation}

To demonstrate that the operator (\ref{e5.116})  really does rotate the spin of the system,
let us consider its effect on $\langle S_x\rangle$. Under rotation, this
expectation value changes as follows:
\begin{equation}
\langle S_x\rangle \rightarrow \langle A_R| \,S_x\, |A_R \rangle
= \langle A| \,T_z^{\dag}\, S_x \,T_z \,|A\rangle.
\end{equation}
Thus, we need to compute
\begin{equation}\label{e5.119}
\exp(\,{\rm i}\,S_z\, {\mit\Delta}\varphi/\hbar)\, S_x \,
\exp(-{\rm i}\,S_z \,{\mit\Delta}\varphi/\hbar).
\end{equation}
This can be achieved in two different ways. 

First, we can use the explicit formula for $S_x$ given in Equation~(\ref{e5.108a}). We find
that Equation~(\ref{e5.119}) becomes
\begin{equation}
\frac{\hbar}{2} \,\exp(\,{\rm i}\,S_z\, {\mit\Delta}\varphi/\hbar)\,
(\,|+\rangle \langle -| + |-\rangle \langle +|\,)\,
\exp(-{\rm i}\,S_z \,{\mit\Delta}\varphi/\hbar),
\end{equation}
or 
\begin{equation}
\frac{\hbar}{2} 
\left( {\rm e}^{\,{\rm i}\,{\mit\Delta}\varphi/2}\,|+\rangle \langle -|\,
{\rm e}^{\,{\rm i}\,{\mit\Delta}\varphi/2} + {\rm e}^{\,-{\rm i}\,{\mit\Delta}\varphi/2}
\,|-\rangle \langle +|\,{\rm e}^{\,-{\rm i}\,{\mit\Delta}\varphi/2}\right),
\end{equation}
which reduces to
\begin{equation}
S_x\,\cos{\mit\Delta}\varphi - S_y\,\sin{\mit\Delta}\varphi,
\end{equation}
where use has been made of Equations~(\ref{e5.108a})--(\ref{e5.108c}).

A second approach is to use the so called {\em Baker-Hausdorff lemma}. This
takes the form
\begin{align}
\exp(\,{\rm i}\, G\,\lambda)\,A\, \exp(-{\rm i} \,G \,\lambda)&= A + {\rm i} \,\lambda\,
[G,A] + \left(\frac{{\rm i}^2 \lambda^2}{2!}\right) [G, [G,A]]+\nonumber\\[0.5ex]
&+   
\left(\frac{{\rm i}^3\lambda^3}{3!}\right)[G, [G, [G,A]]]+\cdots,\label{e5.123}
\end{align}
where $G$ is an Hermitian operator, and $\lambda$ a real parameter. The proof
of this lemma is left as an exercise. Applying the Baker-Hausdorff lemma
to Equation~(\ref{e5.119}), we obtain 
\begin{equation}
S_x + \left(\frac{{\rm i}\,{\mit\Delta}\varphi}{\hbar}\right) [S_z, S_x]
+ \left(\frac{1}{2!}\right) \left(\frac{{\rm i}\,{\mit\Delta}\varphi}{\hbar}\right)^2
[S_z, [S_z, S_x]] + \cdots,
\end{equation}
which reduces to 
\begin{equation}
S_x\left[ 1- \frac{({\mit\Delta}\varphi)^2}{2!} + \frac{({\mit\Delta}\varphi)^4}{4!} + \cdots\right] - S_y \left[\varphi - 
\frac{({\mit\Delta}\varphi)^3}{3!}+  \frac{({\mit\Delta}\varphi)^5}{5!} +\cdots\right],
\end{equation}
or
\begin{equation}
S_x\,\cos{\mit\Delta}\varphi - S_y\,\sin{\mit\Delta}\varphi,
\end{equation}
where  use has been made of Equation~(\ref{e5.100}). The second
proof is more general than the first, because it only uses the fundamental
 commutation relation (\ref{e5.100}), and is, therefore, valid for systems with spin
angular momentum higher than one-half.

For a  spin one-half system, both methods imply that
\begin{equation}\label{e5.127}
\langle S_x \rangle \rightarrow \langle S_x\rangle \,\cos{\mit\Delta}\varphi
- \langle S_y\rangle\,\sin{\mit\Delta}\varphi
\end{equation}
under the action of the rotation operator (\ref{e5.116}). It is straightforward to
show that 
\begin{equation}
\langle S_y \rangle \rightarrow \langle S_y\rangle \,\cos{\mit\Delta}\varphi
+ \langle S_x\rangle\,\sin{\mit\Delta}\varphi.
\end{equation}
Furthermore, 
\begin{equation}\label{e5.129}
\langle S_z \rangle \rightarrow\langle S_z\rangle,
\end{equation}
because $S_z$ commutes with the rotation operator. Equations~(\ref{e5.127})--(\ref{e5.129})
 demonstrate that
the operator (\ref{e5.116}) rotates the expectation value of ${\bf S}$ by an
angle ${\mit\Delta} \varphi$ about the $z$-axis. In fact, the expectation value
of the spin operator behaves like a classical  vector under rotation:
\begin{equation}
\langle S_k \rangle \rightarrow \sum_l R_{k\,l}\, \langle S_l\rangle,
\end{equation}
where the $R_{k\,l}$ are the elements of the  conventional rotation matrix 
for the rotation in question. It is clear, from our  second derivation of
the result (\ref{e5.127}), that this property is not restricted to the spin operators of
a spin one-half system. In fact, we have effectively demonstrated that
\begin{equation}
\langle J_k \rangle \rightarrow \sum_l R_{k\,l} \,\langle J_l\rangle,
\end{equation}
where the $J_k$ are the generators of rotation, satisfying the fundamental
commutation relation ${\bf J}\times {\bf J} = {\rm i}\,\hbar\, {\bf J}$,
and the rotation operator about the $k$th axis is written
$R_k ({\mit\Delta}\varphi) = \exp(-{\rm i}\,J_k\, {\mit\Delta}\varphi/\hbar)$. 

Consider the effect of the rotation operator (\ref{e5.116}) on the state ket (\ref{e5.115}).
It is easily seen that
\begin{equation}\label{e5.132}
T_z({\mit\Delta}\varphi)\,|A\rangle = {\rm e}^{-{\rm i}\,{\mit\Delta}\varphi/2}\,
\langle +|A\rangle |+\rangle +  {\rm e}^{\,{\rm i}\,{\mit\Delta}\varphi/2}\,
\langle -|A\rangle |-\rangle.
\end{equation}
Consider a rotation by $2\pi$ radians. We find that
\begin{equation}\label{e5.133}
|A\rangle \rightarrow T_z(2\pi)\,|A\rangle = -|A\rangle.
\end{equation}
Note that a ket rotated by $2\pi$ radians differs from the original ket by a
{\em minus}\/ sign. In fact, a rotation by $4\pi$ radians is needed to transform a ket
into itself. The minus sign does not affect the expectation value of
${\bf S}$, since ${\bf S}$ is sandwiched between $\langle A|$ and $| A\rangle$,
both of which change sign. Nevertheless, the minus sign does give rise to
observable consequences, as we shall see  presently. 

\section{Magnetic Moments}\label{s5.5c}
Consider a particle of electric charge $q$ and speed $v$
performing  a  circular orbit of radius $r$ 
in the $x$-$y$ plane. The charge is equivalent to a current loop of radius $r$
in the $x$-$y$ plane carrying  current $I=q\,v/2\pi\, r$. The magnetic moment
$\bmu$  of
the loop is of magnitude $\pi\,  r^2\, I$ and is directed along the $z$-axis.
Thus, we can write
\begin{equation}
\bmu = \frac{q}{2}\, {\bf x} \times {\bf v},
\end{equation}
where ${\bf x}$ and ${\bf v}$ are the vector position and velocity of the particle,
respectively. However, we know that ${\bf p} = {\bf v} /m$, where ${\bf p}$
is the vector momentum of the particle, and $m$ is its mass. We also know that
${\bf L} = {\bf x}\times {\bf p}$, where ${\bf L}$ is the orbital angular momentum.
It follows that
\begin{equation}
\bmu = \frac{q}{2\,m} \,{\bf L}.
\end{equation}
Using the  usual analogy between classical and quantum mechanics, we 
expect the above relation to also hold between the quantum mechanical operators,
$\bmu$ and ${\bf L}$, which represent magnetic moment and orbital angular momentum,
respectively.
This is indeed found to the the case. 

Spin angular momentum also gives rise to a contribution to the magnetic
moment of a charged particle.  In fact, relativistic quantum
mechanics  predicts that a charged particle possessing spin  must also
possess a corresponding magnetic moment (this was first demonstrated by Dirac---see Chapter~\ref{c11}). We can write
\begin{equation}\label{e5.44r}
\bmu  = \frac{q}{2\,m} \left({\bf L} + g \,{\bf S}\right),
\end{equation}
where $g$ is called the {\em gyromagnetic ratio}. For an electron this ratio
is found to be
\begin{equation}
g_e = 2\left( 1 + \frac{1}{2\pi} \frac{e^2}{4\pi \,\epsilon_0\,\hbar \,c} \right).
\end{equation}
The factor 2 is correctly predicted by Dirac's relativistic theory of the electron (see Chapter~\ref{c11}).
The small correction $1/(2\pi\, 137)$, derived  originally by Schwinger, is due to
quantum field effects. We shall ignore this correction in the following,
so
\begin{equation}\label{e5.138}
\bmu  \simeq - \frac{e}{2\,m_e} \left({\bf L} + 2 \,{\bf S}\right)
\end{equation}
for an electron (here, $e>0$).

\section{Spin Precession}
The Hamiltonian for an electron at rest in a $z$-directed  magnetic field, ${\bf B}=
B\,{\bf e}_z$,
is
\begin{equation}\label{e5.139}
H = - \bmu \cdot {\bf B} = \left(\frac{e}{m_e}\right) {\bf S} \cdot {\bf B}
= \omega\, S_z,
\end{equation}
where
\begin{equation}\label{e5.140}
\omega = \frac{e\,B}{m_e}.
\end{equation}
According to Equation~(\ref{e4.28}), the time evolution operator for this system is
\begin{equation}
T(t,0) = \exp(-{\rm i} \,H\, t/\hbar) = \exp(-{\rm i} \,S_z\, \omega\, t/\hbar).
\end{equation}
It can be seen, by comparison with Equation~(\ref{e5.116}), that the time evolution operator
is precisely the same as the rotation operator for spin, with ${\mit\Delta}\varphi$ set
equal to $\omega \,t$. It is immediately clear that the Hamiltonian (\ref{e5.139})
 causes the electron 
spin to precess about the $z$-axis with angular frequency $\omega$. In fact,
Equations~(\ref{e5.127})--(\ref{e5.129}) imply  that
\begin{align}
\langle S_x\rangle_t &= \langle S_x\rangle_{t=0} \cos(\omega \,t) - 
\langle S_y\rangle_{t=0} \sin(\omega\, t),\\[0.5ex]
\langle S_y\rangle_t &= \langle S_y\rangle_{t=0} \cos(\omega\, t) +
\langle S_x\rangle_{t=0} \sin(\omega\, t),\\[0.5ex]
\langle S_z\rangle_t&= \langle S_z\rangle_{t=0}.
\end{align}
The time evolution of the state ket is given by analogy with Equation~(\ref{e5.132}):
\begin{equation}
|A, t\rangle = {\rm e}^{-{\rm i}\,\omega \,t/2}\,
\langle +|A, 0\rangle |+\rangle +  {\rm e}^{\,{\rm i}\,\omega \,t/2}\,
\langle -|A, 0\rangle |-\rangle.
\end{equation}
Note that it takes time $t= 4\pi/\omega$ for the state ket to return  to its
original state. 
By contrast, it only takes times $t=2\pi/\omega$ for the spin vector to point
in its original direction. 

We now describe an experiment to detect the minus sign in Equation~(\ref{e5.133}). An almost
 monoenergetic beam of neutrons is split in two, sent along two different
paths, $A$ and $B$, and then recombined. Path $A$ goes through a magnetic field
free region. However, path $B$ enters a small  region where a static magnetic
field is present. As a result, a neutron state ket going along path
$B$ acquires a phase-shift $\exp(\mp{\rm i}\, \omega \,T/2)$ (the $\mp$
signs correspond to $s_z = \pm 1/2$ states). Here, $T$ is the
time spent in the magnetic field, and $\omega$ is the spin precession frequency
\begin{equation}
\omega = \frac{g_n\, e\,B}{m_p}.
\end{equation}
This frequency is defined in an analogous manner to Equation~(\ref{e5.140}). The gyromagnetic
ratio for a neutron is found experimentally to be $g_n = -1.91$. 
(The magnetic moment of a neutron is entirely a quantum field effect).
When neutrons from path $A$ and path $B$ meet they undergo interference. We
expect the observed  neutron intensity in the interference region to
exhibit a $\cos( \pm \omega\, T/2 + \delta)$ variation, 
where $\delta$ is the phase difference 
between paths $A$ and $B$ in the absence of a magnetic field. In experiments,
the time of flight $T$ through the magnetic field region is kept constant, while
the field-strength $B$ is varied. It follows that the change in magnetic
field required to produce successive maxima is 
\begin{equation}
{\mit\Delta} B = \frac{4\pi \,\hbar}{e\, g_n\, \lambdabar\, l},
\end{equation}
where $l$ is the path-length through the magnetic field region, and $\lambdabar$
is the de Broglie wavelength over $2\pi$ of the neutrons. The above prediction has been verified
experimentally to within a fraction of a percent. This prediction depends crucially
on the fact that it takes a $4\pi$ rotation to return a state ket to its
original state. If it only took a $2\pi$ rotation then ${\mit\Delta} B$ would be half
of the value given above, which does not agree with the experimental data. 

\section{Pauli Two-Component Formalism}\label{spauli}
We have seen, in Section~\ref{s5.4}, that the eigenstates of orbital angular momentum
can be conveniently represented as spherical harmonics. In this
representation, the orbital  angular momentum 
operators take the form of  differential operators involving only
angular coordinates. It is conventional to represent the eigenstates of spin
angular momentum as column (or row) matrices. In this representation,
the spin angular momentum operators take the form of matrices. 

The matrix representation of a  spin one-half system was introduced by Pauli in 1926. 
Recall, from Section~\ref{s5.9}, that a general spin ket can be expressed as
a linear combination of the two eigenkets of $S_z$ belonging to the
eigenvalues $\pm \hbar/2$. These are denoted  $|\pm\rangle$. Let us
represent these basis eigenkets as column vectors:
\begin{align}
|+\rangle&\rightarrow \left(\!\begin{array}{c}1\\0\end{array}\!\right) \equiv \chi_+,\\[0.5ex]
|-\rangle &\rightarrow \left(\!\begin{array}{c}0\\1\end{array}\!\right) \equiv \chi_-.
\end{align}
The corresponding eigenbras are represented as row vectors:
\begin{align}
\langle +| &\rightarrow (1, 0) \equiv \chi_+^{\dag}, \\[0.5ex]
\langle - |&\rightarrow (0, 1) \equiv \chi_-^{\dag}.
\end{align}
In this scheme, a general  ket takes the form 
\begin{equation}\label{e5.148}
|A\rangle = \langle +|A\rangle |+\rangle + \langle -|A\rangle |-\rangle
\rightarrow
\left(\!\begin{array}{c}\langle +|A\rangle\\
\langle -|A\rangle\end{array}\!\right),
\end{equation}
and a general bra becomes
\begin{equation}\label{e5.149}
\langle A| =\langle A|+\rangle \langle +| + \langle A|-\rangle \langle -|
\rightarrow (\langle A|+\rangle, \langle A|-\rangle).
\end{equation}
The column vector (\ref{e5.148})  is called a two-component {\em spinor}, and can be written
\begin{equation}
\chi \equiv \left(\!\begin{array}{c}\langle +|A\rangle\\
\langle -|A\rangle\end{array}\!\right) =\left(\!\begin{array}{c}
c_+\\ c_- \end{array}\!\right) = c_+\, \chi_+ + c_- \,\chi_-,
\end{equation}
where the $c_\pm$ are complex numbers. The row vector (\ref{e5.149}) becomes
\begin{equation}
\chi^{\dag} \equiv (\langle A|+\rangle, \langle A|-\rangle) =(c_+^{~\ast}, c_-^{~\ast}) =c_+^{~\ast}\, \chi_+^{\dag} + c_-^{~\ast}\,\chi_-^{\dag}.
\end{equation}

Consider the ket obtained by the action of a  spin operator on 
ket $A$:
\begin{equation}\label{e5.152}
|A'\rangle = S_k \,|A\rangle.
\end{equation}
This ket is represented as
\begin{equation}
|A'\rangle 
\rightarrow
\left(\!\begin{array}{c}\langle +|A'\rangle\\
\langle -|A'\rangle\end{array}\!\right)\equiv \chi'.
\end{equation}
However,
\begin{align}
\langle + |A'\rangle &= \langle + |\,S_k\,| +\rangle \langle +|A\rangle
+ \langle +|\,S_k\, |-\rangle \langle -|A\rangle,\\[0.5ex]
\langle - |A'\rangle &= \langle - |\,S_k\,| +\rangle \langle +|A\rangle
+ \langle -|\,S_k\, |-\rangle \langle -|A\rangle,
\end{align}
or
\begin{equation}
\left(\!\begin{array}{c}\langle +|A'\rangle\\[0.5ex]
\langle -|A'\rangle\end{array}\!\right) = 
\left(\!\begin{array}{cc}
\langle + |\,S_k\,| +\rangle&\langle +|\,S_k\, |-\rangle\\[0.5ex]
\langle - |\,S_k\,| +\rangle& \langle -|\,S_k\, |-\rangle\end{array}\!\right)
\left(\!\begin{array}{c}\langle +|A\rangle\\[0.5ex]
\langle -|A\rangle\end{array}\!\right).\label{e5.155}
\end{equation}
It follows that we can represent the operator/ket relation
(\ref{e5.152})  as the matrix relation
\begin{equation}
\chi' =\left( \frac{\hbar}{2}\right)\sigma_k \,\chi,
\end{equation}
where the $\sigma_k$ are the matrices of the $\langle \pm |\,S_k\,|\pm \rangle$
values divided by $\hbar/2$. These matrices, which are called the
{\em Pauli matrices}, can easily be evaluated using  the explicit forms for the
spin operators given in Equations~(\ref{e5.108a})--(\ref{e5.108c}). We find that
\begin{align}\label{e5.157a}
\sigma_1 &= \left(\!\begin{array}{rr} 0 &1\\1&0\end{array}\!\right),\\[0.5ex]
\sigma_2 &= \left(\!\begin{array}{rr} 0 &-{\rm i}\\{\rm i}&0\end{array}\!\right),\\[0.5ex]
\sigma_3 &= \left(\!\begin{array}{rr} 1 &0\\0&-1\end{array}\!\right).\label{e5.157c}
\end{align}
Here, 1, 2, and 3 refer to $x$, $y$, and $z$, respectively. Note that, in this
scheme, we are effectively representing the spin operators in terms
of the Pauli matrices:
\begin{equation}\label{e5.158}
S_k \rightarrow \left(\frac{\hbar}{2} \right)\sigma_k.
\end{equation}
The expectation value of $S_k$ can be written in terms of spinors
and the Pauli matrices:
\begin{equation}\label{e5.159}
\langle S_k \rangle = \langle A|\,S_k \,|A\rangle = \sum_\pm
\langle A|\pm \rangle \langle \pm |\,S_k\,|\pm \rangle \langle \pm |A\rangle
= \left(\frac{\hbar}{2}\right) \,\chi^{\dag}\, \sigma_k\, \chi.
\end{equation}

The fundamental commutation relation for angular momentum, Equation~(\ref{e5.100}), can
be combined with (\ref{e5.158}) to give the following commutation relation
for the Pauli matrices:
\begin{equation}\label{e5.160}
\bsigma\times \bsigma = 2\,{\rm i}\,\bsigma.
\end{equation}
It is easily seen that the matrices (\ref{e5.157a})--(\ref{e5.157c}) actually satisfy these relations
({\rm i.e.},  $\sigma_1\, \sigma_2 - \sigma_2\,\sigma_1 = 2\,{\rm i} \,\sigma_3$, plus
all cyclic permutations). It is also easily seen that the Pauli matrices
satisfy the anti-commutation relations
\begin{equation}\label{e5.161}
\{ \sigma_i, \sigma_j \} = 2 \,\delta_{ij}.
\end{equation}
Here, $\{a,b\}\equiv a\,b+b\,a$. 

Let us examine how the Pauli scheme can be extended to take into account the
position of a spin one-half particle. Recall, from Section~\ref{s5.8},
that we can represent a general basis ket  as the product
of basis kets in position space and  spin space:
\begin{equation}
|x', y', z', \pm\rangle = |x',y',z'\rangle |\pm \rangle = |\pm \rangle|
x',y',z'\rangle .
\end{equation}
The ket corresponding to state $A$ is denoted $||A\rangle\rangle$, and resides
in the product space of the position and spin ket spaces. State $A$ is completely
specified by the two wavefunctions
\begin{align}
\psi_+(x', y', z') &= \langle x',y', z'|\langle +||A\rangle\rangle,\\[0.5ex]
\psi_-(x', y', z') &= \langle x',y',z'|\langle -||A\rangle\rangle.
\end{align}
Consider the operator relation
\begin{equation}\label{e5.164}
||A'\rangle\rangle = S_k\, ||A\rangle\rangle.
\end{equation}
It is easily seen that 
\begin{align}
\langle x', y', z'| \langle +|A'\rangle\rangle &=
\langle + |\,S_k\, |+\rangle \langle x',y',z'|\langle +||A\rangle\rangle+\langle + |\,S_k \,|-\rangle \langle x',y',z'|\langle -||A\rangle\rangle,\\[0.5ex]
\langle x', y', z'| \langle -|A'\rangle\rangle &=
\langle - |\,S_k \,|+\rangle \langle x',y',z'|\langle +||A\rangle\rangle+\langle -
|\,S_k\, |-\rangle \langle x',y',z'|\langle -||A\rangle\rangle,
\end{align}
where use has been made of the fact that the spin operator $S_k$  commutes with the
eigenbras $\langle x', y', z'|$. 
It is fairly obvious  that we can represent the operator relation (\ref{e5.164}) as a matrix relation
if we generalize our definition of a spinor by writing
\begin{equation}\label{e5.166}
||A\rangle\rangle \rightarrow \left(\! \begin{array}{c}\psi_+({\bf x}') \\
\psi_-({\bf x}')\end{array}\!\right)\equiv \chi,
\end{equation}
and so on. The components of a spinor are now wavefunctions, instead of 
complex numbers. In this scheme, the operator equation (\ref{e5.164})  becomes simply
\begin{equation}
\chi' = \left(\frac{\hbar}{2}\right) \sigma_k \,\chi.
\end{equation}

Consider the operator relation
\begin{equation}\label{e5.168}
||A'\rangle\rangle = p_k\, ||A\rangle\rangle.
\end{equation}
In the Schr\"{o}dinger representation, we have
\begin{align}
\langle x', y', z'|\langle + |A'\rangle\rangle& =
\langle x', y', z'|\,p_k \langle +||A\rangle\rangle = -{\rm i}\,\hbar\frac{\partial}
{\partial x_k'} \langle x', y', z'| \langle +||A\rangle\rangle,\\[0.5ex]
\langle x', y', z'|\langle - |A'\rangle\rangle& =
\langle x', y', z'|\,p_k \langle -||A\rangle\rangle= -{\rm i}\,\hbar\frac{\partial}
{\partial x_k'} \langle x', y', z'| \langle -||A\rangle\rangle,
\end{align}
where use has been made of Equation~(\ref{e3.67}). The above equation reduces to
\begin{equation}
\left(\! \begin{array}{c} \psi_+'({\bf x}')\\
\psi_-' ({\bf x}') \end{array}\!\right) = 
\left(\begin{array}{c} -{\rm i}\,\hbar \,\partial \psi_+({\bf x}') /\partial x_k'\\
 -{\rm i}\,\hbar \,\partial \psi_-({\bf x}')/\partial x_k'\end{array}\!\right).
\end{equation}
Thus, the operator equation (\ref{e5.168})
can be written
\begin{equation}
\chi' = p_k\, \chi,
\end{equation}
where 
\begin{equation}\label{e5.172}
p_k \rightarrow -{\rm i}\,\hbar\,\frac{\partial}{\partial x_k'}\, {\bf 1}.
\end{equation}
Here, ${\bf 1}$ is the $2\times 2$ unit matrix. In fact, any position operator
({\rm e.g.}, $p_k$ or $L_k$) is represented in the Pauli scheme as some differential
operator of the position eigenvalues multiplied by the $2\times2$ unit matrix. 

What about combinations of position and spin operators? The most
commonly occurring combination  is a dot product: {\rm e.g.}, 
${\bf S}\cdot {\bf L} = (\hbar/2)\,\bsigma \cdot {\bf L}$. 
Consider the hybrid
operator  $\bsigma \cdot {\bf a}$, where ${\bf a} \equiv (a_x, a_y, a_z)$ is
some  vector position  operator. This quantity is represented as 
a $2\times 2$ matrix:
\begin{equation}\label{e5.173}
\bsigma \cdot {\bf a} \equiv \sum_k a_k \,\sigma_k = 
\left(\!\begin{array}{cc} +a_3 & a_1 -{\rm i}\,a_2\\[0.5ex]
a_1 + {\rm i}\,a_2 & -a_3 \end{array}\!\right).
\end{equation}
Since, in the Schr\"{o}dinger representation, a general position operator takes
the form of a differential operator in $x'$, $y'$, or $z'$, it is clear that
the above quantity must be regarded as a matrix differential operator that 
acts on spinors of the general form (\ref{e5.166}).
The important identity
\begin{equation}\label{e5.174}
(\bsigma \cdot {\bf a} ) \,(\bsigma \cdot {\bf b}) = {\bf a} \cdot {\bf b}
+{\rm i}\,\bsigma\cdot ({\bf a} \times {\bf b} )
\end{equation}
follows from the commutation and anti-commutation relations (\ref{e5.160}) and (\ref{e5.161}). Thus,
\begin{align}
\sum_j \sigma_j \,a_j \sum_k \sigma_k \,b_k &= \sum_j \sum_k \left(\frac{1}{2}\,
\{\sigma_j, \sigma_k\} + \frac{1}{2} [\sigma_j, \sigma_k]\right) a_j \,b_k\nonumber\\[0.5ex]
&= \sum_j \sum_k (\delta_{j\,k} + {\rm i}\,\epsilon_{j\,k\,l} \,\sigma_l)\,a_j \,b_k\nonumber\\[0.5ex]
&= {\bf a} \cdot {\bf b}
+{\rm i}\,\bsigma\cdot ({\bf a} \times {\bf b} ).
\end{align}

A general rotation operator in spin space is written
\begin{equation}
T ({\mit\Delta}\phi) = \exp\left(-{\rm i} \,{\bf S}\cdot{\bf n}\,{\mit\Delta}\varphi/\hbar\right),
\end{equation}
by analogy with Equation~(\ref{e5.116}), where ${\bf n}$ is a unit vector pointing along
the axis of rotation, and ${\mit\Delta}\varphi$ is the angle of rotation.
Here, ${\bf n}$ can be regarded as a trivial position operator.  The
rotation operator is represented 
\begin{equation}
\exp\left(-{\rm i} \,{\bf S}\cdot{\bf n}\,{\mit\Delta}\varphi/\hbar\right)\rightarrow
\exp\left(-{\rm i} \,\bsigma\cdot{\bf n}\,{\mit\Delta}\varphi/2\right)
\end{equation}
in the Pauli scheme. 
The term on the right-hand side of the above  expression is the exponential
of a matrix. This can easily be evaluated using the Taylor series for an exponential,
plus the rules
\begin{align}
(\bsigma\cdot {\bf n})^k &= 1 \mbox{\hspace{1.88cm}for $k$ even},\\[0.5ex]
(\bsigma\cdot {\bf n})^k &=(\bsigma\cdot {\bf n})
 \mbox{\hspace{1cm}for $k$ odd}.
\end{align}
These rules follow trivially from the identity (\ref{e5.174}).  Thus, we can write
\begin{align}
\exp\left(-{\rm i} \,\bsigma\!\cdot\!{\bf n}\,{\mit\Delta}\varphi/2\right)&= 
\left[ 1 - \frac{(\bsigma \cdot {\bf n})^2}{2!} \left(\frac{{\mit\Delta}\varphi}{2}
\right)^2 + \frac{(\bsigma  \cdot  {\bf n})^4}{4!} \left(\frac{{\mit\Delta}\varphi}{2}
\right)^4 + \cdots\right]\nonumber\\[0.5ex]
&- {\rm i} \left[ (\bsigma  \cdot {\bf n} )\left( \frac{{\mit\Delta}\varphi}{2}\right)-
\frac{(\bsigma  \cdot  {\bf n})^3}{3!} \left(\frac{{\mit\Delta}\varphi}{2}\right)^3
+ \cdots\right]\nonumber\\[0.5ex]
&= \cos({\mit\Delta}\varphi/2)\,{\bf 1}- {\rm i}\,\sin ({\mit\Delta}\varphi/2)\,\bsigma\cdot{\bf n}.
\end{align}
The explicit $2\times 2$ form of this matrix is
\begin{equation}
\left(\begin{array}{rr} 
\cos({\mit\Delta}\varphi/2) - {\rm i}\,n_z \sin({\mit\Delta}\varphi/2)&
(-{\rm i}\,n_x - n_y) \sin({\mit\Delta}\varphi/2) \\[0.5ex]
(-{\rm i}\,n_x + n_y) \sin({\mit\Delta}\varphi/2) &
\cos({\mit\Delta}\varphi/2) + {\rm i}\, n_z \sin({\mit\Delta}\varphi/2)
\end{array} \right).
\end{equation}
Rotation matrices act on spinors in much the same manner as the corresponding
rotation operators act on state kets. Thus,
\begin{equation}
\chi' = \exp\left(-{\rm i} \,\bsigma\cdot{\bf n}\,{\mit\Delta}\varphi/2\right) \chi,
\end{equation}
where $\chi'$ denotes the  spinor  obtained after rotating the spinor
$\chi$ an angle ${\mit\Delta}\varphi$ about the ${\bf n}$-axis. 
The Pauli matrices remain unchanged under rotations. 
However, the quantity $\chi^\dagger \,\sigma_k \,\chi$ is proportional to the expectation
value of $S_k$ [see Equation~(\ref{e5.159})], so we would expect it to transform like a
vector under rotation (see Section~\ref{s5.9}). In fact, we
require
\begin{equation}
(\chi^\dagger \,\sigma_k \,\chi)' \equiv (\chi^\dagger)' \sigma_k\, \chi' = \sum_l R_{k\,l}\,
(\chi^\dagger \sigma_l\, \chi),
\end{equation}
where the $R_{kl}$ are the elements of a conventional rotation matrix. This
is easily demonstrated, because
\begin{equation}
\exp\left(\frac{\,{\rm i}\,\sigma_3 \,{\mit\Delta}\varphi}{2}\right) \sigma_1 \exp\left(
\frac{-{\rm i}\,\sigma_3 \,{\mit\Delta}\varphi}{2}\right) = \sigma_1 \cos{\mit\Delta}\varphi
-\sigma_2 \sin{\mit\Delta}\varphi
\end{equation}
plus all cyclic permutations. The above expression is the $2\times 2$ matrix analogue
of (see Section~\ref{s5.9})
\begin{equation}
\exp\left(\frac{\,{\rm i}\,S_z \,{\mit\Delta}\varphi}{\hbar}\right) S_x \exp\left(
\frac{-{\rm i}\,S_z \,{\mit\Delta}\varphi}{\hbar}\right) = S_x\,  \cos{\mit\Delta}\varphi
-S_y\, \sin{\mit\Delta}\varphi.
\end{equation}
The previous two formulae can both be validated using the Baker-Hausdorff lemma,
(\ref{e5.123}), which holds for Hermitian matrices, in addition to Hermitian operators. 

\section{Spin Greater Than One-Half Systems}
In the absence of spin, the Hamiltonian can be written as some function
of the position and momentum operators. Using the Schr\"{o}dinger representation,
in which ${\bf p} \rightarrow -{\rm i}\,\hbar\,\nabla$, the energy eigenvalue
problem,
\begin{equation}
H\,|E\rangle = E\,|E\rangle,
\end{equation}
can be transformed into a partial differential equation for the wavefunction
$\psi({\bf x}') \equiv \langle {\bf x'}|E\rangle$. This function specifies the
probability density for observing the particle at a given position, ${\bf x}'$. 
In general, we find
\begin{equation}
H \,\psi = E\, \psi,
\end{equation}
where $H$ is now a partial differential operator.
 The boundary conditions (for a bound state) are obtained
 from the normalization constraint
\begin{equation}
\int d^3x'\, |\psi|^{\,2}= 1.
\end{equation}

This is all very familiar. However, we now know how to generalize this scheme
to  deal with  a spin one-half particle. Instead of representing the 
state of the particle by a single wavefunction, we use {\em two}\/ wavefunctions.
The first, $\psi_+({\bf x'})$, specifies the probability density of
observing the particle at position ${\bf x}'$
with spin angular momentum $+\hbar/2$ 
in the  $z$-direction. The second, $\psi_-({\bf x'})$, specifies the
 probability density of
observing the particle at position ${\bf x}'$  with spin angular momentum $-\hbar/2$ 
in the $z$-direction. In the Pauli scheme, these wavefunctions
are combined into a {\em spinor}, $\chi$, which is simply the column vector of $\psi_+$ and $\psi_-$. 
In general, the Hamiltonian is a function of the position, momentum, and spin
operators. Adopting the Schr\"{o}dinger representation, and the Pauli scheme,
the energy eigenvalue problem reduces to 
\begin{equation}\label{e5.188}
H \,\chi = E \,\chi,
\end{equation}
where $\chi$ is a spinor ({\rm i.e.}, a $2\times 1$
 matrix of wavefunctions)
 and $H$ is a $2\times 2$ matrix partial differential operator [see Equation~(\ref{e5.173})]. 
The above spinor equation can always be  written out explicitly  as {\em two
coupled partial differential equations}\/ for $\psi_+$ and $\psi_-$. 

Suppose that the Hamiltonian has no dependence on the spin operators. In this
case, the Hamiltonian is represented as {\em diagonal}\/ $2\times 2$ matrix partial
differential operator in the Schr\"{o}dinger/Pauli scheme [see Equation~(\ref{e5.172})].
In other words, the partial differential equation for $\psi_+$ decouples
from that for $\psi_-$. In fact, both equations have the same form, so there
is only really one differential equation.  In this
situation, the most general  solution to Equation~(\ref{e5.188})  can be written
\begin{equation}\label{e5.189}
\chi = \psi({\bf x}') \left(\!\begin{array}{c} c_+\\ c_-\end{array}\!\right).
\end{equation}
Here, $\psi({\bf x}')$ is determined by the solution of the differential equation,
and the $c_\pm$ are arbitrary complex numbers. The physical significance of
the above expression is clear. The Hamiltonian determines the relative probabilities
of finding the particle at various different positions, but the direction
of its spin angular momentum remains undetermined.

Suppose that the Hamiltonian depends only on the spin operators. In this
case, the Hamiltonian is represented as a $2\times 2$ matrix of complex numbers 
in the Schr\"{o}dinger/Pauli scheme [see Equation~(\ref{e5.158})], and the spinor eigenvalue
equation (\ref{e5.188})  reduces to a straightforward matrix eigenvalue problem. 
The most general solution can again be written
\begin{equation}\label{e5.190}
\chi = \psi({\bf x}') \left(\!\begin{array}{c} c_+\\ c_-\end{array}\!\right).
\end{equation}
Here, the ratio $c_+/c_-$ is determined by the matrix eigenvalue problem,
and the wavefunction $\psi({\bf x}')$ is arbitrary. Clearly, the Hamiltonian
determines the direction of the particle's spin angular momentum, but leaves
its position undetermined. 

In general, of course, the Hamiltonian is a function of both position and
spin operators. In this case, it is not possible to decompose the 
spinor as in Equations~(\ref{e5.189})  and (\ref{e5.190}).
 In other words, a general Hamiltonian causes the
direction of the particle's spin angular momentum to vary with position in
some specified manner. This can only be represented as a spinor involving 
different wavefunctions, $\psi_+$ and $\psi_-$. 

But, what happens if we have a spin one  or a spin three-halves particle? 
It turns out that we can generalize the Pauli two-component scheme in a fairly
straightforward manner. Consider a spin-$s$ particle: {\rm i.e.}, a particle for which
the eigenvalue of $S^2$ is $s\,(s+1)\,\hbar^2$. Here, $s$ is either an integer, or a half-integer. The eigenvalues of $S_z$ are written $s_z\,\hbar$, where
$s_z$ is allowed to take the values $s, s-1, \cdots, -s+1, -s$. In fact,
there are $2\,s+1$ distinct allowed values of $s_z$. Not surprisingly, we can represent
the state of the particle by $2\,s+1$ different wavefunctions, denoted $\psi_{s_z}
({\bf x}')$. Here, $\psi_{s_z}({\bf x}')$ specifies the probability density 
for observing the particle at position ${\bf x'}$ with spin angular
momentum $s_z\,\hbar$ in the $z$-direction. More exactly,
\begin{equation}
\psi_{s_z}({\bf x}') = \langle {\bf x'}|\langle s, s_z| |A\rangle\rangle,
\end{equation}
where $||A\rangle\rangle$ denotes a state ket in the product space of the position
and spin operators. The state of the particle can be represented more
succinctly by a spinor, $\chi$, which is simply the $2\,s+1$ component column
vector  of the $\psi_{s_z}({\bf x}')$.
Thus, a spin one-half particle is represented by a two-component spinor,
a spin one particle by a three-component spinor, a spin three-halves particle
by a four-component spinor, and so on. 

In this extended Schr\"{o}dinger/Pauli
scheme, position space operators take the form of {\em diagonal}\/ $(2\,s+1) \times (2\,s+1)$
matrix differential operators. Thus, we can represent the momentum operators
as [see
Equation~(\ref{e5.172})]
\begin{equation}
p_k \rightarrow -{\rm i}\,\hbar \,\frac{\partial}{\partial x_k'}\, {\bf 1},
\end{equation}
where ${\bf 1}$ is the $(2\,s+1)\times (2\,s+1)$ unit matrix.
 We represent the spin
operators as
\begin{equation}
S_k \rightarrow s\,\hbar \,\sigma_k,
\end{equation}
where the $(2\,s+1)\times (2\,s+1)$ extended Pauli matrix $\sigma_k$ has elements
\begin{equation}\label{e5.194}
(\sigma_k)_{j\,l} = \frac{ \langle s, j|\,S_k\, | s, l\rangle}{s\,\hbar}.
\end{equation}
Here, $j, l$ are integers, or half-integers, lying in the range $-s$ to $+s$. 
But, how can we evaluate the brackets $\langle s, j|\,S_k \,| s, l\rangle$
and, thereby, construct the extended Pauli matrices? In fact, it is trivial
to construct the $\sigma_z$ matrix. By definition,
\begin{equation}
S_z\, | s, j\rangle = j\,\hbar\, | s, j\rangle.
\end{equation}
Hence, 
\begin{equation}\label{e5.196}
(\sigma_3)_{j\,l} = \frac{\langle s, j|\,S_z\, | s, l\rangle}{s\,\hbar}
= \frac{j}{s}\, \delta_{j\,l},
\end{equation}
where use has been made of the orthonormality property of the $| s, j\rangle$.
Thus, $\sigma_z$ is the suitably normalized diagonal matrix of the eigenvalues
of $S_z$. The matrix elements of $\sigma_x$ and $\sigma_y$ are most easily
obtained by considering the shift operators,
\begin{equation}\label{e5.197}
S^\pm = S_x \pm {\rm i}\, S_y.
\end{equation}
We know, from Equations~(\ref{e5.44a})--(\ref{e5.44b}), that
\begin{align}\label{e5.198a}
S^+\, |s, j\rangle &= [s\,(s+1) - j \,(j+1)]^{1/2} \,\hbar\, |s, j+1\rangle,\\[0.5ex]
S^- \,|s, j\rangle &= [s\,(s+1) - j \,(j-1)]^{1/2}\, \hbar \,|s, j-1\rangle.\label{e5.198b}
\end{align} 
It follows from Equations~(\ref{e5.194}), and (\ref{e5.197})--(\ref{e5.198b}), that
\begin{align}\label{e5.199a}
(\sigma_1)_{j\,l} &= \frac{[s\,(s+1) - j\,(j-1)]^{1/2} }{2\,s}\,\delta_{j\,\, l+1}+ \frac{[s\,(s+1) - j\,(j+1)]^{1/2} }{2\,s}\,\delta_{j\,\, l-1},\\[0.5ex]
(\sigma_2)_{j\,l} &= \frac{[ s\,(s+1) - j\,(j-1)]^{1/2} }{2\,{\rm i}\,s}\,\delta_{j\,\, l+1}- \frac{[s\,(s+1) - j\,(j+1)]^{1/2} }{2\,{\rm i}\,s}\,\delta_{j\,\, l-1}.\label{e5.199b}
\end{align}
According to Equations~(\ref{e5.196}) and (\ref{e5.199a})--(\ref{e5.199b}), the Pauli matrices for a spin one-half 
($s=1/2$)
particle are 
\begin{align}
\sigma_1 &= \left(\!\begin{array}{rr} 0 &1\\1&0\end{array}\!\right),\\[0.5ex]
\sigma_2 &= \left(\!\begin{array}{rr} 0 &-{\rm i}\\{\rm i}&0\end{array}\!\right),\\[0.5ex]
\sigma_3 &= \left(\!\begin{array}{rr} 1 &0\\0&-1\end{array}\!\right),
\end{align}
as we have seen previously. For a spin one ($s=1$) particle, we find that
\begin{align}
\sigma_1 &=\frac{1}{\sqrt{2}}\left(\!
\begin{array}{rrr} 0 &1&0\\1&0&1\\0&1&0\end{array}\!\right),\\[0.5ex]
\sigma_2 &= \frac{1}{\sqrt{2}}
\left(\!\begin{array}{rrr} 0 &-{\rm i}&0\\{\rm i}&0&{-\rm i}\\
0&{\rm i}& 0\end{array}\!\right),\\[0.5ex]
\sigma_3 &= \left(\!\begin{array}{rrr} 1 &0&0\\0&0&0\\0&0&-1\end{array}\!\right).
\end{align}
In fact, we can now construct the Pauli matrices for a spin anything particle. 
This means that we can convert the general energy eigenvalue problem for a spin-$s$ particle, where the Hamiltonian is some function of position and spin operators,
into $2\,s+1$ coupled partial differential equations involving the 
$2\,s+1$ wavefunctions
$\psi_{s_z}({\bf x'})$. Unfortunately, such a system
of equations is generally too complicated
to solve exactly.

\subsection*{Exercises}
\begin{enumerate}[label=\thechapter.\arabic*,leftmargin=*,widest=9.20]

\item Demonstrate that the operators defined in Equations~(\ref{e5.108a})--(\ref{e5.108c}) are Hermitian, and
satisfy
the commutation relations (\ref{e5.100}). 

\item Prove the Baker-Hausdorff lemma, (\ref{e5.123}). 

\item  Find the Pauli representations of the normalized eigenstates of $S_x$ and $S_y$ for
a spin-$1/2$ particle. 

\item Suppose that a spin-$1/2$ particle
has a spin vector that lies in the $x$-$z$ plane, making an
angle $\theta$ with the $z$-axis. Demonstrate that a measurement of $S_z$
yields $\hbar/2$ with probability $\cos^2(\theta/2)$, and $-\hbar/2$
with probability $\sin^2(\theta/2)$. 

\item An electron is in the spin-state 
$$
\chi = A\,\left(\begin{array}{c}1-2\,{\rm i}\\2\end{array}\right)
$$
in the Pauli representation. Determine the constant $A$ by normalizing
$\chi$. If a measurement of $S_z$ is made, what values will be
obtained, and with what probabilities? What is the expectation
value of $S_z$? Repeat the above calculations for $S_x$ and $S_y$. 

\item Consider a spin-$1/2$ system represented by the normalized spinor
$$
\chi =\left(\begin{array}{c}\cos\alpha\\\sin\alpha\,\exp(\,{\rm i}\,\beta)\end{array}\right)
$$
in the Pauli representation, where $\alpha$ and $\beta$ are real. What is the probability that a measurement of
$S_y$ yields $-\hbar/2$? 

\item An electron is at rest in an oscillating magnetic field
$$
{\bf B} = B_0\,\cos(\omega\,t)\,{\bf e}_z,
$$
where $B_0$ and $\omega$ are real positive constants. 
\begin{enumerate}
\item Find the Hamiltonian of the system.
\item If the electron starts in the spin-up state with respect to the
$x$-axis, determine the spinor $\chi(t)$ that represents the state
of the system in the Pauli representation at all subsequent times.
\item Find the probability that a measurement of $S_x$ yields
the result $-\hbar/2$ as a function of time.
\item What is the minimum value of $B_0$ required to force a
complete flip in $S_x$?
\end{enumerate}
\end{enumerate}