\chapter{Time-Independent Perturbation Theory}
% !TEX root = ../Quantum.tex

\section{Introduction}
We have developed techniques by which the general energy eigenvalue problem
can be reduced to a set of coupled partial differential equations involving
various wavefunctions. Unfortunately, the number of such problems that yield
exactly soluble equations is comparatively small. It is, therefore,  necessary to develop techniques for finding
approximate solutions to otherwise intractable problems. 

Consider the following problem, which is very common. The Hamiltonian of a
system  is  written
\begin{equation}
H = H_0 + H_1.
\end{equation}
Here, $H_0$ is a simple Hamiltonian for which we know
the {\em exact}\/ eigenvalues and eigenstates. $H_1$ introduces some
interesting additional physics into the problem, but it is sufficiently
complicated that when we add it to $H_0$ we can no longer find the exact
energy eigenvalues and eigenstates. However, $H_1$ can, in some sense
(which we shall specify more exactly later on), be regarded as 
being {\em small}\/ compared to $H_0$. Let us try to find approximate  eigenvalues
and eigenstates of the modified Hamiltonian, $H_0+H_1$, by performing a perturbation analysis about the   eigenvalues
and eigenstates of the original Hamiltonian, $H_0$. 


\section{Two-State System}
Let us start by considering {\em time-independent perturbation theory},
in which the modification to the Hamiltonian, $H_1$, has no explicit
dependence on time. It is usually assumed that the unperturbed 
Hamiltonian, $H_0$, is also time-independent. 

Consider the simplest non-trivial system, in which there are only {\em two}\/ 
independent  eigenkets of the unperturbed Hamiltonian. These are denoted
\begin{eqnarray}
H_0 \,|1\rangle &=& E_1 \,|1\rangle,\\[0.5ex]
H_0 \,|2\rangle &=& E_2 \,|2\rangle.
\end{eqnarray}
It is assumed that these states, and their associated eigenvalues, are known.
Because $H_0$ is, by definition,  an Hermitian operator,
 its two eigenkets are mutually orthogonal 
and form a complete set. The lengths of these
eigenkets are both normalized to unity. 
Let us now try to solve the modified energy eigenvalue problem
\begin{equation}\label{e6.4}
(H_0 + H_1) \,|E\rangle = E\,|E\rangle.
\end{equation}
In fact, we can solve this problem exactly. Since the eigenkets of $H_0$ form a
complete set, we can write
\begin{equation}
|E\rangle = \langle 1|E\rangle |1\rangle  + \langle 2|E\rangle |2\rangle.
\end{equation}
Right-multiplication of  Equation~(\ref{e6.4}) by $\langle 1|$ and $\langle 2|$ yields two
coupled equations, which can be written in matrix form:
\begin{equation}\label{e6.6}
\left( \begin{array}{c c}
E_1 -E + e_{11}   & e_{12} \\
e_{12}^{\,\ast} & E_2 -E + e_{22} 
\end{array} \right)\left(\!
\begin{array}{c}\langle 1|E\rangle\\
\langle 2|E \rangle\end{array}
\!\right)= \left(\!\begin{array}{c}0\\
0 \end{array}\!
\right).
\end{equation}
Here,
\begin{eqnarray}
e_{11} &=& \langle 1|\,H_1\, | 1\rangle,\\[0.5ex]
e_{22} &=& \langle 2 |\,H_1\, |2\rangle, \\[0.5ex]
e_{12} &=& \langle 1|\,H_1\,|2\rangle.
\end{eqnarray}
In the special (but common) case of a perturbing Hamiltonian whose diagonal
matrix elements (in the unperturbed eigenstates) are zero, so that
\begin{equation}
e_{11} = e_{22} = 0,
\end{equation}
the solution of Equation~(\ref{e6.6}) (obtained by setting the determinant of the matrix
equal to zero) is
\begin{equation}
E = \frac{(E_1+E_2) \pm \sqrt{(E_1-E_2)^{\,2} + 4\,|e_{12}|^{\,2}}}{2}.
\end{equation}
Let us expand in the supposedly small parameter
\begin{equation}
\epsilon = \frac{|e_{12}|}{|E_1-E_2|}.
\end{equation}
We obtain
\begin{equation}\label{e6.13}
E\simeq \frac{1}{2} \,(E_1+E_2) \pm \frac{1}{2}\,(E_1-E_2)\,(1+2\,\epsilon^2 + \cdots).
\end{equation}
The above expression  yields the modifications to the energy eigenvalues due to
the perturbing Hamiltonian:
\begin{eqnarray}
E_1' &=& E_1 + \frac{|e_{12}|^{\,2}}{E_1-E_2} + \cdots,\\[0.5ex]
E_2' &=& E_2 - \frac{|e_{12}|^{\,2}}{E_1-E_2} + \cdots.
\end{eqnarray}
Note that $H_1$ causes the upper eigenvalue to rise, and the lower
eigenvalue to fall. It is easily demonstrated that the modified eigenkets
take the form
\begin{eqnarray}
|1\rangle' &=& |1\rangle + \frac{e_{12}^{~\ast}}{E_1-E_2}\, |2\rangle + \cdots,
\\[0.5ex]
|2\rangle' &=& |2\rangle - \frac{e_{12}}{E_1-E_2}\, |1\rangle +\cdots.
\end{eqnarray}
Thus, the modified  energy eigenstates consist of one of the unperturbed eigenstates
with a slight admixture of the other. Note that the series expansion in Equation~(\ref{e6.13})
only converges if $2\,|\epsilon|<1$. This suggests that the condition for the 
validity of the perturbation expansion is
\begin{equation}
|e_{12}| < \frac{|E_1-E_2|}{2}.
\end{equation}
In other words, when we say that $H_1$ needs to be small compared to $H_0$,
what we really mean is that the above inequality needs to be satisfied.

\section{Non-Degenerate Perturbation Theory}\label{s6.3}
Let us now generalize our perturbation analysis to deal with systems
possessing  more than two energy eigenstates. The energy eigenstates of the
unperturbed Hamiltonian, $H_0$, are denoted
\begin{equation}
H_0\, |n\rangle = E_n\, |n\rangle,
\end{equation}
where $n$ runs from 1 to $N$. The eigenkets $|n\rangle$ are orthogonal,
form a complete set, and have their lengths normalized to unity. 
Let us now try to solve the energy eigenvalue
problem for the perturbed Hamiltonian:
\begin{equation}\label{e6.20}
(H_0 + H_1) \,|E\rangle = E\, |E\rangle.
\end{equation}
We can express $|E\rangle$ as a linear superposition of the unperturbed energy
eigenkets,
\begin{equation}
|E\rangle = \sum_k \langle k | E\rangle |k\rangle,
\end{equation}
where the summation is from $k=1$ to $N$. Substituting the above
equation into Equation~(\ref{e6.20}), and right-multiplying by $\langle m|$, we obtain
\begin{equation}\label{e6.22}
(E_m + e_{mm} - E)\, \langle m|E\rangle + \sum_{k\neq m} e_{mk}\, \langle k|E\rangle = 0,
\end{equation}
where
\begin{equation}
e_{mk} = \langle m |\,H_1\,| k\rangle.
\end{equation}

Let us now develop our perturbation expansion. We assume that
\begin{equation}
\frac{|e_{mk}|}{E_m - E_k} \sim O(\epsilon),
\end{equation}
for all $m\neq k$, where $\epsilon\ll 1$ is our expansion parameter. We also
assume that
\begin{equation}
\frac{|e_{mm}|}{E_m} \sim O(\epsilon),
\end{equation} 
for all $m$. Let us search for a modified version of the $n$th unperturbed energy
eigenstate, for which
\begin{equation}
E= E_n + O(\epsilon),
\end{equation}
and
\begin{eqnarray}
\langle n|E\rangle &=& 1,\\[0.5ex]
\langle m|E\rangle &\sim& O(\epsilon),
\end{eqnarray}
for $m\neq n$. Suppose that we 
write out Equation~(\ref{e6.22}) for $m\neq n$, neglecting terms that 
are $O(\epsilon^2)$ according to our expansion scheme. We find that
\begin{equation}
(E_m - E_n) \,\langle m |E \rangle + e_{mn} \simeq 0,
\end{equation} 
giving 
\begin{equation}
\langle m|E\rangle \simeq -\frac{e_{mn}}{E_m - E_n}.
\end{equation}
Substituting the above expression into Equation~(\ref{e6.22}),
evaluated  for $m=n$, and neglecting $O(\epsilon^3)$ terms, we obtain
\begin{equation}
(E_n + e_{nn} - E)  - \sum_{k\neq n} \frac{|e_{nk}|^{\,2}}
{E_k-E_n} \simeq 0.
\end{equation}
Thus, the modified $n$th energy eigenstate possesses an  eigenvalue
\begin{equation}\label{e6.32}
E_n' = E_n + e_{nn} +  \sum_{k\neq n} \frac{|e_{nk}|^{\,2}}
{E_n-E_k} + O(\epsilon^3),
\end{equation}
and a eigenket
\begin{equation}\label{e6.33}
|n\rangle' = |n\rangle   +\sum_{k\neq n}\frac{e_{kn}}{E_n - E_k}\,|k\rangle + O(\epsilon^2).
\end{equation}
Note that
\begin{equation}
\langle m|n\rangle' = \delta_{mn} + \frac{e_{nm}^{\,\ast}}{E_m-E_n} + \frac{e_{mn}}
{E_n-E_m} + O(\epsilon^2) = \delta_{mn} + O(\epsilon^2).
\end{equation}
Thus, the modified eigenkets remain  orthogonal and properly normalized
to $O(\epsilon^2)$. 

\section{Quadratic Stark Effect}\label{s6.4}
Suppose that a one-electron atom [{\rm i.e.}, either a hydrogen atom, or an alkali metal
atom (which possesses  one valance electron orbiting outside a closed, spherically
symmetric, shell)] is subjected to a uniform electric field in the positive
$z$-direction. The Hamiltonian of the system can be split into two
parts. The unperturbed Hamiltonian,
\begin{equation}
H_0 = \frac{{p}^2}{2\,m_e} + V(r),
\end{equation}
and the perturbing Hamiltonian,
\begin{equation}
H_1=  e\, |{\bf E}|\, z.
\end{equation}
Here, we are neglecting the small difference between the reduced mass, $\mu$, and the electron mass, $m_e$. 

It is assumed that the unperturbed energy eigenvalues and eigenstates are completely
known. The electron spin is irrelevant in this problem (because the spin operators
all commute with $H_1$), so we can ignore the spin degrees of freedom of the system.
This implies that the system possesses no degenerate energy eigenvalues. Actually, this is
not true for the $n\neq 1$ energy levels of the hydrogen atom, due to the special
properties of a pure Coulomb potential. 
It is necessary to deal with this case separately, because
the perturbation theory presented in Section~\ref{s6.3} breaks down  for  degenerate
unperturbed energy levels. 

An  energy eigenket of the unperturbed Hamiltonian is characterized by three quantum numbers---the radial quantum number $n$, and the two angular quantum numbers $l$ and
$m$ (see Section~\ref{s5.6}). Let us denote such a ket $|n,l,m\rangle$, and let its
energy  be $E_{nlm}$. According to Equation~(\ref{e6.32}), the change in this
energy induced by a {\em small}\/ electric field is given by
\begin{equation}
{\mit\Delta} E_{nlm}=e\,|{\bf E}|\, \langle n,l,m |\,z\,|n,l,m\rangle+ e^2 \,|{\bf E}|^{\,2}\,\sum_{n',l',m'\neq n,l,m} \frac{|\langle 
n,l,m|\,z\,| n,'l',m'\rangle|^{\,2}}{ E_{nlm}-E_{n'l'm'} }.\label{e6.38}
\end{equation}

Now, since
\begin{equation}
L_z = x\,p_y - y\, p_x,
\end{equation}
it follows that
\begin{equation}
[L_z, z] = 0.
\end{equation}
Thus,
\begin{equation}
\langle n,l, m|\, [L_z, z]\, | n',l',m'\rangle = 0,
\end{equation}
giving 
\begin{equation}
(m - m')\, \langle  n,l, m|\,z\,| n',l',m'\rangle = 0,
\end{equation}
because $|n,l,m\rangle$ is, by definition, an eigenstate of $L_z$ with eigenvalue
$m\,\hbar$. It is clear, from the above relation, that
the matrix element $\langle  n,l, m|\,z\,| n',l',m'\rangle$ is zero unless $m'=m$. 
This is termed the {\em selection rule}\/ for the quantum number $m$.

Let us now determine the selection rule for $l$. We have
\begin{align}
[L^2, z] &= [L_x^{\,2}, z] + [L_y^{\,2}, z] = L_x\,[L_x, z] + [L_x, z] \,L_x + L_y\,[L_y, z] + [L_y, z]\, L_y\nonumber\\[0.5ex]
         &= {\rm i}\,\hbar\left( -L_x\, y - y\, L_x + L_y \,x + x \,L_y\right)= 2 \,{\rm i} \, \hbar \,( L_y \,x - L_x \,y + {\rm i}\,\hbar\,z)\nonumber\\[0.5ex]
	 &= 2 \,{\rm i}\, \hbar \,( L_y\, x - y \,L_x ) =  2\, {\rm i}\, \hbar\,
 ( x\, L_y - L_x \,y),
\end{align}
where use has been made of Equations~(\ref{e5.1a})--(\ref{e5.2c}). 
Similarly, 
\begin{align}
[L^2, y] &= 2\,{\rm i}\, \hbar \,(  L_x\, z  - x\, L_z  ),\\[0.5ex]
[L^2, x] &= 2\,{\rm i}\, \hbar\, ( y \,L_z - L_y \,z).
\end{align}
Thus,
\begin{align}
[L^2, [L^2, z]] &= 2 \,{\rm i} \, \hbar \left( L^2, L_y \,x - L_x\, y + {\rm i}\,\hbar\,z
\right)= 2\,{\rm i}\,\hbar \left( L_y\, [L^2, x] - L_x\, [ L^2, y] + {\rm i}\,\hbar\,
[L^2, z]\right)\nonumber\\[0.5ex]
&= - 4\, \hbar^2 \,L_y\,(y\, L_z - L_y \,z) + 4\,\hbar^2 \,L_x\,(L_x\, z - x\, L_z)- 2\, \hbar^2\,(L^2 \,z - z\, L^2).
\end{align}
This reduces to
\begin{equation}
[L^2, [L^2, z]] = - \hbar^2 \left[4\,(L_x\, x + L_y \,y + L_z \,z)\, L_z - 4\,
(L_x^{\,2} + L_y^{\,2} + L_z^{\,2})\, z + 2 \,(L^2\, z - z\, L^2)\right].
\end{equation}
However, it is clear from Equations~(\ref{e5.1a})--(\ref{e5.1c}) that
\begin{equation}
L_x \,x + L_y \,y + L_z \,z = 0.
\end{equation}
Hence, we obtain
\begin{equation}
[L^2, [L^2, z]] = 2 \,\hbar^2\, (L^2\, z + z \,L^2),
\end{equation}
which can be expanded to give
\begin{equation}\label{e6.50}
L^4 \,z - 2\, L^2 \,z\, L^2 + z\, L^4 - 2\, \hbar^2\, (L^2 \,z +z \,L^2) = 0.
\end{equation}
Equation (\ref{e6.50}) implies that
\begin{equation}
\langle n,l,m|\, L^4\, z - 2\, L^2\, z \,L^2 + z\, L^4 - 2 \,\hbar^2 \,(L^2\, z +z\, L^2)\, |n',l',m'
\rangle = 0.
\end{equation}
This expression yields
\begin{equation}
\left[l^{\,2}\, (l+1)^2 - 2\, l\,(l+1)\,l'\,(l'+1) + l'^{\,2}\,(l'+1)^2 - 2\, l\,(l+1)
- 2\,l'\,(l'+1)\right] \langle n,l,m|\,z\,|n',l',m' \rangle = 0,
\end{equation}
which reduces to
\begin{equation}
(l+l'+2)\,(l+l')\,(l-l'+1)\,(l-l'-1)\,\langle n,l,m|\,z\,| n',l',m' \rangle = 0.
\end{equation}
According to the above formula, the matrix element 
$\langle n,l,m|\,z\,| n',l',m' \rangle$
vanishes unless $l=l'=0$ or $l' = l\pm 1$. This matrix element can be written
\begin{equation}\label{e6.54}
\langle n,l,m|\,z\,| n',l',m' \rangle = \int\!\int\!\int dV'\,
 \psi^\ast_{nlm}(r',\theta',\varphi')\,
r'\cos\theta'\, \psi_{n'm'l'}(r',\theta',\varphi'),
\end{equation}
where $\psi_{nlm}({\bf x}') = \langle {\bf x}'|n,l,m\rangle$. Recall, however,
that the wavefunction of an $l=0$ state is spherically symmetric (see Section~\ref{s5.3}):
{\rm i.e.}, $\psi_{n00}({\bf x}') = \psi_{n00}(r')$. It follows from Equation~(\ref{e6.54})
that the matrix element
vanishes by symmetry when $l=l'=0$. In conclusion, the matrix element
$\langle n,l,m|\,z\,| n',l',m' \rangle$ is zero unless $l'=l\pm 1$. This is
 the selection rule for the quantum number $l$. 

Application of the selection rules to Equation~(\ref{e6.38}) yields
\begin{equation}
{\mit\Delta} E_{nlm} = e^2 \,|{\bf E}|^{\,2} \sum_{n'}\sum_{l'=l\pm 1}
\frac{|\langle n,l,m|\,z\,|n',l',m\rangle|^{\,2}}{E_{nlm} - E_{n'l' m}}.
\end{equation}
Note that all of the terms in Equation~(\ref{e6.38}) that vary linearly with
 the electric field-strength
vanish by symmetry, according to the selection rules.
 Only those terms that  vary {\em quadratically}\/ with the
field-strength survive. The electrical polarizability, $\alpha$, of an atom is defined in terms 
of the electric-field induced energy-shift of a given atomic state as follows:
\begin{equation}
{\mit\Delta} E = - \frac{1}{2} \,\alpha \,|{\bf E}|^{\,2}.
\end{equation}
Consider the ground state of a hydrogen atom. (Recall, that we cannot address
the $n>1$ excited states because they are degenerate, and our theory cannot
handle this at present). The polarizability  of this state is given by
\begin{equation}
\alpha = 2 \,e^2  \sum_{n>1} 
\frac{|\langle 1,0,0|\,z\,|n,1,0\rangle|^{\,2}}{E_{n00}-E_{100}}.
\end{equation}
Here, we have made use of the fact that $E_{n10} = E_{n00}$ for a hydrogen atom.

The sum in the above expression can be evaluated approximately by noting that
[see Equation~(\ref{e5.95})]
\begin{equation}
E_{n00} = - \frac{e^2}{8\pi\,\epsilon_0\, a_0\,n^2} 
\end{equation}
for a hydrogen atom,
where
\begin{equation}
a_0 = \frac{4\pi \,\epsilon_0 \,\hbar^2}{m_e \,e^2}
\end{equation}
is the  Bohr radius. We can write
\begin{equation}
E_{n00}-E_{100} \geq E_{200} - E_{100} = \frac{3}{4}
 \frac{e^2}{8\pi\,\epsilon_0\, a_0}.
\end{equation}
Thus, 
\begin{equation}
\alpha < \frac{16}{3}\, 4\pi \,\epsilon_0\, a_0  \sum_{n>1} 
|\langle 1,0,0|\,z\,|n,1,0\rangle|^{\,2}.
\end{equation}
However,
\begin{equation}
\sum_{n>1} 
|\langle 1,0,0|\,z\,|n,1,0\rangle|^{\,2} = \sum_{n',l',m'}
\langle 1,0,0|\,z\,|n',l',m'\rangle\langle n',m',l'|\,z\,|1,0,0\rangle= \langle 1,0,0|\,z^2\,|1,0,0\rangle,
\end{equation}
where we have made use of the fact that the wavefunctions of a hydrogen atom
form a complete set. It is easily demonstrated from the 
actual form of the ground-state wavefunction
that
\begin{equation}
\langle 1,0,0|\,z^2\,|1,0,0\rangle = a_0^{\,2}.
\end{equation}
Thus, we conclude that
\begin{equation}
\alpha <  \frac{16}{3} \,4\pi\, \epsilon_0\, a_0^{\,3} \simeq 5.3\,4\pi\, \epsilon_0 \,a_0^{\,3}.
\end{equation}
The exact result is
\begin{equation}
\alpha = \frac{9}{2}\, 4\pi\, \epsilon_0\, a_0^{\,3} = 4.5\,4\pi \,\epsilon_0 \,a_0^{\,3}.
\end{equation}
It is  possible to obtain this result, without recourse to perturbation
theory, by solving Schr\"{o}dinger's equation in parabolic coordinates.

\section{Degenerate Perturbation Theory}\label{s6.5}
Let us now consider systems  in which the eigenstates of
the unperturbed Hamiltonian, $H_0$,  possess
{\em degenerate}\/ energy levels. It is always possible to
represent degenerate energy eigenstates
as the simultaneous eigenstates 
of the Hamiltonian and some other Hermitian operator (or group
of operators). Let us denote this operator (or group of operators) $L$.
We can write
\begin{equation}
H_0\, |n, l\rangle = E_n\, |n, l\rangle,
\end{equation}
and
\begin{equation}
L\,|n,l\rangle = L_{n\,l}\, |n, l\rangle,
\end{equation}
where $[H_0, L] = 0$. Here, the $E_n$ and the $L_{n\,l}$ are real numbers that 
depend on the quantum numbers $n$, and $n$ and $l$, respectively.
 It is always possible
to find a sufficient number of operators which commute with the Hamiltonian
in order to ensure 
that the $L_{n\,l}$ are all different. In other words, we can
choose $L$ such that the quantum numbers
$n$ and $l$ {\em uniquely}\/ specify each eigenstate. Suppose that for each value
of $n$ there are $N_n$ different values of $l$: {\rm i.e.}, the $n$th energy eigenstate
is $N_n$-fold degenerate. 

In general, $L$ {\em does not}\/ commute with the perturbing Hamiltonian, $H_1$.
This implies that the modified energy eigenstates are {\em not}\/ eigenstates
of $L$. In this situation, we expect the perturbation to split the degeneracy
of the energy levels, so that each modified eigenstate $|n,l\rangle'$ acquires
a unique energy eigenvalue $E_{nl}'$. Let us naively attempt to use the standard
perturbation theory of Section~\ref{s6.3} to evaluate the modified 
energy eigenstates
and energy levels. A direct generalization of Equations~(\ref{e6.32}) and (\ref{e6.33}) yields
\begin{equation}\label{e6.68}
E_{nl}' = E_n + e_{nlnl} + \sum_{n', l' \neq n,l}
\frac{|e_{n'l'nl}|^{\,2}}{E_n - E_{n'}} + O(\epsilon^3),
\end{equation}
and
\begin{equation}\label{e6.69}
|n, l\rangle' = |n,l\rangle + \sum_{n', l'\neq n, l}
\frac{e_{n'l'nl}}{E_n-E_{n'}}\,|n',l'\rangle + O(\epsilon^2),
\end{equation}
where 
\begin{equation}
e_{n'l'nl} = \langle n',l'|\,H_1\,|n,l\rangle.
\end{equation}
It is fairly obvious that the summations in Equations~(\ref{e6.68}) and (\ref{e6.69}) are not
well-behaved if the $n$th energy level is degenerate. The problem terms
are those involving unperturbed eigenstates labeled by the same value of $n$, but different
values of $l$: {\rm i.e.}, those states whose unperturbed energies are $E_n$.  These
terms give rise to singular factors $1/(E_n - E_n)$ in the summations. 
Note, however, that this problem would not exist if the matrix
elements, $e_{nl'nl}$, of the perturbing Hamiltonian between distinct, 
degenerate, unperturbed energy eigenstates 
corresponding to the eigenvalue $E_n$ were zero. In other words, if
\begin{equation}\label{e6.71}
\langle n, l' |\,H_1\,| n, l\rangle = \lambda_{n\,l}\, \delta_{l\,l'},
\end{equation}
then all of the singular terms in Equations~(\ref{e6.68}) and (\ref{e6.69}) would vanish. 

In general, Equation~(\ref{e6.71}) is not satisfied. Fortunately, we can always redefine
the unperturbed energy eigenstates belonging to the eigenvalue $E_n$ in such
a manner that Equation~(\ref{e6.71}) is satisfied. Let us define $N_n$ new states
that  are linear combinations of the $N_n$ original degenerate
eigenstates corresponding
to the eigenvalue $E_n$:
\begin{equation}
|n,l^{(1)}\rangle = \sum_{k=1,N_n} \langle n,k|n,l^{(1)}\rangle |n,k\rangle.
\end{equation}
Note that these new states are also degenerate energy eigenstates 
of the unperturbed Hamiltonian corresponding to the eigenvalue $E_n$. 
The $|n,l^{(1)}\rangle$ are chosen in
such a manner that they are eigenstates of the perturbing
Hamiltonian, $H_1$. Thus,
\begin{equation}\label{e6.71a}
H_1\, |n, l^{(1)}\rangle = \lambda_{n\,l} \,|n, l^{(1)}\rangle.
\end{equation}
The $|n,l^{(1)}\rangle$ are also chosen so that they are orthogonal,
and have unit lengths.
It follows that
\begin{equation}
\langle n, l'^{(1)} | \,H_1\,|n, l^{(1)}\rangle = \lambda_{n\,l}\,\delta_{l\,l'}.
\end{equation}
Thus, if we use the new eigenstates, instead of the old ones, then we can employ
Equations~(\ref{e6.68}) and (\ref{e6.69}) directly, because all of the singular terms vanish.
The only remaining difficulty is to determine the new eigenstates in terms of
the original ones.  

Now 
\begin{equation}
\sum_{l=1,N_n} |n,l\rangle \langle n,l| = 1,
\end{equation}
where 1 denotes the identity operator in the sub-space of all unperturbed
energy eigenkets corresponding to the eigenvalue $E_n$. Using this completeness
relation, the operator eigenvalue equation (\ref{e6.71a}) can be transformed into a 
straightforward matrix eigenvalue equation:
\begin{equation}
\sum_{l''=1,N_n}\langle n, l'|H_1|n, l''\rangle \langle n, l''|n, l^{(1)}\rangle
= \lambda_{n\,l}\, \langle n, l'| n, l^{(1)}\rangle. 
\end{equation}
This can be written more transparently as
\begin{equation}\label{e6.77}
{\bf U} \,{\bf x} = \lambda \,{\bf x},
\end{equation}
where the elements of the $N_n\times N_n$  Hermitian matrix ${\bf U}$ are
\begin{equation}\label{e6.78}
U_{j\,k} = \langle n, j|\, H_1\,| n, k\rangle.
\end{equation}
Provided that the determinant of ${\bf U}$ is non-zero, Equation~(\ref{e6.77})  can always be solved to
give $N_n$ eigenvalues $\lambda_{n\,l}$ (for $l=1$ to $N_n$), with
$N_n$ corresponding eigenvectors ${\bf x}_{n\,l}$. The eigenvectors specify the
weights of the new eigenstates in terms of the original eigenstates: {\rm i.e.},
\begin{equation}
({\bf x}_{n\,l})_k = \langle n, k|n, l^{(1)}\rangle,
\end{equation}
for $k=1$ to $N_n$. In our new scheme, Equations~(\ref{e6.68}) and (\ref{e6.69}) yield
\begin{equation}
E_{nl}' = E_n + \lambda_{n\,l} + \sum_{n'\neq n, l'}
\frac{|e_{n'l'nl}|^{\,2}}{E_n - E_{n'}} + O(\epsilon^3),
\end{equation}
and
\begin{equation}
|n, l^{(1)}\rangle' = |n,l^{(1)}\rangle + \sum_{n'\neq n, l'}
\frac{e_{n'l'nl}}{E_n-E_{n'}}\,|n',l'\rangle + O(\epsilon^2).
\end{equation}
There are no singular terms in these expressions, because the summations
are over $n'\neq n$: {\rm i.e.}, they specifically exclude
the problematic, 
degenerate, unperturbed energy eigenstates corresponding to the eigenvalue $E_n$. 
Note that the first-order energy-shifts are equivalent to the eigenvalues
of the matrix equation, (\ref{e6.77}).

\section{Linear Stark Effect}\label{s6.6}
Let us examine the effect of an electric field on the excited  energy
levels of a hydrogen atom. For instance, consider the $n=2$ states. 
There is a single $l=0$ state, usually referred to as $2s$, and three $l=1$
states (with $m=-1,0,1$), usually referred to as $2p$. All of these states
possess the same energy, $E_{200} = -e^2/(32\pi\,\epsilon_0 \,a_0)$. As in Section~\ref{s6.4}, the
perturbing Hamiltonian is 
\begin{equation}
H_1 = e \,|{\bf E}| \,z.
\end{equation}
In order to apply  perturbation theory, we have to solve
the matrix eigenvalue equation
\begin{equation}
{\bf U} \,{\bf x} = \lambda\,{\bf x},
\end{equation}
where ${\bf U}$ is the array of the matrix elements of $H_1$ between the
degenerate  $2s$ and
$2p$ states. Thus,
\begin{equation}
{\bf U} = e \,|{\bf E}| \left(
\begin{array}{cccc}
0&\langle 2,0,0|\,z\,|2,1,0\rangle & 0&0\\
\langle 2,1,0|\,z\,|2,0,0\rangle&0&0&0\\
0&0&0&0\\
0&0&0&0
\end{array}\right),
\end{equation}
where the rows and columns correspond to the $|2,0,0\rangle$, $|2,1,0\rangle$,
$|2,1,1\rangle$, and $|2,1,-1\rangle$ states, respectively. Here, we have made use
of the selection rules, which tell us that the matrix element of $z$ between
two hydrogen atom states  is zero unless the states 
possess the same $m$ quantum number, 
and $l$ quantum numbers that differ by unity. It is easily demonstrated,
from the exact forms of the $2s$ and $2p$ wavefunctions, that
\begin{equation}
\langle 2,0,0|\,z\,|2,1,0\rangle = \langle 2,1,0|\,z\,|2,0,0\rangle = 3\,a_0.
\end{equation}

It can be seen, by inspection, that the eigenvalues of ${\bf U}$ are
$\lambda_1= 3\,e\,a_0\,|{\bf E}|$, $\lambda_2 = -  3\,e\,a_0\,|{\bf E}|$, $\lambda_3=0$,
and $\lambda_4 =0$. The corresponding eigenvectors are
\begin{align}
{\bf x}_1 &= \left( \begin{array}{c} 1/\sqrt{2} \\ 1/\sqrt{2} \\ 0 \\ 0 \end{array}
\right),\\[0.5ex]
{\bf x}_2 &= \left( \begin{array}{c} 1/\sqrt{2} \\- 1/\sqrt{2} \\ 0 \\ 0 \end{array}
\right),\\[0.5ex]
{\bf x}_3 &= \left( \begin{array}{c} 0 \\0 \\ 1 \\0 \end{array}
\right),\\[0.5ex]
{\bf x}_4 &= \left( \begin{array}{c} 0 \\ 0\\ 0 \\ 1\end{array}
\right).
\end{align}
It follows from Section~\ref{s6.5} that the 
simultaneous eigenstates of the unperturbed Hamiltonian and the
perturbing Hamiltonian take the form
\begin{align}
|1\rangle &= \frac{|2,0,0\rangle + |2,1,0\rangle}{\sqrt{2}},\\[0.5ex]
|2\rangle &= \frac{|2,0,0\rangle - |2,1,0\rangle}{\sqrt{2}},\\[0.5ex]
|3\rangle &= |2,1,1\rangle,\\[0.5ex]
|4\rangle &= |2,1,-1\rangle.
\end{align}
In the absence of an electric field, all of these states possess the
same energy, $E_{200}$. 
 The first-order energy-shifts induced by an electric field are
given by
\begin{align}
{\mit\Delta} E_1 &= +3\,e\,a_0\, |{\bf E}|,\\[0.5ex]
{\mit\Delta} E_2 &= -3\,e\,a_0 \,|{\bf E}|,\\[0.5ex]
{\mit\Delta} E_3 &= 0,\\[0.5ex]
{\mit\Delta} E_4 &= 0.
\end{align}
Thus, the energies of states 1 and 2 are shifted upwards and downwards, respectively, 
by an amount $3\,e\,a_0\, |{\bf E}|$  in the presence of an electric field.
States 1 and 2 are orthogonal linear combinations of the original
$2s$ and $2p(m=0)$ states. 
 Note that
the energy-shifts are {\em linear}\/ in the electric field-strength, so this
is a much larger effect that the quadratic  effect described in Section~\ref{s6.4}.
The energies of states 3 and 4 (which are equivalent to the
original  $2p(m=1)$  and $2p(m=-1)$ states, respectively) 
are not affected to first order. Of course, to second order the energies of these states are shifted by an amount that depends on the
square of the electric field-strength. 

Note that the linear Stark effect depends crucially on the degeneracy of
the $2s$ and
$2p$ states. This degeneracy is a special property of
a pure Coulomb potential, and, therefore, only applies to a hydrogen atom.
Thus, alkali metal atoms do not exhibit the linear Stark effect. 
 
\section{Fine Structure} \label{s7.7x}
Let us now consider the energy levels of hydrogen-like atoms ({\rm i.e.}, alkali
metal atoms) in more detail. The outermost electron moves in a spherically
symmetric potential $V(r)$ due to the nuclear charge and the charges of the
other electrons (which occupy spherically symmetric closed shells). The
shielding effect of the inner electrons causes $V(r)$ to depart from
the pure Coulomb form. This splits the degeneracy of states characterized by the
same value of $n$, but different values of $l$. In fact, higher $l$ states 
have higher energies. 

Let us examine a phenomenon known as {\em fine structure}, which is due to
interaction between the spin and orbital angular momenta of the outermost 
electron. This electron experiences an electric field
\begin{equation}
{\bf E} = \frac{\nabla V}{e}.
\end{equation}
However, a non-relativistic charge moving in an electric field also experiences an effective
magnetic field
\begin{equation}
{\bf B} = - \frac{{\bf v} \times {\bf E}}{c^2}.
\end{equation}
Now, an electron possesses a spin magnetic moment [see Equation~(\ref{e5.138})]
\begin{equation}\label{e7.99d}
{\bmu} = - \frac{e\, {\bf S}}{m_e}.
\end{equation}
We, therefore, expect a spin-orbit contribution to the Hamiltonian of
the form
\begin{equation}
H_{LS} = - {\bmu}\cdot {\bf B} 
= - \frac{e \,{\bf S}}{m_e\,c^2} \cdot {\bf v} \times \left(\frac{1}{e} \frac{\bf x}{r}
\frac{d V}{dr}\right)
= \frac{1}{m_e^{\,2}\,c^2\,r}  \frac{d V}{dr}\, {\bf L}\cdot {\bf S},\label{e6.101}
\end{equation}
where ${\bf  L} = m_e \,{\bf  x}\times{\bf v}$ is the orbital angular momentum.
Actually, when the above expression is compared to the observed spin-orbit interaction,
it is found to be too large by a factor of two. There is a classical explanation
for this, due to spin precession, which we need not go into. The correct 
quantum mechanical explanation requires a relativistically covariant
treatment of electron dynamics (this is achieved  using the so-called {\em
Dirac equation}---see Chapter~\ref{c11}). 

Let us now apply perturbation theory to a hydrogen-like atom, using $H_{LS}$
as the perturbation (with $H_{LS}$ taking one half of the value given above), and
\begin{equation}
H_0 = \frac{{p}^2}{2\,m_e} + V(r)
\end{equation}
as the unperturbed Hamiltonian. We have two choices for the energy
eigenstates of $H_0$. We can adopt the simultaneous eigenstates of 
$H_0, L^2, S^2, L_z$ and $S_z$, or the simultaneous eigenstates of
$H_0, L^2, S^{\,2}, J^{\,2},$ and $J_z$, where ${\bf J} = {\bf L} + {\bf S}$ is
the total angular momentum. Although the departure of $V(r)$ from a pure
$1/r$ form splits the degeneracy of  same $n$, different $l$, states,
those states characterized by the same values of $n$ and $l$, but different
values of $m_l$, are still degenerate.
(Here, $m_l, m_s,$ and $m_j$ are the quantum numbers
corresponding to $L_z, S_z,$ and $J_z$, respectively.)
 Moreover, with the addition of spin
degrees of freedom, each state is doubly degenerate due to the two possible
orientations of the electron spin ({\rm i.e.}, $m_s = \pm 1/2$). Thus, we are still
dealing with a
highly degenerate system. We know, from Section~\ref{s6.6}, that the application of
perturbation theory to a degenerate system is greatly simplified if the
basis eigenstates of the unperturbed Hamiltonian are also eigenstates
of the perturbing Hamiltonian. Now, the perturbing Hamiltonian,
$H_{LS}$, is proportional to ${\bf L}\cdot {\bf S}$, where
\begin{equation}\label{e6.103}
{\bf L} \cdot{\bf S} = \frac{J^{\,2} - L^2 - S^{\,2}}{2}.
\end{equation}
It is fairly obvious
 that the first group of operators ($H_0, L^2, S^2, L_z$ and $S_z$)
{\em does not}\/ commute with $H_{LS}$,  whereas the second group
($H_0, L^2, S^{\,2}, J^{\,2},$ and $J_z$) does. In fact, ${\bf L} \cdot{\bf S}$
is just a combination of operators appearing in the second group. Thus, it is
advantageous to work in terms of the eigenstates of the second group of
operators, rather than those of the first group.

We now need to find the simultaneous eigenstates of $H_0, L^2, S^{\,2}, J^{\,2},$ and $J_z$.
This is equivalent to finding the eigenstates of the total angular momentum
resulting from the  addition of  two angular momenta: $j_1=l$, and $j_2 = s = 1/2$. 
According to Equation~(\ref{e5.217}), the allowed values of the total angular
momentum are $j=l+1/2$ and $j=l-1/2$. We can write
\begin{align}
|l+1/2, m\rangle &= \cos\alpha\, |m-1/2, 1/2\rangle + \sin\alpha\,
|m+1/2, -1/2\rangle,~~\\[0.5ex]
|l-1/2, m\rangle &= -\sin\alpha\, |m-1/2, 1/2\rangle + \cos\alpha\,
|m+1/2, -1/2\rangle.~~~~
\end{align}
Here, the kets on the left-hand side are $|j,m_j\rangle $ kets, whereas
those on the right-hand side are $|m_l, m_s\rangle$ kets
(the $j_1, j_2$ labels have been dropped, for the sake of clarity). We have made use
of the fact that the Clebsch-Gordon coefficients are automatically
zero unless $m_j=m_l+m_s$. We have also made use of the fact that
both the $|j,m_j\rangle $  and  $|m_l, m_s\rangle$ kets are orthonormal,
and have unit lengths. We now need to determine 
\begin{equation}
\cos\alpha = \langle m-1/2,1/2|l+1/2, m\rangle,
\end{equation}
where the Clebsch-Gordon coefficient is written in $\langle m_l, m_s| j, m_j\rangle$
form. 

Let us now employ the recursion relation for Clebsch-Gordon coefficients, Equation~(\ref{e5.223}),
with $j_1=l, j_2 = 1/2, j = l+1/2, m_1=m-1/2, m_2=1/2$ (lower sign). 
We obtain 
\begin{align}
&[(l+1/2)\,(l+3/2)-m\,(m+1)]^{1/2} \,\langle m-1/2, 1/2|l+1/2, m\rangle\nonumber\\[0.5ex]
&= [l\,(l+1)-(m-1/2)\,(m+1/2)]^{1/2}\, \langle m+1/2, 1/2|l+1/2, m+1\rangle,
\end{align}
which reduces to
\begin{equation}
\langle m-1/2, 1/2|l+1/2, m\rangle = \sqrt{\frac{l+m+1/2}{l+m+3/2}}\,
 \langle m+1/2, 1/2|l+1/2, m+1\rangle.
\end{equation}
We can use this formula to successively increase the value of $m_l$. For
instance,
\begin{equation}
\langle m-1/2, 1/2|l+1/2, m\rangle =\sqrt{\frac{l+m+1/2}{l+m+3/2}}
\sqrt{\frac{l+m+3/2}{l+m+5/2}}  \langle m+3/2, 1/2|l+1/2, m+2\rangle.
\end{equation}
This procedure can be continued until $m_l$ attains its maximum possible value,
$l$. Thus,
\begin{equation}\label{e6.110}
\langle m-1/2, 1/2|l+1/2, m\rangle = \sqrt{\frac{l+m+1/2}{2\,l+1}}\,
 \langle l, 1/2|l+1/2, l+1/2\rangle.
\end{equation}

Consider the situation in which $m_l$ and $m$ both take their maximum values,
$l$ and $1/2$, respectively. The corresponding value of $m_j$ is
$l+1/2$. This value is possible when  $j=l+1/2$, but not when $j=l-1/2$. 
Thus, the $|m_l, m_s\rangle$ ket $|l,1/2\rangle$ must be equal to
the $|j,m_j\rangle$ ket $|l+1/2, l+1/2\rangle$, up to an arbitrary phase-factor.
By convention, this factor is taken to be unity, giving
\begin{equation}
\langle l, 1/2|l+1/2, l+1/2\rangle = 1.
\end{equation}
It follows from Equation~(\ref{e6.110}) that
\begin{equation}
\cos\alpha=\langle m-1/2, 1/2|l+1/2, m\rangle = \sqrt{\frac{l+m+1/2}{2\,l+1}}.
\end{equation}
Hence, 
\begin{equation}
\sin^2\alpha = 1 - \frac{l+m+1/2}{2\,l+1} = \frac{l-m+1/2}{2\,l+1}.
\end{equation}

We now need to determine the sign of $\sin\alpha$. A careful examination 
of the recursion relation, Equation~(\ref{e5.223}), shows that the plus sign is
appropriate. Thus,
\begin{align}
|l+1/2, m\rangle &=\sqrt{\frac{l+m+1/2}{2\,l+1}}\,|m-1/2, 1/2\rangle+\sqrt{\frac{l-m+1/2}{2\,l+1}}\,|m+1/2, -1/2\rangle,\label{e6.114}\\[0.5ex]
|l-1/2, m\rangle &= - \sqrt{\frac{l-m+1/2}{2\,l+1}} \,|m-1/2,1/2\rangle
+  \sqrt{\frac{l+m+1/2}{2\,l+1}} \,|m+1/2, -1/2\rangle.\label{e6.115}
\end{align}
It is convenient to define so called {\em spin-angular functions} using the
Pauli two-component formalism:
\begin{align}
{\cal Y}_{l\,m}^{j=l\pm 1/2}\equiv {\cal Y}_{l\,m}^{\pm} &= \pm \sqrt{\frac{l\pm m+1/2}{2\,l+1}}\,
Y_{l\,\,m-1/2}(\theta, \varphi) \,\chi_+ +\sqrt{\frac{l\mp m+1/2}{2\,l+1}} \,Y_{l\,\,m+1/2}(\theta,\varphi)\, \chi_-
\nonumber\\[0.5ex]
&= \frac{1}{\sqrt{2\,l+1}}\left( \begin{array}{c}
\pm \sqrt{l\pm m +1/2}\,\,Y_{l\,\,m-1/2}(\theta,\varphi)\\[0.5ex]
\sqrt{l\mp m+1/2}\,\,Y_{l\,\,m+1/2}(\theta, \varphi) \end{array}
\right).
\end{align}
These functions are eigenfunctions of the total angular momentum for spin
one-half particles, just as the spherical harmonics are  eigenfunctions
of the orbital angular momentum. A general wavefunction for an energy
eigenstate in a hydrogen-like atom is written
\begin{equation}\label{e6.117}
\psi_{nlm\pm} = R_{n\,l}(r)\, {\cal Y}_{l\,m}^\pm.
\end{equation}
The radial part of the wavefunction, $R_{n\,l}(r)$, depends on the radial
quantum number $n$ and the angular quantum number $l$. The wavefunction 
is also
labeled by $m$, which is the quantum number associated with $J_z$. 
For a given choice of $l$, the quantum number $j$
({\rm i.e.}, the quantum number associated with $J^{\,2}$) can take the values
 $l\pm 1/2$.  

The $|l\pm 1/2, m\rangle$ kets are eigenstates of ${\bf L}\cdot{\bf S}$,
according to Equation~(\ref{e6.103}).
Thus,
\begin{equation}
{\bf L} \cdot{\bf S}\, |j=l\pm 1/2,m_j= m\rangle = \frac{\hbar^2}{2}
\left[ j\,(j+1) - l\,(l+1) - 3/4\right]|j,m\rangle,
\end{equation}
giving
\begin{align}
{\bf L} \cdot{\bf S}\, |l+ 1/2, m\rangle&=
\frac{l \,\hbar^2}{2}\, |l+ 1/2, m\rangle,\\[0.5ex]
{\bf L} \cdot{\bf S}\, |l- 1/2, m\rangle&=
-\frac{(l+1)\, \hbar^2}{2}\, |l- 1/2, m\rangle.
\end{align}
It follows that
\begin{align}\label{e6.121}
\oint  d{\mit\Omega}\,({\cal Y}_{l\,m}^+)^\dagger \,{\bf L} \cdot{\bf S}\,
{\cal Y}_{l \,m}^+& = \frac{l \,\hbar^2}{2},\\[0.5ex]
\oint d{\mit\Omega}\,({\cal Y}_{l \,m}^-)^\dagger \,{\bf L} \cdot{\bf S}\,
{\cal Y}_{l\,m}^-& = -\frac{(l+1) \,\hbar^2}{2},\label{e6.122}
\end{align}
where the integrals are over all solid angle, $d{\mit\Omega} = \sin\theta\,d\theta\,d\varphi$. 

Let us now apply degenerate perturbation theory to evaluate the
 shift in energy  of a  state whose wavefunction is $\psi_{nlm\pm}$ 
due to the spin-orbit Hamiltonian, $H_{LS}$. To first order, the energy-shift is given by
\begin{equation}
{\mit\Delta} E_{nlm\pm} = \int dV\,(\psi_{nlm\pm})^\dagger\, H_{LS}\,\psi_{nlm\pm},
\end{equation}
where the integral is over all space, $dV = r^2\,d{\mit\Omega}$.  Equations~(\ref{e6.101}) (remembering the
factor of two), (\ref{e6.117}), and (\ref{e6.121})--(\ref{e6.122}) yield
\begin{align}\label{e6.124}
{\mit\Delta} E_{nlm+} &= +\frac{1}{2\,m_e^{\,2}\,c^2} \left\langle \frac{1}{r}\frac{dV}{dr}
\right\rangle \frac{l\,\hbar^2}{2},\\[0.5ex]
{\mit\Delta} E_{nlm-} &=- \frac{1}{2\,m_e^{\,2}\,c^2} \left\langle \frac{1}{r}\frac{dV}{dr}
\right\rangle \frac{(l+1)\,\hbar^2}{2},\label{e6.125}
\end{align}
where
\begin{equation}\label{e6.126}
 \left\langle \frac{1}{r}\frac{dV}{dr}
\right\rangle = \int_0^\infty dr\, r^2\,(R_{n\,l})^\ast \,\frac{1}{r}\frac{dV}{dr}\, R_{n\,l}.
\end{equation}

Let us now apply the above result to the case of a sodium atom. 
In chemist's notation, the ground state is written
\begin{equation}
(1s)^2 (2s)^2(2p)^6(3s).
\end{equation}
The inner ten electrons effectively form a spherically symmetric electron
cloud. We are interested in the excitation of the eleventh electron
from $3s$ to some higher energy state. The closest (in energy) unoccupied
state is $3p$. This state has a higher energy than $3s$ due to the deviations
of the potential from the pure Coulomb form. In the absence of spin-orbit
interaction, there are six degenerate $3p$ states. The spin-orbit
interaction breaks the degeneracy of these states. The modified states are
labeled $(3p)_{1/2}$ and $(3p)_{3/2}$, where the subscript refers to the
value of $j$. The four $(3p)_{3/2}$ states lie at a slightly higher
energy level than the two $(3p)_{1/2}$ states,
because the radial integral (\ref{e6.126}) is positive.  The splitting of
the $(3p)$ energy levels of the sodium atom can be observed
using a  spectroscope.
The well-known sodium D line is associated with transitions between 
the $3p$ and $3s$ states. The fact that there are two  slightly different
$3p$ energy levels (note that spin-orbit coupling does not split
the $3s$ energy levels) means that the sodium D line actually consists
of two very closely spaced spectroscopic lines. It is easily
demonstrated that the ratio of the typical spacing of
 Balmer lines to the splitting 
brought about by spin-orbit interaction is about $1 : \alpha^2$,
where
\begin{equation}
\alpha = \frac{e^2}{2\,\epsilon_0\, h\, c} = \frac{1}{137}
\end{equation}
is the  fine structure constant. Actually, Equations~(\ref{e6.124})--(\ref{e6.125}) are not
entirely correct, because  we have neglected an effect (namely, the
relativistic mass correction of the electron) that  is the same
order of magnitude as spin-orbit coupling. (See Exercises~\ref{ex7.1} and \ref{ex7.2}.)

\section{Zeeman Effect}
Consider a hydrogen-like atom placed in a uniform $z$-directed
magnetic field. The change in energy of the outermost electron is
\begin{equation}
H_B = -\bmu \cdot  {\bf B},
\end{equation}
where 
\begin{equation}
\bmu = - \frac{e}{2\,m_e}\, ({\bf L} + 2 \,{\bf S})
\end{equation}
is its magnetic moment, including both the spin  and
orbital  contributions.
Thus,
\begin{equation}
H_B = \frac{e\, B}{2\, m_e}\, (L_z + 2\, S_z).
\end{equation}

Suppose that the energy-shifts induced by  the magnetic field are much smaller
than those  induced by spin-orbit interaction. In this situation,
we can treat $H_B$ as a small perturbation acting on the 
eigenstates of $H_0 + H_{LS}$. 
Of course, these states are the 
simultaneous eigenstates of $J^{\,2}$ and $J_z$. Let us consider one
of these states, labeled by the quantum numbers $j$ and $m$, where $j=l\pm 1/2$.   
From  standard perturbation theory, the first-order energy-shift 
in the presence of a  magnetic field is
\begin{equation}
{\mit\Delta} E_{nlm\pm} = \langle l\pm 1/2, m|\, H_B\, | l\pm 1/2, m\rangle.
\end{equation}
Because 
\begin{equation}
L_z + 2 \,S_z = J_z + S_z,
\end{equation}
we find that
\begin{equation}
{\mit\Delta} E_{nlm\pm} = \frac{e\, B}{2 \,m_e}\, \left(m\,\hbar + \langle  
l\pm 1/2, m|\, S_z\,| l\pm 1/2, m\rangle\,
\right).
\end{equation}
Now, from Equations~(\ref{e6.114})--(\ref{e6.115}),
\begin{equation}
|l\pm 1/2, m\rangle = \pm \sqrt{\frac{l\pm m +1/2}{2\,l+1}}\,
|m-1/2, 1/2\rangle+\sqrt{\frac{l\mp m+1/2}{2\,l+1}}\, |m+1/2, -1/2\rangle.
\end{equation}
It follows that
\begin{equation}
 \langle  
l\pm 1/2, m|\, S_z\,| l\pm 1/2, m\rangle = \frac{\hbar}{2\,(2\,l+1)}
\left[(l\pm m+1/2) - (l\mp m + 1/2) \right] = \pm \frac{m\,\hbar}{2\,l+1}.
\end{equation}
Thus, we obtain the so-called {\em Lande formula}\/ for the energy-shift induced by  a
weak magnetic field:
\begin{equation}\label{e6.137}
{\mit\Delta} E_{nlm\pm} = \frac{e\, \hbar\, B}{2\, m_e}\,m \left( 1 \pm \frac{1}{2\,l+1}
\right).
\end{equation}

Let us apply this theory to the sodium atom. We have already seen that
the non-Coulomb potential splits the degeneracy of the $3s$ and $3p$ states,
the latter states acquiring  a  higher energy. The spin-orbit interaction
splits the six $3p$ states into two groups, with four $j=3/2$ states
lying at a slightly higher energy than two $j=1/2$ states. According to
Equation~(\ref{e6.137}), a magnetic field splits the $(3p)_{3/2}$ quadruplet of states,
each state acquiring a different energy. In fact, the energy of  each state
becomes  dependent  on the quantum number $m$, which measures the
projection of the total angular momentum along the $z$-axis. States with
higher $m$ values have higher energies. 
A magnetic field also splits the $(3p)_{1/2}$ doublet of states. However,
it is evident from Equation~(\ref{e6.137}) that these states are split by a lesser
amount than the $j=3/2$ states. 

Suppose that we increase the strength of the magnetic
field, so that the energy-shift due to the magnetic field becomes
comparable to the energy-shift induced by spin-orbit interaction. 
Clearly, in this situation, it does not make much sense to think
of $H_B$ as a small interaction term operating on the eigenstates
of $H_0 + H_{LS}$. In fact, this intermediate case is very difficult
to analyze. Let us, instead, consider the extreme limit in which the energy-shift
due to the magnetic field greatly exceeds that induced by spin-orbit effects.
This  is called the {\em Paschen-Back limit}. 

In the Paschen-Back
limit, we can think of the spin-orbit Hamiltonian, $H_{LS}$, as
a small interaction term operating on the eigenstates of
$H_0 + H_B$. Note that the magnetic Hamiltonian, $H_B$, commutes
with $L^2, S^{\,2}, L_z, S_z$, but does not commute with $L^2, S^{\,2}, J^{\,2},
J_z$. Thus, in an intense magnetic field, the energy eigenstates of
a hydrogen-like atom are approximate eigenstates of the 
spin and orbital angular momenta,  but are not eigenstates of the
total angular momentum. We can label each state by the quantum
numbers $n$ (the energy quantum number), $l$, $m_l$, and $m_s$. 
Thus, our energy eigenkets are written $|n, l,m_l, m_s\rangle$. 
The unperturbed Hamiltonian, $H_0$, causes states with different
values of the quantum numbers $n$ and $l$ to have different energies. 
However, states with the same value of $n$ and $l$, but different
values of $m_l$ and $m_s$, are degenerate.
The shift in energy due to the magnetic field is simply
\begin{equation}
{\mit\Delta} E_{n\,l\,m_l\, m_s}= \langle n,l,m_l, m_s|\, H_B\,| n,l,m_l, m_s\rangle
= \frac{e\, \hbar \,B} {2 \,m_e} \,(m_l + 2 \,m_s).
\end{equation}
Thus, states with different values of $m_l + 2\, m_s$ acquire different
energies. 

Let us apply this result to a sodium atom. In the absence of
a magnetic field, the six $3p$ states form two groups of four and
two states, depending on the values of their total angular momentum.
In the presence of an intense magnetic field, the $3p$  states are split
into five groups. There is  a state with $m_l+2\,m_s = 2$,
a state with $m_l+2\,m_s = 1$, two states with $m_l+2\,m_s = 0$,
a state with $m_l+2\,m_s = -1$, and a state with
$m_l+2\,m_s = -2$.  These groups are equally spaced in energy,
the energy difference between adjacent groups being 
$e \,\hbar\, B/ 2\,m_e$. 

The energy-shift induced by the spin-orbit Hamiltonian is
given by
\begin{equation}
{\mit\Delta} E_{n\,l\,m_l \,m_s} = \langle  n,l,m_l, m_s|\,H_{LS}\,| n,l,m_l, m_s\rangle,
\end{equation}
where
\begin{equation}
H_{LS} = \frac{1}{2 \,m_e^{\,2}\,c^2} \frac{1}{r} \frac{dV}{dr} \,{\bf L}\cdot
 {\bf S}.
\end{equation}
Now,
\begin{equation}
\langle {\bf L}\cdot {\bf S}\rangle = \langle\, L_z \,S_z + (L^+\, S^- 
+ L^- \,S^+)/2\,
\rangle= \hbar^2\, m_l\, m_s,
\end{equation}
since
\begin{equation}
\langle L^\pm \rangle = \langle S^\pm\rangle = 0
\end{equation}
for expectation values taken between the simultaneous eigenkets of
$L_z$ and $S_z$. 
Thus,
\begin{equation}
{\mit\Delta} E_{n\,l\, m_l\, m_s} = \frac{\hbar^2\, m_l \,m_s}
{2 \,m_e^{\,2}\,c^2} \left\langle \frac{1}{r} \frac{d V}{dr}\right\rangle.
\end{equation}

Let us apply the above result to a sodium atom. In the presence of
an intense magnetic field, the $3p$ states are split into five
groups with ($m_l,m_s$) quantum numbers $(1,1/2)$, $(0,1/2)$, $(1,-1/2)$
or $(-1, 1/2)$, $(0,-1/2)$, and $(-1,-1/2)$, respectively, in order of
decreasing energy. The spin-orbit term increases the energy of
the highest energy state, does not affect the next highest energy state,
decreases, but does not split, the energy of the doublet, does
not affect the next lowest energy state, and increases 
the energy of the lowest
energy state. The net result is that the five groups of states are no
longer equally spaced in energy. 

The typical magnetic field-strength needed to access the Paschen-Bach limit
is 
\begin{equation}
B_{PB} \sim \alpha^2 \frac{e\, m_e}{\epsilon_0\, h \,a_0} 
\simeq 25\,\,\,{\rm tesla}.
\end{equation}

\section{Hyperfine Structure}
The proton in a hydrogen atom is a spin one-half charged particle, and therefore
possesses a magnetic moment. By analogy with Equation~(\ref{e7.99d}),
we can write
\begin{equation}\label{e12.151}
\bmu_p = \frac{g_p\,e}{2\,m_p}\,{\bf S}_p,
\end{equation}
where $\bmu_p$ is the proton magnetic
moment, ${\bf S}_p$ is the proton spin, and the proton gyromagnetic ratio $g_p$ is found experimentally to take that value $5.59$. Note that the
magnetic moment of a proton is much smaller (by a factor of order $m_e/m_p$)
than that of an electron.
 According
to classical electromagnetism, the proton's magnetic moment generates a
magnetic field of the form
\begin{equation}
{\bf B} = \frac{\mu_0}{4\pi\,r^3}\,\left[3\,(\bmu_p\cdot{\bf e}_r)\,{\bf e}_r - \bmu_p\right] + \frac{2\,\mu_0}{3}\,\bmu_p\,\delta({\bf x}),
\end{equation}
where ${\bf e}_r = {\bf x}/r$, and $r=|{\bf x}|$. We can understand the origin of the delta-function term
in the above expression by thinking of the proton as a tiny current loop centred on the origin.
All magnetic field-lines generated by the loop must pass through the loop.
Hence, if the size of the  loop goes to zero then the field will be infinite at the origin, and this contribution is what is reflected by the delta-function term. Now, the Hamiltonian of the electron in the magnetic
field generated by the proton is simply
\begin{equation}
H_1 = - \bmu_e\cdot {\bf B},
\end{equation}
where
\begin{equation}
\bmu_e = - \frac{e}{m_e}\,{\bf S}_e.
\end{equation}
Here, $\bmu_e$ is the electron  magnetic moment [see Equation~(\ref{e7.99d})], and ${\bf S}_e$ the electron spin. Thus, the
perturbing Hamiltonian is written
\begin{equation}
H_1=\frac{\mu_0\,g_p\,e^2}{8\pi\,m_p\,m_e}\left(\frac{3\,({\bf S}_p\cdot{\bf e}_r)\,({\bf S}_e\cdot{\bf e}_r) - {\bf S}_p\cdot{\bf S}_e}{r^3}\right) + \frac{\mu_0\,g_p\,e^2}{3\,m_p\,m_e}\,{\bf S}_p\cdot{\bf S}_e\,\delta({\bf x}).
\end{equation}
Note that, because we have neglected coupling between the proton
spin and the magnetic field generated by the electron's orbital motion,
the above expression is only valid for $l=0$ states.

According to standard first-order perturbation theory, the energy-shift induced
by spin-spin coupling between the proton and the electron is the expectation
value of the perturbing Hamiltonian. Hence,
\begin{equation}
{\mit\Delta} E = \frac{\mu_0\,g_p\,e^2}{8\pi\,m_p\,m_e}\left\langle\frac{3\,({\bf S}_p\cdot{\bf e}_r)\,({\bf S}_e\cdot{\bf e}_r) - {\bf S}_p\cdot{\bf S}_e}{r^3}\right\rangle+ \frac{\mu_0\,g_p\,e^2}{3\,m_p\,m_e}\,\langle{\bf S}_p\cdot{\bf S}_e\rangle\,|\psi(0)|^{\,2}.
\end{equation}
For the ground-state of hydrogen, which is spherically symmetric,
the first term in the above expression vanishes by symmetry. 
Moreover, it is easily demonstrated that $|\psi_{000}(0)|^{\,2}=
1/(\pi\,a_0^{\,3})$. Thus, we obtain
\begin{equation}
{\mit\Delta} E_{000} = \frac{\mu_0\,g_p\,e^2}{3\pi\,m_p\,m_e\,a_0^{\,3}}\,\langle{\bf S}_p\cdot{\bf S}_e\rangle.
\end{equation}

Let
\begin{equation}
{\bf S} = {\bf S}_e + {\bf S}_p
\end{equation}
be the total spin. We can show that
\begin{equation}
{\bf S}_p\cdot{\bf S}_e = \frac{1}{2}\,(S^2-S_e^{\,2}-S_p^{\,2}).
\end{equation}
Thus, the simultaneous eigenstates of the perturbing Hamiltonian
and the main Hamiltonian are the simultaneous eigenstates of $S_e^{\,2}$,
$S_p^{\,2}$, and $S^2$. However, both the proton and
the electron are spin one-half particles. According to Chapter~\ref{c6},
when two spin one-half particles are combined (in the absence of orbital
angular momentum) the net state  has either spin 1 or spin 0.
In fact, there are three spin 1 states, known as triplet states, and a single
spin 0 state, known as the singlet state. For all states,
the eigenvalues of $S_e^{\,2}$ and $S_p^{\,2}$ are $(3/4)\,\hbar^2$.
The eigenvalue of $S^2$ is 0 for the singlet state, and $2\,\hbar^2$
for the triplet states. Hence,
\begin{equation}
\langle {\bf S}_p\cdot{\bf S}_e\rangle = - \frac{3}{4}\,\hbar^2
\end{equation}
for the singlet state, and
\begin{equation}
\langle {\bf S}_p\cdot{\bf S}_e\rangle =  \frac{1}{4}\,\hbar^2
\end{equation}
for the triplet states. 

It follows, from the above analysis, that spin-spin coupling breaks
the degeneracy of the two $(1s)_{1/2}$ states of the hydrogen atom, lifting the
energy of the triplet configuration, and lowering that of the singlet.
This splitting is known as {\em hyperfine structure}.
The net energy difference between the singlet and the triplet states
is
\begin{equation}
{\mit\Delta} E_{000} = \frac{8}{3}\,g_p\,\frac{m_e}{m_p}\,\alpha^2\,|E_0| = 5.88\times 10^{-6}\,{\rm eV},
\end{equation}
where $|E_0|=13.6\,{\rm eV}$ is the (magnitude of the) ground-state energy.
Note that the hyperfine energy-shift is much smaller, by a factor $m_e/m_p$, than
a typical fine structure energy-shift (see Exercise~\ref{ex7.2}).
If we convert the above energy into a wavelength then we obtain
\begin{equation}
\lambda = 21.1\,{\rm cm}.
\end{equation}
This is the wavelength of the radiation emitted by a hydrogen atom
that is collisionally excited from the singlet to the triplet
state, and then decays back to the lower energy singlet state. 
The 21\,cm line is famous in  radio astronomy because it was used to
map out the spiral structure of our galaxy in the 1950's. 

\subsection*{Exercises}
\begin{enumerate}[label=\thechapter.\arabic*,leftmargin=*,widest=9.20]
\item Calculate the energy-shift in the ground state of the one-dimensional harmonic
oscillator when the perturbation
$$
V = \lambda\,x^4
$$
is added to
$$
H = \frac{p_x^{\,2}}{2\,m} + \frac{1}{2}\,m\,\omega^2\,x^2.
$$
The properly normalized ground-state wavefunction is 
$$
\psi(x) = \left(\frac{m\,\omega}{\pi\,\hbar}\right)^{1/4}\,\exp\left(-\frac{m\,\omega^2\,x^2}{2\,\hbar}\right).
$$

\item Calculate the energy-shifts due to the first-order Stark effect in the $n=3$ state of a hydrogen atom. You do not
need to perform all of the integrals, but you should construct the correct linear combinations of states. 

\item The Hamiltonian of the valence electron in a hydrogen-like atom can be written\label{ex7.1}
$$
H = \frac{p^2}{2\,m_e} + V(r) - \frac{p^4}{8\,m_e^{\,3}\,c^2}.
$$
Here, the final term on the right-hand side is the first-order correction due to the electron's relativistic mass
increase. Treating this term as a small perturbation, deduce that it causes an energy-shift in the energy eigenstate
characterized by the standard quantum numbers $n$, $l$, $m$ of
$$
{\mit\Delta}E_{nlm} = -\frac{1}{2\,m_e\,c^2}\left(E_n^{\,2} - 2\,E_n\,\langle V\rangle + \langle V^{\,2}\rangle\right),
$$
where $E_n$ is the unperturbed energy, and $\alpha$ the fine structure constant. 

\item Consider an energy eigenstate of the hydrogen atom characterized by the standard quantum numbers $n$, $l$, and $m$. 
Show that if the energy-shift due to spin-orbit coupling (see Section~\ref{s7.7x}) is added to that due to the electron's relativistic mass increase (see previous exercise) then the
net fine structure energy-shift can be written\label{ex7.2}
$$
{\mit\Delta} E_{nlm} = \frac{\alpha^2\,E_n}{n^2}\left(\frac{n}{j+1/2}-\frac{3}{4}\right).
$$
Here,  $E_n$ is the unperturbed energy,  $\alpha$ the fine structure constant, and $j=l\pm 1/2$ the quantum number associated
with the magnitude of the sum of the electron's orbital and spin angular momenta.  You will need to use the following standard results for a hydrogen atom:
\begin{align}
\left\langle \frac{a_0}{r}\right\rangle &= \frac{1}{n^2},\nonumber\\[0.5ex]
\left\langle \frac{a_0^{\,2}}{r^2}\right\rangle &= \frac{1}{(l+1/2)\,n^3},\nonumber\\[0.5ex]
\left\langle \frac{a_0^{\,3}}{r^3}\right\rangle &= \frac{1}{l\,(l+1/2)\,(l+1)\,n^3}.\nonumber
\end{align}
Here, $a_0$ is the Bohr radius. 
Assuming that the above formula for the energy shift is valid for $l=0$ (which it is), show that fine structure causes the
energy of the $(2p)_{3/2}$ states of a hydrogen atom to exceed those of the $(2p)_{1/2}$ and $(2s)_{1/2}$ states by
$4.5\times 10^{-5}\,{\rm eV}$. 
\end{enumerate}


