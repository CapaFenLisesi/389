\chapter{Identical Particles}
% !TEX root = ../Quantum.tex

\section{Permutation Symmetry}
Consider a system consisting of a collection of identical particles. 
In classical mechanics, it is, in principle, possible to continuously monitor the position of each particle as a function of time. 
Hence, the constituent particles  can be unambiguously labeled. In quantum mechanics, on the other hand, this is not possible because continuous position
measurements would disturb the system. It follows that identical particles cannot be unambiguously labeled in quantum
mechanics. 

Consider a  quantum system consisting of two identical particles. Suppose that one of the particles---particle 1, say---is characterized by the state ket $|k'\rangle$. Here, $k'$
represents the eigenvalues of the complete set of commuting observables associated with the particle. Suppose that the other particle---particle 2---is
characterized by the state ket $|k''\rangle$. The state ket for the whole system can be written in the product form
\begin{equation}
|k'\rangle\,|k''\rangle,
\end{equation}
where it is understood that the first ket corresponds to particle 1, and the second to particle 2. We can also
consider the ket
\begin{equation}
|k''\rangle\,|k'\rangle,
\end{equation}
which corresponds to a state in which particle 1 has the eigenvalues $k''$, and particle $2$ the eigenvalues $k'$. 

Suppose that we were to measure all of the simultaneously measurable properties of our two-particle system. We might obtain the results $k'$ for one particle, and $k''$ for the other.
However, we have no way of knowing whether the corresponding state ket is $|k'\rangle\,|k''\rangle$ or $|k''\rangle\,|k'\rangle$, or any
linear combination of these two kets. In other words, all state kets of the form
\begin{equation}
c_1\,|k'\rangle\,|k''\rangle + c_2\,|k''\rangle\,|k'\rangle,
\end{equation}
correspond to an identical set of results when the properties of the system are measured. This phenomenon is
known as {\em exchange degeneracy}. Such degeneracy is problematic because the specification of a complete set of  observable eigenvalues in a system of identical particles does not seem to uniquely determine the corresponding state ket. Fortunately,  nature has a way of avoiding this difficulty. 

Consider the permutation operator $P_{12}$, which is defined such that
\begin{equation}
P_{12}\,|k'\rangle\,|k''\rangle = |k''\rangle\,|k'\rangle.
\end{equation}
In other words, $P_{12}$ swaps the identities of particles $1$ and $2$. It is
easily seen that
\begin{align}
P_{21} &= P_{12},\\[0.5ex]
P_{12}^{\,2} &= 1.\label{e10.6t}
\end{align}
Now, the Hamiltonian of a system of two identical particles must necessarily be a symmetric function of each particle's observables (because
exchange of identical particles could not possibly affect the overall energy of the system). For instance, 
\begin{equation}
H = \frac{{\bf p}_1^{\,2}}{2\,m} + \frac{{\bf p}_2^{\,2}}{2\,m} + V_{\rm pair} (|{\bf x}_1-{\bf x}_2|) + V_{\rm ext}({\bf x}_1)+ V_{\rm ext}({\bf x}_2).
\end{equation}
Here, we have separated the mutual interaction of the two particles from their interaction with an external potential. It follows that if
\begin{equation}\label{e10.8t}
H\,|k'\rangle\,|k''\rangle = E\,|k'\rangle\,|k''\rangle
\end{equation}
then
\begin{equation}\label{e10.9t}
H\,|k''\rangle\,|k'\rangle = E\,|k''\rangle\,|k'\rangle,
\end{equation}
where $E$ is the total energy. 
Operating on both sides of (\ref{e10.8t}) with $P_{12}$, and making use of Equation~(\ref{e10.6t}), we obtain
\begin{equation}
P_{12}\,H\,P_{12}^{\,2}\,|k'\rangle\,|k''\rangle = E\,P_{12}\,|k'\rangle\,|k''\rangle,
\end{equation}
or
\begin{equation}
P_{12}\,H\,P_{12}\,|k''\rangle\,|k'\rangle = E\,|k''\rangle\,|k'\rangle = H\,|k''\rangle\,|k'\rangle,
\end{equation}
where use has been made of (\ref{e10.9t}). 
We deduce that
\begin{equation}
P_{12}\,H\,P_{12} = H,
\end{equation}
which implies [from (\ref{e10.6t})] that
\begin{equation}
[H,P_{12}] = 0.
\end{equation}
In other words, an eigenstate of the Hamiltonian is a simultaneous eigenstate of the permutation operator $P_{12}$. 

Now, according to Equation~(\ref{e10.6t}), the permutation operator possesses  the eigenvalues $+1$ and $-1$, respectively. The corresponding properly normalized eigenstates are
\begin{equation}
|k'\,k''\rangle_+ = \frac{1}{\sqrt{2}}\,\left(|k'\rangle\,|k''\rangle + |k''\rangle\,|k'\rangle\right),
\end{equation}
and
\begin{equation}
|k'\,k''\rangle_- = \frac{1}{\sqrt{2}}\,\left(|k'\rangle\,|k''\rangle - |k''\rangle\,|k'\rangle\right).
\end{equation}
Here, it is assumed that $\langle k'|k''\rangle = \delta_{k'\,k''}$. Note that
$|k'\,k''\rangle_+$ is  symmetric with respect to interchange of particles---i.e., 
\begin{equation}
|k''\,k'\rangle_+ =+ |k'\,k''\rangle_+,
\end{equation}
whereas $|k'\,k''\rangle_-$ is  antisymmetric---i.e.,  
\begin{equation}
|k''\,k'\rangle_- =- |k'\,k''\rangle_-.
\end{equation}

Let us now consider a  system of three identical particles. We can represent the overall state ket as
\begin{equation}
|k'\,k''\,k'''\rangle,
\end{equation}
where $k'$, $k''$, and $k'''$ are the eigenvalues of particles 1, 2, and 3, respectively. We can also
define the two-particle permutation operators
\begin{align}
P_{12}\,|k'\,k''\,k'''\rangle &= |k''\,k'\,k'''\rangle,\\[0.5ex]
P_{23}\,|k'\,k''\,k'''\rangle &= |k'\,k'''\,k''\rangle,\\[0.5ex]
P_{31}\,|k'\,k''\,k'''\rangle &= |k'''\,k''\,k'\rangle.
\end{align}
It is easily demonstrated that
\begin{align}
P_{21}&= P_{12},\\[0.5ex]
P_{32}&= P_{23},\\[0.5ex]
P_{13}&= P_{31},
\end{align}
and
\begin{align}\label{e10.25t}
P_{12}^{\,2} &= 1,\\[0.5ex]
P_{23}^{\,2} &= 1,\\[0.5ex]
P_{31}^{\,2} &= 1.\label{e10.27t}
\end{align}
As before, the Hamiltonian of the system must be a symmetric function of the particle's observables: i.e.,
\begin{align}\label{e10.28t}
H\,|k'\,k''\,k'''\rangle &= E\,|k'\,k''\,k'''\rangle,\\[0.5ex]
H\,|k''\,k'''\,k'\rangle &= E\,|k''\,k'''\,k'\rangle,\\[0.5ex]
H\,|k'''\,k'\,k''\rangle &= E\,|k'''\,k'\,k''\rangle,\\[0.5ex]
H\,|k''\,k'\,k'''\rangle &= E\,|k''\,k'\,k'''\rangle,\\[0.5ex]
H\,|k'\,k'''\,k''\rangle &= E\,|k'\,k'''\,k''\rangle,\\[0.5ex]
H\,|k'''\,k''\,k'\rangle &= E\,|k'''\,k''\,k'\rangle,\label{e10.33t}
\end{align}
where $E$ is the total energy. 
Using analogous arguments to those employed for the two-particle system, 
we deduce that 
\begin{equation}
[H,P_{12}] = [H,P_{23}] = [H,P_{31}] = 0.
\end{equation}
Hence, an eigenstate of the Hamiltonian is a simultaneous eigenstate of the permutation operators $P_{12}$, $P_{23}$, and $P_{31}$. 
However, according to Equations~(\ref{e10.25t})--(\ref{e10.27t}), the possible eigenvalues of these operators are $\pm 1$. 

Let us define the cyclic permutation operator $P_{123}$, where
\begin{equation}
P_{123} \,|k'\,k''\,k'''\rangle =|k'''\,k'\,k''\rangle.
\end{equation}
It follows that
\begin{equation}\label{e10.36t}
P_{123} = P_{12}\,P_{31}=P_{23}\,P_{12}=P_{31}\,P_{23}.
\end{equation}
It is also clear from Equations~(\ref{e10.28t}) and (\ref{e10.33t}) that
\begin{equation}
[H,P_{123}] = 0.
\end{equation}
Thus,  an eigenstate of the Hamiltonian is a simultaneous eigenstate of the permutation operators $P_{12}$, $P_{23}$, $P_{31}$,
and $P_{123}$. Let $\lambda_{12}$, $\lambda_{23}$, $\lambda_{31}$ and $\lambda_{123}$ represent  the eigenvalues of these operators,
respectively. We know that $\lambda_{12}=\pm 1$, $\lambda_{23}=\pm 1$, and $\lambda_{31}=\pm 1$.  Moreover, it follows from
(\ref{e10.36t}) that
\begin{equation}
\lambda_{123} = \lambda_{12}\,\lambda_{31} = \lambda_{23}\,\lambda_{12}=\lambda_{31}\,\lambda_{23}.
\end{equation}
The above equations imply that
\begin{equation}
\lambda_{123} = +1,
\end{equation}
and
either
\begin{equation}
\lambda_{12} = \lambda_{23} = \lambda_{31}= +1,
\end{equation}
or
\begin{equation}
\lambda_{12} = \lambda_{23} = \lambda_{31}= -1.
\end{equation}
In other words, the multi-particle state ket must be either totally symmetric, or totally antisymmetric, with respect to swapping the
identities of any given  pair
of particles. 
Thus, in terms of properly normalized single particle kets, the properly normalized totally symmetric and totally antisymmetric kets are 
\begin{align}
|k'\,k''\,k'''\rangle_+ &= \frac{1}{\sqrt{3!}}\left(|k'\rangle\,|k''\rangle\,|k'''\rangle + |k'''\rangle\,|k'\rangle\,|k''\rangle+|k''\rangle\,|k'''\rangle\,|k'\rangle \right.\nonumber\\[0.5ex]
&\left.+ |k'''\rangle\,|k''\rangle\,|k'\rangle + |k'\rangle\,|k'''\rangle\,|k''\rangle+ |k''\rangle\,|k'\rangle\,|k'''\rangle\right), 
\end{align}
and
\begin{align}\label{e10.43t}
|k'\,k''\,k'''\rangle_- &= \frac{1}{\sqrt{3!}}\left(|k'\rangle\,|k''\rangle\,|k'''\rangle + |k'''\rangle\,|k'\rangle\,|k''\rangle+|k''\rangle\,|k'''\rangle\,|k'\rangle \right.\nonumber\\[0.5ex]
&\left.-|k'''\rangle\,|k''\rangle\,|k'\rangle - |k'\rangle\,|k'''\rangle\,|k''\rangle- |k''\rangle\,|k'\rangle\,|k'''\rangle\right), 
\end{align}
respectively. 

The above arguments can be generalized to systems of more than three identical particles in a straightforward manner. 

\section{Symmetrization Postulate}
We have seen that the exchange degeneracy of a  system of identical particles is such that a specification of a complete set of
observable eigenvalues does not uniquely determine the corresponding state ket. However, we have also seen that there are only two
possible state kets: i.e., a ket that is totally symmetric with respect to particle interchange, or a ket that is totally antisymmetric. 
It turns out that systems of identical particles possessing integer-spin (e.g., spin 0, or spin 1) always choose the totally symmetric
ket, whereas systems of identical particles possessing half-integer-spin (e.g., spin 1/2) always choose the totally antisymmetric ket. 
This additional piece of information ensures that the specification of a complete set of
observable eigenvalues of a system of identical particles does, in fact,  uniquely determine the corresponding state ket. 

Systems of identical particles whose state kets are totally symmetric with respect to particle interchange are said to
obey {\em Bose-Einstein statistics}. Moreover, such particles are termed {\em bosons}. On the other hand,
systems of identical particles whose state kets are totally antisymmetric with respect to particle interchange are said to
obey {\em Fermi-Dirac statistics}, and the constituent particles are called {\em fermions}. In non-relativistic quantum mechanics, the rule that all integer-spin particles are bosons, whereas all half-integer spin
particles are fermions, must be accepted as an empirical fact. However, in relativistic quantum mechanics, it is possible
to prove that half-integer-spin particles cannot be bosons, and integer-spin particles cannot be fermions. 
Incidentally, electrons, protons, and neutrons are all fermions.

The {\em Pauli exclusion principle} is an immediate consequence of the fact that electrons obey Fermi-Dirac statistics. This principle 
states that no two electrons in a multi-electron system can possess identical sets of observable eigenvalues. For instance,
in the case of a three-electron system, the state ket is [see Equation~(\ref{e10.43t})]
\begin{align}
|k'\,k''\,k'''\rangle_- &= \frac{1}{\sqrt{3!}}\left(|k'\rangle\,|k''\rangle\,|k'''\rangle + |k'''\rangle\,|k'\rangle\,|k''\rangle+|k''\rangle\,|k'''\rangle\,|k'\rangle \right.\nonumber\\[0.5ex]
&\left.-|k'''\rangle\,|k''\rangle\,|k'\rangle - |k'\rangle\,|k'''\rangle\,|k''\rangle- |k''\rangle\,|k'\rangle\,|k'''\rangle\right).
\end{align}
Note, however, that
\begin{equation}
|k'\,k'\,k'''\rangle = |k'\,k''\,k''\rangle = |k'''\,k''\,k'''\rangle = |0\rangle.
\end{equation}
In other words, if two of the electrons in the system possess the same set of observable eigenvalues then the state ket becomes the null ket, which corresponds to
the absence of a state. 

\section{Two-Electron System}
Consider a system consisting of two electrons. Let ${\bf x}_1$ and ${\bf S}_1$ represent the position and spin operators of the first electron, respectively, 
and let ${\bf x}_2$ and ${\bf S}_2$ represent the corresponding operators for the second electron. Furthermore,
let ${\bf S}={\bf S}_1+{\bf S}_2$ represent the total spin operator for the system. Suppose that the Hamiltonian commutes with $S^{2}$, as is often the case.
It follows that the state of the system is specified by the position eigenvalues ${\bf x}_1'$ and ${\bf x}_2'$, as well as the total spin
quantum numbers $s$ and $m$. As usual, the eigenvalue of $S^{2}$ is $s\,(s+1)\,\hbar^2$, and the eigenvalue of $S_z$ is $m\,\hbar$. 
Moreover, $s, m$ can only takes the values $1,1$ or $1,0$ or $1,-1$ or $0,0$. (See Chapter~\ref{c6}.)
The overall wavefunction of the system can be written
\begin{equation}\label{e10.46t}
\psi({\bf x}_1',{\bf x}_2'; s,m) = \phi({\bf x}_1', {\bf x}_2')\,\chi(s,m),
\end{equation}
where
\begin{align}
\chi(1,1) &= \chi_+\,\chi_+,\\[0.5ex]
\chi(1,0)&= \frac{1}{\sqrt{2}}\,(\chi_+\,\chi_-+\chi_-\,\chi_+),\\[0.5ex]
\chi(1,-1)&=  \chi_-\,\chi_-,\\[0.5ex]
\chi(0,0) &= \frac{1}{\sqrt{2}}\,(\chi_+\,\chi_--\chi_-\,\chi_+).
\end{align}
Here, the spinor $\chi_+\,\chi_-$ denotes a state in which $m_1=1/2$ and $m_2=-1/2$, etc., where $m_1\,\hbar$ and $m_2\,\hbar$ are the eigenvalues
of  $S_{1\,z}$ and $S_{2\,z}$,  respectively. The three $s=1$ spinors are usually referred to as {\em triplet}\/ spinors, whereas the single
$s=0$ spinor is called the {\em singlet}\/ spinor. Note that the triplet spinors are all symmetric with respect to exchange of particles, whereas the
singlet spinor is antisymmetric. 

Fermi-Dirac statistics requires the overall wavefunction to be antisymmetric with respect to exchange of particles. Now, according to (\ref{e10.46t}), 
the overall wavefunction can be written as a product of a spatial wavefunction and a spinor. Moreover, when the system is in the spin triplet
state (i.e., $s=1$) the spinor is symmetric with respect to exchange of particles. On the other hand, when the system is in the spin singlet
state (i.e., $s=0$) the spinor is antisymmetric. It follows that, to maintain the overall antisymmetry
of the wavefunction, the triplet spatial wavefunction must be antisymmetric with respect to exchange of particles, whereas the singlet
spatial wavefunction must be symmetric. In other words, in the spin triplet state, the spatial wavefunction takes the form
\begin{equation}
\phi({\bf x}_1',{\bf x}_2') = \frac{1}{\sqrt{2}}\left[\omega_A({\bf x}_1')\,\omega_B({\bf x}_2')-\omega_B({\bf x}_1')\,\omega_A({\bf x}_2')\right],
\end{equation}
whereas in the spin singlet state the spatial wavefunction is written
\begin{equation}
\phi({\bf x}_1',{\bf x}_2') = \frac{1}{\sqrt{2}}\left[\omega_A({\bf x}_1')\,\omega_B({\bf x}_2')+\omega_B({\bf x}_1')\,\omega_A({\bf x}_2')\right].
\end{equation}
The probability of observing one electron in the volume element $d^3 x_1$ around position ${\bf x}_1$, and the
other in the volume element $d^3x_2$ around position ${\bf x}_2$, is $|\phi({\bf x}_1,{\bf x}_2)|^{\,2}\,d^3x_1\,d^3x_2$,  or
\begin{equation}
\frac{1}{2}\left\{|\omega_A({\bf x}_1)|^{\,2}\,|\omega_B({\bf x}_2)|^{\,2}+|\omega_A({\bf x}_2)|^{\,2}\,|\omega_B({\bf x}_1)|^{\,2}
\pm 2\,{\rm Re}\left[\omega_A({\bf x}_1)\,\omega_B({\bf x}_2)\,\omega_A^{\,\ast}({\bf x}_2)\,\omega_B^{\,\ast}({\bf x}_1)\right]\right\}d^3 x_1\,d^3x_2.
\end{equation}
Here, the plus sign corresponds to the spin singlet state, whereas the minus sign corresponds to the spin triplet state. We can immediately see that in the
spin triplet state the probability of finding the two electrons at the same point in space is zero. In other words, the two electrons have
a tendency to avoid one another in the triplet state. On the other hand, in the spin singlet state there is an enhanced probability of finding the
two electrons at the same point in space, because of the final term in the previous expression. In other words, the two electrons are attracted to
one another in the singlet state. 
Note, however, that the spatial probability distributions
associated with the singlet and triplet states only differ substantially when the two single particle spatial wavefunctions $\omega_A({\bf x})$ and $\omega_B({\bf x})$
overlap: i.e., when there exists a region of space in which the two wavefunctions are simultaneously nonnegligible. 

\section{Helium Atom}
Consider the helium atom, which is a good example of a two-electron system. The Hamiltonian is written
\begin{equation}\label{e10.54t}
H = \frac{{\bf p}_1^{\,2}}{2\,m_1} + \frac{{\bf p}_2^{\,2}}{Z\,m_2} - \frac{Z\,e^2}{4\pi\,\epsilon_0\,r_1} -\frac{Z\,e^2}{4\pi\,\epsilon_0\,r_2}
+ \frac{e^2}{4\pi\,\epsilon_0\,r_{12}},
\end{equation}
where $Z=2$, $r_1=|{\bf x}_1|$, $r_2=|{\bf x}_2|$, and $r_{12} = |{\bf x}_1-{\bf x}_2|$. Suppose that the final term on the right-hand side of the above expression were absent. In this case, the overall spatial wavefunction can be formed from products of  hydrogen atom wavefunctions calculated with
$Z=2$, instead of $Z=1$. Each of these wavefunctions is characterized by the usual triplet of quantum numbers, $n$, $l$, and $m$. 
Now, the total spin of the system is a constant of the motion (since ${\bf S}$ obviously commutes with the Hamiltonian), 
so the overall spin state is either the singlet or the triplet state. The corresponding spatial wavefunction is symmetric in the former case, and
antisymmetric in the latter. Suppose that one electron has the quantum numbers $n$, $l$, $m$ whereas the other has the quantum numbers $n'$ ,$l'$, $m'$. 
The corresponding spatial wavefunction is
\begin{equation}\label{e10.55t}
\phi({\bf x}_1,{\bf x}_2)=\frac{1}{\sqrt{2}}\left[\psi_{nlm}({\bf x}_1)\,\psi_{n'l'm'}({\bf x}_2)\pm \psi_{nlm}({\bf x}_2)\,\psi_{n'l'm'}({\bf x}_1)\right],
\end{equation}
where the plus and minus signs correspond to the singlet and triplet spin states, respectively. Here, $\psi_{nlm}({\bf x})$
is a standard hydrogen atom wavefunction (calculated with $Z=2$). For the special case in which the two sets of spatial quantum numbers, $n$, $l$, $m$
and $n'$,$l'$,$m'$, are the same, the triplet spin state does not exist (because the associated spatial wavefunction is null). Hence, only
singlet spin state is allowed, and the spatial wavefunction reduces to
\begin{equation}
\phi({\bf x}_1,{\bf x}_2)=\psi_{nlm}({\bf x}_1)\,\psi_{nlm}({\bf x}_2).
\end{equation}
In particular, the ground state ($n=n'=1$, $l=l'=0$, 	$m=m'=0$) can only exist as a singlet spin state (i.e., a state of overall spin 0), and has the
spatial wavefunction
\begin{equation}\label{e10.57t}
\phi({\bf x}_1,{\bf x}_2) =\psi_{100}({\bf x}_1)\,\psi_{100}({\bf x}_2)= \frac{Z^{\,3}}{\pi\,a_0^{\,3}}\,\exp\left[\frac{-Z\,(r_1+r_2)}{a_0}\right],
\end{equation}
where  $a_0$ is the Bohr radius. This follows because
\begin{equation}
\psi_{100}({\bf x})= \frac{1}{\sqrt{4\pi}}\left(\frac{Z}{a_0}\right)^{3/2}2\,\exp\left(\frac{-Z\,r}{a_0}\right).
\end{equation}
The energy of this state
is
\begin{equation}
E = 2\,Z^{\,2}\,E_0=-108.8\,{\rm eV},
\end{equation}
where $E_0=-13.6\,{\rm eV}$ is the ground state energy of a hydrogen atom. In the above expression, the factor of $2$ comes from the fact that
there are two electrons in a helium atom. 

The above estimate for the ground state energy of a helium atom completely ignores the final term on the right-hand side of Equation~(\ref{e10.54t}),
which describes the mutual interaction between the two electrons. We can obtain a better estimate for the ground state energy by treating (\ref{e10.57t})
as the unperturbed wavefunction, and $e^2/(4\pi\,\epsilon_0\,r_{12})$ as a perturbation. According to standard first-order perturbation theory, the
correction to the ground state energy is
\begin{equation}
{\mit\Delta} E = \left\langle \frac{e^2}{4\pi\,\epsilon_0\,r_{12}}\right\rangle=\int\int d^3x_1\,d^3 x_2\,\frac{Z^{\,6}}{\pi^2\,a_0^{\,6}}\,\exp\left[\frac{-2\,Z\,(r_1+r_2)}{a_0}\right]\frac{e^2}{4\pi\,\epsilon_0\,r_{12}}.
\end{equation}
This can be written
\begin{equation}
\frac{{\mit\Delta E}}{|E_0|} = \frac{2\,Z^{\,6}}{\pi^2}\int\int \frac{d^3 x_1}{a_0^{\,3}}\,\frac{d^3 x_2}{a_0^{\,3}}\,\exp\left(-2\,Z\,\frac{r_1+r_2}{a_0}\right)
\frac{a_0}{r_{12}},
\end{equation}
since $E_0=-e^2/(8\pi\,\epsilon_0\,a_0)$. Now, 
\begin{equation}
\frac{1}{r_{12}} = \frac{1}{(r_1^{\,2}+r_2^{\,2}-2\,r_1\,r_2\,\cos\gamma)^{1/2} }= \sum_{l=0,\infty} \frac{r_<^{\,l}}{r_>^{\,l+1}}\,P_l(\cos\gamma),
\end{equation}
where $r_>$ ($r_<$) is the larger (smaller) of $r_1$ and $r_2$, and $\gamma$ is the angle subtended between ${\bf x}_1$ and ${\bf x}_2$. 
Moreover, the {\em addition theorem}\/ for spherical harmonics states that
\begin{equation}
P_l(\cos\gamma) = \frac{4\pi}{2\,l+1}\sum_{m=-l,+l} Y_{l\,m}^\ast(\theta_1,\varphi_1)\,Y_{l\,m}(\theta_2,\varphi_2).
\end{equation}
However, 
\begin{equation}
\int d{\mit\Omega}  \,Y_{l\,m}(\theta,\phi) = \sqrt{4\pi}\,\delta_{l\,0}\,\delta_{m\,0},
\end{equation}
so we obtain 
\begin{align}
\frac{{\mit\Delta E}}{|E_0|}& = 32\,Z\int_0^\infty d x_1\,x_1^{\,2}\left[\int_0^{x_1} d x_2\,\frac{x_2^{\,2}}{x_1}\,{\rm e}^{-2\,(x_1+x_2)}
+\int_{x_1}^\infty dx_2\,x_2\,{\rm e}^{-2\,(x_1+x_2)}\right]\nonumber\\[0.5ex]
&= \frac{5\,Z}{4} = \frac{5}{2}.
\end{align}
 Here, $x_1=Z\,r_1/a_0$ and $x_2=Z\,r_2/a_0$, and $Z=2$.  
Thus, our improved estimate for the ground state energy of the helium atom is
\begin{equation}
E = \left(8 - \frac{5}{2}\right)E_0 = -74.8\,{\rm eV}.
\end{equation}
This is much closer to the experimental value of $-78.8\,{\rm eV}$ than our previous estimate. 

Consider an exited state of the helium atom in which one electron is in the ground state, while the other is in a state characterized by the quantum
numbers $n$, $l$, $m$. We can write the energy of this state as
\begin{equation}
E = Z^{\,2}\,E_{100} + Z^{\,2}\,E_{nlm} + {\mit\Delta E},
\end{equation}
where $E_{nlm}$ is the energy of a hydrogen atom electron whose quantum numbers are $n$, $l$, $m$. 
According to first-order perturbation theory, ${\mit\Delta E}$ is the expectation value of $e^2/(4\pi\,\epsilon_0\,r_{12})$. It follows from
(\ref{e10.55t}) (with $n,l,m=1,0,0$ and $n',l',m'=n,l,m$) that
\begin{equation}\label{e10.68t}
{\mit\Delta}E = I\pm J,
\end{equation}
where
\begin{align}
I &= \int d^3 x_1\int d^3 x_2\, |\psi_{100}({\bf x}_1)|^{\,2}\,|\psi_{nlm}({\bf x}_2)|^{\,2}\,\frac{e^2}{4\pi\,\epsilon_0\,r_{12}},\\[0.5ex]
J&= \int d^3 x_1\int d^3 x_2 \,\psi_{100}({\bf x}_1)\,\psi_{nlm}({\bf x}_2)\,\frac{e^2}{4\pi\,\epsilon_0\,r_{12}}\,\psi_{100}^{\,\ast}({\bf x}_2)\,\psi_{nlm}^{\,\ast}({\bf x}_1).
\end{align}
Here, the plus sign in (\ref{e10.68t}) corresponds to the spin singlet state, whereas the minus sign corresponds to the spin triplet state. The
integral $I$---which is known as the  {\em direct integral}---is obviously positive. The integral $J$---which is known as the {\em exchange integral}---can
be shown to also be positive. Hence, we conclude that in excited states of helium  the spin singlet state has a higher energy
than the spin triplet state. Incidentally, helium in the spin singlet state is known as {\em para-helium}, whereas helium in the triplet state is
called {\em ortho-helium}. As we have seen, for the ground state, only para-helium is possible. 

The fact that para-helium energy levels lie slightly above corresponding ortho-helium levels is interesting because our original Hamiltonian does not
depend on spin. Nevertheless, there is a spin dependent effect---i.e., a helium atom has a lower energy when its electrons possess parallel spins---as a consequence of
Fermi-Dirac statistics. To be more exact, the energy is lower in the spin triplet state because the corresponding spatial wavefunction is antisymmetric,
causing the electrons to tend to avoid one another (thereby reducing their electrostatic repulsion). 

\subsection*{Exercises}
\begin{enumerate}[label=\thechapter.\arabic*,leftmargin=*,widest=9.20]

\item Demonstrate that the particle interchange operator, $P_{12}$, in a system of two identical particles is Hermitian.

\item Consider two identical spin-$1/2$ particles of mass $m$ confined in a cubic box of dimension $L$. Find the possible
energies and wavefunctions of this system in the case of no interaction between the particles. 

\item Consider a system of two spin-$1$ particles with no orbital angular momentum (i.e., both particles are  in $s$-states). What are
the possible eigenvalues of the total spin angular momentum of the system, as well as its projection along the $z$-direction, in the
cases in which the particles are non-identical and identical?

\end{enumerate}