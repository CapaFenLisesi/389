\chapter{Scattering Theory}
% !TEX root = ../Quantum.tex

\section{Introduction}
Historically,  data regarding quantum phenomena has been
obtained from two main sources---the study of spectroscopic lines,
and scattering experiments. 
We have already developed theories that account
for some aspects of the spectra  of hydrogen-like atoms.
Let us now examine the quantum theory of
scattering. 

\section{Fundamental Equations}
Consider time-independent scattering theory, for which the Hamiltonian
of the system is written
\begin{equation}
H= H_0 + H_1,
\end{equation}
where $H_0$ is the Hamiltonian of a free particle of mass $m$,
\begin{equation}
H_0 = \frac{p^2}{2\,m},
\end{equation}
and $H_1$ represents the non-time-varying source of the scattering. Let $|\phi\rangle$
be an energy eigenket of $H_0$,
\begin{equation}\label{e7.3}
H_0\, |\phi\rangle = E\, |\phi\rangle,
\end{equation}
whose wavefunction $\langle {\bf x}'|\phi\rangle$ is
$\phi({\bf x}')$. This state is   assumed to be a plane wave state or, possibly, a
spherical wave state. 
Schr\"{o}dinger's equation for the scattering problem is
\begin{equation}\label{e7.4}
(H_0 + H_1)\, |\psi\rangle = E\,|\psi\rangle,
\end{equation}
where $|\psi\rangle$ is an energy eigenstate of the total Hamiltonian
whose wavefunction $\langle {\bf x}'|\psi\rangle$ is $\psi({\bf x}')$. 
In general, both $H_0$ and $H_0 + H_1$ have continuous energy
spectra: {\rm i.e.}, their  energy eigenstates are unbound.
We require a solution of Equation~(\ref{e7.4}) that satisfies the
boundary condition $|\psi\rangle \rightarrow |\phi\rangle$ as
$H_1\rightarrow 0$. Here, $|\phi\rangle$ is a solution of the free particle
Schr\"{o}dinger equation, (\ref{e7.3}), corresponding to the same energy eigenvalue.

Adopting the Schr\"{o}dinger representation, we can write the scattering
problem (\ref{e7.4}) in the form
\begin{equation}\label{e9.8u}
(\nabla^2 + k^2)\,\psi({\bf x}) = \frac{2\,m}{\hbar^2}\, \langle {\bf x} |\,H_1\,|
\psi\rangle,
\end{equation}
where
\begin{equation}
E = \frac{\hbar^2 k^2}{2\,m}.
\end{equation}
Equation~(\ref{e9.8u})  is called the {\em Helmholtz equation}, and can be inverted
using standard Green's function techniques. Thus,
\begin{equation}\label{e7.10}
\psi({\bf x}) = \phi({\bf x}) + \frac{2\,m}{\hbar^2} \int d^3 x'\,G({\bf x}, {\bf x}')
\,\langle {\bf x}' |\,H_1\,|\psi\rangle,
\end{equation}
where
\begin{equation}
(\nabla^2 + k^2)\,G({\bf x}, {\bf x}') = \delta({\bf x} -{\bf x}').
\end{equation}
Note that the solution (\ref{e7.10}) satisfies the boundary condition $|\psi\rangle
\rightarrow |\phi\rangle$ as $H_1\rightarrow 0$. 
As is well-known, the Green's function for the Helmholtz problem is
given by
\begin{equation}
G({\bf x}, {\bf x}') = -\frac{\exp(\pm {\rm i}\,k\,
|{\bf x} - {\bf x}'|\,)}{4\pi\,|{\bf x} - {\bf x}'|}.
\end{equation}
Thus, Equation~(\ref{e7.10}) becomes
\begin{equation}
\psi^\pm({\bf x}) = \phi({\bf x}) - \frac{2\,m}{\hbar^2} \int d^3 x'\,\frac{\exp(\pm {\rm i}\,k\,
|{\bf x} - {\bf x}'|\,)}{4\pi\,|{\bf x} - {\bf x}'|}\, \langle {\bf x}' |\,H_1\,|\psi^\pm\rangle.\label{e7.13}
\end{equation}

Let us suppose that the scattering Hamiltonian, $H_1$, is only a function
of the position operators. This implies that
\begin{equation}\label{e7.15}
\langle {\bf x}'|\,H_1\,|{\bf x}\rangle = V({\bf x})\, \delta({\bf x} -{\bf x}').
\end{equation}
 We can write
\begin{equation}
\langle {\bf x}'|\,H_1\,| \psi^\pm\rangle = \int d^3 x''\,\langle
{\bf x}'|\,H_1\,|{\bf x}''\rangle \langle {\bf x}'' |\psi^\pm\rangle\,
= V({\bf x}') \,\psi^\pm ({\bf x}').
\end{equation}
Thus, the integral equation (\ref{e7.13}) simplifies to
\begin{equation}\label{e7.17}
\psi^\pm({\bf x}) = \phi({\bf x}) - \frac{2\,m}{\hbar^2} \int d^3 x'\,\frac{\exp(\pm {\rm i}\,k\,
|{\bf x} - {\bf x}'|)}{4\pi\,|{\bf x} - {\bf x}'|}\, V({\bf x}')\,
\psi^\pm({\bf x}').
\end{equation}

Suppose that the initial state $|\phi\rangle$ is a plane wave with  wavevector  ${\bf k}$ ({\rm i.e.}, a stream of particles of
definite momentum ${\bf p} = \hbar \,{\bf k}$). The ket corresponding to
this  state is denoted $|{\bf k}\rangle$. The associated wavefunction 
takes the form 
\begin{equation}
\langle {\bf x} | {\bf k}\rangle = \frac{
\exp(\,{\rm i}\,{\bf k}\cdot{\bf x}) }{(2\pi)^{3/2}}.
\end{equation}
The wavefunction is normalized such that
\begin{equation}
\langle {\bf k}|{\bf k}'\rangle =\int d^3 x\, \langle {\bf k}|{\bf x}\rangle
\langle {\bf x} |{\bf k}'\rangle
= \int d^3 x\, \frac{ \exp[-{\rm i}\, {\bf x}\cdot({\bf k} -{\bf k}')]}
{(2\pi )^3} = \delta ({\bf k} - {\bf k'}).
\end{equation}

Suppose that the scattering potential $V({\bf x})$ is only non-zero in some
relatively localized region centered on the origin (${\bf x} = {\bf 0}$).
Let us calculate the wavefunction $\psi({\bf x})$ a long way from
the scattering region. In other words, let us adopt the ordering
$r\gg r'$. It is easily demonstrated that
\begin{equation}
|{\bf x} - {\bf x}'| \simeq r - {\bf e}_r\cdot{\bf x}'
\end{equation}
to first order in $r'/r$, where
\begin{equation}
{\bf e}_r = \frac{\bf x}{r}
\end{equation}
is a unit vector that points from the scattering region to the
observation point. Here, $r=|{\bf x}|$ and $r'=|{\bf x}'|$. Let us define
\begin{equation}
{\bf k}' = k\,{\bf e}_r.
\end{equation}
Clearly, ${\bf k}'$ is the wavevector for particles that possess the
same energy as the incoming particles ({\rm i.e.}, $k'=k$), but propagate
from the scattering region to the observation point. Note that
\begin{equation}
\exp(\pm {\rm i}\, k\,|{\bf x} - {\bf x}' |\,) \simeq
\exp(\pm {\rm i}\, k \,r) \exp(\mp {\rm i}\, {\bf k}' \cdot {\bf x}').
\end{equation}

In the large-$r$ limit, Equation~(\ref{e7.17}) reduces to
\begin{equation}
\psi^\pm({\bf x}) \simeq \frac{\exp(\,{\rm i}\,{\bf k}\cdot{\bf x})}{
(2\pi)^{3/2}} -\frac{m}{2\pi\,\hbar^2} \frac{\exp(\pm{\rm i}\,k\,r)}{r}
\int d^3 x'\, \exp(\mp {\rm i} \,{\bf k}' \cdot {\bf x}')\,
V({\bf x}')\, \psi^\pm  ({\bf x}').
\end{equation}
The first term on the right-hand side is the incident wave. The second term
represents a spherical wave centred on the scattering region. The
plus sign (on $\psi^\pm$) corresponds to a wave propagating away from the
scattering region, whereas the minus sign corresponds to a
wave propagating towards the scattering region. It is obvious that
the former represents the physical solution. 
Thus, the wavefunction a long way from the scattering region can be
written
\begin{equation}
\psi({\bf x}) = \frac{1}{(2\pi)^{3/2}} \left[\exp(\,{\rm i}\,{\bf k}\cdot{\bf x}) + \frac{\exp(\,{\rm i}\,k\,r)}{r} f({\bf k}', {\bf k}) \right],
\end{equation}
where
\begin{equation}
f({\bf k}', {\bf k}) = - \frac{(2\pi)^2 \,m}{\hbar^2} \int d^3{\bf x}' \,
\frac{\exp(-{\rm i}\,{\bf k}'\cdot {\bf x}'  ) }{(2\pi)^{3/2}}\, V({\bf x}')\,\psi({\bf x}')  = - \frac{(2\pi)^2 \,m}{\hbar^2}\, \langle {\bf k}'|\,H_1\,|\psi\rangle.
\end{equation}

Let us define the differential cross-section, $d\sigma/d{\mit\Omega}$, as
the number of particles per unit time scattered into an element of
solid angle $d{\mit\Omega}$, divided by the incident flux of particles. 
Recall, from Chapter~\ref{s4}, that the probability current 
({\rm i.e.}, the particle flux) associated with a
wavefunction $\psi$ is
\begin{equation}
{\bf j} = \frac{\hbar}{m}\, {\rm Im}(\psi^\ast\, \nabla \psi).
\end{equation}
Thus, the probability flux associated with the incident wavefunction,
\begin{equation}
\frac{ \exp(\,{\rm i} \,{\bf k}\cdot {\bf x})}{(2\pi)^{3/2}},
\end{equation}
is
\begin{equation}
{\bf j}_{\rm inc} = \frac{\hbar}{(2\pi)^{3}\,m} \,{\bf k}.
\end{equation}
Likewise, the probability flux associated with the scattered wavefunction,
\begin{equation}
\frac{ \exp(\,{\rm i} \,k\,r)}{(2\pi)^{3/2}}\frac{
 f({\bf k}', {\bf k})}{r},
\end{equation}
 is
\begin{equation}
{\bf j}_{\rm sca}=\frac{\hbar}{(2\pi)^{3}\,m}
\frac{|f( {\bf k}', {\bf k})|^{\,2}}{r^2} \, k\, {\bf e}_r.
\end{equation}
Now,
\begin{equation}
\frac{d\sigma}{d {\mit\Omega}} \,d{\mit\Omega}  = 
\frac{ r^2\,d{\mit\Omega} \, |{\bf j}_{\rm sca}|}{|{\bf j}_{\rm inc}|},
\end{equation}
giving
\begin{equation}\label{e7.33}
\frac{d\sigma}{d {\mit\Omega}} = |f({\bf k}', {\bf k})|^{\,2}.
\end{equation}
Thus, $|f({\bf k}', {\bf k})|^{\,2}$ gives the differential cross-section for particles with incident momentum $\hbar\,{ \bf k}$ to be scattered
into states whose momentum vectors are directed in a range of solid angles
$d{\mit\Omega}$ about $\hbar\,{ \bf k}'$. Note that the scattered particles possess
the same energy as the incoming particles ({\rm i.e.}, $k'=k$). This is always
the case for scattering Hamiltonians of the form specified in Equation~(\ref{e7.15}). 

\section{Born Approximation}
Equation~(\ref{e7.33}) is not particularly useful, as it stands, because the
quantity $f({\bf k}', {\bf k})$ depends on the unknown ket $|\psi\rangle$.
Recall that $\psi({\bf x})=\langle {\bf x}|\psi\rangle$
 is the solution of the integral equation
\begin{equation}
\psi({\bf x}) = \phi({\bf x})-\frac{m}{2\pi\,\hbar^2}
 \frac{\exp(\,{\rm i}\,k\,r)}{r}
\int d^3 x'\, \exp(- {\rm i} \,{\bf k}' \cdot  {\bf x}')\,
V({\bf x}')\, \psi  ({\bf x}'),
\end{equation}
where $\phi({\bf x})$ is the wavefunction of the incident state. 
According to the above equation, the total wavefunction is a superposition
of the incident wavefunction and lots of spherical waves emitted from
the scattering region. The strength of the spherical wave emitted at
a given point is proportional to the local value of the scattering
potential, $V$, as well as the local value of the wavefunction, $\psi$.

Suppose that the scattering is not particularly  strong. In this case, it is
reasonable to suppose that the total wavefunction, $\psi({\bf x})$, does
not differ substantially from the incident wavefunction, $\phi({\bf x})$.
Thus, we can obtain an expression for $f({\bf k}', {\bf k})$ by making
the substitution
\begin{equation}
\psi({\bf x}) \rightarrow \phi({\bf x}) = 
\frac{\exp(\,{\rm i}\,{\bf k} \cdot  {\bf x}  ) }{(2\pi)^{3/2}}.
\end{equation}
This  is called the {\em Born approximation}.

The Born approximation yields
\begin{equation}
f({\bf k}', {\bf k}) \simeq - \frac{m}{2\pi\, \hbar^2} \int d^3  x'\,\exp\left[\,
{\rm i}\, ({\bf k} - {\bf k}')\cdot {\bf x}'\right] 
V({\bf x}').
\end{equation}
Thus, $f({\bf k}', {\bf k})$ is proportional to the Fourier transform
of the scattering potential $V({\bf x})$ with respect to the wavevector
${\bf q} \equiv {\bf k} - {\bf k}'$.

For a spherically symmetric potential, 
\begin{equation}
f({\bf k}', {\bf k}) \simeq  - \frac{m}{2\pi\, \hbar^2} \int_0^\infty\!\int_0^\pi\!\int_0^{2\pi} dr' \,d\theta'\,d\phi'\,r'^{\,2}\,\sin\theta'
\,\exp(\,{\rm i} \, q \,r'\cos\theta') \, V(r'),
\end{equation}
giving
\begin{equation}\label{e7.38}
f({\bf k}', {\bf k}) \simeq  - \frac{2\,m}{\hbar^2\,q}
\int_0^\infty  dr'\,r' \,V(r') \sin(q \,r').
\end{equation}
Note that $f({\bf k}', {\bf k})$ is just a function of $q$ for a
spherically symmetric potential.
It is easily demonstrated that
\begin{equation}
q \equiv |{\bf k} - {\bf k}'| = 2\, k \,\sin (\theta/2),
\end{equation}
where $\theta$ is the angle subtended between the vectors
${\bf k}$ and ${\bf k}'$. In other words, $\theta$ is the angle of
scattering. Recall that the
vectors ${\bf k}$ and ${\bf k}'$ have the same length, as a consequence of  energy conservation.

Consider scattering by a Yukawa potential
\begin{equation}
V(r) = \frac{V_0\,\exp(-\mu \,r)}{\mu \,r},
\end{equation}
where $V_0$ is a constant, and $1/\mu$ measures the ``range'' of the
potential. It follows from Equation~(\ref{e7.38}) that
\begin{equation}
f(\theta) = - \frac{2\,m \,V_0}{\hbar^2\,\mu} \frac{1}{q^2 + \mu^2},
\end{equation}
because 
\begin{equation}
\int_0^\infty  dr'\, \exp(-\mu \,r') \,\sin(q\,r') = \frac{q}{q^2+\mu^2}.
\end{equation}
Thus, in the Born approximation, the differential cross-section
for scattering by a Yukawa potential is
\begin{equation}
\frac{d\sigma}{d {\mit\Omega}} \simeq \left(\frac{2\,m \,V_0}{ \hbar^2\,\mu}\right)^2
\frac{1}{[4\,k^2\,\sin^2(\theta/2) + \mu^2]^{\,2}}.
\end{equation}

The Yukawa potential reduces to the familiar Coulomb potential as
$\mu \rightarrow 0$, provided that $V_0/\mu \rightarrow
Z\,Z'\, e^2 / 4\pi\,\epsilon_0$. In this limit, the Born differential cross-section becomes
\begin{equation}
\frac{d\sigma}{d{\mit\Omega}} \simeq \left(\frac{2\,m \,Z\, Z'\, e^2}{4\pi\,\epsilon_0\,\hbar^2}\right)^2
\frac{1}{ 16 \,k^4\, \sin^4( \theta/2)}.
\end{equation}
Recall that $\hbar\, k$ is equivalent to $|{\bf p}|$, so the above
equation can be rewritten 
\begin{equation}\label{e7.46}
 \frac{d\sigma}{d{\mit\Omega}} \simeq\left(\frac{Z \,Z'\, e^2}{16\pi\,\epsilon_0\,E}\right)^2
\frac{1}{\sin^4(\theta/2)},
\end{equation}
where $E= p^2/2\,m$ is the kinetic energy of the incident particles. 
Equation~(\ref{e7.46}) is identical to the classical Rutherford scattering cross-section formula.

The Born approximation is valid provided that $\psi({\bf x})$ is
not too different from $\phi({\bf x})$ in the scattering region. 
It follows, from Equation~(\ref{e7.17}), that the condition for  $\psi({\bf x})
\simeq \phi({\bf x})$ in the vicinity of ${\bf x} = {\bf 0}$ is 
\begin{equation}\label{e7.47}
\left| \frac{m}{2\pi\, \hbar^2} \int d^3  x'\,\frac{ \exp(\,{\rm i}\, k \,r')}{r'} 
\,V({\bf x}') \right| \ll 1.
\end{equation}
 Consider the special case of the Yukawa potential. At low energies,
({\rm i.e.}, $k\ll \mu$) we can replace $\exp(\,{\rm i}\,k\, r')$ by unity,
giving
\begin{equation}
\frac{2\,m}{\hbar^2} \frac{|V_0|}{\mu^2} \ll 1
\end{equation}
as the condition for the validity of the Born approximation.
The condition for the Yukawa potential to develop a bound state
is
\begin{equation}
\frac{2\,m}{\hbar^2} \frac{|V_0|} {\mu^2} \geq 2.7,
\end{equation}
where $V_0$ is negative. Thus, if the potential is strong enough to
form a bound state then the Born approximation is likely to break
down. In the high-$k$ limit, Equation~(\ref{e7.47}) yields
\begin{equation}
\frac{2\,m}{\hbar^2} \frac{|V_0|}{\mu \,k} \ll 1.
\end{equation}
This inequality becomes progressively easier to satisfy as $k$ increases,
implying that the Born approximation is more accurate at high
incident particle energies.

\section{Partial Waves}
We can assume, without loss of generality, that the incident wavefunction
is characterized by  a wavevector ${\bf k}$ that  is aligned parallel to the $z$-axis.
The scattered wavefunction is characterized by a wavevector ${\bf k}'$
that has the same magnitude as ${\bf k}$, but, in general, points
in a different direction. The direction of ${\bf k}'$ is specified
by the polar angle $\theta$ ({\rm i.e.}, the angle subtended between the
two wavevectors), and an azimuthal angle $\varphi$ about the $z$-axis.
Equation~(\ref{e7.38}) strongly suggests that for a spherically symmetric
scattering potential [{\rm i.e.}, $V({\bf x}) = V(r)$] the scattering amplitude
is a function of $\theta$ only: i.e., 
\begin{equation}
f(\theta, \varphi) = f(\theta).
\end{equation}
It follows that neither the incident wavefunction,
\begin{equation}\label{e7.52}
\phi({\bf x}) = \frac{\exp(\,{\rm i}\,k\,z)}{(2\pi)^{3/2}}= \frac{\exp(\,{\rm i}\,k\,r\cos\theta)}{(2\pi)^{3/2}},
\end{equation}
nor the total wavefunction,
\begin{equation}\label{e7.53}
\psi({\bf x})  = \frac{1}{(2\pi)^{3/2}}
\left[ \exp(\,{\rm i}\,k\,r\cos\theta) + \frac{\exp(\,{\rm i}\,k\,r)\, f(\theta)}
{r} \right],
\end{equation}
depend on the azimuthal angle $\varphi$. 

Outside the range of the scattering potential, both $\phi({\bf x})$ and
$\psi({\bf x})$ satisfy the free space Schr\"{o}dinger equation 
\begin{equation}\label{e7.54}
(\nabla^2 + k^2)\,\psi = 0.
\end{equation}
Consider the most general solution to this equation in spherical polar
coordinates that does not depend on the azimuthal angle $\varphi$.
Separation of variables yields
\begin{equation}\label{e7.55}
\psi(r,\theta) = \sum_{l=0,\infty} R_l(r)\, P_l(\cos\theta),
\end{equation}
since the Legendre polynomials $P_l(\cos\theta)$ form a complete
set in $\theta$-space. The Legendre polynomials are related to the
spherical harmonics  introduced in Chapter~\ref{ch4} via
\begin{equation}
P_l(\cos\theta) = \sqrt{\frac{4\pi}{2\,l+1}}\, Y_{l\,0}(\theta,\varphi).
\end{equation}
Equations~(\ref{e7.54}) and (\ref{e7.55}) can be combined to give
\begin{equation}
r^2\frac{d^2 R_l}{dr^2} + 2\,r \frac{dR_l}{dr} +  [k^2 \,r^2 -
l\,(l+1)]\,R_l = 0.
\end{equation}
The two independent solutions to this equation are the 
{\em spherical Bessel function}, $j_l(k\,r)$, and the  {\em Neumann function},
$\eta_l(k\,r)$, 
where 
\begin{align}\label{e7.58a}
j_l(y) &= y^l\left(-\frac{1}{y}\frac{d}{dy}\right)^l \frac{\sin y}{y},
\\[0.5ex]\label{e7.58b}
\eta_l(y) &= -y^l\left(-\frac{1}{y}\frac{d}{dy}\right)^l \frac{\cos y}{y}.
\end{align}
Note that  spherical Bessel functions are well-behaved in the limit
$y\rightarrow 0$ , whereas  Neumann functions become singular.
The asymptotic behaviour of these functions in the limit $y\rightarrow
\infty$ is
\begin{align}\label{e7.59a}
j_l(y) &\rightarrow \frac{\sin(y - l\,\pi/2)}{y},\\[0.5ex]
\eta_l(y) &\rightarrow  - \frac{\cos(y-l\,\pi/2)}{y}.\label{e7.59b}
\end{align}

We can write
\begin{equation}
\exp(\,{\rm i}\,k\,r \cos\theta) = \sum_{l=0,\infty} a_l\, j_l(k\,r)\, P_l(\cos\theta),
\end{equation}
where the $a_l$ are constants. Note there are no  Neumann functions in
this expansion, because they are not well-behaved  as $r \rightarrow 0$. 
The Legendre polynomials  are orthogonal,
\begin{equation}\label{e7.61}
\int_{-1}^1 d\mu\,P_n(\mu) \,P_m(\mu) = \frac{\delta_{n\,m}}{n+1/2},
\end{equation}
so we can invert the above expansion to give
\begin{equation}
a_l \,j_l(k\,r) = (l+1/2)\int_{-1}^1 d\mu\,\exp(\,{\rm i}\,k\,r \,\mu) \,P_l(\mu).
\end{equation}
It is well-known that
\begin{equation}
j_l(y) = \frac{(-{\rm i})^l}{2} \int_{-1}^1 d\mu\, \exp(\,{\rm i}\, y\,\mu)
\,P_l(\mu),
\end{equation}
where $l=0, \infty$.  Thus,
\begin{equation}
a_l = {\rm i}^l \,(2\,l+1),
\end{equation}
giving
\begin{equation}
\exp(\,{\rm i}\,k\,r \cos\theta) = \sum_{l=0,\infty} {\rm i}^l\,
(2\,l+1)\, j_l(k\,r)\, P_l(\cos\theta).
\end{equation}
The above expression  tells us how to decompose
a plane wave  into
a series of spherical waves (or ``partial waves'').

The most general solution for the total wavefunction outside the
scattering region is
\begin{equation}
\psi({\bf x}) = \frac{1}{(2\pi)^{3/2}} \sum_{l=0,\infty}\left[
A_l\,j_l(k\,r) + B_l\,\eta_l(k\,r)\right] P_l(\cos\theta),
\end{equation}
where the $A_l$ and $B_l$ are constants. 
Note that the Neumann functions are allowed to appear 
in this expansion, because
its region of validity does not include the origin. In the large-$r$
limit, the total wavefunction reduces to
\begin{equation}
\psi ({\bf x} ) \simeq \frac{1}{(2\pi)^{3/2}} \sum_{l=0,\infty}\left[A_l\,
\frac{\sin(k\,r - l\,\pi/2)}{k\,r} - B_l\,\frac{\cos(k\,r -l\,\pi/2)}{k\,r}
\right] P_l(\cos\theta),
\end{equation}
where use has been made of Equations~(\ref{e7.59a})--(\ref{e7.59b}). The above expression can also
be written
\begin{equation}\label{e7.68}
\psi ({\bf x} ) \simeq \frac{1}{(2\pi)^{3/2}} \sum_l C_l\,
\frac{\sin(k\,r - l\,\pi/2+ \delta_l)}{k\,r}\, P_l(\cos\theta),
\end{equation}
where the sine and cosine functions have been combined to give a
sine function that is phase-shifted by $\delta_l$. 

Equation~(\ref{e7.68}) yields
\begin{equation}
\psi({\bf x}) \simeq \frac{1}{(2\pi)^{3/2}} \sum_l C_l\,
\frac{\exp[\,{\rm i}\,(k\,r - l\,\pi/2+ \delta_l)]
-\exp[-{\rm i}\,(k\,r - l\,\pi/2+ \delta_l)] }{2\,{\rm i}\,k\,r}\,  P_l(\cos\theta),\label{e7.69}
\end{equation}
which contains both incoming and outgoing spherical waves. What is the
source of the incoming waves? Obviously, they must be part of
the large-$r$ asymptotic expansion of the incident wavefunction. In fact,
it is easily seen that
\begin{equation}
\phi({\bf x}) \simeq \frac{1}{(2\pi)^{3/2}} \sum_{l=0,\infty} {\rm i}^l\,
(2l+1)\, \frac{
\exp[\,{\rm i}\,(k\,r - l\,\pi/2)]
-\exp[-{\rm i}\,(k\,r - l\,\pi/2)]}{2\,{\rm i}\,k\,r} \, P_l(\cos\theta),\label{e7.70}
\end{equation}
in the large-$r$ limit. Now, Equations~(\ref{e7.52}) and (\ref{e7.53}) give
\begin{equation}\label{e7.71}
(2\pi)^{3/2}[\psi({\bf x} )- \phi({\bf x})  ] = 
\frac{\exp(\,{\rm i}\,k\,r)}{r}\,
f(\theta).
\end{equation}
Note that the right-hand side consists only of an outgoing spherical
wave. This implies that the coefficients of the incoming spherical waves
in the large-$r$  expansions of $\psi({\bf x})$ and $\phi({\bf x})$
must be equal. It follows from Equations~(\ref{e7.69}) and (\ref{e7.70}) that
\begin{equation}
C_l = (2\,l+1)\,\exp[\,{\rm i}\,(\delta_l + l\,\pi/2)].
\end{equation} 
Thus, Equations~(\ref{e7.69})--(\ref{e7.71}) yield
\begin{equation}\label{e7.73}
f(\theta) = \sum_{l=0,\infty} (2\,l+1)\,\frac{\exp(\,{\rm i}\,\delta_l)}
{k} \,\sin\delta_l\,P_l(\cos\theta).
\end{equation}
Clearly, determining the scattering amplitude
$f(\theta)$  via  a decomposition into
partial waves ({\rm i.e.}, spherical waves) is equivalent to determining
the phase-shifts $\delta_l$.

\section{Optical Theorem}
The differential scattering cross-section $d\sigma/d{\mit\Omega}$ is simply
the modulus squared of the scattering amplitude $f(\theta)$. The
total cross-section is given by
\begin{align}
\sigma_{\rm total}& = \oint d{\mit\Omega}\,|f(\theta)|^{\,2}\nonumber\\[0.5ex]
&= \frac{1}{k^2} \oint d\varphi \int_{-1}^{1} d\mu
\sum_l \sum_{l'} (2\,l+1)\,(2\,l'+1) 
\exp[\,{\rm i}\,(\delta_l-\delta_{l'})]\,  \sin\delta_l \,\sin\delta_{l'}\,
P_l(\mu)\, P_{l'}(\mu),
\end{align}
where $\mu = \cos\theta$. It follows that
\begin{equation}\label{e7.75}
\sigma_{\rm total} = \frac{4\pi}{k^2} \sum_{l=0,\infty} (2\,l+1)\,\sin^2\delta_l,
\end{equation}
where use has been made of Equation~(\ref{e7.61}). A comparison of this result with
Equation~(\ref{e7.73}) yields 
\begin{equation}
\sigma_{\rm total} = \frac{4\pi}{k}\, {\rm Im}\left[f(0)\right],
\end{equation}
since $P_l(1) = 1$. This result is known as the {\em optical theorem}.
It is a reflection of the fact that the very existence of scattering
requires scattering in the forward ($\theta=0$) direction
in order to interfere with the incident wave, and thereby reduce the
probability current in this direction.

It is usual to write
\begin{equation}
\sigma_{\rm total} = \sum_{l=0,\infty} \sigma_l,
\end{equation}
where 
\begin{equation}\label{e7.78}
\sigma_l = \frac{4\pi}{k^2}\, (2\,l+1)\, \sin^2\delta_l
\end{equation}
is the $l$th partial cross-section: {\rm  i.e.}, the contribution to the
total cross-section from the $l$th partial wave. Note that the maximum 
value for the $l$th partial cross-section occurs when the phase-shift  $\delta_l$ takes the value $\pi/2$.

\section{Determination of Phase-Shifts}
Let us now consider how the phase-shifts $\delta_l$ can be
 evaluated. Consider a spherically symmetric potential $V(r)$ that 
vanishes for $r>a$, where $a$ is termed the range of the potential.
In the region $r>a$, the wavefunction $\psi({\bf x})$ 
satisfies the free-space Schr\"{o}dinger equation (\ref{e7.54}). The
most general solution that is consistent with no incoming spherical waves is
\begin{equation}
\psi({\bf x}) = \frac{1}{(2\pi)^{3/2}} \sum_{l=0,\infty}
{\rm i}^l\, (2\,l+1) \, A_l(r)\, P_l(\cos\theta),
\end{equation}
where
\begin{equation}\label{e7.80}
A_l(r) = \exp(\,{\rm i} \,\delta_l)\,
\left[ \cos\delta_l \,j_l(k\,r) -\sin\delta_l\, \eta_l(k\,r)\right].
\end{equation}
Note that  Neumann functions are allowed to appear in the above
expression, because its region of validity does not include the origin
(where $V\neq 0$). The logarithmic derivative of the $l$th 
radial wavefunction
$A_l(r)$ just outside the range of the potential is given by
\begin{equation}
\beta_{l+} = k\,a \left[\frac{ \cos\delta_l\,j_l'(k\,a) -
\sin\delta_l\, \eta_l'(k\,a)}{\cos\delta_l \,
j_l(k\,a) - \sin\delta_l\,\eta_l(k\,a)}\right],
\end{equation}
where $j_l'(x)$ denotes $dj_l(x)/dx$, {\rm etc}. The above equation
can be inverted to give
\begin{equation}\label{e7.82}
\tan \delta_l = \frac{ k\,a\,j_l'(k\,a) - \beta_{l+}\, j_l(k\,a)}
{k\,a\,\eta_l'(k\,a) - \beta_{l+}\, \eta_l(k\,a)}.
\end{equation}
Thus, the problem of determining the phase-shift $\delta_l$ is equivalent
to that of determining $\beta_{l+}$. 

The most general solution to Schr\"{o}dinger's equation inside 
the range of the potential ($r<a$) that does not depend on the
azimuthal angle $\varphi$ is
\begin{equation}
\psi({\bf x}) = \frac{1}{(2\pi)^{3/2}}\sum_{l=0,\infty}
{\rm i}^l \,(2\,l+1)\,R_l(r)\,P_l(\cos\theta),
\end{equation}
where
\begin{equation}
R_l (r) = \frac{u_l(r)}{r},
\end{equation}
and
\begin{equation}\label{e7.85}
\frac{d^2 u_l}{d r^2} +\left[k^2 - \frac{2m}{\hbar^2} \,V - \frac{l\,(l+1)}
{r^2}\right] u_l = 0.
\end{equation}
The boundary condition 
\begin{equation}\label{e7.86}
u_l(0) = 0
\end{equation}
 ensures that the radial wavefunction is well-behaved at the
origin. 
We can launch a well-behaved solution of the above equation from 
$r=0$, integrate out to $r=a$, and form the logarithmic derivative
\begin{equation}
\beta_{l-} = \left.\frac{1}{(u_l/r)} \frac{d(u_l/r)}{dr}\right|_{r=a}.
\end{equation}
Because $\psi({\bf x})$ and its first derivatives are necessarily continuous for
physically acceptible wavefunctions, it follows that
\begin{equation}
\beta_{l+} = \beta_{l-}.
\end{equation}
The phase-shift $\delta_l$ is obtainable from Equation~(\ref{e7.82}).

\section{Hard Sphere Scattering}
Let us test out this scheme using a particularly simple example. Consider
scattering by a hard sphere, for which  the potential is infinite 
for $r<a$, and zero for $r>a$. It follows that $\psi({\bf x})$ is
zero in the region $r<a$, which implies that  $u_l =0$ for all $l$. 
Thus,
\begin{equation}
\beta_{l-} = \beta_{l+} = \infty,
\end{equation}
for all $l$. It follows from Equation~(\ref{e7.82}) that
\begin{equation}\label{e7.90}
\tan \delta_l = \frac{j_l(k\,a)}{\eta_l(k\,a)}.
\end{equation}

Consider the $l=0$ partial wave, which is usually referred to as the $s$-wave.
Equation~(\ref{e7.90}) yields
\begin{equation}
\tan\delta_0 = \frac{\sin (k\,a)/k\,a}{-\cos (k\,a)/ka} = -\tan (k\,a),
\end{equation}
where use has been made of Equations~(\ref{e7.58a})--(\ref{e7.58b}). It follows that
\begin{equation}\label{e7.92}
\delta_0 = -k\,a.
\end{equation}
The  $s$-wave radial wave function
is
\begin{equation}
A_0(r) = \exp(-{\rm i}\, k\,a) \left[\frac{\cos (k\,a) \,\sin (k\,r)
-\sin (k\,a) \,\cos( k\,r)}{k\,r}\right]
=\exp(-{\rm i}\, k\,a)\, \frac{ \sin[k\,(r-a)]}{k\,r}.
\end{equation}
The corresponding radial wavefunction for the incident wave 
takes the form
\begin{equation}
\tilde{A}_0(r) = \frac{ \sin (k\,r)}{k\,r}.
\end{equation}
It is clear that the actual $l=0$ radial wavefunction is similar to the
incident $l=0$ wavefunction, except that it is phase-shifted by $k\,a$. 

Let us consider the low and high energy asymptotic limits of $\tan\delta_l$.
Low energy corresponds to $k\,a\ll 1$. In this limit, the spherical Bessel functions
and Neumann functions reduce to:
\begin{align}
j_l(k\,r) &\simeq  \frac{(k\,r)^l}{(2\,l+1)!!},\\[0.5ex]
\eta_l(k\,r) &\simeq  -\frac{(2\,l-1)!!}{(k\,r)^{l+1}},
\end{align}
where $n!! = n\,(n-2)\,(n-4)\cdots 1$. It follows that
\begin{equation}
\tan\delta_l = \frac{-(k\,a)^{2\,l+1}}{(2\,l+1) \,[(2\,l-1)!!]^{\,2}}.
\end{equation}
It is clear that we can neglect  $\delta_l$, with $l>0$, with respect to
$\delta_0$. In other words, at low energy only $s$-wave scattering
({\rm i.e.}, spherically symmetric scattering) is important. It follows
from Equations~(\ref{e7.33}), (\ref{e7.73}), and (\ref{e7.92})  that 
\begin{equation}
\frac{d\sigma}{d{\mit\Omega}} = \frac{\sin^2 (k\,a)}{k^2} \simeq a^2
\end{equation}
for $k\,a\ll 1$. Note that the total cross-section
\begin{equation}
\sigma_{\rm total} = \oint d{\mit\Omega}\,\frac{d\sigma}{d{\mit\Omega}} = 4\pi \,a^2
\end{equation}
is {\em four times}\/ the {\em geometric cross-section}\/ $\pi \,a^2$
({\rm i.e.}, the cross-section for classical particles bouncing off a
hard sphere of radius $a$). 
However, 
low energy scattering implies relatively long wavelengths, so we do not
expect to obtain the  classical result  in this limit. 

Consider the high energy limit $k\,a\gg 1$. At high energies, all partial
waves up to $l_{\rm max} = k\,a$ contribute significantly to
the scattering cross-section. It follows from Equation~(\ref{e7.75}) that
\begin{equation}\label{e7.99}
\sigma_{\rm total} = \frac{4\pi}{k^2} \sum_{l=0,l_{\rm max}}
(2\,l+1)\,\sin^2\delta_l.
\end{equation}
With so many $l$ values contributing, it is legitimate to replace
$\sin^2\delta_l$ by its average value $1/2$. Thus,
\begin{equation}
\sigma_{\rm total} = \sum_{l=0,k\,a} \frac{2\pi}{k^2} \,(2\,l+1) \simeq 
2\pi \,a^2.
\end{equation}
This is {\em twice}\/ the classical result, which is  somewhat surprizing,
because we might expect to obtain the classical result in the short
wavelength limit. For hard sphere scattering, incident waves with
impact parameters less than $a$ must be deflected. However, in order to
produce a ``shadow'' behind the sphere there must be scattering
in the forward direction (recall the optical theorem) to produce
destructive interference with the incident plane wave. In fact, the
interference is not completely destructive, and the shadow has a bright
spot in the forward direction. The effective cross-section associated with
this bright spot is $\pi \,a^2$ which, when combined with the
cross-section for classical reflection, $\pi \,a^2$, gives the actual
cross-section of $2\pi \,a^2$.
 
\section{Low Energy Scattering}
At low energies ({\rm i.e.}, when $1/k$ is much larger than the range
of the potential) partial waves with $l>0$, in general, make a
negligible contribution to the scattering cross-section. It follows
that, at these energies, with a finite range potential, only $s$-wave
scattering is important.

As a specific example, let us consider scattering by  a finite
potential well, characterized by $V=V_0$ for $r<a$, and
$V=0$ for $r\geq a$. Here, $V_0$ is a constant. The potential
is repulsive for $V_0>0$, and attractive for $V_0<0$. 
The external  wavefunction is given by [see Equation~(\ref{e7.80})]
\begin{equation}
A_0(r) = \exp(\,{\rm i}\, \delta_0)\,\left[
j_0(k\,r) \cos\delta_0 - \eta_0(k\,r) \sin\delta_0\right]
= \frac{ \exp(\,{\rm i} \,\delta_0)\, \sin(k\,r+\delta_0)}{k\,r},
\end{equation}
where use has been made of Equations~(\ref{e7.58a})--(\ref{e7.58b}).
The internal wavefunction follows from Equation~(\ref{e7.85}). We obtain
\begin{equation}\label{e7.103}
A_0(r) = B \,\frac{\sin (k'\,r)}{r},
\end{equation}
where use has been made of the boundary condition (\ref{e7.86}).
Here, $B$ is a constant, and 
\begin{equation}
E - V_0 = \frac{\hbar^2 \,k'^{\,2}}{2\,m}.
\end{equation}
Note that Equation~(\ref{e7.103}) only applies when $E>V_0$. For $E<V_0$, we have
\begin{equation}
A_0(r) = B \,\frac{\sinh(\kappa\, r)}{r},
\end{equation}
where
\begin{equation}
V_0 - E = \frac{\hbar^2 \kappa^2}{2\,m}.
\end{equation}
Matching $A_0(r)$, and its radial derivative at $r=a$, yields
\begin{equation}\label{e7.107}
\tan(k\,a+\delta_0) = \frac{k}{k'} \,\tan (k'\,a)
\end{equation}
for $E>V_0$, and
\begin{equation}
\tan(k\,a+ \delta_0) = \frac{k}{\kappa} \,\tanh (\kappa\, a)
\end{equation}
for $E<V_0$.

Consider an attractive potential, for which $E>V_0$. Suppose that 
$|V_0|\gg E$ ({\rm i.e.}, the depth of the potential well is much larger than
the energy of the incident particles), so that $k' \gg k$. It follows
from Equation~(\ref{e7.107}) that, unless $\tan (k'\,a)$ becomes extremely large, the right-hand side is much less that unity, so replacing the tangent of a
small quantity with the quantity itself, we obtain
\begin{equation}
k\,a + \delta_0 \simeq \frac{k}{k'}\,\tan (k'\,a).
\end{equation}
This yields
\begin{equation}
\delta_0 \simeq k\,a \left[ \frac{\tan( k'\,a)}{k'\,a} -1\right].
\end{equation}
According to Equation~(\ref{e7.99}), the scattering cross-section is given by
\begin{equation}\label{e7.111}
\sigma_{\rm total} \simeq \frac{4\pi}{k^2} \sin^2\delta_0
=4\pi \,a^2\left[\frac{\tan (k'\,a)}{(k'\,a)} -1\right]^{\,2}.
\end{equation}
Now, 
\begin{equation}\label{e7.112}
k'a = \sqrt{ k^2 \,a^2 + \frac{2 \,m \,|V_0|\, a^2}{\hbar^2}},
\end{equation}
so for sufficiently small values of $k\,a$,
\begin{equation}
k' a \simeq \sqrt{\frac{2\, m \,|V_0|\, a^2}{\hbar^2}}.
\end{equation}
It follows that the total ($s$-wave) scattering cross-section is independent
of the energy of the incident particles (provided that this energy is
sufficiently small). 

Note that there are values of $k'a$ ({\rm e.g.}, $k'a\simeq 4.49$) at which
$\delta_0\rightarrow \pi$, and 
the scattering cross-section (\ref{e7.111}) vanishes, despite the very strong
attraction of the potential. In reality, the cross-section is not
exactly zero, because of contributions from $l>0$ partial waves. But,
at low incident energies, these contributions are small. It follows that
there are certain values of $V_0$ and $k$ that  give rise to almost perfect 
transmission of the incident wave. This is called the {\em Ramsauer-Townsend
effect}, and has been observed experimentally. 

\section{Resonance Scattering}
There is a significant exception to the independence of the cross-section on energy. Suppose that the quantity $\sqrt{2\,m \,|V_0|\,a^2/\hbar^2}$
is slightly less than $\pi/2$. As the incident energy increases, $k'a$,
which is given by Equation~(\ref{e7.112}), can reach the value $\pi/2$. In this case,
$\tan( k'\,a)$ becomes infinite, so we can no longer assume that the right-hand side of Equation~(\ref{e7.107}) is small. In fact,
at the value of the incident energy
when $k'a = \pi/2$, it follows from Equation~(\ref{e7.107}) that $k\,a+\delta_0 = \pi/2$,
or $\delta_0 \simeq \pi/2$ (because we are assuming that $k\,a\ll 1$). 
This implies that
\begin{equation}
\sigma_{\rm total} = \frac{4\pi}{k^2} \sin^2\delta_0 = 4\pi \,a^2
\left(\frac{1}{k^2 \,a^2}\right).
\end{equation}
Note that the cross-section now depends on the energy. Furthermore, the
magnitude of the cross-section is much larger than that given in Equation~(\ref{e7.111})
for $k'a\neq \pi/2$ (since $k\,a\ll 1$). 

The origin of this rather strange behaviour is quite simple. The condition
\begin{equation}
\sqrt{\frac{2\,m\,|V_0 |\,a^2}{\hbar^2} } = \frac{\pi}{2}
\end{equation}
is equivalent to the condition that a spherical well of depth
$V_0$ possesses a bound state at zero energy. Thus, for a potential
well that satisfies the above equation, the energy of the scattering system
is essentially the same as the energy of the bound state. In this situation,
an incident particle would like to form a bound state in the potential
well. However, the bound state is not stable, because  the system has a small
positive energy. Nevertheless, this sort of {\em resonance scattering}\/
is best understood as the capture of an incident particle to form
a metastable bound state, and the subsequent decay of the bound state
and release of the particle. The cross-section for resonance scattering
is generally far higher than that for non-resonance scattering.

We have seen that there is a resonant effect when the phase-shift of
the $s$-wave takes the value $\pi/2$.  There is nothing special about
the $l=0$ partial wave, so it is reasonable to assume that there
is a similar resonance when the phase-shift of the $l$th partial
wave is $\pi/2$. Suppose that $\delta_l$ attains the value
$\pi/2$ at the incident energy $E_0$, so that
\begin{equation}
\delta_l(E_0) = \frac{\pi}{2}.
\end{equation}
Let us expand $\cot \delta_l$ in the vicinity of the resonant energy:
\begin{equation}
\cot \delta_l(E) = \cot \delta_l(E_0) +\left(
\frac{ d \cot\delta_l}{d E}\right)_{E=E_0}(E-E_0) + \cdots=- \left(\frac{1}{\sin^2\delta_l}\frac{d\delta_l}{d E}\right)_{E=E_0}
(E-E_0)+\cdots.
\end{equation}
Defining
\begin{equation}
\left(\frac{d \delta_l(E)}{d E} \right)_{E=E_0} = \frac{2}{{\mit\Gamma}},
\end{equation}
we obtain
\begin{equation}
\cot\delta_l(E) = - \frac{2}{{\mit\Gamma}} \,(E-E_0) + \cdots.
\end{equation}
Recall, from Equation~(\ref{e7.78}), that the contribution of the $l$th partial wave
to the scattering cross-section is
\begin{equation}
\sigma_l = \frac{4\pi}{k^2} \,(2\,l+1)\,\sin^2\delta_l 
= \frac{4\pi}{k^2} \,(2\,l+1)\,\frac{1}{1+\cot^2\delta_l}.
\end{equation}
Thus,
\begin{equation}
\sigma_l \simeq \frac{4\pi}{k^2} \,(2\,l+1)\,
\frac{{\mit\Gamma}^{\,2}/4}{(E-E_0)^{\,2} + {\mit\Gamma}^{\,2}/4}.
\end{equation}
This is the famous {\em Breit-Wigner}\/ formula. The variation of
the partial cross-section $\sigma_l$ with the incident energy has
the form of a classical resonance curve. The quantity ${\mit\Gamma}$ is
the width of the resonance (in energy). We can interpret the
Breit-Wigner formula as describing the absorption of an incident particle
to form a metastable state, of energy $E_0$, and lifetime $\tau = \hbar/
{\mit\Gamma}$.

\subsection*{Exercises}
\begin{enumerate}[label=\thechapter.\arabic*,leftmargin=*,widest=9.20]
\item Consider a scattering potential of the form
$$
V(r)=V_0\,\exp(-r^2/a^2).
$$
Calculate the differential scattering cross-section, $d\sigma/d{\mit\Omega}$, using the Born approximation.

\item Consider a scattering potential that takes the constant value $V_0$ for $r<R$, and is zero
for $r>R$, where $V_0$ may be either positive or negative. Using the method of partial waves, show
that for $|V_0|\ll E=\hbar^2\,k^2/2\,m$, and $k\,R\ll 1$, the differential cross-section is isotropic, and that
the total cross-section is
$$
\sigma_{\rm tot} = \left(\frac{16\pi}{9}\right) \frac{m^2\,V_0^{\,2}\,R^{\,6}}{\hbar^4}.
$$
Suppose that  the energy is slightly raised. Show that the angular distribution can then
be written in the form
$$
\frac{d\sigma}{d{\mit\Omega} }= A + B\,\cos\theta.
$$
Obtain an approximate expression for $B/A$. 

\item Consider scattering by a repulsive $\delta$-shell potential:
$$
V(r) = \left(\frac{\hbar^2}{2\,m}\right)\gamma\,\delta(r-R),
$$
where $\gamma>0$. 
Find the equation that determines the $s$-wave phase-shift, $\delta_0$, as a function of $k$ (where $E=\hbar^2\,k^2/2\,m$). 
Assume that  $\gamma\gg R^{-1}$, $k$. Show that if $\tan(k\,R)$ is not close to zero then the $s$-wave phase-shift
resembles the hard sphere result discussed in the text. Furthermore, show that if $\tan(k\,R)$ is close to zero then resonance
behavior is possible: i.e., $\cot \delta_0$ goes through zero from the positive side as $k$ increases. Determine the
approximate positions of the resonances (retaining terms up to order $1/\gamma$). Compare the resonant
energies with the bound state energies for a particle confined within an infinite spherical well of radius $R$. 
Obtain an approximate expression for the resonance width
$$
{\mit\Gamma} = - \frac{2}{[d(\cot\delta_0)/dE]_{E=E_r}}.
$$
Show that the resonances become extremely sharp as $\gamma\rightarrow \infty$. 

\item Show that the differential cross-section for the elastic scattering of a fast electron by the ground-state of a hydrogen
atom is 
$$
\frac{d\sigma}{d{\mit\Omega}} = \left(\frac{2\,m_e\,e^2}{4\pi\,\epsilon_0\,\hbar^{\,2}\,q^2}\right)^2\left(1-\frac{16}{[4+(q\,a_0)^2]^{\,2}}\right),
$$
where $q=|{\bf k}-{\bf k}'|$, and $a_0$ is the Bohr radius. 
\end{enumerate}


