\chapter{Position and Momentum}
% !TEX root = ../Quantum.tex

\section{Introduction}
So far, we have considered {\em general}\/ dynamical variables represented by 
{\em general}\/ linear
operators acting in ket space. However, in classical mechanics, the most important
dynamical variables are those involving
position and momentum. Let us investigate 
the role of such  variables  in quantum mechanics. 

In classical mechanics, the position $q$ and momentum $p$ 
of some component of a dynamical  system are represented as {\em real numbers} which,
by definition, commute.
In quantum mechanics, these quantities are represented
as   {\em non-commuting}\/ linear Hermitian operators acting in a ket space
that represents all of the possible states of the system. Our first task is
to discover a quantum mechanical replacement for the classical result
$q\,p-p\,q = 0$.  

\section{Poisson Brackets}
Consider a dynamical system whose state at a particular time $t$ is
fully specified 
by $N$ independent classical  coordinates $q_i$ (where $i$ runs from 1 to $N$). 
Associated with each generalized coordinate $q_i$ is a 
classical canonical momentum
$p_i$. For instance, a Cartesian coordinate  has an associated linear
momentum, an angular coordinate has an associated angular momentum, {\rm etc}.
As is well-known, the behavior of a classical system can be specified in terms
of Lagrangian or Hamiltonian dynamics. For instance, in Hamiltonian dynamics,
\begin{align}\label{e2.1e}
\frac{d q_i}{d t} &= ~\frac{\partial H}{\partial p_i},\\[0.5ex]
\frac{d p_i}{dt} &= - \frac{\partial H}{\partial q_i},\label{e2.2e}
\end{align}
where the function $H(q_i, p_i, t)$ is the system energy  at time $t$
expressed in terms of the 
classical coordinates and canonical momenta. This function is
usually referred to as the {\em Hamiltonian}\/ of the system.

We are interested in 
finding some
construct in classical dynamics that consists of
{\em products}\/ of dynamical variables. If such a construct exists then we hope to 
generalize it somehow to obtain  a
rule describing how dynamical variables
commute with one another in quantum mechanics. There is, indeed,
 one well-known construct
in classical dynamics that involves products of dynamical variables. The classical 
{\em Poisson bracket}\/ of two dynamical variables $u$ and $v$ is defined
\begin{equation}\label{e3.2}
[u, v]_{cl} = \sum_{i=1,N} \left(\frac{\partial u}{\partial q_i}\frac{\partial v}
{\partial p_i} - \frac{\partial u}{\partial p_i}\frac{\partial v}{\partial q_i}
\right),
\end{equation}
where $u$ and $v$ are regarded as functions of the coordinates 
and momenta, $q_i$ and $p_i$. It is easily demonstrated that
\begin{align}\label{e3.3a}
[q_i, q_j]_{cl} &= 0,\\[0.5ex]
[p_i, p_j]_{cl} &= 0,\\[0.5ex]
[q_i, p_j]_{cl} &= \delta_{ij}.\label{e3.3c}
\end{align}
 The time evolution of a dynamical variable can also
be
written in terms of a Poisson bracket by noting that
\begin{equation}
\frac{du}{dt} = \sum_{i=1,N} \left(\frac{\partial u}{\partial q_i}\frac{d q_i}{dt}
+ \frac{\partial u}{\partial p_i}\frac{dp_i}{dt}\right)
= \sum_{i=1,N} \left(\frac{\partial u}{\partial q_i}\frac{\partial H}{\partial p_i}
-\frac{\partial u}{\partial p_i}\frac{\partial H}{\partial q_i}\right)
=[u, H]_{cl},\label{e3.4}
\end{equation}
where use has been made of Hamilton's equations, (\ref{e2.1e})--(\ref{e2.2e}).

Can we construct a quantum mechanical Poisson bracket in which $u$ and
$v$ are non-commuting operators, instead of functions? Well, the main properties
of the classical Poisson bracket are as follows:
\begin{align}\label{e3.5a}
[u, v]_{cl} &= - [v, u]_{cl},\\[0.5ex]
[u, c]_{cl} &= 0,\\[0.5ex]
[u_1+ u_2, v]_{cl} &= [u_1, v]_{cl} + [u_2, v]_{cl},\\[0.5ex]
[u, v_1 + v_2]_{cl}&= [u, v_1] _{cl}+ [u, v_2]_{cl},\\[0.5ex]
[u_1\, u_2, v]_{cl} &= [u_1, v]_{cl} \,u_2 + u_1\, [u_2, v]_{cl},\label{e3.5e}\\[0.5ex]
[u, v_1 \,v_2]_{cl} &= [u, v_1]_{cl} \,v_2 + v_1 \,[u, v_2]_{cl},\label{e3.5f}\\[0.5ex]
[u, [v, w]_{cl} ]_{cl}+ [v, [w, u]_{cl} ]_{cl} + [w, [u, v]_{cl}]_{cl} &= 0.\label{e3.6}
\end{align}
The last relation is known as the {\em Jacobi identity}. In the above,
$u$, $v$, $w$, {\rm etc.}, represent dynamical variables, and $c$ represents a number.
Can we find some combination of non-commuting operators $u$ and $v$, {\rm etc.}, 
that satisfies all of the above relations? We shall refer to such a combination as a quantum mechanical Poisson bracket. 

Actually, we can evaluate the quantum mechanical Poisson bracket $[u_1 \,u_2, v_1 \,v_2]_{qm}$ in
two different ways, because we can employ either of the formulae (\ref{e3.5e}) or
(\ref{e3.5f}) first. Thus,
\begin{align}
[u_1\, u_2, v_1\, v_2]_{qm} &= [u_1, v_1 \,v_2]_{qm}\,u_2 + u_1\,[u_2, v_1\, v_2]_{qm}\nonumber\\[0.5ex]
&=\left([u_1, v_1]_{qm}\,v_2 + v_1\,[u_1, v_2]_{qm}\right) u_2 
+u_1\left([u_2, v_1]_{qm}\,v_2 + v_1\,[u_2, v_2]_{qm}\right)\nonumber\\[0.5ex]
&= [u_1, v_1]_{qm}\, v_2 \,u_2 + v_1\,[u_1, v_2]_{qm} \,u_2 + u_1\,[u_2, v_1]_{qm}\,v_2
+ u_1 \,v_1\,[u_2, v_2]_{qm},
\end{align}
and
\begin{align}
[u_1 \,u_2, v_1\, v_2]_{qm} &= [u_1 \,u_2, v_1 ]_{qm}\,v_2 + v_1\,[u_1\, u_2, v_2]_{qm}\nonumber\\[0.5ex]
&= [u_1, v_1]_{qm} \,u_2 \,v_2 + u_1\,[u_2, v_1]_{qm} \,v_2 + v_1\,[u_1, v_2]_{qm}\,u_2
+ v_1\, u_1\,[u_2, v_2]_{qm}.
\end{align}
Note that the order of the various factors has been preserved, because they
now represent {\em non-commuting}\/ operators. Equating the above two results
yields
\begin{equation}
[u_1, v_1]_{qm}\, (u_2 \,v_2 - v_2 \,u_2) = (u_1 \,v_1-v_1\, u_1)\,[u_2, v_2]_{qm}.
\end{equation}
Since this relation must hold for $u_1$ and $v_1$ quite independent of
$u_2$ and $v_2$, it follows that
\begin{align}
u_1 \,v_1 - v_1\, u_1 &= {\rm i}\,\hbar \,[u_1, v_1]_{qm},\\[0.5ex]
u_2\, v_2 - v_2\, u_2 &= {\rm i} \,\hbar \,[u_2, v_2]_{qm},
\end{align}
where $\hbar$ does not depend on $u_1$, $v_1$, $u_2$, $v_2$, and also
commutes with $(u_1\, v_1- v_1 \,u_1)$. Because $u_1$, {\rm etc.}, are  general
operators, it follows that $\hbar$ is just a number. We want the quantum 
mechanical Poisson
bracket of two Hermitian operators to be a Hermitian operator itself, because 
the classical Poisson bracket of two real dynamical variables is real. This
requirement is satisfied if $\hbar$ is a real number. Thus, the
quantum mechanical Poisson bracket of two dynamical variables $u$ and $v$
is given by
\begin{equation}
[u, v]_{qm} = \frac{u\,v - v\,u}{{\rm i} \,\hbar},
\end{equation}
where $\hbar$ is a new universal constant of nature. Quantum mechanics agrees with
experiments provided that  $\hbar$ takes the value $h/2\pi$, where
\begin{equation}
h = 6.6261 \times 10^{-34}~~{\rm J\,s}
\end{equation}
is {\em Planck's constant}. The notation
$[u, v]$ is
conventionally  reserved for the commutator $u\,v-v\,u$ in quantum mechanics. 
Thus,
\begin{equation}\label{e3.13}
[u, v]_{qm} = \frac{[u, v]}{{\rm i}\, \hbar}.
\end{equation}
It is easily demonstrated that the quantum mechanical Poisson bracket, as defined above,
satisfies all of the relations (\ref{e3.5a})--(\ref{e3.6}).

The strong analogy we have found between the classical Poisson bracket, defined
in Equation~(\ref{e3.2}), and the quantum mechanical 
Poisson bracket, defined in Equation~(\ref{e3.13}), leads
us to  assume that the quantum mechanical bracket has the same
value as the corresponding classical bracket, at least for the simplest
cases. In other words, we are assuming that Equations~(\ref{e3.3a})--(\ref{e3.3c}) hold for quantum
mechanical as well as classical Poisson brackets. This argument yields the
fundamental commutation relations
\begin{align}
[q_i, q_j] &= 0,\label{e3.14a}\\[0.5ex]
[p_i, p_j]&= 0, \label{e3.14b}\\[0.5ex]
[q_i, p_j] &= {\rm i} \,\hbar \,\delta_{ij}.\label{e3.14c}
\end{align}
These results provide us with the basis for calculating commutation
relations between general dynamical variables. For instance, if
two dynamical variables,  $\xi$ and $\eta$,
can both  be written as a power series in the
 $q_i$ and $p_i$ then repeated application of Equations~(\ref{e3.5a})--(\ref{e3.5f})
allows $[\xi, \eta]$ to be expressed in terms of the fundamental
commutation relations (\ref{e3.14a})--(\ref{e3.14c}). 

Equations~(\ref{e3.14a})--(\ref{e3.14c}) provide the foundation for the analogy between quantum mechanics
and classical mechanics. Note that the classical result (that everything commutes)
is obtained in the limit $\hbar\rightarrow 0$. Thus, {\em classical mechanics
can be regarded as the limiting case of quantum mechanics when
$\hbar$ goes to zero}. 
In classical mechanics, each 
pair of generalized coordinate and its conjugate momentum, $q_i$ and
$p_i$, correspond to a different classical degree of freedom of the system. 
It is clear from Equations~(\ref{e3.14a})--(\ref{e3.14c}) that in quantum mechanics the {\em dynamical
variables corresponding to different degrees of freedom all commute}. 
It is only those variables corresponding to the same degree of freedom that 
may fail to commute.

\section{Wavefunctions}
Consider a simple system with one classical degree of freedom, which corresponds to
the Cartesian coordinate $x$. Suppose that $x$ is free to take any value ({\rm e.g.},
$x$ could be the position of a free particle). The classical dynamical variable
$x$ is represented in quantum 
mechanics  as a linear Hermitian operator which is also called  $x$.
Moreover, the operator $x$ possesses eigenvalues $x'$
lying in the {\em continuous}
range $-\infty< x'<+\infty$ (since the eigenvalues
correspond to all the possible results of a measurement of $x$). We can
span ket space using the suitably normalized eigenkets of $x$.
An eigenket corresponding to the eigenvalue $x'$ is denoted $|x'\rangle$.
Moreover, 
[see Equation~(\ref{e2.77})]
\begin{equation}\label{e3.15}
\langle x' | x''\rangle = \delta(x'-x'').
\end{equation}
The eigenkets satisfy the extremely useful relation [see Equation~(\ref{e2.78})]
\begin{equation}\label{e3.16}
\int_{-\infty}^{+\infty} d x' \, |x'\rangle\langle x'|= 1.
\end{equation}
This formula expresses the fact that the eigenkets are complete, mutually
orthogonal, and suitably normalized.

A state ket $|A\rangle$ (which represents a general  state $A$ of the system)
can be expressed as a linear superposition of the eigenkets of the position
operator using Equation~(\ref{e3.16}). Thus,
\begin{equation}\label{e3.17}
|A\rangle = \int_{-\infty}^{+\infty} dx' \,\langle x'|A\rangle |x'\rangle
\end{equation}
The quantity $\langle x'|A\rangle$ is a complex function of the position eigenvalue
$x'$. We can write
\begin{equation}\label{e3.18}
\langle x'|A\rangle = \psi_A(x').
\end{equation}
Here, $\psi_A(x')$ is the famous {\em wavefunction}\/ of quantum mechanics. 
Note that state $A$ is completely specified by its wavefunction $\psi_A(x')$
[because the wavefunction can be used to reconstruct the state ket $|A\rangle$
using Equation~(\ref{e3.17})].
It is clear that the wavefunction of state $A$ is simply the collection
of the weights of the corresponding state ket $|A\rangle$,
when it is expanded in terms of the eigenkets of the
position operator. Recall, from Section~\ref{s2.10}, that the probability of
a measurement of a dynamical variable $\xi$ yielding the result $\xi'$ 
when the system is in state $A$ is given by
$|\langle \xi'|A\rangle|^{\,2}$, assuming that
the
eigenvalues of $\xi$ are discrete. This result is easily generalized to dynamical
variables possessing continuous eigenvalues. In fact, the probability of
a measurement of $x$ yielding a result lying in the range $x'$ to $x'+dx'$
when the system is in a state $|A\rangle$ is $|\langle x'|A\rangle|^{\,2}\,dx'$. 
In other words, the probability of a measurement of position yielding a
result in the range $x'$ to $x'+dx'$ when the wavefunction of the system is
$\psi_A(x')$ is
\begin{equation}\label{e3.19}
P(x', dx') = |\psi_A(x')|^{\,2}\, dx'.
\end{equation}
This formula is only valid if the state ket $|A\rangle$ is properly normalized:
{\rm i.e.}, if $\langle A|A\rangle = 1$. The corresponding normalization for
the wavefunction is
\begin{equation}\label{e3.20}
\int_{-\infty}^{+\infty} dx'\, |\psi_A(x')|^{\,2}= 1.
\end{equation}
Consider a second state $B$ represented by a state ket $|B\rangle$ and
a wavefunction $\psi_B(x')$. The inner product $\langle B| A\rangle$ 
can be written
\begin{equation}
\langle B| A\rangle = \int_{-\infty}^{+\infty} dx'\,\langle B| x'\rangle
\langle x' | A \rangle 
 = \int_{-\infty}^{+\infty} dx'\,\psi_B^\ast (x') \,\psi_A'(x'),
\end{equation}
where use has been made of Equations~(\ref{e3.16}) and (\ref{e3.18}). Thus, the inner product of two states is
related to the overlap integral of their wavefunctions. 

Consider a general function $f(x)$ of the observable $x$ [{\rm e.g.}, $f(x)=x^2$]. 
If $|B\rangle = f(x)\,|A\rangle$ then it follows that
\begin{equation}
\psi_B(x') = \langle x'| f(x) \int_{-\infty}^{+\infty} dx''\,\psi_A(x'')\,|x''\rangle
= \int_{-\infty}^{+\infty} dx''\, f(x'')\,\psi_A(x'') \,\langle x'|x''\rangle,
\end{equation}
giving
\begin{equation}
\psi_B(x') = f(x')\, \psi_A(x'),
\end{equation}
where use has been made of Equation~(\ref{e3.15}). Here, $f(x')$ is the same function
of the position eigenvalue $x'$ that $f(x)$ is of the position operator $x$:
{\rm i.e.}, if $f(x)=x^2$ then $f(x') = x'^{\,2}$. It follows, from the above result,
that a general state ket $|A \rangle$ can be written
\begin{equation}
|A\rangle = \psi_A(x) \rangle,
\end{equation}
where $\psi_A(x)$ is the same function of the operator $x$ that the wavefunction
$\psi_A(x')$ is of the position eigenvalue $x'$, and the ket $\rangle$ has the
wavefunction $\psi(x') =1$. The ket $\rangle$ is termed the {\em standard ket}. 
The dual of the standard ket is termed the {\em standard bra}, and is
denoted $\langle$.  It is
easily seen that
\begin{equation}\label{e3.25}
\langle \psi_A^{\,\ast}(x) \stackrel{\rm DC}{\longleftrightarrow} \psi_A(x)\rangle.
\end{equation}
Note, finally, that $\psi_A(x)\rangle$ is often shortened to $\psi_A\rangle$, leaving
the dependence on the position operator $x$ tacitly understood. 

\section{Schr\"{o}dinger Representation}\label{s3.4}
Consider the simple system described in the previous section. A general
state ket can be written $\psi(x)\rangle$, where $\psi(x)$ is a general function of
the position operator $x$, and $\psi(x')$ is the associated wavefunction. 
Consider the ket whose wavefunction is $d\psi(x')/dx'$. This ket is
denoted $d\psi/dx\rangle$. The new ket is clearly a linear function of
the  original ket, so we can think of it as the result of some linear
operator acting on $\psi\rangle$. Let us denote this operator
$d/dx$. It follows that
\begin{equation}
\frac{d}{dx}\, \psi\rangle = \frac{d\psi}{dx}\rangle.
\end{equation}

Any linear operator that acts on ket vectors can also act on bra vectors. 
Consider $d/dx$ acting on a general bra $\langle \phi(x)$. According to
Equation~(\ref{e2.32}), the bra $\langle \phi \,d/dx$ satisfies
\begin{equation}
\left( \langle \phi \,\frac{d}{dx} \right) \psi\rangle = \langle \phi\left(
\frac{d}{dx}\,\psi\rangle\right).
\end{equation}
Making use of Equations~(\ref{e3.16}) and (\ref{e3.18}), we can write
\begin{equation}
\int_{-\infty}^{+\infty}dx' \,\langle \phi \,\frac{d}{dx}\, |x'\rangle
\, \psi(x') = \int_{-\infty}^{+\infty} dx'\,\phi(x')\,  \frac{d\psi(x')}{dx'}.
\end{equation}
The right-hand side can be transformed via integration by parts to give
\begin{equation}\label{e3.29}
\int_{-\infty}^{+\infty} \,dx' \,\langle \phi\, \frac{d}{dx}\, |x'\rangle
\,\psi(x') = -\int_{-\infty}^{+\infty}dx' \, \frac{d \phi(x')}{dx'}\,  \psi(x'),
\end{equation}
assuming that the contributions from the limits of integration vanish. 
It follows that
\begin{equation}
\langle \phi\, \frac{d}{dx}\, | x'\rangle = - \frac{d \phi(x')}{dx'},
\end{equation}
which implies that 
\begin{equation}\label{e3.31}
\langle \phi\, \frac{d}{dx} = - \langle \frac{d\phi}{dx}.
\end{equation}
The neglect of contributions from the limits of integration in Equation~(\ref{e3.29}) is
reasonable because
 physical wavefunctions are square-integrable [see Equation~(\ref{e3.20})].
Note that
\begin{equation}
\frac{d}{dx}\, \psi\rangle = \frac{d\psi}{dx}\rangle \stackrel{\rm DC}{\longleftrightarrow}
\langle \frac{d\psi^\ast}{dx} = -\langle \psi^\ast \,\frac{d}{dx},
\end{equation}
where use has been made of Equation~(\ref{e3.31}). 
It follows, by comparison with Equations~(\ref{e2.33}) and (\ref{e3.25}), that
\begin{equation}\label{e3.33}
\left(\frac{d}{dx}\right)^{\dag }= - \frac{d}{dx}.
\end{equation}
Thus, $d/dx$ is an {\em anti-Hermitian}\/  operator. 

Let us evaluate the commutation relation between the operators $x$ and $d/dx$.
We have
\begin{equation}
\frac{d}{dx}\, x\, \psi\rangle = \frac{d (x\,\psi)}{dx}\rangle = x\,\frac{d}{dx}\,\psi\rangle
+ \psi\rangle.
\end{equation}
Since this holds for any ket $\psi\rangle$, it follows that
\begin{equation}\label{e3.35}
\frac{d}{dx} \,x - x \,\frac{d}{dx} = 1.
\end{equation}
Let $p_x$ be the momentum conjugate to $x$
 (for the simple system under consideration
 $p_x$ is a straightforward linear momentum). According to Equation~(\ref{e3.14c}),
$x$ and $p$ satisfy the commutation relation
\begin{equation}\label{e3.36}
x\,p_x - p_x\,x = {\rm i} \,\hbar.
\end{equation}
It can be seen, by comparison with Equation~(\ref{e3.35}), that 
the Hermitian operator {\em $ -{\rm i}\,\hbar \,d/dx$
satisfies the same commutation relation with $x$ that $p_x$ does}. 
The most general conclusion which may be drawn from a comparison of Equations~(\ref{e3.35})
and (\ref{e3.36}) is that
\begin{equation}\label{e3.37}
p_x = -{\rm i}\,\hbar\,\frac{d}{dx} + f(x),
\end{equation}
since (as is easily demonstrated) a general function $f(x)$ of the position
operator  automatically commutes with $x$. 

We have chosen to normalize the eigenkets and
eigenbras of the position operator such that they satisfy
the normalization condition (\ref{e3.15}). However, this choice of normalization does not
uniquely determine the eigenkets and
eigenbras. Suppose that we transform to a new
set of eigenbras which are related to the old set via
\begin{equation}\label{e3.38}
\langle x'|_{\rm new} = {\rm e}^{\,{\rm i}\,\gamma'}\langle x'|_{\rm old},
\end{equation}
where $\gamma'\equiv \gamma(x')$ is a real function of $x'$.
This transformation amounts to a rearrangement of the relative
phases of the eigenbras.  The new
normalization condition is
\begin{align}
\langle x'|x''\rangle_{\rm new} &= \langle x'|\, {\rm e}^{\,{\rm i}\,\gamma'}
 {\rm e}^{-{\rm i}\,\gamma''}|x''\rangle_{\rm old} 
= {\rm e}^{\,{\rm i}\,(\gamma'-\gamma'')}
\langle x'|x''\rangle_{\rm old}\nonumber\\[0.5ex]
 &={ \rm e}^{\,{\rm i}\,(\gamma'-\gamma'')}\,\delta(x'-x'')= \delta(x'-x'').
\end{align}
Thus, the new eigenbras satisfy the same normalization condition
 as the old eigenbras. 

By definition, the standard ket $\rangle$ satisfies $\langle x'|\rangle = 1$.
It follows from Equation~(\ref{e3.38}) that the new standard ket is related to the
old standard ket via
\begin{equation}\label{e3.40}
\rangle_{\rm new} = {\rm e}^{-{\rm i}\, \gamma} \rangle_{\rm old},
\end{equation}
where $\gamma\equiv \gamma(x)$ is a real function of the position operator $x$. 
The dual of the above equation yields the transformation rule for
the standard bra,
\begin{equation}\label{e3.41}
\langle_{\rm new} = \langle_{\rm old} \,{\rm e}^{\,{\rm i}\, \gamma}.
\end{equation}
The transformation rule for a general operator $A$ follows from Equations~(\ref{e3.40}) and (\ref{e3.41}),
plus the requirement that the triple product $\langle A\rangle$ remain
invariant (this must be the case, otherwise the probability of a measurement
yielding a certain result would depend on the choice of eigenbras). Thus,
\begin{equation}\label{e3.42}
A_{\rm new} = {\rm e}^{-{\rm i}\,\gamma} A_{\rm old}\,{\rm e}^{\,{\rm i}\,\gamma}.
\end{equation}
Of course, if $A$ commutes with $x$ then $A$ is invariant under the transformation.
In fact, $d/dx$ is the only operator (that we know of) which does not commute
with $x$, so Equation~(\ref{e3.42}) yields
\begin{equation}
\left(\frac{d}{dx}\right)_{\rm new} =  {\rm e}^{-{\rm i}\,\gamma}\frac{d}{dx}\,{\rm e}^{\,{\rm i}\,\gamma} = \frac{d}{dx} + {\rm i}\, \frac{d\gamma}{dx},
\end{equation}
where the subscript ``old'' is taken as read. It follows, from Equation~(\ref{e3.37}), that
the momentum operator $p_x$ can be written
\begin{equation}
p_x = -{\rm i} \,\hbar\,\left(\frac{d}{dx}\right)_{\rm new} - \hbar\,\frac{d\gamma}{dx} + f(x).
\end{equation}
Thus, the special choice
\begin{equation}\label{e3.45}
\hbar\,\gamma(x) = \int^x dx'\, f(x')
\end{equation}
 yields
\begin{equation}\label{e3.46}
p_x =  -{\rm i} \,\hbar\,\left(\frac{d}{dx}\right)_{\rm new}.
\end{equation}
Equation~(\ref{e3.45}) fixes $\gamma$ to
within an arbitrary additive constant: {\rm i.e.}, the special eigenkets and eigenbras 
for which Equation~(\ref{e3.46}) is true are determined to within an arbitrary common phase-factor. 

In conclusion, it is possible to find a set of basis eigenkets and eigenbras
of the position operator $x$ that satisfy the normalization condition (\ref{e3.15}),
and for which the momentum conjugate to $x$ can be represented as the operator
\begin{equation}
p_x = -{\rm i}\,\hbar\,\frac{d}{dx}.
\end{equation}
A general state ket is written $\psi(x)\rangle$, where the standard ket $\rangle$
satisfies $\langle x'|\rangle = 1$, and where $\psi(x')= \langle x'|
\psi(x)\rangle$ is the wavefunction. 
This scheme of things is known as the {\em Schr\"{o}dinger representation}, and is the
basis of wave mechanics. 

\section{Generalized Schr\"{o}dinger Representation}\label{s3.5}
In the preceding section, we developed the Schr\"{o}dinger  representation
for the case of a single 
operator $x$ corresponding to a classical Cartesian coordinate. However, this scheme
can easily be extended. Consider a system with $N$ generalized coordinates, 
$q_1\cdots q_N$, which can all be {\em simultaneously}\/ measured. These are represented 
as $N$ {\em  commuting}\/ operators, $q_1\cdots q_N$, each with a continuous range
of eigenvalues, $q_1'\cdots q_N'$.
 Ket space is conveniently spanned  by  the simultaneous
eigenkets of $q_1\cdots q_N$, which are denoted $|q_1'\cdots q_N'\rangle$. These 
eigenkets must form a complete set, otherwise the $q_1\cdots q_N$ would not be 
simultaneously observable. 

The orthogonality condition for the eigenkets [{\rm i.e.}, the generalization of
Equation~(\ref{e3.15})] is
\begin{equation}
\langle q_1'\cdots q_N'| q_1''\cdots q_N''\rangle = \delta(q_1'-q_1'')\,\delta(q_2'-q_2'')\cdots
\delta(q_N'-q_N'').
\end{equation}
The completeness condition [{\rm i.e.}, the generalization of Equation~(\ref{e3.16})] is
\begin{equation}
\int_{-\infty}^{+\infty} \cdots\int_{-\infty}^{+\infty} dq_1' \cdots dq_N'\,
|q_1'\cdots q_N'\rangle \langle q_1'\cdots q_N'| = 1.
\end{equation}
The standard ket $\rangle$ is defined such that
\begin{equation}\label{e3.50}
\langle q_1'\cdots q_N'|\rangle = 1.
\end{equation}
The standard bra $\langle$ is the dual of the standard ket. A general state 
ket is written
\begin{equation}
\psi(q_1\cdots q_N)\rangle.
\end{equation}
The associated wavefunction is
\begin{equation}
\psi(q_1'\cdots q_N') = \langle q_1'\cdots q_N'|\psi\rangle.
\end{equation}
Likewise, a general state bra is written
\begin{equation}
\langle \phi(q_1\cdots q_N),
\end{equation}
where
\begin{equation}
\phi(q_1'\cdots q_N') = \langle \phi|q_1'\cdots q_N'\rangle.
\end{equation}
The probability of an observation of the system simultaneously finding the first coordinate in
the range $q_1'$ to $q_1'+dq_1'$, the second coordinate in the range $q_2'$ to $q_2'
+dq_2'$, {\rm etc.}, is
\begin{equation}
P(q_1'\cdots q_N'; dq_1'\cdots dq_N') = |\psi(q_1'\cdots q_N')|^{\,2}\,dq_1'\cdots 
dq_N'.
\end{equation}
Finally, the normalization condition for a physical wavefunction is
\begin{equation}
\int_{-\infty}^{+\infty} \cdots \int_{-\infty}^{+\infty}dq_1'\cdots dq_N' \,
|\psi(q_1'\cdots q_N')|^{\,2}= 1.
\end{equation}

The $N$ linear operators $\partial/\partial q_i$ (where $i$ runs from 1 to $N$)
are defined
\begin{equation}\label{e3.57}
\frac{\partial}{\partial q_i} \,\psi\rangle =\frac{\partial\psi}{\partial q_i}
\rangle.
\end{equation}
These linear operators can also act on bras (provided the associated wavefunctions
are square integrable) in accordance with [see Equation~(\ref{e3.31})]
\begin{equation}
\langle \phi\, \frac{\partial }{\partial q_i} = -\langle \frac{\partial \phi}
{\partial q_i}.
\end{equation}
Corresponding to Equation~(\ref{e3.35}), we can derive the commutation relations
\begin{equation}
\frac{\partial}{\partial q_i}\, q_j - q_j \,\frac{\partial}{\partial q_i} = \delta_{ij}.
\end{equation}
It is also clear that
\begin{equation}
\frac{\partial}{\partial q_i}\,\frac{\partial}{\partial q_j}\, \psi\rangle
= \frac{\partial^2\psi}{\partial q_i \partial q_j} \rangle = \frac{\partial }{\partial
q_j} \,\frac{\partial }{\partial q_i}\, \psi\rangle,
\end{equation}
showing that
\begin{equation}
\frac{\partial}{\partial q_i}\,\frac{\partial}{\partial q_j} = 
\frac{\partial }{\partial q_j}\,
\frac{\partial }{\partial q_i}.
\end{equation}

It can be seen, by comparison with Equations~(\ref{e3.14a})--(\ref{e3.14c}), that the linear operators
$-{\rm i}\,\hbar\, \partial/\partial q_i$ satisfy the same commutation relations
with the $q$'s
and with each other that the $p$'s do. The most general conclusion
we can draw from this coincidence of commutation relations is (see Dirac)
\begin{equation}
p_i = -{\rm i} \,\hbar \frac{\partial}{\partial q_i} +
\frac{\partial  F(q_1\cdots q_N) }{\partial q_i}.
\end{equation}
However, the function $F$
 can be transformed away via  a suitable readjustment of the phases
of the basis eigenkets (see Section~\ref{s3.4}, and Dirac). Thus, we can always construct
a set of simultaneous eigenkets of $q_1\cdots q_N$ for  which
\begin{equation}\label{e3.63}
p_i = -{\rm i}\,\hbar \frac{\partial}{\partial q_i}.
\end{equation}
This is the generalized Schr\"{o}dinger representation. 

It follows from Equations~(\ref{e3.50}), (\ref{e3.57}), and (\ref{e3.63}) that
\begin{equation}
p_i \rangle = 0.
\end{equation}
Thus, the standard ket in the Schr\"{o}dinger representation is a simultaneous eigenket
of all the momentum operators belonging to the eigenvalue zero. Note that
\begin{equation}
\langle q_1'\cdots q_N'|\,\frac{\partial}{\partial q_i}\, \psi\rangle = 
\langle q_1'\cdots q_N'| \,\frac{\partial \psi}{\partial q_i}\rangle 
=\frac{\partial \psi(q_1'\cdots q_N')}{\partial q_i'} = \frac{\partial }
{\partial q_i'}\,\langle q_1'\cdots q_N'|\psi\rangle.
\end{equation}
Hence,
\begin{equation}
\langle q_1'\cdots q_N'|\, \frac{\partial}{\partial q_i} = 
\frac{\partial}{\partial q_i'}\, \langle q_1'\cdots q_N'|,
\end{equation}
so that
\begin{equation}\label{e3.67}
\langle q_1'\cdots q_N'|\, p_i = -{\rm i}\,\hbar\,\frac{\partial}{\partial q_i'}\,
\langle q_1'\cdots q_N'|.
\end{equation}
The dual of the above equation gives
\begin{equation}
p_i\,|q_1'\cdots q_N'\rangle = {\rm i}\,\hbar\,\frac{\partial}{\partial q_i'}\, |q_1'\cdots
q_N'\rangle.
\end{equation}

\section{Momentum Representation}
Consider a system with one degree of freedom, describable in terms of a coordinate
$x$ and its conjugate momentum $p_x$, both of which have a continuous range of
eigenvalues. We have seen that it is possible to represent the system in terms
of the eigenkets of $x$. 
This is termed the Schr\"{o}dinger representation.
However, it is also possible to represent the system in 
terms of the eigenkets of $p_x$. 

Consider the eigenkets of $p_x$ which belong to the eigenvalues $p_x'$. These are
denoted $|p_x'\rangle$. The orthogonality relation for the momentum eigenkets is
\begin{equation}\label{e3.69}
\langle p_x'|p_x''\rangle = \delta(p_x'-p_x''),
\end{equation}
and the corresponding completeness relation is
\begin{equation}\label{e3.70}
\int_{-\infty}^{+\infty} dp_x'\, |p_x'\rangle \langle p_x'| = 1.
\end{equation}
A general state ket can be written
\begin{equation}
\phi(p)\rangle
\end{equation}
where the standard ket $\rangle $  satisfies 
\begin{equation}
\langle p_x'| \rangle = 1.
\end{equation}
Note that the standard ket in this representation is quite
different to that  in the Schr\"{o}dinger representation. 
The momentum space wavefunction $\phi(p_x')$ satisfies
\begin{equation}
\phi(p_x') = \langle p_x'|\phi \rangle.
\end{equation}
The probability that a measurement of the momentum yields a result
lying  in the range $p_x'$ to
$p_x'+dp_x'$ is given by
\begin{equation}
P(p_x', dp_x') = |\phi(p_x')|^{\,2}\,dp_x'.
\end{equation}
Finally, the normalization condition for a physical momentum space wavefunction is
\begin{equation}
\int_{-\infty}^{+\infty}dp_x'\, |\phi(p_x')|^{\,2} = 1.
\end{equation}

The fundamental commutation relations (\ref{e3.14a})--(\ref{e3.14c}) exhibit a 
particular symmetry between coordinates and their conjugate momenta. If all the
coordinates are transformed into their conjugate momenta, and {\rm vice versa}, and
$\rm i$ is  then replaced by $-\rm i$,  then the commutation relations are
unchanged. It follows from this symmetry that we can always choose the eigenkets
of $p_x$ in such a manner that  the coordinate 
$x$ can be represented as (see Section~\ref{s3.4})
\begin{equation}
x = {\rm i}\,\hbar\, \frac{d}{dp_x}.
\end{equation}
This is termed the momentum representation. 

The above result is easily generalized to a system with more than one degree of
freedom. Suppose the system is specified by $N$ coordinates, $q_1\cdots q_N$, and
$N$ conjugate momenta, $p_1\cdots p_N$. Then, in the momentum representation, the
coordinates can be written as 
\begin{equation}
q_i = {\rm i}\,\hbar\, \frac{\partial}{\partial p_i}.
\end{equation}
We also have
\begin{equation}
q_i \rangle = 0,
\end{equation}
and
\begin{equation}
\langle p_1' \cdots p_N'|\, q_i = {\rm i}\,\hbar \,\frac{\partial}{\partial p_i'}\,
\langle p_1'\cdots p_N'|.
\end{equation}

The momentum representation is less useful than the Schr\"{o}dinger representation
for  a very simple reason. The energy operator ({\rm i.e.}, the Hamiltonian) of
most simple systems takes the form of a sum of quadratic terms in the momenta
({\rm i.e.}, the kinetic energy) plus a complicated function of the coordinates
({\rm i.e.}, the potential energy). In the Schr\"{o}dinger  representation,
the eigenvalue problem for the energy translates into a second-order differential
equation in the coordinates, with a complicated potential function. In the
momentum representation, the problem transforms into a high-order differential
equation in the momenta, with a quadratic potential. With the mathematical
tools at our disposal, we are far better able to solve the former type of problem
than the latter. Hence, the Schr\"{o}dinger  representation is
generally more useful than the momentum representation. 

\section{Uncertainty Relation}
How is a momentum space wavefunction related to the corresponding coordinate
space wavefunction? To answer this question, let us consider the
representative $\langle x'|p_x'\rangle$ of the
 momentum eigenkets $|p'\rangle$ in the Schr\"{o}dinger  representation
for a system with a single degree of freedom. This representative satisfies
\begin{equation}
p_x'\, \langle x'|p_x'\rangle = \langle x'|\,p_x\,|p_x'\rangle = -{\rm i}\,\hbar\,\frac{d}{dx'}\,
\langle x'|p_x'\rangle,
\end{equation}
where use has been made of Equation~(\ref{e3.67}) (for the case of a system with one
degree of freedom). The solution of the above  differential equation is
\begin{equation}
\langle x' | p_x'\rangle = c' \exp(\,{\rm i}\, p_x'\, x'/\hbar),
\end{equation}
where $c' = c'(p_x')$. It is easily demonstrated that
\begin{equation}
\langle p_x'|p_x''\rangle = \int_{-\infty}^{+\infty} dx'\,\langle p_x'|x'\rangle \,
\langle x'|p_x''\rangle = c'^{\,\ast}\, c'' \int_{-\infty}^{\infty}dx'\,\exp[-{\rm i}\,
(p_x'-p_x'')\,x'/\hbar].
\end{equation}
The well-known mathematical result
\begin{equation}\label{e3.83}
\int_{-\infty}^{+\infty}dx \,\exp(\,{\rm i}\, a\,x) = 2\pi\,\delta (a),
\end{equation}
yields
\begin{equation}
\langle p_x'|p_x''\rangle = |c'|^{\,2} \, h\, \delta(p_x'-p_x'').
\end{equation}
This is consistent with Equation~(\ref{e3.69}), provided that $c' = h^{-1/2}$. Thus,
\begin{equation}\label{e3.85}
\langle x'|p_x'\rangle = h^{-1/2}\, \exp(\,{\rm i}\, p_x'\, x'/\hbar).
\end{equation}

Consider a general state ket $|A\rangle$ whose coordinate wavefunction is $\psi(x')$,
and whose momentum wavefunction is ${\mit\Psi}(p_x')$. In other words,
\begin{align}
\psi(x') &= \langle x'|A\rangle, \\[0.5ex]
{\mit\Psi}(p_x') &= \langle p_x'|A\rangle.
\end{align}
It is easily demonstrated that
\begin{equation}
\psi(x') = \int_{-\infty}^{+\infty} dp_x'\, \langle x'|p_x'\rangle \langle p_x'|A\rangle
= \frac{1}{h^{1/2}} \int_{-\infty}^{+\infty} dp_x'\,{\mit\Psi}(p_x') \,
\exp(\,{\rm i} \,p_x'\,x'/\hbar)\label{e3.87}
\end{equation}
and 
\begin{equation}
{\mit\Psi}(p_x') = \int_{-\infty}^{+\infty} dx' \,\langle p_x'|x'\rangle \langle x'|A\rangle
=\frac{1}{h^{1/2}} \int_{-\infty}^{+\infty} dx'\,\psi(x')\, \exp(-{\rm i}\, p_x' x'/\hbar),
\end{equation}
where use has been made of Equations~(\ref{e3.16}), (\ref{e3.70}), (\ref{e3.83}), and (\ref{e3.85}). 
Clearly, the momentum
space wavefunction is the {\em Fourier transform}\/
 of the coordinate space wavefunction. 

Consider a state whose coordinate space wavefunction is a {\em wavepacket}. 
In other words, the wavefunction only has non-negligible amplitude in some
spatially localized region of extent ${\mit\Delta} x$. As is well-known, the Fourier
transform of a wavepacket fills up a wavenumber band of approximate extent
${\mit\Delta} k \sim 1/{\mit\Delta} x$. Note that in Equation~(\ref{e3.87}) the role of the wavenumber
$k$ is played by the quantity $p_x'/\hbar$. It follows that the momentum space
wavefunction corresponding to a wavepacket in coordinate space extends over
a range of momenta ${\mit\Delta} p_x \sim \hbar /{\mit\Delta} x$. Clearly, a measurement 
of $x$ is almost certain to give a result lying in a
range of width ${\mit\Delta} x$. Likewise, measurement of $p_x$ is almost certain to
yield a result lying in a range of width ${\mit\Delta} p_x$. The product of these two
uncertainties is 
\begin{equation}
{\mit\Delta} x \,{\mit\Delta} p_x \sim \hbar.
\end{equation}
This result is called the {\em Heisenberg uncertainty principle}. 

Actually, it is possible to write the Heisenberg uncertainty principle
 more exactly by making use of
Equation~(\ref{e2.75}) and the commutation relation (\ref{e3.36}). We obtain
\begin{equation}
\langle ({\mit\Delta} x)^2\rangle \,\langle ({\mit\Delta} p_x)^2\rangle \geq \frac{\hbar^2}{4}
\end{equation}
for any general state. It is easily demonstrated that the minimum uncertainty states,
for which the equality sign holds in the above relation, correspond to Gaussian
wavepackets in both coordinate and momentum space. 

\section{Displacement Operators}\label{s3.8}
Consider a system with one degree of freedom corresponding to the Cartesian
coordinate $x$. Suppose that we displace this system some distance along the $x$-axis.
We could imagine that the system is on wheels, and we just give it a little
push. The final state of the system is completely determined by its initial state,
together with the direction and magnitude of the displacement.
Note that the type of displacement we are considering is one in which
{\em everything}\/ to do with the system is displaced. So, if the system is
subject to an external potential then the potential must be  displaced.

The situation is not so clear with state kets. The final state
of the system only determines the {\em direction}\/ of the displaced state ket. Even if
we adopt the  convention that all state kets have unit norms, the final ket is
still not completely determined, because it can be multiplied by a constant phase-factor. However, we know that the superposition relations between states 
remain invariant under the displacement. This follows because the superposition
relations have a physical significance that is unaffected by a displacement of
the system. 
Thus, if 
\begin{equation}\label{e3.91}
|R\rangle = |A\rangle + |B\rangle
\end{equation}
in the undisplaced system, and the displacement causes ket $|R\rangle$ to
transform to ket $|Rd\rangle$, {\rm etc}., then in the displaced system we have
\begin{equation}\label{e3.92}
|Rd\rangle = |Ad\rangle + |Bd\rangle.
\end{equation}
Incidentally, this determines the displaced kets to within a single arbitrary phase-factor to be multiplied into all of them. The displaced kets cannot be multiplied by 
individual phase-factors, because this would wreck the superposition relations. 

Since Equation~(\ref{e3.92}) holds in the displaced system whenever Equation~(\ref{e3.91}) holds in the
undisplaced system, it follows that the displaced ket $|Rd\rangle$ must be the
result of some linear operator acting on the undisplaced ket
$|R\rangle$. In other
words,
\begin{equation}
|R d\rangle = D \,|R\rangle,
\end{equation}
where $D$ an  operator that depends only on the
nature of the displacement. The arbitrary phase-factor by which all
displaced kets may be multiplied results in $D$ being undetermined to an arbitrary
multiplicative constant of modulus unity. 

We now adopt the ansatz that any combination of bras, kets, and dynamical
variables that possesses a physical significance is invariant under a displacement
of the system. The normalization condition 
\begin{equation}
\langle A|A\rangle = 1
\end{equation}
for a state ket $|A\rangle$ certainly has a physical significance. Thus, we must
have 
\begin{equation}
\langle Ad|Ad\rangle = 1.
\end{equation}
Now, $|Ad\rangle = D\,|A\rangle$ and $\langle Ad| = \langle A|\,D^{\dag}$, so
\begin{equation}
\langle A|\, D^{\dag} \,D\,|A\rangle = 1.
\end{equation}
Because this must hold for any state ket $|A\rangle$, it follows that
\begin{equation}\label{e3.97}
D^{\dag} \,D = 1.
\end{equation}
Hence, the displacement operator is {\em unitary}. 
Note that the above relation implies that
\begin{equation}
|A\rangle = D^{\dag}\, |A d\rangle.
\end{equation}

The  equation
\begin{equation}
v \,|A\rangle = |B\rangle,
\end{equation}
where the operator
$v$ represents a dynamical variable, has some physical significance. Thus,
we require that
\begin{equation}
v_d\, |Ad\rangle = |Bd\rangle,
\end{equation}
where $v_d$ is the displaced operator. It follows that
\begin{equation}
v_d\, |Ad\rangle = D\, |B\rangle = D \,v \,|A\rangle = D\, v\, D^{\dag}\, |Ad\rangle.
\end{equation}
Since this is true for any ket $|Ad\rangle$, we have
\begin{equation}\label{e3.102}
v_d = D\, v\, D^{\dag}.
\end{equation}
Note that the arbitrary numerical factor in $D$ does not affect either of the
results (\ref{e3.97}) and (\ref{e3.102}).

Suppose, now, that the system is displaced an {\em infinitesimal} distance $\delta x$
along the $x$-axis. We expect that the displaced ket $|A d\rangle$ should
approach the undisplaced ket 
$|A\rangle$ in the limit as $\delta x\rightarrow 0$. Thus,
we expect the limit
\begin{equation}
\lim_{\delta x\rightarrow 0 } \frac{|A d\rangle - |A\rangle}{\delta x}
= \lim_{\delta x\rightarrow 0 }\frac{D-1}{\delta x}\,|A\rangle
\end{equation}
to exist. Let
\begin{equation}
d_x =  \lim_{\delta x\rightarrow 0 }\frac{D-1}{\delta x},
\end{equation}
where $d_x$ is denoted the {\em displacement operator}\/ along the $x$-axis. The fact 
that $D$ can be replaced by $D \,\exp(\,{\rm i}\,\gamma)$, where $\gamma$ is a real
phase-angle, implies that $d_x$ can be replaced by
\begin{equation}
 \lim_{\delta x\rightarrow 0 }\frac{D\,\exp({\rm i}\,\gamma)-1}{\delta x}=
 \lim_{\delta x\rightarrow 0 }\frac{D-1+{\rm i}\, \gamma}{\delta x}= d_x + {\rm i}\,a_x,
\end{equation}
where $a_x$ is the limit of $\gamma/\delta x$. We have assumed, as seems
reasonable,  that $\gamma$ tends
to zero as $\delta x\rightarrow 0$. It is clear that the displacement operator
is undetermined to an arbitrary imaginary additive constant. 

For small $\delta x$, we have
\begin{equation}\label{e3.106}
D = 1 + \delta x\,d_x.
\end{equation}
It follows from Equation~(\ref{e3.97}) that
\begin{equation}
(1+ \delta x\,d_x^{\,\dag}) \,(1+  \delta x\,d_x) = 1.
\end{equation}
Neglecting order $(\delta x)^{\,2}$, we obtain
\begin{equation}
d_x^{~\dag} + d_x = 0.
\end{equation}
Thus, the displacement operator is {\em anti-Hermitian}. Substituting into
Equation~(\ref{e3.102}), and again neglecting order $(\delta x)^2$, we find that
\begin{equation}
v_d = (1+ \delta x\,d_x)\, v\, (1- \delta x\,d_x) = v + \delta x\,( d_x\, v - v\, d_x),
\end{equation}
which implies
\begin{equation}\label{e3.110}
\lim_{\delta x\rightarrow 0} \frac{v_d -v}{\delta x} = d_x \,v -v\, d_x.
\end{equation}

Let us consider a specific example. 
Suppose that a state has a wavefunction $\psi(x')$. If the system is displaced
a distance $\delta x$ along the $x$-axis then the new wavefunction is
$\psi(x'-\delta x)$ ({\rm i.e.}, the same shape shifted in the $x$-direction
by a distance $\delta x$). Actually, the new wavefunction can be multiplied by
an arbitrary number of modulus unity. It can be seen that the new wavefunction  
is obtained from the old wavefunction according to the 
prescription  $x'\rightarrow x'- \delta x$. Thus,
\begin{equation}
x_d = x -\delta x.
\end{equation}
A comparison  with Equation~(\ref{e3.110}), using  $x=v$, yields
\begin{equation}
d_x \,x - x\,d_x = -1.
\end{equation}
It follows that ${\rm i}\,\hbar\, d_x$ obeys the same commutation relation with 
$x$ that $p_x$, the momentum conjugate to $x$, does [see Equation~(\ref{e3.14c})]. 
The most general conclusion we can draw from this observation is that
\begin{equation}
p_x = {\rm i}\,\hbar\, d_x + f(x),
\end{equation}
where $f$ is Hermitian (since $p_x$ is Hermitian). However, the fact that $d_x$
is undetermined to an arbitrary additive imaginary constant (which could be a function
of $x$) enables us to transform the function $f$ out of the above equation, leaving
\begin{equation}\label{e3.114}
p_x = {\rm i}\,\hbar\, d_x.
\end{equation}
Thus, the displacement operator in the $x$-direction is proportional to the
momentum conjugate to $x$. We say that $p_x$ is the {\em generator}\/ of translations
along the $x$-axis. 

A finite translation along the $x$-axis can be constructed from 
a series of very many
infinitesimal translations. Thus, the operator $D({\mit\Delta} x)$ which translates the
system a distance ${\mit\Delta} x$ along the $x$-axis is
written
\begin{equation}
D({\mit\Delta} x) =
\lim_{N\rightarrow \infty}
 \left(1-{\rm i}\, \frac{{\mit\Delta} x}{N} \frac{p_x}{\hbar}\right)^N,
\end{equation}
where use has been made of Equations~(\ref{e3.106}) and (\ref{e3.114}). It follows that
\begin{equation}\label{e3.116}
D({\mit\Delta} x) = \exp\left({-\rm i} \,p_x\,{\mit\Delta} x /\hbar\right).
\end{equation}
The unitary nature of the operator is now clearly apparent.

We can also construct displacement operators which translate the system along
the $y$- and $z$-axes. Note that a displacement a distance ${\mit\Delta} x$ 
along the
$x$-axis {\em commutes} with a displacement a distance ${\mit\Delta} y$ along the
 $y$-axis. 
In other words, if the system is moved ${\mit\Delta} x$ along the $x$-axis, and then
${\mit\Delta} y$ along the $y$-axis, then it ends up in the same state as if it were moved
${\mit\Delta} y$ along the $y$-axis, and then ${\mit\Delta} x$ along the $x$-axis. The fact that
translations in independent directions commute is clearly associated with the
fact that the conjugate momentum operators
 associated with these directions also commute
[see Equations~(\ref{e3.14b}) and (\ref{e3.116})].

\subsection*{Exercises}
\begin{enumerate}[label=\thechapter.\arabic*,leftmargin=*,widest=9.20]
\item Demonstrate that
\begin{align}
[q_i, q_j]_{cl} &= 0,\nonumber\\[0.5ex]
[p_i, p_j]_{cl} &= 0,\nonumber\\[0.5ex]
[q_i, p_j]_{cl} &= \delta_{ij},\nonumber
\end{align}
where $[\cdots,\cdots]_{cl}$ represents a classical Poisson bracket. Here, the $q_i$ and $p_i$ are
the coordinates and corresponding canonical momenta of a classical, many degree of freedom, dynamical system.
\item Verify that 
\begin{align}
[u, v] &= - [v, u],\nonumber\\[0.5ex]
[u, c] &= 0,\nonumber\\[0.5ex]
[u_1+ u_2, v] &= [u_1, v] + [u_2, v],\nonumber\\[0.5ex]
[u, v_1 + v_2]&= [u, v_1] + [u, v_2],\nonumber\\[0.5ex]
[u_1\, u_2, v] &= [u_1, v] \,u_2 + u_1\, [u_2, v],\nonumber\\[0.5ex]
[u, v_1 \,v_2] &= [u, v_1] \,v_2 + v_1 \,[u, v_2],\nonumber
[u, [v, w] ]+ [v, [w, u] ] + [w, [u, v]] = 0,\nonumber
\end{align}
where $[\cdots,\cdots]$ represents either a classical or a quantum mechanical Poisson bracket. 
Here, $u$, $u$, $w$, etc., represent dynamical variables (i.e., functions of the coordinates and
canonical momenta of a dynamical system), and $c$ represents a number.

\item Consider a Gaussian wavepacket whose corresponding wavefunction is 
$$
\psi(x') =\psi_0\,\exp\left[-\frac{(x'-x_0)^{\,2}}{4\,\sigma^2}\right],
$$
where $\psi_0$, $x_0$, and $\sigma$ are constants. Demonstrate that
\begin{enumerate}
\item
$$\langle x\rangle = x_0,$$
\item
$$\langle ({\mit\Delta x})^2\rangle = \sigma^2,$$
\item
$$ \langle p\rangle = 0,$$
\item 
$$
\langle ({\mit\Delta} p)^2\rangle = \frac{\hbar^2}{4\,\sigma^2}.
$$
\end{enumerate}
 Here, $x$ and $p$ are a position operator and its conjugate momentum operator, respectively.
 
 \item Suppose that we displace a one-dimensional quantum mechanical system a distance $a$ along the $x$-axis. The
 corresponding displacement operator is\label{ex2.4}
 $$
 D(a) = \exp\left(-{\rm i}\,p_x\,a/\hbar\right),
 $$
 where $p_x$ is the momentum conjugate to the position operator $x$.  Demonstrate that
 $$
 D(a)\,x\,D(a)^{\,\dag} = x - a.
 $$
 [Hint: Use the momentum
 representation, $x = {\rm i}\,\hbar\,d/dp_x$.] Similarly, demonstrate that
 $$
 D(a)\,x^m\,D(a)^{\,\dag} = (x-a)^m.
 $$
 Hence, deduce that 
 $$
 D(a)\,V(x)\,D(a)^{\,\dag} = V(x-a),
 $$
 where $V(x)$ is a general function of $x$. 
 
 Let $k=p_x/\hbar$, and let $|k'\rangle$ denote an eigenket of the $k$ operator belonging to the eigenvalue $k'$. 
 Demonstrate that
 $$
 |A\rangle = \sum_{n=-\infty,\infty}c_n\,|k'+n\,k_a\rangle,
 $$
 where the $c_n$ are arbitrary complex coefficients, and $k_a=2\pi/a$, is an eigenket of the $D(a)$ operator
 belonging to the eigenvalue $\exp(-{\rm i}\,k'\,a)$. Show that the corresponding wavefunction can
 be written
 $$
 \psi_A(x') = {\rm e}^{\,{\rm i}\,k'\,x'}\,u(x'),
 $$
 where $u(x'+a)=u(x')$ for all $x'$. 
\end{enumerate}
 