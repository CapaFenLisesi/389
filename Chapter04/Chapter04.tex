\chapter{Quantum Dynamics}\label{s4}
\section{Schr\"{o}dinger's Equations of Motion}\label{s4.1}
Up to now, we have only considered systems at one particular instant of time. 
Let us now investigate how quantum mechanical systems evolve with time. 

Consider a system in a state $A$ which evolves in time. At time
$t$ the state of the system is represented by the ket $|At\rangle$. The label
$A$ is needed to distinguish the ket from any other ket ($|Bt\rangle$, say)
which is evolving in time. The label $t$ is needed to distinguish the different
states of the system at different times. 

The final state of the system at time $t$ is completely determined by its
initial state at time $t_0$ plus the time interval $t-t_0$ (assuming that 
the system is left undisturbed during this time interval). However, the
final state only determines the {\em direction} of the final state ket. 
Even if we adopt the convention that all state kets have unit norms,
the final ket is still not completely determined, since it can be multiplied
by an arbitrary
 phase-factor. However, we expect that if a superposition relation 
holds for certain states at time $t_0$ then the same relation
should hold between the corresponding time-evolved states at time $t$, assuming
that the system is left undisturbed between times $t_0$ and $t$.
In other words,
if
\begin{equation}\label{e4.1}
|Rt_0\rangle = |At_0\rangle + |B t_0\rangle
\end{equation}
for any three kets, then we should  have
\begin{equation}\label{e4.2}
|Rt\rangle = |At\rangle + |B t\rangle.
\end{equation}
This rule determines the time-evolved kets to within a single arbitrary phase-factor to be multiplied into all of them. The evolved kets cannot be multiplied
by individual phase-factors, since this would invalidate the superposition
relation at later times. 

According to Eqs.~(\ref{e4.1}) and (\ref{e4.2}), the final ket $|Rt\rangle$ depends linearly
on the initial ket $|Rt_0\rangle$. Thus, the final ket can be regarded as the
result of some linear operator acting on the initial ket: {\em, i.e.},
\begin{equation}\label{e4.3}
|Rt\rangle = T |Rt_0\rangle,
\end{equation}
where $T$ is a linear operator which depends only on the times $t$
and $t_0$. 
The arbitrary phase-factor by which all time evolved kets may be multiplied
results in $T(t, t_0)$ being undetermined to an arbitrary multiplicative constant
of modulus unity. 

Since we have adopted a convention in which the norm of any state ket is unity,
it make sense to define the time evolution operator $T$ in such a manner that
it preserves the length of any ket upon which it acts
({\em i.e.}, if a ket is properly normalized at time $t$ then it will remain normalized at
all subsequent times $t>t_0$).
This is always possible,
since the length of a ket possesses no physical significance. Thus,
we require that
\begin{equation} 
\langle A t_0|A t_0\rangle =\langle A t|A t\rangle
\end{equation}
for any ket $A$, 
which immediately yields 
\begin{equation}\label{e4.5}
T^{\dag}\,T = 1.
\end{equation}
Hence, the time evolution operator $T$ is a {\em unitary} operator. 

Up to now, the time evolution operator $T$ looks very much like the 
spatial displacement
operator $D$ introduced in the previous section. However, there are some
important differences between time evolution and spatial displacement. In general,
we {\em do} expect the expectation value of some observable $\xi$ to
evolve with time, even if the system is left in a state of undisturbed motion
(after all, time evolution has no meaning unless something {\em observable}
changes with time). The triple product $\langle A|\xi|A\rangle$ can evolve
either because the ket $|A\rangle$ evolves and the operator $\xi$ stays constant,
the ket $|A\rangle$ stays constant and the operator $\xi$ evolves, or both
the ket $|A\rangle$ and the operator $\xi$ evolve. 
Since we are already committed to evolving state kets, according to Eq.~(\ref{e4.3}),
let us assume that the time evolution operator $T$ can be chosen in such a
manner that the operators representing the dynamical variables of the
system {\em do not}
 evolve in time (unless they contain some specific time dependence). 

We expect, from physical continuity, that as $t\rightarrow t_0$ then
$|At\rangle\rightarrow |A t_0\rangle$ for any ket $A$. Thus, the
limit
\begin{equation}\label{e4.6}
\lim_{t\rightarrow t_0} \frac{|At\rangle - |At_0\rangle}{t-t_0} =
\lim_{t\rightarrow t_0}\frac{T-1}{t-t_0}|At_0\rangle
\end{equation}
should exist. Note that this limit is simply the derivative of
$|A t_0\rangle$ with respect to $t_0$. Let 
\begin{equation}\label{e4.7}
\tau(t_0) = \lim_{t\rightarrow t_0}\frac{T(t, t_0)-1}{t-t_0}.
\end{equation}
It is easily demonstrated from Eq.~(\ref{e4.5}) that $\tau$ is anti-Hermitian:
{\em i.e.},
\begin{equation}
\tau^{\dag} + \tau = 0.
\end{equation}
The fact that $T$ can be replaced by $T\exp({\rm i}\,\gamma)$ (where $\gamma$ is
real) implies that $\tau$ is undetermined to an arbitrary {\em imaginary} additive
constant (see previous section). Let us
define the Hermitian operator $H(t_0)= {\rm i}\,\hbar \,\tau$. This operator is
undetermined to an arbitrary {\em  real} additive constant. It follows from Eqs.~(\ref{e4.6})
and (\ref{e4.7}) that
\begin{equation}
{\rm i}\,\hbar \,\frac{d|At_0\rangle}{dt_0} = {\rm i}\,\hbar\lim_{t\rightarrow t_0}
\frac{ |At\rangle - |At_0\rangle}{t-t_0} = {\rm i}\,\hbar\,\tau(t_0)|At_0\rangle
= H(t_0) |At_0\rangle.
\end{equation}
When written for general $t$ this equation becomes
\begin{equation}\label{e4.10}
{\rm i}\,\hbar\, \frac{d|At\rangle}{dt} = H(t)|At\rangle.
\end{equation}

Equation~(\ref{e4.10}) gives the general law for the time evolution  of a state
ket in a scheme in which the operators representing the dynamical variables remain
fixed. This equation is denoted {\em Schr\"{o}dinger's equation of motion}. 
It involves a Hermitian operator $H(t)$ which is, presumably, a characteristic
of the dynamical system under investigation. 

We saw, in the previous section, that
if the operator $D(x, x_0)$ displaces the system along the $x$-axis from $x_0$ to $x$
then
\begin{equation}
p_x = {\rm i}\,\hbar\,\lim_{x\rightarrow x_0} \frac{D(x,x_0)-1}{x-x_0},
\end{equation}
where $p_x$ is the operator representing the momentum conjugate to $x$. We now
have that if the operator $T(t, t_0)$ evolves the system in time from $t_0$ to $t$
then
\begin{equation}
H(t_0) = {\rm i}\,\hbar\,\lim_{t\rightarrow t_0} \frac{T(t, t_0)-1}{t-t_0}.
\end{equation}
Thus, the 
dynamical variable corresponding to
the operator $H$ stands  to time $t$ as the momentum $p_x$ stands  to the
coordinate $x$. By analogy
with classical physics, this suggests that $H(t)$ is the operator representing
the total {\em energy} of the system. (Recall that, in classical physics,
if the equations of motion of a system are invariant under an $x$-displacement
of the system then this implies that the system conserves momentum in the
$x$-direction. Likewise, if the equations of motion are invariant under
a temporal displacement then this implies that the system conserves energy.)
The operator $H(t)$ is usually called the {\em Hamiltonian} of the system. 
The fact that the Hamiltonian is undetermined to an arbitrary real additive
constant is related to the well-known phenomenon that  energy is
undetermined to an arbitrary additive constant in physics ({\em i.e.}, the zero
of potential energy is not well-defined). 

Substituting $|At\rangle = T |At_0\rangle$ into Eq.~(\ref{e4.10}) yields
\begin{equation}
{\rm i}\,\hbar \,\frac{d T}{dt}|At_0\rangle = H(t)\,T|At_0\rangle.
\end{equation}
Since this must hold for any initial state $|At_0\rangle$ we conclude that
\begin{equation}\label{e4.14}
{\rm i} \,\hbar\, \frac{dT}{dt} = H(t) \,T.
\end{equation}
This  equation can be integrated to give
\begin{equation}\label{e4.15}
T(t, t_0) = \exp\left(-{\rm i} \,\int_{t_0}^t H(t') \,dt'/\hbar\right),
\end{equation}
where use has been made of Eqs.~(\ref{e4.5}) and (\ref{e4.6}). 
(Here, we assume that Hamiltonian operators
evaluated at  different times commute with one another). It is now clear  how
the fact that $H$ is undetermined to an arbitrary real additive constant leaves
$T$ undetermined to a phase-factor. Note that, in the above
analysis,  time is {\em not} an
operator (we cannot observe time, as such), it is just a parameter (or, more 
accurately, a continuous label). Since we are only dealing with non-relativistic
quantum mechanics, the fact that position is an operator, but time is only a
label, need not worry us unduly. In relativistic quantum mechanics, time and space
coordinates are treated on the same footing by relegating position from being
an operator to being just a label. 

\section{Heisenberg's Equations of Motion}\label{s4.2}
We have seen that in Schr\"{o}dinger's scheme the dynamical variables of the
system remain fixed during a period of undisturbed motion, whereas the state kets
evolve according to Eq.~(\ref{e4.10}). However, this is not the only way in which
to represent the time evolution of the system. 

Suppose that a general state ket $A$ is subject to the transformation 
\begin{equation}\label{e4.16}
|A_t \rangle = T^{\dag}(t, t_0) |A\rangle.
\end{equation}
This is a time-dependent transformation,  since  the operator $T(t, t_0)$ obviously
depends on time. The subscript $t$ is used to remind us that the transformation
is time-dependent. 
 The time evolution of the transformed state ket is given by
\begin{equation}
|A_t t\rangle = T^{\dag} (t, t_0)|At\rangle = T^{\dag} (t, t_0)\,T (t, t_0)
|At_0\rangle = |A_t t_0\rangle,
\end{equation}
where use has been made of Eqs.~(\ref{e4.3}), (\ref{e4.5}), and the fact that $T(t_0, t_0)=1$. 
Clearly, the transformed state ket {\em does not} evolve in time. Thus, the
transformation (\ref{e4.16}) has the effect of bringing all kets representing
states of undisturbed motion of the system to rest. 

The transformation  must also be applied to bras. The dual of Eq.~(\ref{e4.16})
yields
\begin{equation}
\langle A_t| =\langle A| T.
\end{equation}
The transformation rule for a general observable $v$ is obtained from the requirement
that the expectation value $\langle A|v|A\rangle$ 
should remain invariant. It is easily
seen that
\begin{equation}\label{e4.19}
v_t = T^{\dag} \,v \,T.
\end{equation}
Thus, a dynamical variable, which corresponds to a fixed linear operator in
Schr\"{o}\-dinger's scheme, corresponds to a moving linear operator in this
new scheme. It is clear that the transformation (\ref{e4.16}) leads us to a scenario
in which the state  of the system is represented by
a
 fixed vector, and the dynamical variables
are represented by moving linear operators. This is termed the {\em Heisenberg
picture}, as opposed to the {\em Schr\"{o}dinger picture},
which is  outlined in Sect.~\ref{s4.1}.

Consider a dynamical variable $v$ corresponding to a fixed linear operator in
the Schr\"{o}dinger picture. According to Eq.~(\ref{e4.19}), we can write
\begin{equation}
T \,v_t = v\, T.
\end{equation}
Differentiation with respect to time yields
\begin{equation}
\frac{d T}{dt} \,v_t + T \,\frac{dv_t}{dt} = v\, \frac{dT}{dt}.
\end{equation}
With the help of Eq.~(\ref{e4.14}), this reduces to
\begin{equation}
H\,T\, v_t +{\rm i}\,\hbar\, T\,\frac{d v_t}{dt} = v\, H\,T,
\end{equation}
or
\begin{equation}\label{e4.23}
{\rm i}\,\hbar\ \frac{d v_t}{dt} = T^{\dag} \,v \,H\,T - T^{\dag} \,H\,T \,v_t
= v_t\, H_t - H_t\, v_t,
\end{equation}
where
\begin{equation}
H_t = T^{\dag}\, H\, T.
\end{equation}
Equation~(\ref{e4.23})
can be written
\begin{equation}\label{e4.25}
{\rm i}\,\hbar\ \frac{d v_t}{dt} = [v_t, H_t].
\end{equation}

Equation~(\ref{e4.25}) shows how the dynamical variables of the system evolve in the
Heisenberg  picture. It is denoted {\em Heisenberg's equation of motion}. 
Note that the time-varying dynamical variables in the Heisenberg picture
are usually called {\em Heisenberg dynamical variables} to distinguish them
from {\em Schr\"{o}dinger dynamical variables} ({\em i.e.}, the corresponding variables in
the Schr\"{o}dinger picture), which do not evolve in time. 

According to Eq.~(\ref{e3.13}), the Heisenberg equation of motion can be written
\begin{equation}\label{e4.26}
\frac{dv_t}{dt} = [v_t, H_t]_{\rm quantum},
\end{equation}
where $[\cdots]_{\rm quantum}$ denotes the quantum Poisson bracket. 
Let us compare this equation
with the classical time evolution
equation for a general dynamical variable $v$, which can be written
in the form [see Eq.~(\ref{e3.4})]
\begin{equation}\label{e4.27}
\frac{dv}{dt} = [v,H]_{\rm classical}.
\end{equation}
Here, $[\cdots]_{\rm classical}$ is the classical Poisson bracket, and $H$ denotes
the classical Hamiltonian. The strong resemblance between 
Eqs.~(\ref{e4.26}) and (\ref{e4.27}) 
provides us with further justification for our identification
of the linear operator $H$ with the energy of the system in quantum mechanics. 

Note that if the Hamiltonian does not explicitly depend on time ({\em i.e.}, the system is
not subject to some time-dependent external force) then Eq.~(\ref{e4.15}) yields
\begin{equation}\label{e4.28}
T(t, t_0) = \exp\left[-{\rm i}\, H\,(t-t_0)/\hbar \right].
\end{equation}
This operator manifestly commutes with $H$, so 
\begin{equation}
H_t = T^{\dag}\, H \,T = H.
\end{equation}
Furthermore, Eq.~(\ref{e4.25}) gives
\begin{equation}
{\rm i}\,\hbar \,\frac{dH}{dt} = [H, H] = 0.
\end{equation}
Thus, if the energy of the system  has no explicit time-dependence then it is 
represented by the same non-time-varying operator $H$ in both the Schr\"{o}dinger
and Heisenberg pictures. 

Suppose that $v$ is an observable which commutes with the Hamiltonian
(and, hence, with the time evolution operator $T$). It follows from Eq.~(\ref{e4.19})
that $v_t= v$. Heisenberg's equation of motion yields
\begin{equation}
{\rm i}\,\hbar \,\frac{d v}{dt} = [v, H] = 0.
\end{equation}
Thus, {\em any observable which commutes with the Hamiltonian is a constant
of the motion} (hence, it is represented by the same fixed operator in
both the Schr\"{o}dinger and Heisenberg pictures). Only those observables
which {\em do not} commute with the Hamiltonian evolve 
in time in the Heisenberg picture.

\section{Ehrenfest's Theorem}
We have now derived all of the basic elements of quantum mechanics. The only
thing which is lacking is some rule  to determine the form of the 
quantum mechanical Hamiltonian. For a physical system which possess a classical
analogue, we generally assume that the Hamiltonian has the same form as
in classical physics ({\em i.e.}, we replace the classical coordinates and conjugate
momenta by the corresponding quantum mechanical operators). This scheme guarantees
that quantum mechanics yields the correct classical equations of motion
in the classical limit. Whenever an ambiguity arises because of 
non-commuting
observables, this can usually be resolved by requiring the Hamiltonian $H$ to
be an Hermitian operator. For instance, we would write the
quantum mechanical analogue of the classical product $x\,p$, appearing in the
Hamiltonian, as the Hermitian product $(1/2)(x\,p + p\,x)$. When the system
in question has no classical analogue then we are reduced to guessing a form
for $H$ which reproduces the observed behaviour of the system. 

Consider a three-dimensional system characterized by three independent Cartesian
position coordinates $x_i$ (where $i$ runs from 1 to 3), with three corresponding
conjugate momenta $p_i$. These are represented by three commuting position
operators $x_i$, and three commuting momentum operators $p_i$, respectively. The 
commutation
relations  satisfied by the position and momentum operators
are [see Eq.~(\ref{e3.14c})]
\begin{equation}\label{e4.32}
[x_i, p_j] = {\rm i}\,\hbar\, \delta_{ij}.
\end{equation}
It is helpful to denote $(x_1, x_2, x_3)$ as ${\bf x}$ and $(p_1, p_2, p_3)$ as 
${\bf p}$. The following useful formulae,
\begin{eqnarray}\label{e4.33a}
[x_i, F({\bf p})]& =& {\rm i}\,\hbar \,\frac{\partial F}{\partial p_i},\\[0.5ex]
[p_i, G({\bf x})]& = &- {\rm i}\,\hbar\,\frac{\partial G}{\partial x_i},\label{e4.33b}
\end{eqnarray}
where $F$ and $G$ are functions which can be expanded as  power series, are
easily proved using  the fundamental commutation relations Eq.~(\ref{e4.32}).

Let us now consider the three-dimensional motion of a free particle of mass
$m$ in the
Heisenberg picture. The Hamiltonian is assumed to have the same form as in
classical physics:
\begin{equation}
H = \frac{{\bf p}^2}{2\,m} = \frac{1}{2\,m} \sum_{i=1}^3 p_i^{~2}.
\end{equation}
In the following, all dynamical variables are assumed to be Heisenberg dynamical
variables, although we will omit the subscript $t$ for the sake of clarity. 
The time evolution of the momentum operator $p_i$ follows from Heisenberg's
equation of motion (\ref{e4.25}). We find that
\begin{equation}
\frac{dp_i}{dt} = \frac{1}{{\rm i}\,\hbar}[p_i, H] = 0,
\end{equation}
since $p_i$ automatically commutes with any function of the momentum operators. 
Thus, for a free particle the momentum operators are constants of the motion,
which means that $p_i(t) = p_i(0)$ at all times $t$ (for $i$ is 1 to 3). 
The time evolution of the position operator $x_i$ is given by
\begin{equation}
\frac{d x_i}{dt} = \frac{1}{{\rm i}\,\hbar} [x_i, H] = \frac{1}{{\rm i}\,\hbar}
\frac{1}{2\,m}\, {\rm i}\,\hbar\,\frac{\partial}{\partial p_i}\!
\left(\sum_{j=1}^3 p_j^{~2}\right) = \frac{p_i}{m} = \frac{p_i(0)}{m},
\end{equation}
where use has been made of Eq.~(\ref{e4.33a}).  It follows that
\begin{equation}
x_i(t) = x_i(0) + \left[\frac{p_i(0)}{m}\right] t,
\end{equation}
which is analogous to the equation of motion of a classical free particle.
Note that even though
\begin{equation}
[x_i(0), x_j(0)] = 0,
\end{equation}
where the position operators are evaluated at equal times, the $x_i$ {\em do not}
commute when evaluated at different times. For instance,
\begin{equation}
[x_i(t), x_i(0)] = \left[ \frac{p_i(0)\,t}{m}, x_i(0)\right] = 
\frac{-{\rm i}\,\hbar \,t}
{m}.
\end{equation}
Combining the above commutation relation with the uncertainty relation (\ref{e2.75}) yields
\begin{equation}
\langle (\Delta x_i)^2\rangle_t \langle (\Delta x_i)^2\rangle_{t=0} \geq
\frac{\hbar^2\, t^2}{4\, m^2}.
\end{equation}
This result implies that even if a particle is well-localized at $t=0$, its
position becomes progressively more uncertain with time. This conclusion
can also be obtained by studying the propagation of wave-packets in
wave mechanics.

Let us now add a potential $V({\bf x})$ to our free particle Hamiltonian:
\begin{equation}
H = \frac{{\bf p}^2}{2\,m} + V({\bf x}).
\end{equation}
Here, $V$ is some function of the $x_i$ operators. Heisenberg's equation of
motion gives
\begin{equation}
\frac{d p_i}{dt} = \frac{1}{{\rm i}\,\hbar} [p_i, V({\bf x})] = - \frac{\partial
V({\bf x})}{\partial x_i},
\end{equation}
where use has been made of Eq.~(\ref{e4.33b}). On the other hand, the result
\begin{equation}
\frac{d x_i}{dt} = \frac{p_i}{m}
\end{equation}
still holds, because the $x_i$ all commute with the new term $V({\bf x})$ in the
Hamiltonian. We can use the Heisenberg equation of motion a second time
to deduce that
\begin{equation}
\frac{d^2 x_i}{dt^2} = \frac{1}{{\rm i}\,\hbar} \left[\frac{dx_i}{dt}, H\right]
= \frac{1}{{\rm i}\,\hbar}\left[ \frac{p_i}{m}, H \right]
=\frac{1}{m}\frac{d p_i}{dt} = - \frac{1}{m} \frac{\partial V({\bf x})}{\partial x_i}.
\end{equation}
In vectorial form, this equation becomes
\begin{equation}\label{e4.45}
m\,\frac{d^2 {\bf x}}{d t^2} = \frac{d{\bf p}}{dt} =
- \nabla V({\bf x}).
\end{equation}
This is the quantum mechanical equivalent of Newton's second law of motion.
Taking the expectation values of both sides with respect to a Heisenberg
state ket that does {\em not}\/ move with time, we obtain
\begin{equation}\label{e4.46}
m \,\frac{d^2\langle {\bf x}\rangle}{dt^2}= \frac{d\langle{\bf p}\rangle}{dt}
= - \langle \nabla V({\bf x})\rangle.
\end{equation}
This is known as {\em Ehrenfest's theorem}. When written in terms of expectation
values, this result is independent of whether we are using the Heisenberg or
Schr\"{o}dinger picture. In contrast, the operator equation (\ref{e4.45}) only holds
if ${\bf x}$ and ${\bf p}$ are understood to be Heisenberg dynamical variables.
Note that Eq.~(\ref{e4.46}) has no dependence on $\hbar$. 
In fact, it guarantees to  us that the centre of
a wave-packet always moves like a classical particle.

\section{Schr\"{o}dinger's Wave Equation}
Let us now consider the motion of a particle 
in three dimensions in the Schr\"{o}dinger picture. The fixed dynamical variables of
the system are the position operators ${\bf x}\equiv
(x_1, x_2, x_3)$, and the momentum operators ${\bf p}\equiv (p_1, p_2, p_3)$.
The state of the system is represented as some time evolving ket $|At\rangle$.

Let $|{\bf x'}\rangle$ represent a simultaneous eigenket of the position operators
belonging to the eigenvalues ${\bf x'} \equiv (x_1', x_2', x_3')$. Note that, since
the position operators  are {\em fixed} in the Schr\"{o}dinger picture, we do not
expect the $|{\bf x}'\rangle$ to evolve in time. The wave-function of the system
at time $t$ is defined
\begin{equation}
\psi({\bf x'}, t) = \langle {\bf x'}| At\rangle.
\end{equation}
The Hamiltonian of the system is taken to be
\begin{equation}\label{e4.48}
H = \frac{{\bf p}^2}{2\,m} + V({\bf x}).
\end{equation}

Schr\"{o}dinger's equation of motion (\ref{e4.10}) yields 
\begin{equation}\label{e4.49}
{\rm i}\,\hbar \,\frac{\partial \langle {\bf x'}|At\rangle}{\partial t}
= \langle {\bf x'}|H| At\rangle,
\end{equation}
where use has been made of the time independence of the $|{\bf x'}\rangle$. 
We adopt Schr\"{o}d\-inger's representation in which the momentum conjugate
to the position operator $x_i$ is written  [see Eq.~(\ref{e3.63})]
\begin{equation}\label{e4.50}
p_i = -{\rm i}\,\hbar\frac{\partial}{\partial x_i}.
\end{equation}
Thus,
\begin{equation}\label{e4.51}
\left\langle {\bf x}'\left|\frac{{\bf p}^2}{2\,m}\right| At\right\rangle = 
- \left(\frac{\hbar^2}{2\,m}\right)
\nabla'^2\langle {\bf x'} |At\rangle,
\end{equation}
where use has been made of Eq.~(\ref{e3.67}). Here, $\nabla'\equiv(\partial/\partial x',
\partial/\partial y', \partial/\partial z')$ denotes the gradient operator written
in terms of the position eigenvalues. We can also
write
\begin{equation}\label{e4.52}
\langle {\bf x'}|V({\bf x}) = V({\bf x'})\langle {\bf x'} |,
\end{equation}
where $V({\bf x'})$ is  a scalar function of the position eigenvalues. Combining
Eqs.~(\ref{e4.48}), (\ref{e4.49}), (\ref{e4.51}), and (\ref{e4.52}), we obtain
\begin{equation}
{\rm i}\,\hbar \,\frac{\partial \langle {\bf x}'|At\rangle}{\partial t}
= - \left(\frac{\hbar^2}{2\,m}\right)
\nabla'^2\langle{\bf x'} |At\rangle + V({\bf x}') \langle {\bf x}'|At\rangle,
\end{equation}
which can also be written
\begin{equation}\label{e4.54}
{\rm i}\,\hbar\,\frac{\partial \psi({\bf x}', t)}{\partial t}
= -  \left(\frac{\hbar^2}{2\,m}\right)
\nabla'^2\psi({\bf x}', t) + V({\bf x'})\, \psi({\bf x'}, t).
\end{equation}
This is Schr\"{o}dinger's famous wave-equation, and is the basis of
wave mechanics. Note, however, that the wave-equation is 
just one of many possible representations of quantum mechanics. It just happens
to give a type of equation which we know how to solve. In deriving the wave-equation, we have chosen to represent the system in terms of the eigenkets of
the position operators, instead of those of the momentum operators. We have
also fixed the relative phases of the $|{\bf x'}\rangle$ according to 
Schr\"{o}dinger's representation, so that Eq.~(\ref{e4.50}) is valid.  Finally, we
have chosen to work in the Schr\"{o}dinger picture, in which state kets evolve
and dynamical variables are fixed, instead of the Heisenberg picture,
in which the opposite is true. 

Suppose that the ket $|At\rangle$ is an eigenket of the Hamiltonian
belonging to the eigenvalue $H'$:
\begin{equation}
H|At\rangle = H'|At\rangle.
\end{equation}
Schr\"{o}dinger's equation of motion (\ref{e4.10}) yields
\begin{equation}
{\rm i}\,\hbar \,\frac{d |At\rangle}{dt} = H' |At\rangle.
\end{equation}
This can be integrated to give
\begin{equation}
|At\rangle = \exp[ -{\rm i}\,H'(t-t_0)/\hbar] |At_0\rangle.
\end{equation}
Note that $|At\rangle$ only differs from $|At_0\rangle$ by a phase-factor. The direction of the vector remains fixed in ket space. This
suggests that if the system is initially in an eigenstate of the
Hamiltonian then it remains in this state for ever, as long as the system
is undisturbed. Such a state is called a {\em stationary state}. The wave-function
of a stationary state satisfies 
\begin{equation}
\psi({\bf x}', t) = \psi({\bf x'}, t_0) \exp[ -{\rm i}\,H'\,(t-t_0)/\hbar].
\end{equation}

Substituting the above relation into Schr\"{o}dinger's wave equation
(\ref{e4.54}), we
obtain
\begin{equation}\label{e4.59}
-\left(\frac{\hbar^2}{2\,m}\right)\nabla'^2\psi_0({\bf x'}) + 
(V({\bf x'})-E)\,\psi_0({\bf x'}) =0,
\end{equation}
where $\psi_0({\bf x'}) \equiv \psi({\bf x'}, t_0)$,
and $E= H'$ is the energy of the system. This is Schr\"{o}dinger's
time-independent wave-equation. A {\em bound state} solution of the
above equation, in which the particle is confined within a finite region
of space, satisfies the boundary condition
\begin{equation}\label{e4.60}
\psi_0({\bf x}') \rightarrow 0 \mbox{\hspace{1cm}as $|{\bf x}'|\rightarrow \infty$}.
\end{equation}
Such a solution is only possible if
\begin{equation}
E < \lim_{|{\bf x}'|\rightarrow \infty} V({\bf x'}).
\end{equation}
Since it is conventional to set the potential at infinity equal to zero, the above
relation implies that bound states are equivalent to negative energy states.
The boundary condition (\ref{e4.60}) is sufficient to uniquely specify the solution
of Eq.~(\ref{e4.59}).

The quantity $\rho({\bf x'}, t)$, defined by
\begin{equation}
\rho({\bf x'}, t) = |\psi({\bf x'}, t)|^2,
\end{equation}
is termed the {\em probability density}. Recall, from Eq.~(\ref{e3.19}), that the
probability of observing the particle in some volume element $d^3 x'$ 
around position ${\bf x'}$ is proportional to $\rho({\bf x'}, t)\,d^3 x'$.
The probability is {\em equal} to $\rho({\bf x'}, t)\,d^3 x'$ if the wave-function
is properly normalized, so that
\begin{equation}\label{e4.63}
\int \rho({\bf x'}, t)\, d^3 x' = 1.
\end{equation}

Schr\"{o}dinger's time-dependent wave-equation, (\ref{e4.54}), can easily be
written in the form of a conservation equation for the probability
density:
\begin{equation}\label{e4.64}
\frac{\partial \rho}{\partial t} + \nabla'\cdot {\bf j} = 0.
\end{equation}
The {\em probability
current}\/ ${\bf j}$ takes the form
\begin{equation}\label{e4.65}
{\bf j}({\bf x}', t) = - \left(\frac{{\rm i}\, \hbar}{2\,m}\right)
\left[ \psi^\ast\, \nabla' \psi - (\nabla' \psi^\ast)\,\psi\right]
= \left(\frac{\hbar}{m} \right){\rm Im} (\psi^\ast \nabla' \psi).
\end{equation}
We can integrate Eq.~(\ref{e4.64}) over all space, using the divergence theorem,
and the boundary condition $\rho\rightarrow 0$ as $|{\bf x'}|\rightarrow
\infty$, to obtain 
\begin{equation}
\frac{\partial}{\partial t} \int \rho({\bf x'}, t)\, d^3 x' = 0.
\end{equation}
Thus, Schr\"{o}dinger's wave-equation {\em  conserves}\/
 probability. In particular, if the
wave-function starts off properly normalized, according to Eq.~(\ref{e4.63}), then it
remains properly normalized at all subsequent times. It is easily
demonstrated that
\begin{equation}
\int {\bf j}({\bf x}', t) \,d^3 x' = \frac{\langle {\bf p }\rangle_t}{m},
\end{equation}
where $\langle {\bf p} \rangle_t$ denotes the expectation value of the momentum
evaluated at time $t$.  Clearly, the probability current is indirectly related to
the particle momentum.

In deriving Eq.~(\ref{e4.64}) we have, naturally, assumed that the potential $V({\bf x}')$
is real. Suppose, however, that the potential has an imaginary component. 
In this case, Eq.~(\ref{e4.64}) generalizes to
\begin{equation}
\frac{\partial \rho}{\partial t} + \nabla'\cdot {\bf j} = \frac{2\,{\rm Im}(V)}{\hbar}
\rho,
\end{equation}
giving
\begin{equation}
\frac{\partial}{\partial t} \int \rho({\bf x'}, t)\, d^3 x' = \frac{2}{\hbar}\,
{\rm Im}\!\int
V({\bf x}') \,\rho({\bf x'}, t)\, d^3 x'.
\end{equation}
Thus, if ${\rm Im}(V)<0$ then the total probability of observing the particle
anywhere in space 
decreases monotonically with time. Thus, an imaginary potential can be
used to account for the disappearance of a particle. Such a potential
is often employed to model nuclear reactions in which incident particles can be
absorbed by nuclei.

The wave-function can always be written in the form
\begin{equation}\label{e4.70}
\psi({\bf x}', t) = \sqrt{\rho({\bf x}', t)} \exp\left[\frac{{\rm i}\,S({\bf x}',t)}{
\hbar}\right],
\end{equation}
where 
$\rho$ and $S$ are both real functions. The interpretation of $\rho$ as a probability
density has already been given. What is the interpretation of $S$?
Note that
\begin{equation}
\psi^\ast \nabla'\psi = \sqrt{\rho} \,\nabla'(\sqrt{\rho}) + \left(\frac{{\rm i}}
{\hbar}
\right) \rho \nabla' S.
\end{equation}
It follows from Eq.~(\ref{e4.65}) that
\begin{equation}
{\bf j} = \frac{\rho \,\nabla' S}{m}.
\end{equation}
Thus, the gradient of the phase of the wave-function determines the 
direction of the probability
current. In particular, the probability current is locally
normal to the  contours of the  phase-function $S$. 

Let us substitute Eq.~(\ref{e4.70}) into Schr\"{o}dinger's time-dependent wave-equation. We obtain
\begin{eqnarray}
-\frac{1}{2\,m}\left[ \hbar^2 \nabla'^2 \sqrt{\rho} + 2{\rm i}\,\hbar
\nabla'(\sqrt{\rho})\!\cdot\! \nabla' S - \sqrt{\rho}\, |\nabla' S|^2 
+{\rm i}\,\hbar \sqrt{\rho}\,\nabla'^2 S\right]
+ \sqrt{\rho}\, V& &\nonumber\\[0.5ex]
= \left[ {\rm i}\,\hbar \frac{\partial \sqrt{\rho}}{\partial t}
- \sqrt{\rho}\, \frac{\partial S}{\partial t} \right].&&\label{e4.73}
\end{eqnarray}
Let us treat $\hbar$ as a small quantity. To lowest order, Eq.~(\ref{e4.73})
yields
\begin{equation}
-\frac{\partial S({\bf x}', t)}{\partial t} = \frac{1}{2\,m} |\nabla' S({\bf x}', t)
|^2 + V ({\bf x}', t) = H({\bf x}', \nabla'
S, t),
\end{equation}
where $H({\bf x}, {\bf p}, t)$ is the Hamiltonian operator. The above equation is
known as the {\em Hamilton-Jacobi} equation, and is one of the many forms
in which we can write the equations of classical mechanics. In classical
mechanics, $S$ is the {\em action} ({\em i.e.}, the path-integral of the Lagrangian).
Thus, in the limit $\hbar\rightarrow 0$, wave mechanics reduces to classical
mechanics. It is a good approximation to neglect the terms involving
$\hbar$ in Eq.~(\ref{e4.73}) provided that
\begin{equation}\label{e4.75}
\hbar\, |\nabla'^2 S| \ll |\nabla' S|^2.
\end{equation}
Note that, according to Eq.~(\ref{e4.70}),
\begin{equation}
\lambdabar = \frac{\hbar}{|\nabla' S|},
\end{equation}
where $\lambdabar$ is the de~Broglie wave-length divided by $2\pi$. The inequality
(\ref{e4.75}) is equivalent to
\begin{equation}
|\nabla' \lambdabar| \ll 1.
\end{equation}
In other words, quantum mechanics reduces  to classical mechanics whenever the 
de~Broglie wave-length is small compared to the characteristic distance over which
things (other than the quantum phase)
 vary. This distance is usually set by the variation scale-length of the potential.
