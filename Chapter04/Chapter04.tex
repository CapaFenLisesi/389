\chapter{Orbital Angular Momentum}\label{ch4}
% !TEX root = ../Quantum.tex

\section{Orbital Angular Momentum}\label{s5.1}
Consider a particle described by the Cartesian coordinates 
$(x, y, z)\equiv {\bf x}$
 and their conjugate momenta $(p_x, p_y, p_z)\equiv {\bf p}$. The classical
definition of the orbital angular momentum of such a particle about the
origin is ${\bf L} = {\bf x}\times{\bf p}$, giving
\begin{align}\label{e5.1a}
L_x &= y\, p_z - z\, p_y,\\[0.5ex]
L_y &= z\, p_x - x\, p_z,\\[0.5ex]
L_z &= x\,p_y - y \,p_x.\label{e5.1c}
\end{align}
Let us assume that the operators $(L_x, L_y, L_z)\equiv {\bf L}$ which
represent the components of
orbital angular momentum in quantum mechanics can be defined in
an analogous manner to the corresponding components of
classical angular momentum. In other words, we are 
going to assume that the above equations specify  the angular momentum operators
in terms of the position and linear momentum operators. Note that $L_x$, $L_y$,
and $L_z$ are Hermitian, so they represent things which can, in principle,
be measured. Note, also, that there is no ambiguity regarding the order
in which operators appear in products on the right-hand sides of Equations~(\ref{e5.1a})--(\ref{e5.1c}),
because all of the products consist of operators that commute. 

The fundamental commutation relations satisfied by the
position  and linear momentum operators are [see Equations~(\ref{e3.14a})--(\ref{e3.14c})]
\begin{align}\label{e5.2a}
[x_i, x_j] &=0,\\[0.5ex]
[p_i, p_j] &=0,\\[0.5ex]
[x_i, p_j] &= {\rm i}\,\hbar \,\delta_{ij},\label{e5.2c}
\end{align}
where $i$ and $j$ stand for either $x$, $y$, or $z$. 
Consider the commutator of the operators $L_x$ and $L_z$:
\begin{align}
[L_x, L_y] &= [(y\,p_z-z\,p_y), (z\,p_x-x \,p_z)]
= y\,[p_z, z]\,p_x + x\,p_y\,[z, p_z] \nonumber\\[0.5ex]
&= {\rm i}\,\hbar\,(-y \,p_x+ x\,p_y) = {\rm i}\,\hbar\, L_z.
\end{align}
The cyclic permutations of the above result yield
the fundamental commutation relations satisfied 
by the   components of an orbital angular momentum:
\begin{align}\label{e5.4a}
[L_x, L_y] &= {\rm i}\,\hbar\, L_z,\\[0.5ex]
[L_y, L_z] &= {\rm i}\,\hbar\, L_x,\\[0.5ex]
[L_z, L_x] &= {\rm i}\,\hbar\, L_y.\label{e5.4c}
\end{align}
These can be summed up more succinctly by writing
\begin{equation}\label{e5.5}
{\bf L}\times {\bf L} = {\rm i}\,\hbar \,{\bf L}.
\end{equation}
The three commutation relations (\ref{e5.4a})--(\ref{e5.4c})  are the foundation for the whole
theory of angular momentum in quantum mechanics. Whenever we encounter 
three operators having these commutation relations, we know that the
dynamical variables  that they represent have identical properties
to those of the components of an
 angular momentum (which we are about to derive). In fact,
we shall  assume that {\em any three operators that satisfy the commutation
relations (\ref{e5.4a})--(\ref{e5.4c}) represent the components of some sort of angular momentum}. 

Suppose that there are $N$ particles in the system, with
angular momentum vectors ${\bf L}_i$ (where $i$ runs from 1 to $N$). 
Each of these vectors satisfies Equation~(\ref{e5.5}), so that
\begin{equation}\label{e5.6}
{\bf L}_i\times {\bf L}_i = {\rm i}\,\hbar \,{\bf L}_i.
\end{equation}
However, we expect  the angular momentum operators
 belonging to  different particles to commute, because they represent different
degrees of freedom of the system. So,
we can write
\begin{equation}\label{e5.7}
{\bf L}_i\times {\bf L}_j + {\bf L}_j\times {\bf L}_i =0,
\end{equation}
for $i\neq j$. Consider the total angular momentum of the system, 
${\bf L} = \sum_{i=1,N} {\bf L}_i$. It is clear from Equations~(\ref{e5.6}) and (\ref{e5.7})
 that
\begin{align}
{\bf L} \times {\bf L}& = \sum_{i=1,N} {\bf L}_i\times
\sum_{j=1,N} {\bf L}_j  = \sum_{i=1,N} {\bf L}_i \times
{\bf L}_i +\frac{1}{2}\!\sum_{i,j = 1,N}^{i\neq j}(  {\bf L}_i\times {\bf L}_j + {\bf L}_j\times {\bf L}_i) \nonumber\\[0.5ex]
&= {\rm i}\,\hbar\,\sum_{i=1,N} {\bf L}_i = {\rm i}\,\hbar \,{\bf L}.\label{e5.8}
\end{align}
Thus, the sum of two or more angular momentum vectors satisfies the
same commutation relation as a primitive  angular momentum vector.
In particular, the total angular momentum of the system satisfies the
commutation relation (\ref{e5.5}).

The immediate conclusion which can be drawn from the commutation relations
(\ref{e5.4a})--(\ref{e5.4c}) is that the three components of an angular momentum vector cannot
be specified (or measured) simultaneously. In fact,  once we have specified one
component, the values of other two components become uncertain.  It is
conventional to specify the $z$-component, $L_z$. 

Consider the magnitude squared of the angular momentum vector, $L^2 \equiv
L_x^{\,2} + L_y^{\,2}+L_z^{\,2}$. The commutator of $L^2$ and $L_z$ is
written
\begin{equation}
[L^2, L_z] = [L_x^{\,2}, L_z] + [L_y^{\,2}, L_z] + [L_z^{\,2}, L_z].
\end{equation}
It is easily demonstrated that
\begin{align}
[L_x^{\,2}, L_z] &= -{\rm i}\,\hbar\,(L_x\, L_y + L_y \,L_x),\\[0.5ex]
[L_y^{\,2}, L_z] &= +{\rm i}\,\hbar\,(L_x\,L_y + L_y \,L_x),\\[0.5ex]
[L_z^{\,2}, L_z] &= 0,
\end{align}
so 
\begin{equation}\label{e5.11}
[L^2, L_z] = 0.
\end{equation}
Because there is nothing special about the $z$-direction, we conclude that $L^2$ also
commutes with $L_x$ and $L_y$. It is clear from Equations~(\ref{e5.4a})--(\ref{e5.4c}) and
(\ref{e5.11}) that the best we
can do in quantum mechanics is to specify the 
magnitude of an angular momentum vector 
along with {\em one}\/ of its components (by convention, the $z$-component).

It is convenient to define the {\em shift operators}\/ $L^+$ and $L^-$:
\begin{align}
L^+ &= L_x + {\rm i}\, L_y,\\[0.5ex]
L^- &= L_x -{\rm i} \,L_y.
\end{align}
It can easily be shown that
\begin{align}\label{e5.13a}
[L^+, L_z ] &= -\hbar\,L^+,\\[0.5ex]
[L^-, L_z] &=+\hbar\,L^-,\\[0.5ex]
[L^+, L^-] &= 2\,\hbar\,L_z,
\end{align}
and also that both  shift operators commute with $L^2$. 

\section{Eigenvalues of Orbital Angular Momentum}\label{s5.2}
Suppose that the simultaneous eigenkets of $L^2$ and $L_z$ are completely
specified by two quantum numbers, $l$ and $m$. These  kets are denoted
$|l, m\rangle$. The quantum number $m$ is defined by
\begin{equation}
L_z \,|l, m\rangle = m\,\hbar\, |l, m\rangle.
\end{equation}
Thus, $m$ is the eigenvalue of $L_z$ divided by $\hbar$. It is possible
to write such an equation because $\hbar$ has the dimensions of angular momentum.
Note that $m$ is a real number, because  $L_z$ is an Hermitian operator. 

We can write
\begin{equation}\label{e5.15}
L^2 \,|l, m\rangle = f(l,m)\, \hbar^2\,|l, m\rangle,
\end{equation}
without loss of generality, 
where $f(l,m)$ is some real dimensionless function of $l$ and $m$. Later on,
we will show that $f(l,m) = l\,(l+1)$. 
Now, 
\begin{equation}\label{e5.16}
\langle l, m |\, L^2 - L_z^{\,2}\, |l, m\rangle =\langle l, m|\,
f(l, m) \,\hbar^2 - m^2\, \hbar^2\, |l, m\rangle =[f(l,m) - m^2] \,\hbar^2,
\end{equation}
assuming that the $|l, m\rangle$ have unit norms. However,
\begin{equation}
\langle l, m |\,L^2 - L_z^{\,2}\,|l, m\rangle =\langle l, m|\,
L_x^{\,2} + L_y^{\,2}\,|l, m\rangle = \langle l, m|\,L_x^{\,2}\,|l, m\rangle+
\langle l, m|\,L_y^{\,2}\,|l, m\rangle.
\end{equation}
It is easily demonstrated that
\begin{equation}\label{e5.18}
\langle A|\,\xi^{\,2}\,|A\rangle\geq 0,
\end{equation}
where $|A\rangle$ is a general  ket, and $\xi$ is an Hermitian operator. 
The proof follows from the observation that 
\begin{equation}
\langle A|\,\xi^{\,2}\,|A\rangle
= \langle A|\,\xi^{\dag}\, \xi\,|A\rangle = \langle B| B\rangle, 
\end{equation}
where $|B\rangle = \xi\, |A\rangle$, plus the fact that $\langle B|B\rangle\geq 0$
for a general  ket $|B\rangle$ [see Equation~(\ref{e2.21})]. It follows from
Equations~(\ref{e5.16})--(\ref{e5.18}) that
\begin{equation}\label{e5.20}
m^2 \leq f(l,m).
\end{equation}

Consider the effect of the shift operator $L^+$ on the eigenket $|l, m\rangle$.
It is easily demonstrated that
\begin{equation}
L^2 (L^+ |l, m\rangle) = \hbar^2\, f(l,m)\, (L^+ |l,m\rangle),
\end{equation}
where use has been made of Equation~(\ref{e5.15}), plus
the fact that $L^2$ and $L^+$ commute. 
It follows that the ket $L^+ |l,m\rangle$ has the same
eigenvalue of $L^2$ as the ket $|l,m\rangle$. Thus, the shift operator
$L^+$ does not affect the magnitude of the angular momentum of 
any eigenket it acts upon. However, 
\begin{equation}
L_z \,L^+ |l, m\rangle = (L^+ L_z + [L_z, L^+])\,|l,m\rangle
= (L^+ L_z + \hbar\, L^+) \,|l,m\rangle
= (m+1)\,\hbar \,L^+|l, m\rangle,
\end{equation}
where use has been made of Equation~(\ref{e5.13a}). The above equation implies
that $L^+ |l,m\rangle$ is proportional to $|l, m+1\rangle$. We can
write
\begin{equation}\label{e5.23}
L^+ |l ,m\rangle = c^+_{l\,m}\, \hbar\,|l, m+1\rangle,
\end{equation}
where $c^+_{l\, m}$ is a number. It is clear that if the operator $L^+$ 
acts on a simultaneous  eigenstate of $L^2$ and $L_z$ then 
the eigenvalue of $L^2$ remains unchanged, but the  eigenvalue
of $L_z$ is increased by $\hbar$. For this reason, $L^+$  is called
a {\em raising operator}. 

Using similar arguments to those given above, it is possible
to demonstrate that
\begin{equation}\label{e5.24}
L^-\, |l ,m\rangle = c^-_{l\,m}\,\hbar\, |l, m-1\rangle.
\end{equation}
Hence, $L^-$ is called a {\em lowering operator}. 

The shift operators, $L^+$ and $L^-$, respectively step the value of $m$ up and down by unity
each time they operate on one of the simultaneous eigenkets of 
$L^2$ and $L_z$. It would appear, at first sight, that any value of
$m$ can be obtained by applying these operators a sufficient
number of times. However, according to Equation~(\ref{e5.20}), there is
a definite upper bound to the values that $m^2$ can take. This 
bound is determined by  the eigenvalue of $L^2$
[see Equation~(\ref{e5.15})]. It follows that there is a maximum and a minimum
possible
value which  $m$ can take. 
Suppose that we attempt to raise the value
of $m$ above its maximum value $m_{\rm max}$. Since there is  no
state with $m> m_{\rm max}$, we must have
\begin{equation}
L^+ |l, m_{\rm max}\rangle = |0\rangle.
\end{equation}
This implies that 
\begin{equation}\label{e5.26}
L^-\, L^+ |l, m_{\rm max}\rangle = |0\rangle.
\end{equation}
However,
\begin{equation}
L^-\, L^+ = L_x^{\,2} + L_y^{\,2} + {\rm i}\,[L_x,  L_y]
= L^2 - L_z^{\,2} - \hbar \,L_z,
\end{equation}
so Equation~(\ref{e5.26}) yields
\begin{equation}
(L^2 - L_z^{\,2} - \hbar \,L_z) \,|l, m_{\rm max}\rangle = |0\rangle.
\end{equation}
The above equation can be rearranged to give
\begin{equation}
L^2\, |l, m_{\rm max}\rangle = (L_z^{\,2} + \hbar \,L_z)\, |l, m_{\rm max}\rangle 
= m_{\rm max}(m_{\rm max} + 1) \,\hbar^2\, |l, m_{\rm max}\rangle.
\end{equation}
Comparison of this equation with Equation~(\ref{e5.15}) yields the result
\begin{equation}
f(l, m_{\rm max}) = m_{\rm max} (m_{\rm max} + 1).
\end{equation}
But, when $L^-$ operates on $|n, m_{\rm max}\rangle$ it generates
$|n, m_{\rm max}-1\rangle$, $|n, m_{\rm max}-2\rangle$,  etc. Since
the lowering operator does not change the eigenvalue of $L^2$, all of these states
must correspond to the same value of $f$, namely $m_{\rm max}(m_{\rm max} + 1)$.
Thus,
\begin{equation}
L^2 \,|l, m\rangle = m_{\rm max}(m_{\rm max} + 1)\,\hbar^2\, |l, m\rangle.
\end{equation}
At this stage, we can give the unknown quantum number $l$ the value $m_{\rm max}$,
without loss of generality. 
We can also write the above equation in the form
\begin{equation}
L^2 \,|l, m\rangle = l\,(l+1)\, \hbar^2\, |l, m\rangle.
\end{equation}

It is easily seen that
\begin{equation}
L^- \,L^+ \,|l, m\rangle = (L^2 - L_z^{\,2}-\hbar\, L_z)\,|l, m \rangle
= \hbar^2 \,[l\,(l+1) - m\,(m+1)]\,|l,m\rangle.
\end{equation}
Thus,
\begin{equation}
\langle l,m|\, L^- \,L^+\,| l,m\rangle =\hbar^2 \,[l\,(l+1) - m\,(m+1)].
\end{equation}
However, we also know that
\begin{equation}
\langle l,m| L^- \,L^+ |l,m\rangle = \langle l, m| L^-\, \hbar\, c^+_{l\,m}
|l,m+1\rangle = \hbar^2\,  c^+_{l\,m} \,c^{-}_{l\,\, m+1},
\end{equation}
where use has been made of Equations~(\ref{e5.23}) and (\ref{e5.24}). 
It follows that
\begin{equation}\label{e5.36}
 c^+_{l\,m}\, c^{-}_{l\,\,m+1} = [l\,(l+1) - m\,(m+1)].
\end{equation}

Consider the following:
\begin{align}
\langle l, m|\, L^-\, |l, m+1\rangle &=
\langle l, m|\,L_x\,|l,m+1 \rangle - {\rm i}\, \langle l, m| \,L_y\,|l,m+1 \rangle \nonumber
\\[0.5ex]
&= \langle l, m+1| \,L_x\,|l,m \rangle^\ast - {\rm i}\, \langle l, m+1|\, L_y\,|l,m \rangle^\ast\nonumber\\[0.5ex]
&=( \langle l, m+1|\, L_x\,|l,m \rangle + {\rm i}\, 
\langle l, m+1|\, L_y\,|l,m \rangle)^\ast\nonumber\\[0.5ex]
&= \langle l, m+1|\,L^+\,|l, m\rangle^\ast,
\end{align}
where use has been made of the fact that $L_x$ and $L_y$ are Hermitian.
The above equation reduces to
\begin{equation}\label{e5.38}
c^-_{l\,\, m+1} = (c_{l\,m}^+)^\ast
\end{equation}
with the aid of Equations~(\ref{e5.23}) and (\ref{e5.24}). 

Equations (\ref{e5.36}) and (\ref{e5.38}) can be combined to give
\begin{equation}\label{e5.39}
|c^+_{l\,m}|^{\,2} = [l\,(l+1) - m \,(m+1)].
\end{equation}
The solution of the above equation is
\begin{equation}
c_{l\,m}^+ = \sqrt{l\,(l+1)- m \,(m+1)}.
\end{equation}
Note that $c_{l\,m}^+$ is undetermined to an arbitrary phase-factor
[{\rm i.e.}, we can replace $c_{l\, m}^+$, given above, by $c_{l\, m}^+\exp(\,{\rm i}\,\gamma)$,
where $\gamma$ is real, and we still satisfy Equation~(\ref{e5.39})]. We have made the arbitrary, but convenient, choice that $c_{l\,m}^+$ is real and positive. This is equivalent
to choosing  the relative phases of the eigenkets $|l, m\rangle$. 
According to Equation~(\ref{e5.38}),
\begin{equation}\label{e5.41}
c_{l\, m}^- = (c_{l\, \,m-1}^+)^\ast =  \sqrt{l\,(l+1)- m\, (m-1)}.
\end{equation}

We have already seen that the inequality (\ref{e5.20}) implies that there is a
maximum and a minimum possible value of $m$. The maximum value of $m$
is denoted $l$. What is the minimum value? Suppose that we try
to lower the value of $m$ below its minimum value $m_{\rm min}$. Because 
there is no state with $m<m_{\rm min}$, we must have
\begin{equation}
L^- \,|l, m_{\rm min}\rangle = 0.
\end{equation}
According to Equation~(\ref{e5.24}), this implies that
\begin{equation}
c_{l\,\, m_{\rm min}}^- = 0.
\end{equation}
It can be seen from Equation~(\ref{e5.41})  that $m_{\rm min} = -l$. 
We conclude that $m$ can take a ``ladder'' of discrete values, each rung differing
from its immediate neighbors by unity. The top rung is $l$, and the
bottom rung is  $-l$. There are only two possible choices for $l$.
Either it is an integer ({\rm e.g.}, $l=2$, which  allows $m$ to take the values
$-2, -1, 0, 1, 2$), or it is a half-integer ({\rm e.g.}, $l=3/2$, which  allows
$m$ to take the values $-3/2, -1/2, 1/2, 3/2$). We shall prove in the next
section that an orbital angular momentum can only take integer values
of $l$. 

In summary, using  just  the fundamental commutation relations (\ref{e5.4a})--(\ref{e5.4c}),
plus the fact that $L_x$, $L_y$, and $L_z$ are Hermitian operators, we have
shown that the eigenvalues of $L^2\equiv L_x^{\,2} + L_y^{\,2}+L_z^{\,2}$
can be written $l\,(l+1)\,\hbar^2$, where $l$ is an integer, or a half-integer. 
We have also demonstrated that the eigenvalues of $L_z$ can only
take the values $m\,\hbar$, where $m$ lies in the range $-l, -l+1,\cdots
l-1, l$. Let $|l, m\rangle$ denote a properly normalized simultaneous eigenket
of $L^2$ and $L_z$, belonging to the eigenvalues $l\,(l+1)\,\hbar^2$
and $m\,\hbar$, respectively.
We have shown that 
\begin{align}\label{e5.44a}
L^+ \,|l, m\rangle &= \sqrt{l\,(l+1)-m\,(m+1)}\,\hbar\,|l, m+1\rangle\\[0.5ex]
L^- \,|l,m \rangle &= \sqrt{l\,(l+1)-m\,(m-1)}\,\hbar\,|l, m-1\rangle,\label{e5.44b}
\end{align}
where $L^\pm = L_x \pm {\rm i} \,L_y$ are the so-called shift operators.

\section{Rotation Operators}\label{s5.3}
Consider a particle whose position is described by the spherical polar coordinates
$(r, \theta, \varphi)$. The classical momentum conjugate to the azimuthal
angle $\varphi$ is the $z$-component of angular momentum, $L_z$. 
According to Section~\ref{s3.5}, in quantum mechanics we can always adopt the Schr\"{o}dinger 
representation, for  which ket space is spanned by the simultaneous eigenkets
of the position operators $r$, $\theta$, and $\varphi$, and $L_z$ takes the
form
\begin{equation}\label{e5.45}
L_z = -{\rm i}\,\hbar\, \frac{\partial}{\partial \varphi}.
\end{equation}
We can do this because there is nothing in Section~\ref{s3.5} which specifies that
we have to use  Cartesian coordinates---the representation (\ref{e3.63}) works for 
any well-defined set of coordinates. 

Consider an operator $R({\mit\Delta}\varphi)$ that rotates the system through an angle
${\mit\Delta}\varphi$ about the $z$-axis. This operator is very similar to the
operator $D({\mit\Delta} x)$, introduced in Section~\ref{s3.8}, which translates the system
a distance ${\mit\Delta} x$ along the $x$-axis. 
We were able to demonstrate in Section~\ref{s3.8}
that
\begin{equation}
p_x = {\rm i}\,\hbar\, \lim_{\delta x\rightarrow 0}\frac{D(\delta x)-1}
{\delta x},
\end{equation}
where $p_x$ is the linear momentum conjugate to $x$. There is nothing
in our derivation of this result which specifies that $x$ has to be a Cartesian
coordinate. Thus, the result should apply just as well to an angular
coordinate. We conclude that
\begin{equation}\label{e5.47}
L_z = {\rm i}\,\hbar\, \lim_{\delta \varphi\rightarrow 0}\frac{R(\delta \varphi)-1}
{\delta \varphi}.
\end{equation}

According to Equation~(\ref{e5.47}), we can write
\begin{equation}
R(\delta \varphi) = 1 -{\rm i}\,L_z\,\delta\varphi/\hbar
\end{equation}
in the limit $\delta\varphi\rightarrow 0$. In other words, the angular momentum 
operator $L_z$
can be used to rotate the system about the $z$-axis by an infinitesimal amount.
We say that $L_z$ is the {\em generator}\/ of rotations about the $z$-axis. 
The above equation implies that 
\begin{equation}
R({\mit\Delta}\varphi) = \lim_{N\rightarrow\infty} \left(1-{\rm i} \,\frac{{\mit\Delta}
\varphi}{N} \frac{L_z}{\hbar}\right)^N,
\end{equation}
which reduces to
\begin{equation}\label{e5.50}
R({\mit\Delta}\varphi) = \exp(-{\rm i}\,L_z \,{\mit\Delta}\varphi/\hbar).
\end{equation}
Note that $R({\mit\Delta}\varphi)$ has all of the properties we would expect of
a rotation operator: i.e., 
\begin{align}
R(0) &= 1,\\[0.5ex]
R({\mit\Delta}\varphi)\,R(-{\mit\Delta}\varphi) &= 1,\\[0.5ex]
R({\mit\Delta}\varphi_1)\,R({\mit\Delta}\varphi_2)& = R({\mit\Delta}\varphi_1+ {\mit\Delta}\varphi_2).
\end{align}

Suppose that the system is in a simultaneous eigenstate of $L^2$ and $L_z$. 
As before, this state is represented by the eigenket $|l,m\rangle$, where
the eigenvalue of $L^2$ is $l\,(l+1)\,\hbar^2$, and
the eigenvalue of $L_z$ is $m\,\hbar$.  We expect the wavefunction
 to remain unaltered  if we rotate
the system $2\pi$ degrees about the $z$-axis. Thus,
\begin{equation}
R(2\pi)\,|l,m\rangle = \exp(-{\rm i}\,L_z \,2\pi/\hbar)\,|l,m\rangle 
= \exp(-{\rm i}\, 2\pi\,m) \,|l,m\rangle = |l,m\rangle.
\end{equation}
We conclude that $m$ must be an integer. This implies, from the previous
section, that $l$ must also be an integer. Thus, an {\em orbital}\/ angular momentum
can only take  {\em integer}\/ values of the quantum numbers $l$ and $m$. 

Consider the action of the rotation operator $R({\mit\Delta}\varphi)$ on an eigenstate
possessing zero angular momentum about the $z$-axis ({\rm i.e.}, an $m=0$
state). We have
\begin{equation}
R({\mit\Delta}\varphi)\,|l, 0\rangle =  \exp(0)\,| l, 0\rangle = |l, 0\rangle.
\end{equation}
Thus, the eigenstate is invariant to rotations about the $z$-axis. Clearly,
its wavefunction must be symmetric about the $z$-axis. 

There is nothing special about the $z$-axis, so we can write
\begin{align}\label{e5.54a}
R_x({\mit\Delta}\varphi_x) &= \exp(-{\rm i}\,L_x \,{\mit\Delta}\varphi_x/\hbar),\\[0.5ex]
R_y({\mit\Delta}\varphi_y) &= \exp(-{\rm i}\,L_y \,{\mit\Delta}\varphi_y/\hbar),\\[0.5ex]\label{e5.54c}
R_z({\mit\Delta}\varphi_y) &= \exp(-{\rm i}\,L_z\, {\mit\Delta}\varphi_z/\hbar),
\end{align}
by analogy with Equation~(\ref{e5.50}). Here, $R_x({\mit\Delta}\varphi_x)$ denotes an operator
that rotates the system by an angle ${\mit\Delta}\varphi_x$ about the $x$-axis, {\rm etc}.
Suppose that the system is in an eigenstate of zero overall orbital angular momentum
({\rm i.e.}, an $l=0$ state).
We know that the system is also in an eigenstate of zero orbital angular momentum 
about any particular axis. This follows because $l=0$ implies $m=0$, according
to  the
previous section, and we can choose the $z$-axis to point in any direction. 
Thus,
\begin{align}
R_x({\mit\Delta} \varphi_x)\, |0,0\rangle &=\exp(0)\,|0,0\rangle = |0,0\rangle,\\[0.5ex]
R_y({\mit\Delta} \varphi_y)\, |0,0\rangle &=\exp(0)\,|0,0\rangle = |0,0\rangle,\\[0.5ex]
R_z({\mit\Delta} \varphi_z) \,|0,0\rangle &=\exp(0)\,|0,0\rangle = |0,0\rangle.
\end{align}
Clearly, a zero angular momentum state is invariant to rotations about {\em any}\/
axis.
Such a state  must possess a spherically symmetric wavefunction. 

Note that a rotation about the $x$-axis does not commute with a rotation
about the $y$-axis. In other words, if the system is rotated an angle
${\mit\Delta}\varphi_x$ about the $x$-axis, and then ${\mit\Delta}\varphi_y$ about the
$y$-axis, it ends up in a different state to that obtained by rotating
an angle ${\mit\Delta}\varphi_y$ about the
$y$-axis, and then ${\mit\Delta}\varphi_x$ about the $x$-axis. In quantum
mechanics, this implies that $R_y({\mit\Delta}\varphi_y)\,R_x({\mit\Delta}\varphi_x)
\neq R_x({\mit\Delta}\varphi_x)\,R_y({\mit\Delta}\varphi_y)$, or 
$L_y \,L_x \neq L_x\, L_y$, [see  Equations~(\ref{e5.54a})--(\ref{e5.54c})]. Thus, the noncommuting
nature of the angular momentum operators is a direct consequence
of  the fact that
rotations do not commute. 

\section{Eigenfunctions of Orbital Angular Momentum}\label{s5.4}
In Cartesian coordinates, the three components of orbital angular
momentum can be written 
\begin{align}\label{e4.74v}
L_x &= -{\rm i}\,\hbar\left(y\,\frac{\partial}{\partial z} - z\,\frac{\partial}
{\partial y}\right),\\[0.5ex]
L_y &= -{\rm i}\,\hbar\left(z\,\frac{\partial}{\partial x} - x\,\frac{\partial}
{\partial z}\right),\\[0.5ex]
L_z &= -{\rm i}\,\hbar\left(x\,\frac{\partial}{\partial y} - y\,\frac{\partial}
{\partial x}\right),\label{e4.76v}
\end{align}
using the Schr\"{o}dinger representation. Transforming to standard
spherical polar coordinates,
\begin{align}
x &= r \,\sin\theta\, \cos\varphi,\\[0.5ex]
y &= r\, \sin\theta\, \sin\varphi,\\[0.5ex]
z &= r\,\cos\theta,\label{e4.79v}
\end{align}
we obtain
\begin{align}\label{e5.58a}
L_x &= {\rm i}\,\hbar\,\left(\sin\varphi\, \frac{\partial}{\partial \theta}
+ \cot\theta \cos\varphi\,\frac{\partial}{\partial \varphi}\right),
\\[0.5ex]
L_y &= -{\rm i} \,\hbar\,\left(\cos\varphi\, \frac{\partial}{\partial\theta}
-\cot\theta \sin\varphi \,\frac{\partial}{\partial \varphi}\right),\\[0.5ex]
L_z&= -{\rm i}\,\hbar\,\frac{\partial}{\partial\varphi}.\label{e5.58c}
\end{align}
Note that Equation~(\ref{e5.58c}) accords with Equation~(\ref{e5.45}). The shift
operators $L^\pm = L_x \pm {\rm i}\,L_y$ become
\begin{equation}\label{e5.59}
L^\pm = \pm \hbar\,\exp(\pm{\rm i}\,\varphi)\left(\frac{\partial}{\partial\theta}
\pm{\rm i} \,\cot\theta\,\frac{\partial}{\partial\varphi}\right).
\end{equation}
Now,
\begin{equation}
L^2 = L_x^{\,2}+L_y^{\,2}+L_z^{\,2} = L_z^{\,2} + (L^+\, L^- + L^- \,L^+) /2,
\end{equation}
so
\begin{equation}\label{e5.61}
L^2 = - \hbar^2\left( \frac{1}{\sin\theta}\frac{\partial}{\partial\theta}\,
\sin\theta\,\frac{\partial}{\partial\theta} + \frac{1}{\sin^2\theta}\frac{\partial^2}
{\partial\varphi^2}\right).
\end{equation}

 The eigenvalue problem for
$L^2$ takes the form
\begin{equation}\label{e5.62}
L^2 \,\psi = \lambda \,\hbar^2\, \psi,
\end{equation}
where $\psi(r, \theta, \varphi)$ is the wavefunction, and $\lambda$ is a number. 
Let us write
\begin{equation}
\psi(r, \theta, \varphi) = R(r) \,Y(\theta, \varphi).
\end{equation}
Equation~(\ref{e5.62}) reduces to
\begin{equation}
\left( \frac{1}{\sin\theta}\frac{\partial}{\partial\theta}\,
\sin\theta\,\frac{\partial}{\partial\theta} + \frac{1}{\sin^2\theta}\frac{\partial^2}
{\partial\varphi^2}\right)Y + \lambda \,Y = 0,
\end{equation}
where use has been made of Equation~(\ref{e5.61}). As is well-known,
square integrable solutions to this
equation only exist when $\lambda$ takes the values $l\,(l+1)$, where $l$ is
an integer. These solutions are known as {\em spherical harmonics}, and
can be written
\begin{equation}\label{e5.65}
Y_{l\,m}(\theta, \varphi) = \sqrt{ \frac{2\,l+1}{4\pi} \frac{(l-m)!}{(l+m)!}}
\,(-1)^m\, {\rm e}^{\,{\rm i} \,m\,\varphi}\, P_{l\,m}(\cos\varphi),
\end{equation}
where $m$ is a positive  integer lying in the range $0\leq  m\leq l$. Here, 
$P_{l\,m}(\xi)$ is an {\em associated Legendre function}\/ satisfying the
equation
\begin{equation}
\frac{d}{d\xi}\! \left[ (1-\xi^{\,2})\,\frac{dP_{l\,m}}{d\xi}\right]
- \frac{m^2}{1-\xi^{\,2}}\, P_{l\,m} + l\,(l+1)\,P_{l\,m} = 0.
\end{equation}
We define 
\begin{equation}
Y_{l\,\,-m} = (-1)^m\, Y_{l\,m}^{\,\ast},
\end{equation}
which allows $m$ to take the negative values $-l\leq m< 0$.
The spherical harmonics are {\em orthogonal}\/ functions, and are 
properly normalized with respect to integration over
the entire solid angle:
\begin{equation}
\int_0^\pi \int_0^{2\pi}d\theta\,d\varphi \, \sin\theta\,Y_{l\,m}^{\,\ast} (\theta,\varphi)\,
Y_{l'\,m'}(\theta, \varphi) = \delta_{l \,l'} \,\delta_{m \,m'}.
\end{equation}
The spherical harmonics also form a complete set for representing general functions
of $\theta$ and $\varphi$. 

By definition,
\begin{equation}\label{e5.69}
L^2 \,Y_{l\,m} = l\,(l+1)\,\hbar^2\,Y_{l\,m},
\end{equation}
where $l$ is an integer.
It follows from Equations~(\ref{e5.58c}) and (\ref{e5.65}) that
\begin{equation}
L_z \,Y_{l\,m} = m\,\hbar\,Y_{l\,m},
\end{equation}
where $m$ is an integer lying in the range $-l\leq m \leq l$. Thus, the
wavefunction $\psi(r, \theta, \varphi) = R(r) \,Y_{l,m}(\theta, \phi)$, where
$R$ is a general function,  has
all of the expected features of the wavefunction 
of a simultaneous eigenstate of $L^2$ and $L_z$
belonging to the quantum numbers $l$ and $m$. The well-known formula
\begin{equation}\label{e4.94v}
\frac{d P_{l\,m}}{d\xi} = -\frac{1}{\sqrt{1-\xi^{\,2}}}\,P_{l\,\,m+1}
- \frac{m\,\xi}{1-\xi^{\,2}}\, P_{l\,m} 
=  \frac{(l+m)\,(l-m+1)}{\sqrt{1-\xi^{\,2}}}\,P_{l\,\,m-1} + \frac{m\,\xi}
{1-\xi^{\,2}}\, P_{l\,m}
\end{equation}
can be combined with Equations~(\ref{e5.59}) and (\ref{e5.65}) to give
\begin{align}\label{e4.95v}
L^+ \,Y_{l\,m} &= [l\,(l+1)- m\,(m+1)]^{1/2}\,\hbar\,Y_{l\,\,m+1},\\[0.5ex]
L^- \,Y_{l\,m} &= [l\,(l+1) - m \,(m-1)]^{1/2} \,\hbar \,Y_{l\,\,m-1}.\label{e4.96v}
\end{align}
These equations are equivalent to Equations~(\ref{e5.44a})--(\ref{e5.44b}). Note that a spherical
harmonic  wavefunction
is symmetric about the $z$-axis ({\rm i.e.}, independent of $\varphi$) whenever
$m=0$, and is spherically symmetric whenever $l=0$ (since
$Y_{0\,0} = 1/\sqrt{4\pi}$). 

In summary, by solving directly
for the 
eigenfunctions of $L^2$ and $L_z$  in the Schr\"{o}d\-inger representation, we have been able to reproduce
all of the results of Section~\ref{s5.2}. Nevertheless, the results of Section~\ref{s5.2}
are more general than those obtained in this section, because they still apply
when the quantum number $l$ takes on half-integer values. 

\section{Motion in  Central Field}\label{s5.5}
Consider a particle of mass $M$ moving in a spherically symmetric potential.
The Hamiltonian takes the form
\begin{equation}
H = \frac{{p}^2}{2\,M} + V(r).
\end{equation}
Adopting Schr\"{o}dinger's representation, we can write ${\bf p} = -({\rm i}/\hbar)
\nabla$. Hence,
\begin{equation}
H = -\frac{\hbar^2}{2\,M}\, \nabla^2 + V(r).
\end{equation}
When written in spherical polar coordinates, the above equation becomes
\begin{equation}
H= -\frac{\hbar^2}{2\,M}\left[ \frac{1}{r^2}\frac{\partial}{\partial r}\,
r^2\,\frac{\partial}{\partial r}  + \frac{1}{r^2\sin\theta}
\frac{\partial}{\partial \theta} \,
\sin\theta \,\frac{\partial}{\partial\theta} 
+ \frac{1}{r^2\sin^2\theta} \frac{\partial^2}{\partial\varphi^2}\right]
+ V(r).
\end{equation}
Comparing this equation with Equation~(\ref{e5.61}), we find that
\begin{equation}\label{e5.76}
H= \frac{\hbar^2}{2\,M}\left(- \frac{1}{r^2}\frac{\partial}{\partial r}
\,r^2\,\frac{\partial}{\partial r} + \frac{L^2}{\hbar^2 r^2}\right) +
V(r).
\end{equation}

Now, we know that the three components of angular momentum commute with $L^2$ (see Section~\ref{s5.1}). We also know, from Equations~(\ref{e5.58a})--(\ref{e5.58c}), that $L_x$, $L_y$, and $L_z$ take the
form of partial derivative operators involving only {\em angular}\/ coordinates, 
when written in terms of spherical polar coordinates using the  Schr\"{o}dinger  representation. It follows from Equation~(\ref{e5.76}) that all three components of the angular
momentum commute with the Hamiltonian:
\begin{equation}
[{\bf L}, H] = 0.
\end{equation}
It is also easily seen that $L^2$ (which can be expressed as a purely angular differential operator) commutes with the Hamiltonian:
\begin{equation}
[L^2, H] = 0.
\end{equation}
According to Section~\ref{s4.2},  the previous two equations
ensure that the angular momentum ${\bf L}$ and its magnitude squared $L^2$
are both constants of the motion. This is as expected for a spherically
symmetric potential. 

Consider the energy eigenvalue problem
\begin{equation}
H\,\psi = E\,\psi,
\end{equation}
where $E$ is a number. Since $L^2$ and $L_z$ commute with each other and
the Hamiltonian, it is always possible to represent the state of the
system in terms of the simultaneous eigenstates of $L^2$, $L_z$, and $H$.
But, we already know that the most general form for the wavefunction of
a simultaneous
eigenstate of $L^2$ and $L_z$ is (see previous section)
\begin{equation}\label{e5.80}
\psi(r, \theta, \varphi) = R(r) \,Y_{l\,m}(\theta, \varphi).
\end{equation}
Substituting Equation~(\ref{e5.80}) into Equation~(\ref{e5.76}), and making use of Equation~(\ref{e5.69}), we
obtain
\begin{equation}\label{e5.81}
\left[\frac{\hbar^2}{2\,M} \left(-\frac{1}{r^2} \frac{d}{dr}\,r^2\,\frac{d}{dr}
+\frac{l\,(l+1)}{r^2}\right) + V(r) - E\right] R = 0.
\end{equation}
This is a {\em Sturm-Liouville equation}\/ for the function $R(r)$. We know,
from the general properties of this type of equation, 
that if $R(r)$ is required to be well-behaved  at $r=0$ and as $r\rightarrow
\infty$ then solutions only exist for a discrete set of values of $E$. These
are the energy eigenvalues. In general, the energy eigenvalues depend
on the quantum number $l$, but are independent of the quantum number $m$. 

\section{Energy Levels of  Hydrogen Atom}\label{s5.6}
Consider a hydrogen atom, for which the potential takes the specific form
\begin{equation}
V(r) = -\frac{e^2}{4\pi\,\epsilon_0\,r}.
\end{equation}
The radial eigenfunction $R(r)$ satisfies Equation~(\ref{e5.81}), which can be written
\begin{equation}
\left[\frac{\hbar^2}{2\,\mu} \left(-\frac{1}{r^2} \frac{d}{dr}\,r^2\,\frac{d}{dr}
+\frac{l\,(l+1)}{r^2}\right)   -\frac{e^2}{4\pi\,\epsilon_0\,r}- E\right] R = 0.
\end{equation}
Here, $\mu = m_e \,m_p/(m_e+ m_p)\simeq m_e$ is the {\em reduced mass}, which takes into
account the fact that the electron (of mass $m_e$) and the proton (of mass $m_p$)
both rotate about a common centre, which is equivalent to a particle of
mass $\mu$ rotating about a fixed point. Let us write the product $r\, R(r)$
as the function $P(r)$. The above equation transforms to
\begin{equation}\label{e5.84}
\frac{d^2 P}{d r^2} - \frac{2\,\mu}{\hbar^2}\left[
\frac{l\,(l+1)\,\hbar^2}{2\,\mu \,r^2} - \frac{e^2}{4\pi \,\epsilon_0 \,r}-E\right] P =0,
\end{equation}
which is the one-dimensional Schr\"{o}dinger equation for a particle of
mass $\mu$ moving in the {\em effective potential}\/
\begin{equation}
V_{\rm eff}(r) = -\frac{e^2}{4\pi \,\epsilon_0 \,r} + \frac{l\,(l+1)\,\hbar^2}{2\,\mu\, r^2}.
\end{equation}
The effective potential has a simple physical interpretation. The first part is the
attractive Coulomb potential, and the second part corresponds
to the repulsive centrifugal force.

Let 
\begin{equation}\label{e5.86}
a= \sqrt{\frac{-\hbar^2}{2\,\mu \,E}},
\end{equation}
and $y=r/a$, with
  \begin{equation}\label{e5.87}
P(r) = f(y) \exp(-y).
\end{equation}
Here, it is assumed that the energy eigenvalue $E$ is negative.
Equation~(\ref{e5.84}) transforms to 
\begin{equation}\label{e5.88}
\left(\frac{d^2}{dy^2} -2\,\frac{d}{dy} -\frac{l\,(l+1)}{y^2}
+ \frac{2\,\mu\, e^2\, a}{4\pi\, \epsilon_0\, \hbar^2\,y}\right) f= 0.
\end{equation}
Let us look for a power-law solution of the form
\begin{equation}\label{e5.89}
f(y) = \sum_{n} c_n\, y^{\,n}.
\end{equation}
Substituting this solution into Equation~(\ref{e5.88}), we obtain
\begin{equation}
\sum_n c_n \left[ n\,(n-1)\,y^{\,n-2} - 2\,n\, y^{\,n-1} - l\,(l+1) \,y^{\,n-2} +
\frac{2\,\mu\, e^2 \,a}{4\pi\, \epsilon_0 \,\hbar^2}\, y^{\,n-1} \right] = 0.
\end{equation}
Equating the coefficients of $y^{\,n-2}$ gives
\begin{equation}\label{e5.91}
c_n\,[n\,(n-1) - l\,(l+1)] = c_{n-1} \left [2\,(n-1) - \frac{2\,\mu\, e^2\, a}{4\pi\, \epsilon_0\, \hbar^2}\right].
\end{equation}
Now, the power law series (\ref{e5.89}) must terminate at small $n$, at some positive
value of $n$, otherwise $f(y)$ would behave unphysically as $y\rightarrow 0$. This
is only possible if $[n_{\rm min} (n_{\rm min} -1) - l\,(l+1) ] =0$, where
the first term in the series is $c_{n_{\rm min}}\,y^{\,n_{\rm min}}$. There
are two possibilities: $n_{\rm min} = -l$ or $n_{\rm min} = l+1$. The
former predicts  unphysical behavior of the wavefunction at $y=0$.
Thus, we conclude that $n_{\rm min} = l+1$. Note that for an $l=0$ state
there is a finite probability of finding the electron at the nucleus,
whereas for an $l>0$ state there is zero  probability of finding 
the electron at the nucleus ({\rm i.e.}, $|\psi|^{\,2} =0$ at $r=0$, except when
$l=0$). Note, also,  that it is only possible to obtain sensible behavior of the
wavefunction as $r\rightarrow 0$ if $l$ is an integer. 

For large values of $y$, the ratio of successive terms in the series
(\ref{e5.89}) is
\begin{equation}
\frac{c_n \,y}{c_{n-1}} = \frac{2\, y}{n},
\end{equation}
according to Equation~(\ref{e5.91}). This is the same as the ratio of
successive terms in the series
\begin{equation}
\sum_n \frac{(2\,y)^{\,n}}{n!},
\end{equation}
which converges to $\exp(2\,y)$. We conclude that $f(y)\rightarrow \exp(2\,y)$
as $y\rightarrow \infty$. It follows from Equation~(\ref{e5.87})  that $R(r) \rightarrow 
\exp(r/a) /r $ as $r\rightarrow \infty$. This does not correspond to
physically acceptable behavior of the wavefunction, since $\int d^3x'\, |\psi|^{\,2}$
must be finite. The only way in which we can avoid this unphysical
behavior is if the series (\ref{e5.89}) terminates at some maximum value of $n$.
According to the recursion relation (\ref{e5.91}), this is only possible
if
\begin{equation}
\frac{\mu\, e^2 \,a}{4\pi \,\epsilon_0\, \hbar^2} = n,
\end{equation}
where the last term in the series is $c_n\, y^{\,n}$. It follows from Equation~(\ref{e5.86})
that the energy eigenvalues are {\em quantized}, and can only take the values
\begin{equation}\label{e5.95}
E = \frac{E_0}{n^2},
\end{equation}
where
\begin{equation}
E_0 = - \frac{\mu\, e^4}{32\pi^2\,\epsilon_0^{\,2}\, \hbar^2} = - 13.6\,{\rm eV}
\end{equation}
is the ground state energy. 
Here, $n$ is a positive integer which must exceed the quantum number $l$,
otherwise there would be no terms in the series (\ref{e5.89}).

The properly normalized wavefunction of a hydrogen atom is written
\begin{equation}
\psi(r, \theta, \varphi) = R_{n\,l}(r)\, Y_{l\,m} (\theta, \varphi),
\end{equation}
where
\begin{equation}
R_{n\,l}(r) = {\cal R}_{n\,l}(r/a),
\end{equation}
and
\begin{equation}
a = n\,a_0.
\end{equation}
Here, 
\begin{equation}
a_0 =\frac{4\pi\, \epsilon_0\,\hbar^2}{\mu \,e^2} = 5.3\times 10^{-11}\,\,\,{\rm meters}
\end{equation}
is the {\em Bohr radius}, and 
 ${\cal R}_{n\,l}(x)$ is a well-behaved solution of the differential equation
\begin{equation}\label{e5.98}
\left[\frac{1}{x^2} \frac{d}{dx}\, x^2 \,\frac{d}{dx}-\frac{l\,(l+1)}{x^2}
+ \frac{2\,n}{x} - 1\right] {\cal R}_{n\,l} = 0
\end{equation}
that is consistent with  the normalization constraint
\begin{equation}
\int_0^\infty dr\,r^2\,[R_{n\,l}(r)]^{\,2}= 1.
\end{equation}
Finally, the $Y_{l\,m}$ are spherical harmonics. The restrictions on the quantum numbers
are  $|m| \leq l< n$, where $n$ is a positive integer, $l$ 
a non-negative integer, and $m$ an integer. 

The ground state of hydrogen corresponds to $n=1$. The only permissible values
of the other quantum numbers are $l=0$ and $m=0$. Thus, the ground state is
a spherically symmetric,  zero angular momentum state. The next energy level corresponds to $n=2$. The other quantum numbers are
allowed to take the values $l=0$, $m=0$ or $l=1$, $m=-1, 0, 1$. Thus, there are
$n=2$ states with non-zero angular momentum. Note that the energy levels given
in Equation~(\ref{e5.95}) are independent of the quantum number $l$, despite the fact that
$l$ appears in the radial eigenfunction equation (\ref{e5.98}). This is a special
property of a $1/r$ Coulomb potential. 

In addition to the quantized negative energy states of the
hydrogen atom, which we have just found, there
is  also a continuum of unbound positive energy states. 

\subsection*{Exercises}
\begin{enumerate}[label=\thechapter.\arabic*,leftmargin=*,widest=9.20]
\item Demonstrate directly from the fundamental commutation relations for angular momentum, (\ref{e5.5}),  that
$[L^2, L_z] = 0$, $[L^\pm, L_z] = \mp \,\hbar\,L^\pm$, and 
$[L^+,L^-] = 2\,\hbar\,L_z$.

\item Demonstrate from Equations~(\ref{e4.74v})--(\ref{e4.79v}) that 
\begin{align}
L_x &= {\rm i}\,\hbar\,\left(\sin\varphi\, \frac{\partial}{\partial \theta}
+ \cot\theta \cos\varphi\,\frac{\partial}{\partial \varphi}\right),\nonumber
\\[0.5ex]
L_y &= -{\rm i} \,\hbar\,\left(\cos\varphi\, \frac{\partial}{\partial\theta}
-\cot\theta \sin\varphi \,\frac{\partial}{\partial \varphi}\right),\nonumber\\[0.5ex]
L_z&= -{\rm i}\,\hbar\,\frac{\partial}{\partial\varphi},\nonumber
\end{align}
where $\theta$, $\varphi$ are conventional spherical polar angles. 

\item A system is in the state $\psi(\theta,\varphi)=Y_{l\,m}(\theta,\varphi)$. Evaluate
$\langle L_x\rangle$,  $\langle L_y\rangle$, $\langle L_x^{\,2}\rangle$, and 
$\langle L_y^{\,2}\rangle$.

\item Derive Equations~(\ref{e4.95v}) and (\ref{e4.96v}) from Equation~(\ref{e4.94v}).

\item Find the eigenvalues and eigenfunctions (in terms of the angles $\theta$ and $\varphi$) of $L_x$.

\item Consider a beam of particles with $l=1$. A measurement of $L_x$ yields the result $\hbar$. What
values will be obtained by a subsequent measurement of $L_z$, and with what probabilities? Repeat
the calculation for the cases in which the measurement of $L_x$ yields the results $0$ and
$-\hbar$. 

\item The Hamiltonian for an axially symmetric rotator is given by
$$
H = \frac{L_x^{\,2}+L_y^{\,2}}{2\,I_1} + \frac{L_z^{\,2}}{2\,I_2}.
$$
What are the eigenvalues of $H$?

\item The expectation value of $f({\bf x},{\bf p})$ in any stationary state is a constant.
Calculate
$$
0= \frac{d}{dt}\,(\langle{\bf x}\cdot{\bf p}\rangle) = \frac{{\rm i}}{\hbar}\,\langle[H, {\bf x}\cdot{\bf p}]\rangle
$$
for a Hamiltonian of the form
$$
H = \frac{p^2}{2\,m} + V(r).
$$
Hence, show that
$$
\left\langle\frac{p^2}{2\,m}\right\rangle = \frac{1}{2}\left\langle r\,\frac{dV}{dr}\right\rangle
$$
in a stationary state. This is another form of the {\em Virial theorem}. (See Exercise~\ref{ex3.5}.)

\item Use the Virial theorem of the previous exercise to prove that
$$
\left\langle \frac{1}{r}\right\rangle = \frac{1}{n^2\,a_0}
$$
for an energy eigenstate of the hydrogen atom. 

\item Demonstrate that the  first few properly normalized radial wavefunctions of the hydrogen atom take the form:
\begin{enumerate}
\item 
$$
R_{1\,0}(r) = \frac{2}{a_0^{\,3/2}}\,\exp\left(-\frac{r}{a_0}\right).
$$
\item
$$
R_{2\,0}(r)= \frac{2}{(2\,a_0)^{3/2}}\left(1-\frac{r}{2\,a_0}\right)\exp\left(-\frac{r}{2\,a_0}\right).
$$
\item
$$
R_{2\,1}(r)= \frac{1}{\sqrt{3}\,(2\,a_0)^{3/2}}\,\frac{r}{a_0}\,\exp\left(-\frac{r}{2\,a_0}\right).
$$
\end{enumerate}
\end{enumerate}
