\chapter{Relativistic Electron Theory}\label{c11}
% !TEX root = ../Quantum.tex

\section{Introduction}
The aim of this chapter is to develop a quantum mechanical theory of electron
dynamics that is consistent with special relativity. Such a theory is needed
to explain the origin of electron spin (which is essentially a relativistic effect),
and to account for the fact that the spin contribution to the electron's magnetic
moment is twice what we would naively expect by analogy with (non-relativistic)
classical physics (see Section~\ref{s5.5c}). Relativistic electron theory is
also required to fully understand the fine structure of the hydrogen
atom energy levels (recall, from Section~\ref{s7.7x}, and Exercises~\ref{ex7.1} and \ref{ex7.2}, that the modification to the energy
levels due to spin-orbit coupling is of the same order of magnitude as the
first-order correction due to the electron's relativistic mass increase.)

In the following, we shall use
$x^1$, $x^2$, $x^3$ to represent the Cartesian coordinates  $x$, $y$, $z$, respectively, and $x^0$ to represent $c\,t$. 
The time dependent wavefunction then takes the form $\psi(x^0,x^1,x^2,x^3)$.  Adopting 
standard relativistic notation, we write the four $x$'s as $x^{\,\mu}$, for $\mu= 0,1,2,3$. A space-time
vector with four components that transforms under Lorentz transformation in an analogous manner to the four space-time coordinates 
$x^{\,\mu}$ is termed a {\em 4-vector}, and its components are written like $a^{\,\mu}$ (i.e., with an upper
Greek suffix). We can lower the suffix according to the rules
\begin{align}
a_0&= a^0,\\[0.5ex]
a_1 &=-a^1,\\[0.5ex]
a_2 &= -a^2,\\[0.5ex]
a_3 &= -a^3.
\end{align}
Here, the $a^{\,\mu}$ are called the {\em contravariant}\/ components of the vector $a$, whereas the
$a_{\mu}$ are termed the {\em covariant}\/ components. Two 4-vectors $a^{\,\mu}$ and $b^{\,\mu}$
have the Lorentz invariant scalar product
\begin{equation}\label{e11.5}
a^0\,b^0-a^1\,b^1-a^2\,a^2-a^3\,b^3 = a^{\,\mu}\,b_\mu= a_\mu\,b^{\,\mu},
\end{equation}
a summation being implied over a repeated letter suffix. The metric tenor $g_{\mu\,\nu}$ is defined
\begin{align}
g_{00}&=1,\\[0.5ex]
g_{11}&=-1,\\[0.5ex]
g_{22}&=-1,\\[0.5ex]
g_{33}&=-1,
\end{align}
with all other components zero.
Thus,
\begin{equation}
a_{\mu} = g_{\mu\,\nu}\,a^\nu.
\end{equation}
Likewise,
\begin{equation}
a^{\,\mu} = g^{\,\mu\,\nu}\,a_\nu,
\end{equation}
where $g^{00}=1$, $g^{11}=g^{22}=g^{33}=-1$, with all other components zero. 
Finally, $g_\nu^{~\mu}=g^{\,\mu}_{~\nu} =1$ if $\mu=\nu$, and $g_\nu^{~\mu}=g^{\,\mu}_{~\nu}=0$ otherwise. 

In the Schr\"{o}dinger representation, the momentum of a particle, whose components are written $p_x$, $p_y$, $p_z$, or $p^1$, $p^2$, $p^3$, 
is represented by the operators
\begin{equation}\label{e11.11}
p^{\,i} = -{\rm i}\,\hbar\,\frac{\partial}{\partial x^{\,i}},
\end{equation}
for $i=1,2,3$. Now, the four operators $\partial/\partial x^{\,\mu}$ form the covariant components of a
4-vector whose contravariant components are written $\partial /\partial x_{\mu}$. So, to make
expression (\ref{e11.11}) consistent with relativistic theory,  we must first write it with its
suffixes balanced,
\begin{equation}
p^{\,i} = {\rm i}\,\hbar\,\frac{\partial}{\partial x_i},
\end{equation}
and then extend it to the complete 4-vector equation
\begin{equation}\label{e11.13}
p^{\,\mu} =  {\rm i}\,\hbar\,\frac{\partial}{\partial x_{\mu}}.
\end{equation}
According to standard relativistic theory, the new operator $p^0={\rm i}\,\hbar\,\partial/\partial x_0$, which forms a 4-vector when
combined with the momenta $p^{\,i}$, is interpreted as the energy of the particle divided by $c$, where $c$ is the velocity of
light in vacuum.

\section{Dirac Equation}
Consider the motion of an electron in the absence of an electromagnetic field. In classical relativity, 
electron energy, $E$, is related to  electron momentum, ${\bf p}$,  according to the well-known formula
\begin{equation}
\frac{E}{c}=(p^2+ m_e^2\,c^2)^{1/2},
\end{equation}
where $m_e$ is the electron rest mass. 
The quantum mechanical equivalent of this expression is the wave equation
\begin{equation}\label{e11.15}
\left[p^0  - (p^1\,p^1+p^2\,p^2+p^3\,p^3+m_e^2\,c^2)^{1/2}\right]\psi = 0,
\end{equation}
where the $p$'s are interpreted as differential operators according to Equation~(\ref{e11.13}). The above equation
takes into account the correct relativistic relation between electron energy and momentum, but is nevertheless unsatisfactory from the
point of view of relativistic theory, because it is highly asymmetric between  $p^0$ and the other $p$'s. This  makes the equation
difficult to generalize, in a manifestly Lorentz  invariant manner, in the presence of an electromagnetic  field. We must therefore look for a new equation. 

If we multiply the wave equation (\ref{e11.15}) by the operator $\left[p^0  +(p^1\,p^1+p^2\,p^2+p^3\,p^3+m_e^2\,c^2)^{1/2}\right]$
then we obtain
\begin{equation}\label{e11.16}
\left(p^0\,p^0- p^1\,p^1-p^2\,p^2-p^3\,p^3+m_e^2\,c^2\right)\psi = \left(p^{\,\mu}\,p_\mu+m_e^2\,c^2\right)\psi.
\end{equation}
This equation is manifestly Lorentz invariant, and, therefore, forms a more convenient starting point for relativistic quantum mechanics. 
Note, however, that Equation~(\ref{e11.16}) is not entirely equivalent to Equation~(\ref{e11.15}), because, although every
solution of (\ref{e11.15}) is also a solution of (\ref{e11.16}), the converse is not true. In fact, only those solutions of (\ref{e11.16})
belonging to positive values of $p^0$ are also solutions of (\ref{e11.15}). 

The wave equation (\ref{e11.16}) is quadratic in $p^0$, and is thus not of the form required by the  laws of quantum theory.  (Recall that we showed, from
general principles, in Chapter~\ref{s4},   that the 
time evolution equation for the wavefunction should be linear in the operator $\partial/\partial t$, and, hence, in $p^0$.) We, therefore, seek a wave equation that is
equivalent to (\ref{e11.16}), but is
linear in $p^0$. In order to ensure that this equation transforms in a simple way under a Lorentz transformation, we
shall require it to be rational and linear in $p^1$, $p^2$, $p^3$, as well as $p^0$. We are thus
lead to a wave equation of the form
\begin{equation}\label{e11.17}
\left(p^0 - \alpha_1\,p^1-\alpha_2\,p^2-\alpha_3\,p^3-\beta\,m_e\,c\right)\psi = 0,
\end{equation}
where the $\alpha$'s and $\beta$ are dimensionless, and independent of the $p$'s. Moreover, according to standard relativity, because we are considering the case of no electromagnetic field, all points in space-time
must be equivalent. Hence, the $\alpha$'s and $\beta$ must also be independent of
the $x$'s. This implies that the $\alpha$'s and $\beta$ commute with the $p$'s and the $x$'s. We, therefore,  deduce that the $\alpha$'s and $\beta$
describe an internal degree of freedom that is independent of space-time coordinates. Actually, we shall show later that these operators are related to electron spin. 

Multiplying (\ref{e11.17}) by the operator $p^0 +\alpha_1\,p^1+\alpha_2\,p^2+\alpha_3\,p^3+\beta\,m_e\,c$, we obtain
\begin{equation}
\left[p^0\,p^0-\frac{1}{2}\sum_{i,j=1,3}\{\alpha_i,\alpha_j\}\,p^i\,p^j-\sum_{i=1,3}\{\alpha_i,\beta\}\,p^i\,m_e\,c-\beta^{\,2}\,m_e^2\,c^2\right]\psi= 0,
\end{equation}
where $\{a,b\}\equiv a\,b+b\,a$. 
This equation is equivalent to (\ref{e11.16}) provided that
\begin{align}\label{e11.19}
\{\alpha_i,\alpha_j\}&=2\,\delta_{ij},\\[0.5ex]
\{\alpha_i,\beta\}&= 0,\label{e11.20}\\[0.5ex]
\beta^{\,2} &= 1,\label{e11.21}
\end{align}
for $i,j=1,3$. 
It is helpful to define the $\gamma^{\,\mu}$, for $\mu=0,3$, where
\begin{align}\label{e11.25}
\beta &=\gamma^0,\\[0.5ex]
\alpha_i &= \gamma^0\,\gamma^i,\label{e11.26}
\end{align}
for $i=1,3$. 
Equations~(\ref{e11.19})--(\ref{e11.21}) can then be shown to reduce to
\begin{equation}\label{e11.26a}
\{\gamma^{\,\mu},\gamma^\nu\} = 2\,g^{\,\mu\,\nu}.
\end{equation}
One way of satisfying the above anti-commutation relations is to represent the operators $\gamma^{\,\mu}$ as matrices. However, it turns
out that the smallest  dimension in which the $\gamma^{\,\mu}$ can be realized   is four. In fact, it is easily verified that the  $4\times 4$ matrices 
\begin{align}\label{e11.26t}
\gamma^0 &= \left(\begin{array}{rr} 1& 0\\[0.5ex]0&-1\end{array}\right),\\[0.5ex]
\gamma^i &= \left(\begin{array}{rr} 0& \sigma_i\\[0.5ex]-\sigma_i&0\end{array}\right),\label{e11.27t}
\end{align}
for $i=1,3$, satisfy the appropriate anti-commutation relations. Here, $0$ and $1$ denote $2\times 2$ null and identity matrices, respectively, whereas the $\sigma_i$ represent the $2\times 2$ Pauli matrices
introduced in Section~\ref{spauli}. 
It follows from (\ref{e11.25}) and (\ref{e11.26}) that
\begin{align}\label{e11.28g}
\beta &= \left(\begin{array}{rr} 1& 0\\[0.5ex]0&-1\end{array}\right),\\[0.5ex]
\alpha_i &= \left(\begin{array}{rr} 0& \sigma_i\\[0.5ex]\sigma_i&0\end{array}\right).\label{e11.29g}
\end{align}
Note that  $\gamma^0$, $\beta$, and the $\alpha_i$, are all Hermitian matrices, whereas the $\gamma^{\,\mu}$, for $\mu=1,3$, are anti-Hermitian. 
However, the matrices $\gamma^0\,\gamma^{\,\mu}$, for $\mu=0,3$, are Hermitian. Moreover, it is easily demonstrated that
\begin{equation}\label{e11.30x}
\gamma^{\,\mu\,\dag} = \gamma^0\,\gamma^{\,\mu}\,\gamma^0,
\end{equation}
for $\mu=0,3$. 

Equation~(\ref{e11.17}) can be written in the form
\begin{equation}\label{e11.30}
(\gamma^{\,\mu}\,p_\mu - m_e\,c)\,\psi = ({\rm i}\,\hbar\,\gamma^{\,\mu}\,\partial_\mu-m_e\,c)\,\psi=0,
\end{equation}
where $\partial_\mu\equiv \partial/\partial x^{\,\mu}$.  Alternatively, we can write 
\begin{equation}\label{e11.31}
{\rm i}\,\hbar\,\frac{\partial\psi}{\partial t} = (c\,\balpha\cdot{\bf p} + \beta\,m_e\,c^2)\,\psi,
\end{equation}
where ${\bf p}=(p_x,p_y,p_z)=(p^1,p^2,p^3)$, and $\balpha$ is the vector of the $\alpha_i$ matrices. The previous expression is known as the {\em Dirac equation}. 
Incidentally, it is clear that, corresponding to the four rows and columns of the $\gamma^{\,\mu}$ matrices, the wavefunction $\psi$
must take the form of a $4\times 1$ column matrix, each element of which is, in general, a function of the $x^{\,\mu}$. 
We saw in Section~\ref{spauli} that the spin of the electron requires the wavefunction to have two components. The reason
our present theory requires the wavefunction to have four components is because the wave equation (\ref{e11.16}) has twice
as many solutions as it ought to have, half of them corresponding to negative energy states. 

We can incorporate an electromagnetic field into the above formalism by means of the standard prescription $E\rightarrow E+e\,\phi$, and
$p^{\,i}\rightarrow p^{\,i} +e\,A^{\,i}$, where $e$ is the magnitude of the electron charge, $\phi$ the scalar potential, and ${\bf A}$ the
vector potential. This prescription can be expressed in the Lorentz invariant form
\begin{equation}
p^{\,\mu} \rightarrow p^{\,\mu} + \frac{e}{c}\,{\mit\Phi}^{\,\mu},
\end{equation}
where ${\mit\Phi}^{\,\mu} = (\phi,c\,{\bf A})$ is the potential 4-vector. Thus,
Equation~(\ref{e11.30}) becomes
\begin{equation}\label{e11.33}
\left[\gamma^{\,\mu}\left(p_\mu+\frac{e}{c}\,{\mit\Phi}_\mu\right)-m_e\,c\right]\psi = \left[\gamma^{\,\mu}\left({\rm i}\,\hbar\,\partial_\mu+\frac{e}{c}\,{\mit\Phi}_\mu\right)-m_e\,c\right]\psi=0 ,
\end{equation}
whereas
Equation~(\ref{e11.31}) generalizes to 
\begin{equation}\label{e11.34}
{\rm i}\,\hbar\,\frac{\partial\psi}{\partial t} =\left[-e\,\phi + c\,\balpha\cdot({\bf p}+e\,{\bf A})+ \beta\,m_e\,c^2\right]\psi = 0.
\end{equation}

If we write the wavefunction in the spinor form
\begin{equation}
\psi= \left(\begin{array}{c}\psi_0\\[0.5ex]\psi_1\\[0.5ex]\psi_2\\[0.5ex]\psi_3\end{array}\right)
\end{equation}
then the Hermitian conjugate of Equation~(\ref{e11.34}) becomes
\begin{equation}
-{\rm i}\,\hbar\,\frac{\partial\psi^\dag}{\partial t} =\psi^\dag\left[-e\,\phi + c\,\balpha\cdot({\bf p} +{\bf e}\,{\bf A})+ \beta\,m_e\,c^2\right] = 0,
\end{equation}
where
\begin{equation}
\psi^\dag = \left(\psi_0^{\,\ast}, \psi_1^{\,\ast},\psi_2^{\,\ast},\psi_3^{\,\ast}\right),
\end{equation}
Here, use has been made of the fact that the $\alpha_i$ and $\beta$ are Hermitian matrices that commute with the $p^i$ and $A^i$. 

It follows from $\psi^\dag\,\gamma^0$ times Equation~(\ref{e11.33}) that
\begin{equation}
\psi^\dag\left[\gamma^0\,\gamma^{\,\mu}\left({\rm i}\,\hbar\,\partial_\mu-\frac{e}{c}\,{\mit\Phi}_\mu\right)-\gamma^0\,m_e\,c\right]\psi=0.
\end{equation}
The Hermitian conjugate of this expression is 
\begin{equation}
\psi^\dag\left[\left(-{\rm i}\,\hbar\,\partial_\mu- \frac{e}{c}\,{\mit\Phi}_\mu\right)\gamma^0\,\gamma^{\,\mu}-m_e\,c\,\gamma^0\,\right]\psi=0,
\end{equation}
where $\partial_\mu$ now acts backward on $\psi^\dag$, and use has been made of the fact that the matrices $\gamma^0\,\gamma^{\,\mu}$
and $\gamma^0$ are Hermitian. Taking the difference between the previous two equation, we obtain
\begin{equation}\label{e11.41}
\partial_\mu\,j^{\,\mu} = 0,
\end{equation}
where
\begin{equation}\label{e11.42x}
j^{\,\mu} = c\,\psi^\dag\,\gamma^0\,\gamma^{\,\mu}\,\psi.
\end{equation}
Writing $j^{\,\mu} = (c\,\rho, {\bf j})$, where
\begin{align}
\rho &= \psi^\dag\,\psi,\\[0.5ex]
j^{\,i}& = c\,\psi^\dag\,\gamma^0\,\gamma^i\,\psi = \psi^\dag\,c\,\alpha_i\,\psi,
\end{align}
Equation~(\ref{e11.41}) becomes
\begin{equation}
\frac{\partial\rho}{\partial t} + \nabla\cdot{\bf j} = 0.
\end{equation}
The above expression has the same form as the non-relativistic probability conservation equation (\ref{e4.64}). This
suggests that we can interpret the positive definite real scalar field $\rho({\bf x},t) = |\psi|^{\,2}$ as the relativistic {\em probability density}, and the vector field ${\bf j}({\bf x},t)$
as the relativistic {\em probability current}. Integration of the above expression over all space, assuming that $|\psi({\bf x},t)|\rightarrow 0$ as $|{\bf x}|\rightarrow
\infty$, yields
\begin{equation}
\frac{d}{dt}\!\int d^3 x\,\,\rho({\bf x},t) = 0.
\end{equation}
This ensures that if the wavefunction  is properly normalized at time $t=0$, such that
\begin{equation}
\int d^3 x\,\,\rho({\bf x},0) = 1,
\end{equation}
then the wavefunction remains properly normalized at all subsequent times, as it evolves in accordance with the Dirac equation. 
In fact, if this were not the case then it would be impossible to interpret $\rho$ as a probability density. Now, relativistic
invariance demands that if the wavefunction is properly normalized in one particular inertial frame then it should
be properly normalized in all inertial frames. This is the case provided that Equation~(\ref{e11.41}) is Lorentz invariant (i.e., if it has the property
that if it holds
in one inertial frame then it holds in all inertial frames), 
which is true as long as the $j^{\,\mu}$ transform as the contravariant components of a  4-vector under  Lorentz transformation (see Exercise~\ref{ex11.4}). 

\section{Lorentz Invariance of Dirac Equation}
Consider two inertial frames, $S$ and $S'$. Let the $x^{\,\mu}$ and $x^{\,\mu'}$ be the space-time coordinates of a given event in each frame, respectively. 
These coordinates are  related via a Lorentz transformation, which takes the general form
\begin{equation}
x^{\,\mu'} = a^{\,\mu}_{~\nu}\,x^\nu,
\end{equation}
where the $a^{\,\mu}_{~\nu}$ are real numerical coefficients that are independent of the $x^{\,\mu}$. 
We also have
\begin{equation}
x_{\mu'} = a_{\mu}^{~\nu}\,x_\nu.
\end{equation}
Now, since [see Equation~(\ref{e11.5})]
\begin{equation}
x^{\,\mu'}\,x_{\mu'} = x^{\,\mu}\,x_\mu,
\end{equation}
it follows that
\begin{equation}\label{e11.51}
a^{\,\mu}_{~\nu}\,a_\mu^{~\lambda} = g_\nu^{~\lambda}.
\end{equation}
Moreover, it is easily shown that
\begin{align}
x^{\,\mu} &= a_\nu^{~\mu}\,x^{\nu'},\\[0.5ex]
x_\mu &= a^\nu_{~\mu}\,x_{\nu'}.
\end{align}
By definition,  a 4-vector $p^{\,\mu}$ has analogous transformation properties to the $x^{\,\mu}$. Thus,
\begin{align}
p^{\,\mu'} &= a^{\,\mu}_{~\nu}\,p^\nu,\\[0.5ex]
p^{\,\mu} &= a_\nu^{~\mu}\,p^{\nu'},
\end{align}
etc.

In  frame $S$, the Dirac equation is written
\begin{equation}\label{e11.55}
\left[\gamma^{\,\mu}\left(p_\mu- \frac{e}{c}\,{\mit\Phi}_\mu\right)-m_e\,c\right]\psi = 0.
\end{equation}
Let $\psi'$ be the wavefunction in frame $S'$. Suppose that
\begin{equation}
\psi' = A\,\psi,
\end{equation}
where $A$ is a $4\times 4$ transformation matrix that is independent of the $x^{\,\mu}$. (Hence, $A$ commutes with the $p_\mu$ and the ${\mit\Phi}_\mu$.)
Multiplying (\ref{e11.55}) by $A$, we obtain
\begin{equation}
\left[A\,\gamma^{\,\mu}\,A^{-1}\left(p_\mu- \frac{e}{c}\,{\mit\Phi}_\mu\right)-m_e\,c\right]\psi' = 0.
\end{equation}
Hence, given that the $p_\mu$ and ${\mit\Phi}_\mu$ are the covariant components of 4-vectors, we obtain
\begin{equation}\label{e11.57}
\left[A\,\gamma^{\,\mu}\,A^{-1}\,a^\nu_{~\mu}\left(p_{\nu'}- \frac{e}{c}\,{\mit\Phi}_{\nu'}\right)-m_e\,c\right]\psi' = 0.
\end{equation}
Suppose that
\begin{equation}\label{e11.58}
A\,\gamma^{\,\mu}\,A^{-1}\,a^\nu_{~\mu} = \gamma^\nu,
\end{equation}
which is equivalent to 
\begin{equation}\label{e11.59}
A^{-1}\,\gamma^\nu\,A = a^\nu_{~\mu}\,\gamma^{\,\mu}.
\end{equation}
Here, we have assumed that the $a^\nu_{~\mu}$ commute with $A$ and the $\gamma^{\,\mu}$ (since they are just numbers). If (\ref{e11.58})
holds then (\ref{e11.57}) becomes
\begin{equation}\label{e11.61}
\left[\gamma^{\,\mu}\left(p_{\mu'}- \frac{e}{c}\,{\mit\Phi}_{\mu'}\right)-m_e\,c\right]\psi' = 0.
\end{equation}
A comparison of this equation with (\ref{e11.55}) reveals that the Dirac equation takes the same form in frames $S$ and $S'$. In other words, the
Dirac equation is Lorentz invariant. Incidentally, it is clear from (\ref{e11.55}) and (\ref{e11.61}) that the $\gamma^{\,\mu}$ matrices are
the same in all inertial frames. 

It remains to find a transformation matrix $A$ that satisfies (\ref{e11.59}). Consider an infinitesimal Lorentz transformation, for which
\begin{equation}\label{e11.62}
a_\mu^{~\nu} = g_\mu^{~\nu} + {\mit\Delta}\omega_\mu^{~\nu},
\end{equation}
where the ${\mit\Delta}\omega_\mu^{~\nu}$ are real numerical coefficients that are independent of the $x^{\,\mu}$, and are  also small compared to unity. To first order in small quantities, (\ref{e11.51}) yields
\begin{equation}\label{e11.63}
{\mit\Delta}\omega^{\,\mu\,\nu} + {\mit\Delta}\omega^{\nu\,\mu} = 0.
\end{equation}
Let us write
\begin{equation}\label{e11.64}
A = 1 - \frac{{\rm i}}{4}\,\sigma_{\mu\,\nu}\,{\mit\Delta}\omega^{\,\mu\,\nu},
\end{equation}
where the $\sigma_{\mu\,\nu}$ are ${\cal O}(1)$ $4\times 4$ matrices. To first order in small
quantities,
\begin{equation}\label{e11.65}
A^{-1} = 1 + \frac{{\rm i}}{4}\,\sigma_{\mu\,\nu}\,{\mit\Delta}\omega^{\,\mu\,\nu}.
\end{equation}
Moreover, it follows from (\ref{e11.63}) that
\begin{equation}
\sigma_{\mu\,\nu} = -\sigma_{\nu\,\mu}.
\end{equation}
To first order in small quantities, Equations~(\ref{e11.59}), (\ref{e11.62}), (\ref{e11.64}), and (\ref{e11.65}) yield
\begin{equation}
{\mit\Delta}\omega^\nu_{~\beta}\,\gamma^{\,\beta} = -\frac{\rm i}{4}\,{\mit\Delta}\omega^{\alpha\,\beta}\left(\gamma^\nu\,\sigma_{\alpha\,\beta}- \sigma_{\alpha\,\beta}\,\gamma^\nu\right).
\end{equation}
Hence, making use of the symmetry property (\ref{e11.63}), we obtain
\begin{equation}
{\mit\Delta}\omega^{\alpha\,\beta}\,(g^\nu_{~\alpha}\,\gamma_\beta -g^\nu_{~\beta}\,\gamma_\alpha) = -\frac{\rm i}{2}\,{\mit\Delta}\omega^{\alpha\,\beta}\,(\gamma^\nu\,\sigma_{\alpha\,\beta}-\sigma_{\alpha\,\beta}\,\gamma^\nu),
\end{equation}
where $\gamma_\mu = g_{\mu\,\nu}\,\gamma^\nu$. 
Since this equation must hold for arbitrary ${\mit\Delta}\omega^{\alpha\,\beta}$, we deduce that
\begin{equation}\label{e11.69}
2\,{\rm i}\,(g^\nu_{~\alpha}\,\gamma_\beta -g^\nu_{~\beta}\,\gamma_\alpha) = [\gamma^\nu, \sigma_{\alpha\,\beta}].
\end{equation}
Making use of the anti-commutation relations (\ref{e11.26a}), it can be shown that a suitable solution of the above
equation is
\begin{equation}\label{e11.70}
\sigma_{\mu\,\nu} = \frac{{\rm i}}{2}\,[\gamma_\mu,\gamma_\nu].
\end{equation}
Hence, 
\begin{align}\label{e11.71}
A &= 1 + \frac{1}{8}\,[\gamma_\mu,\gamma_\nu]\,{\mit\Delta}\omega^{\,\mu\,\nu},\\[0.5ex]
A^{-1} &= 1 - \frac{1}{8}\,[\gamma_\mu,\gamma_\nu]\,{\mit\Delta}\omega^{\,\mu\,\nu}.
\end{align}
Now that we have found the correct transformation rules for an infinitesimal Lorentz transformation, we can easily find those for a
finite transformation by building it up from a large number of successive infinitesimal transforms. 

Making use of (\ref{e11.30x}), as well as $\gamma^0\,\gamma^0=1$,  the Hermitian conjugate of (\ref{e11.71}) can be shown
to take the form
\begin{equation}
A^\dag = 1-\frac{1}{8}\,\gamma^0\,[\gamma_\mu,\gamma_\nu]\,\gamma^0\,{\mit\Delta}\omega^{\,\mu\,\nu} = \gamma^0\,A^{-1}\,\gamma^0.
\end{equation}
Hence, (\ref{e11.59}) yields
\begin{equation}
A^\dag\,\gamma^0\,\gamma^{\,\mu}\,A = a^\mu_{~\nu}\,\gamma^0\,\gamma^\nu.
\end{equation}
It follows that
\begin{equation}
\psi^\dag\,A^\dag\,\gamma^0\,\gamma^{\,\mu}\,A\,\psi= a^{\,\mu}_{~\nu}\,\psi^\dag\,\gamma^0\,\gamma^\nu\,\psi,
\end{equation}
or
\begin{equation}
\psi'^{\dag}\,\gamma^0\,\gamma^{\,\mu}\,\psi'= a^{\,\mu}_{~\nu}\,\psi^\dag\,\gamma^0\,\gamma^\nu\,\psi,
\end{equation}
which implies that
\begin{equation}
j^{\,\mu'} = a^{\,\mu}_{~\nu}\,j^{\,\nu},
\end{equation}
where the $j^{\,\mu}$ are defined in Equation~(\ref{e11.42x}).  This proves that the $j^{\,\mu}$ transform as the contravariant components
of a 4-vector. 

\section{Free Electron Motion}
According to Equation~(\ref{e11.31}), the relativistic Hamiltonian of a free electron takes the form
\begin{equation}
H = c\,\balpha\cdot{\bf p} + \beta\,m_e\,c^2.
\end{equation}
Let us use the Heisenberg picture to investigate the motion of such an electron. For the
sake of brevity, we shall omit the suffix $t$ that should be appended to dynamical variables that vary in time, according to the formalism of Section~\ref{s4.2}.

The above Hamiltonian is independent of ${\bf x}$. Hence, momentum ${\bf p}$ commutes with the Hamiltonian, and is therefore a
constant of the motion. The $x$ component of the velocity is 
\begin{equation}\label{e11.80}
\dot{x} = \frac{[x,H]}{{\rm i}\,\hbar} = c\,\alpha_1,
\end{equation} 
where use has been made of the standard commutation relations between position and momentum operators. 
This result is rather surprising, since it implies a relationship between velocity and momentum that is quite different from that
in classical mechanics. This relationship, however, is clearly connected to the expression $j_x = \psi^\dag c\,\alpha_1\,\psi$ for the
$x$ component of the probability current. The operator $\dot{x}$, specified in the above equation, has the eigenvalues $\pm c$,
corresponding to the eigenvalues $\pm 1$ of $\alpha_1$. Since $\dot{y}$ and $\dot{z}$ are similar to $\dot{x}$, we
conclude that a measurement of a velocity component of a free electron is certain to yield the result $\pm c$. As is easily demonstrated, this conclusion also
holds in the presence of an electromagnetic field. 

Of course, electrons are often observed to have velocities considerably less than that of light. Hence, the previous conclusion seems to
be in conflict with experimental observations. The conflict is not real, however, because the theoretical velocity discussed above is
the velocity at one instance in time, whereas observed velocities are always averages over a finite time interval. We shall find, on
further examination of the equations of motion, that the velocity of a free electron is not constant, but oscillates rapidly about a mean value
that agrees with the experimentally observed value. 

In order to understand why a measurement of a velocity component must lead to the result $\pm c$ in a relativistic theory, consider the
following argument. To measure the velocity we must measure the position at two slightly different times, and then divide the
change in position by the time interval. (We cannot just measure the momentum and then apply a formula, because the
ordinary connection between velocity and momentum is no longer valid.) In order that our measured velocity may
approximate to the instantaneous velocity, the time interval between the two measurements of position must be very short, and the
measurements themselves very accurate. However, the great accuracy with which the position of the electron
is known during the time interval leads to an almost complete indeterminacy in its momentum, according to the Heisenberg uncertainty principle. 
This means that almost all values of the momentum are equally likely, so that the momentum is almost certain to be infinite. But, an infinite
value of a momentum component corresponds to the values $\pm c$ for the corresponding velocity component. 

Let us now examine how the election velocity varies in time. We have
\begin{equation}
{\rm i}\,\hbar\,\dot{\alpha}_1  =\alpha_1\,H-H\,\alpha_1.
\end{equation}
Now $\alpha_1$ anti-commutes with all terms in $H$ except $c\,\alpha_1\,p^1$,
so
\begin{equation}
\alpha_1\,H+H\,\alpha_1  = \alpha_1\,c\,\alpha_1\,p^1 + c\,\alpha_1\,p^1\,\alpha_1 = 2\,c\,p^1.
\end{equation}
Here, use has been made of the fact that $\alpha_1$ commutes with $p^1$, and also that $\alpha_1^{\,2}=1$. 
Hence, we get
\begin{equation}\label{e11.83}
{\rm i}\,\hbar\,\dot{\alpha}_1 = 2\,\alpha_1\,H-2\,c\,p^1.
\end{equation}
Since $H$ and $p^1$ are constants of the motion, this equation yields
\begin{equation}
{\rm i}\,\hbar\,\ddot{\alpha}_1 = 2\,\dot{\alpha}_1\,H,
\end{equation}
which can be integrated to give
\begin{equation}
\dot{\alpha}_1(t) = \dot{\alpha}_1(0)\,\exp\left(\frac{-2\,{\rm i}\,H\,t}{\hbar}\right).
\end{equation}
It follows from Equation~(\ref{e11.80}) and (\ref{e11.83}) that
\begin{equation}
\dot{x}(t) = c\,\alpha_1(t) = \frac{{\rm i}\,\hbar\,c}{2}\,\dot{\alpha}_1(0)\,\exp\left(\frac{-2\,{\rm i}\,H\,t}{\hbar}\right)H^{\,-1} + c^2\,p_x\,H^{\,-1},
\end{equation}
and
\begin{equation}
x(t) = x(0)- \frac{\hbar^2\,c}{4}\,\dot{\alpha}_1(0)\,\exp\left(\frac{-2\,{\rm i}\,H\,t}{\hbar}\right)H^{\,-2} + c^2\,p_x\,H^{\,-1}\,t.
\end{equation}
We can see that the $x$-component of velocity consists of two parts, a constant part, $c^2\,p_x\,H^{\,-1}$, connected with the momentum
according to the classical relativistic formula, and an oscillatory part whose frequency, $2\,H/h$,  is high (being at least $2\,m_e\,c^2/h$). 
Only the constant part would be observed in a practical measurement of velocity (i.e., an average over a short time interval that is still much longer than $h/2\,m_e\,c^2$). 
The oscillatory part ensures that the instantaneous value of $\dot{x}$ has the eigenvalues $\pm c$.  Note, finally, that the oscillatory part of $x$ is small, being of
order
$\hbar/m_e\,c$.

\section{Electron Spin}
According to Equation~(\ref{e11.34}),  the relativistic Hamiltonian of an electron in an electromagnetic
field is
\begin{equation}\label{e11.87}
H =-e\,\phi + c\,\balpha\cdot({\bf p}+e\,{\bf A})+ \beta\,m_e\,c^2.
\end{equation}
Hence,
\begin{equation}\label{e11.89}
\left(\frac{H}{c}+\frac{e}{c}\,\phi\right)^2 = \left[\balpha\cdot({\bf p}+e\,{\bf A})+ \beta\,m_e\,c\right]^{\,2} = \left[\balpha\cdot({\bf p}+e\,{\bf A})\right]^{\,2} + m_e^{\,2}\,c^2,
\end{equation}
where use has been made of Equations~(\ref{e11.20}) and (\ref{e11.21}). 
Now, we can write
\begin{equation}
\alpha_i = \gamma^5\,\Sigma_i,
\end{equation}
for $i=1,3$, where
\begin{equation}
\gamma^5 = \left(\begin{array}{cc} 0& 1\\[0.5ex]1 & 0\end{array}\right),
\end{equation}
and
\begin{equation}\label{e11.92f}
\Sigma_i = \left(\begin{array}{cc} \sigma_i& 0\\[0.5ex]0& \sigma_i\end{array}\right).
\end{equation}
Here, $0$ and $1$ denote $2\times 2$ null and identity matrices, respectively, whereas the $\sigma_i$ are conventional
$2\times 2$ Pauli matrices. Note that $\gamma^5\,\gamma^5=1$, and
\begin{equation}
[\gamma^5, \Sigma_i]=0.
\end{equation}
 It follows from (\ref{e11.89}) that
\begin{equation}
\left(\frac{H}{c}+\frac{e}{c}\,\phi\right)^2 =  \left[\bSigma\cdot({\bf p}+e\,{\bf A})\right]^{\,2} + m_e^{\,2}\,c^2.
\end{equation}
Now, a straightforward generalization of Equation~(\ref{e5.174}) gives
\begin{equation}\label{e11.93}
(\bSigma \cdot {\bf a} ) \,(\bSigma \cdot {\bf b}) = {\bf a} \cdot {\bf b}
+{\rm i}\,\bSigma\cdot ({\bf a} \times {\bf b}),
\end{equation}
where ${\bf a}$ and ${\bf b}$ are any two three-dimensional vectors that commute with $\bSigma$. 
It follows that
\begin{equation}
 \left[\bSigma\cdot({\bf p}+e\,{\bf A})\right]^{\,2} = ({\bf p}+e\,{\bf A})^2 + {\rm i}\,\bSigma\cdot ({\bf p}+e\,{\bf A})\times ({\bf p}+e\,{\bf A}).
 \end{equation}
 However,
 \begin{equation}
 ({\bf p}+e\,{\bf A})\times ({\bf p}+e\,{\bf A}) = e\,{\bf p}\times {\bf A} +e\,{\bf A}\times {\bf p} = -{\rm i}\,e\,\hbar\,\nabla\times {\bf A} -{ \rm i}\,e\,\hbar\,{\bf A}\times \nabla = -{\rm i}\,e\,\hbar\,{\bf B},
 \end{equation}
 where ${\bf B} = \nabla\times {\bf A}$ is the magnetic field strength. 
 Hence, we obtain
 \begin{equation}\label{e11.96}
 \left(\frac{H}{c}+\frac{e}{c}\,\phi\right)^2 = ({\bf p}+e\,{\bf A})^2 + m_e^{\,2}\,c^2+e\,\hbar\,\bSigma\cdot{\bf B}.
 \end{equation}
 
 Consider the non-relativistic limit. In this case, we can write
 \begin{equation}
 H = m_e\,c^2+ \delta H,
 \end{equation}
 where $\delta H$ is small compared to $m_e\,c^2$. Substituting into (\ref{e11.96}), and neglecting $\delta H^{\,2}$, and other
 terms involving $c^{\,-2}$, we get
 \begin{equation}
 \delta H\simeq -e\,\phi + \frac{1}{2\,m_e}\,({\bf p} + e\,{\bf A})^2 + \frac{e\,\hbar}{2\,m_e}\,\bSigma\cdot{\bf B}.
 \end{equation}
 This Hamiltonian is the same as the classical Hamiltonian of a non-relativistic electron, except for the final term. 
 This term may be interpreted as arising from the electron having an intrinsic  magnetic moment 
 \begin{equation}\label{e11.99}
 \bmu = - \frac{e\,\hbar}{2\,m_e}\,\bSigma.
 \end{equation}
 
 In order to demonstrate that the electron's intrinsic magnetic moment is associated with an intrinsic angular momentum,
  consider the motion of an electron in a central electrostatic potential: i.e., $\phi=\phi(r)$ and ${\bf A}={\bf 0}$. 
In this case, the Hamiltonian (\ref{e11.87}) becomes
\begin{equation}
H = - e\,\phi(r) + c\,\gamma^5\,\bSigma\cdot{\bf p} + \beta\,m_e\,c^2.
\end{equation}
Consider the $x$ component of the electron's orbital angular momentum,
\begin{equation}
L_x = y\,p_z-z\,p_y = {\rm i}\,\hbar\left(z\,\frac{\partial}{\partial y} - y\,\frac{\partial}{\partial z}\right).
\end{equation}
The Heisenberg equation of motion for this quantity is
\begin{equation}
{\rm i}\,\hbar\,\dot{L}_x = [L_x,H].
\end{equation}
However, it is easily demonstrated that
\begin{align}
[L_x,r] &= 0,\\[0.5ex]
[L_x,p_x] &= 0,\\[0.5ex]
[L_x,p_y] &= {\rm i}\,\hbar\,p_z,\\[0.5ex]
[L_x,p_z]&=  -{\rm i}\,\hbar\,p_y.
\end{align}
Hence, we obtain
\begin{equation}
[L_x,H] = {\rm i}\,\hbar\,c\,\gamma^5\,(\Sigma_2\,p_z-\Sigma_3\,p_y),
\end{equation}
which implies that
\begin{equation}
\dot{L}_x = c\,\gamma^5\,(\Sigma_2\,p_z-\Sigma_3\,p_y).
\end{equation}
It can be seen that $L_x$ is not a constant of the motion. However, the $x$-component of the total angular
momentum of the system must be a constant of the motion (because a central electrostatic potential exerts zero torque on the system). Hence, we deduce that the electron possesses additional
angular momentum that is not connected with its motion through space. Now,
\begin{equation}
{\rm i}\,\hbar\,\dot{\Sigma}_1= [\Sigma_1,H].
\end{equation}
However,
\begin{align}
[\Sigma_1,\gamma^5] &= 0,\\[0.5ex]
[\Sigma_1,\Sigma_1] &=0,\\[0.5ex]
[\Sigma_1,\Sigma_2] &=2\,{\rm i}\,\Sigma_3,\\[0.5ex]
[\Sigma_1,\Sigma_3]&=-2\,{\rm i}\,\Sigma_2,
\end{align}
so
\begin{equation}
[\Sigma_1,H] = 2\,{\rm i}\,c\,\gamma^5\,(\Sigma_3\,p_y-\Sigma_2\,p_z),
\end{equation}
which implies that
\begin{equation}
\frac{\hbar}{2}\,\dot{\Sigma}_1 = -c\,\gamma^5\,(\Sigma_2\,p_z-\Sigma_3\,p_y).
\end{equation}
Hence, we deduce that
\begin{equation}
\dot{L}_x +\frac{\hbar}{2}\,\dot{\Sigma}_1 = 0.
\end{equation}
Since there is nothing special about the $x$ direction, we conclude that the vector ${\bf L} + (\hbar/2)\,\bSigma$ is
a constant of the motion. We can interpret this result by saying that the electron has a spin angular momentum
${\bf S} = (\hbar/2)\,\bSigma$, which must be added to its orbital angular momentum in order to obtain a constant of the motion. 
According to (\ref{e11.99}), the relationship between the electron's spin angular momentum and its intrinsic (i.e., non-orbital) magnetic moment is
\begin{equation}
\bmu = - \frac{e\,g}{2\,m_e}\,{\bf S},
\end{equation}
where the gyromagnetic ratio $g$ takes the value
\begin{equation}
g= 2.
\end{equation}
As explained in Section~\ref{s5.5c}, this is twice the value one would naively predict by analogy with classical physics. 

\section{Motion in  Central Field}
To further study the motion of an electron in a central field, whose Hamiltonian is
\begin{equation}\label{e11.119}
H = - e\,\phi(r) + c\,\balpha\cdot{\bf p} + \beta\,m_e\,c^2,
\end{equation}
it is convenient to  transform to polar coordinates. Let 
\begin{equation}
r = (x^{2}+y^{2}+z^{2})^{1/2},
\end{equation}
and
\begin{equation}\label{e11.222f}
p_r = {\bf x}\cdot{\bf p}.
\end{equation}
It is easily demonstrated that
\begin{equation}\label{e11.122}
[r,p_r] = {\rm i} \,\hbar,
\end{equation}
which implies that in the Schr\"{o}dinger representation 
\begin{equation}\label{e11.123}
p_r = -{\rm i}\,\hbar\,\frac{\partial}{\partial r}.
\end{equation}

Now, by symmetry, an energy eigenstate in a central field is a simultaneous eigenstate of the total angular momentum
\begin{equation}
{\bf J} = {\bf L} + \frac{\hbar}{2}\,\bSigma.
\end{equation}
Furthermore, we know from general principles that the eigenvalues of $J^{\,2}$ are $j\,(j+1)\,\hbar^2$, where $j$ is a positive half-integer (since
$j=|l+1/2|$, where $l$ is the standard non-negative integer quantum number associated with orbital angular momentum.)

It follows from Equation~(\ref{e11.93}) that
\begin{equation}
(\bSigma\cdot {\bf L})\,(\bSigma\cdot{\bf L}) = L^{\,2} + {\rm i}\,\bSigma\times ({\bf L}\times {\bf L}).
\end{equation}
However, because ${\bf L}$ is an angular momentum, its components satisfy the standard commutation relations
\begin{equation}
{\bf L}\times {\bf L} = {\rm i}\,\hbar\,{\bf L}.
\end{equation}
Thus, we obtain
\begin{equation}
(\bSigma\cdot {\bf L})\,(\bSigma\cdot{\bf L}) = L^{\,2} -\hbar\,\bSigma\cdot{\bf L} = J^{\,2} - 2\,\hbar\,\bSigma\cdot{\bf L} -\frac{\hbar^2}{4}\,\Sigma^{\,2}.
\end{equation}
However, $\Sigma^{\,2}=3$, so
\begin{equation}\label{e11.128}
(\bSigma\cdot {\bf L} + \hbar)^2 = J^{\,2}+\frac{1}{4}\,\hbar^2.
\end{equation}

Further application of (\ref{e11.93}) yields
\begin{align}\label{e11.129}
(\bSigma\cdot {\bf L})\,(\bSigma\cdot{\bf p}) &= {\bf L}\cdot {\bf p} + {\rm i}\,\bSigma\cdot{\bf L}\times {\bf p}= {\rm i}\,\bSigma\cdot{\bf L}\times{\bf p},\\[0.5ex]
(\bSigma\cdot {\bf p})\,(\bSigma\cdot{\bf L}) &= {\bf p}\cdot {\bf L} + {\rm i}\,\bSigma\cdot{\bf p}\times {\bf L}= {\rm i}\,\bSigma\cdot{\bf p}\times{\bf L},
\end{align}
However, it is easily demonstrated from the fundamental commutation relations between position and momentum operators that
\begin{equation}
{\bf L}\times {\bf p} + {\bf p}\times{\bf L} = 2\,{\rm i}\,\hbar\,{\bf p}.
\end{equation}
Thus,
\begin{equation}
(\bSigma\cdot{\bf L})\,(\bSigma\cdot{\bf p}) + (\bSigma\cdot{\bf p})\,(\bSigma\cdot{\bf L}) =-2\,\hbar\,\bSigma\cdot{\bf p},
\end{equation}
which implies that
\begin{equation}
\{\bSigma\cdot {\bf L}+ \hbar,\, \bSigma\cdot {\bf p}\} = 0.
\end{equation}
Now, $\gamma^5\,\bSigma = \balpha$. Moreover, $\gamma^5$ commutes with ${\bf p}$, ${\bf L}$, and $\bSigma$. Hence,
we conclude that
\begin{equation}
\{\bSigma\cdot {\bf L}+ \hbar,\, \balpha\cdot {\bf p}\} = 0.
\end{equation}
Finally, since $\beta$ commutes with ${\bf p}$ and ${\bf L}$, but anti-commutes with the components of $\balpha$, we obtain
\begin{equation}\label{e11.135}
[\zeta,\balpha\cdot {\bf p}] = 0,
\end{equation}
where 
\begin{equation}\label{e11.137}
\zeta = \beta\left(\bSigma\cdot{\bf L} + \hbar\right).
\end{equation}

If we repeat the above analysis, starting at Equation~(\ref{e11.129}), but substituting ${\bf x}$ for ${\bf p}$, and 
making use of the easily demonstrated result
\begin{equation}
{\bf L}\times {\bf x} + {\bf x}\times{\bf L} = 2\,{\rm i}\,\hbar\,{\bf x},
\end{equation}
we find that
\begin{equation}
[\zeta,\balpha\cdot {\bf x}] = 0.
\end{equation}

Now, $r$ commutes with $\beta$, as well as the components of $\bSigma$ and ${\bf L}$. Hence,
\begin{equation}
[\zeta,r] = 0.\label{e11.139}
\end{equation}
Moreover, $\beta$ commutes with the components of ${\bf L}$, and can easily be shown to commute with all components of $\bSigma$. 
It follows that
\begin{equation}\label{e11.140}
[\zeta,\beta]=0.
\end{equation}
Hence, Equations~(\ref{e11.119}), (\ref{e11.135}), (\ref{e11.139}), and (\ref{e11.140}) imply that
\begin{equation}
[\zeta, H] =0.
\end{equation}
In other words, an eigenstate of the Hamiltonian is a simultaneous eigenstate of $\zeta$. 
Now,
\begin{equation}
\zeta^{\,2} = [\beta\,(\bSigma\cdot{\bf L}+\hbar)]^{\,2} = (\bSigma\cdot{\bf L}+\hbar)^2 = J^{\,2}+\frac{1}{4}\,\hbar^2,
\end{equation}
where use has been made of Equation~(\ref{e11.128}), as well as $\beta^{\,2}=1$. It follows that the eigenvalues of
$\zeta^{\,2}$ are $j\,(j+1)\,\hbar^2 + (1/4)\,\hbar^2 = (j+1/2)^2\,\hbar^2$. Thus, the eigenvalues of $\zeta$
can be written $k\,\hbar$, where $k=\pm(j+1/2)$ is a non-zero integer. 

Equation~(\ref{e11.93}) implies that 
\begin{align}
(\bSigma\cdot{\bf x})\,(\bSigma\cdot{\bf p}) &= {\bf x}\cdot{\bf p} + {\rm i}\,\bSigma\cdot{\bf x}\times {\bf p} 
= r\,p_r + {\rm i}\,\bSigma\cdot{\bf L}\nonumber\\[0.5ex]
&= r\,p_r + {\rm i}\,(\beta\,\zeta-\hbar),\label{e11.143}
\end{align}
where use has been made of (\ref{e11.222f}) and (\ref{e11.137}).

It is helpful to define the new operator $\epsilon$, where
\begin{equation}\label{e11.144}
r\,\epsilon = \balpha\cdot{\bf x}.
\end{equation}
Moreover, it is evident that
\begin{equation}\label{e11.145}
[\epsilon,r] = 0.
\end{equation}
Hence,
\begin{equation}
r^2\,\epsilon^2 = (\balpha\cdot{\bf x})^2 = \frac{1}{2}\sum_{i,j=1,3}\{\alpha_i,\alpha_j\}\,x^i\,x^j = \sum_{i=1,3}(x^i)^{\,2}=r^2,
\end{equation}
where use has been made of (\ref{e11.19}). 
It follows that
\begin{equation}\label{e11.147}
\epsilon^2 = 1.
\end{equation}
We have already seen that $\zeta$ commutes with $\balpha\cdot{\bf x}$ and $r$. Thus,
\begin{equation}
[\zeta,\epsilon] = 0.
\end{equation}

Equation~(\ref{e11.93}) gives
\begin{equation}
(\bSigma\cdot{\bf x})\,({\bf x}\cdot{\bf p}) - ({\bf x}\cdot{\bf p})\,(\bSigma\cdot{\bf x}) = \bSigma\cdot\left[{\bf x}\,({\bf x}\cdot{\bf p})- ({\bf x}\cdot{\bf p})\,{\bf x}\right] = {\rm i}\,\hbar\,\bSigma\cdot{\bf x},
\end{equation}
where use has been made of the fundamental commutation relations for position and momentum operators. 
However, ${\bf x}\cdot{\bf p}=r\,p_r$ and $\bSigma\cdot{\bf x} = \gamma^5\,r\,\epsilon$, so, multiplying through by $\gamma^5$, we get
\begin{equation}
r^2\,\epsilon\,p_r - r\,p_r\,r\,\epsilon = {\rm i}\,\hbar\,r\,\epsilon.
\end{equation}
Equation~(\ref{e11.122}) then yields
\begin{equation}
[\epsilon,p_r]= 0.
\end{equation}

Equation~(\ref{e11.143}) implies that
\begin{equation}
(\balpha\cdot{\bf x})\,(\balpha\cdot{\bf p}) = r\,p_r+{\rm i}\,(\beta\,\zeta-\hbar).
\end{equation}
Making use of Equations~(\ref{e11.139}), (\ref{e11.144}), (\ref{e11.145}), and (\ref{e11.147}), we get
\begin{equation}
\balpha\cdot {\bf p} = \epsilon\,(p_r-{\rm i}\,\hbar/r) + {\rm i}\,\epsilon\,\beta\,\zeta/r.
\end{equation}
Hence, the Hamiltonian (\ref{e11.119}) becomes
\begin{equation}
H= - e\,\phi(r) + c\,\epsilon\,(p_r-{\rm i}\,\hbar/r) + {\rm i}\,c\,\epsilon\,\beta\,\zeta/r + \beta\,m_e\,c^2.
\end{equation}
Now, we wish to solve the energy eigenvalue problem
\begin{equation}
H\,\psi = E\,\psi,
\end{equation}
where $E$ is the energy eigenvalue. However, we have already shown that  an eigenstate of the Hamiltonian is a simultaneous eigenstate of the $\zeta$
operator belonging to the eigenvalue $k\,\hbar$, where $k$ is a non-zero integer. Hence, the eigenvalue problem reduces to
\begin{equation}
\left[- e\,\phi(r) + c\,\epsilon\,(p_r-{\rm i}\,\hbar/r) + {\rm i}\,c\,\hbar\,k\,\epsilon\,\beta/r + \beta\,m_e\,c^2\right]\psi = E\,\psi,
\end{equation}
which only involves the radial coordinate $r$. It is easily demonstrated that $\epsilon$ anti-commutes with $\beta$. Hence, given that
$\beta$ takes the form (\ref{e11.28g}), and  that $\epsilon^2=1$, we can represent $\epsilon$ as the matrix
\begin{equation}
\epsilon = \left(\begin{array}{rr}0&-{\rm i}\\[0.5ex]{\rm i}&0\end{array}\right).
\end{equation}
Thus, writing $\psi$ in the spinor form 
\begin{equation}
\psi = \left(\begin{array}{c} \psi_a(r)\\[0.5ex]\psi_b(r)\end{array}\right),
\end{equation}
and making use of (\ref{e11.123}), 
the energy eigenvalue problem for an electron in a central field reduces to the following two coupled radial differential equations:
\begin{align}\label{e11.159}
\hbar\,c\left(\frac{d}{d r} + \frac{k+1}{r}\right)\psi_b + (E-m_e\,c^2+e\,\phi)\,\psi_a&= 0,\\[0.5ex]
\hbar\,c\left(\frac{d}{d r}-\frac{k-1}{r}\right)\psi_a - (E+m_e\,c^2+e\,\phi)\,\psi_b &=0.\label{e11.160}
\end{align}

\section{Fine Structure of Hydrogen Energy Levels}
For the case of a hydrogen atom,
\begin{equation}
\phi(r) = \frac{e}{4\pi\,\epsilon_0\,r}.
\end{equation}
Hence, Equations~(\ref{e11.159}) and (\ref{e11.160}) yield
\begin{align}
\left(\frac{1}{a_1} - \frac{\alpha}{y}\right)\psi_a - \left(\frac{d}{dy} + \frac{k+1}{y}\right)\psi_b &=0,\\[0.5ex]
\left(\frac{1}{a_2} +\frac{\alpha}{y}\right)\psi_b - \left(\frac{d}{dy} - \frac{k-1}{y}\right)\psi_a &=0,
\end{align}
where $y=r/a_0$,  and
\begin{align}
a_1 &= \frac{\alpha}{1-{\cal E}},\\[0.5ex]
a_2&= \frac{\alpha}{1+{\cal E}},
\end{align}
with
 ${\cal E} = E/(m_e\,c^2)$. Here, $a_0= 4\pi\,\epsilon_0\,\hbar^2/(m_e\,e^2)$ is the Bohr radius, and
$\alpha=e^2/(4\pi\,\epsilon_0\,\hbar\,c)$ the fine structure constant. 
Writing
\begin{align}\label{e11.166}
\psi_a(y) = \frac{{\rm e}^{-y/a}}{y}\,f(y),\\[0.5ex]
\psi_b(y) = \frac{{\rm e}^{-y/a}}{y}\,g(y),\label{e11.167}
\end{align}
where
\begin{equation}
a = (a_1\,a_2)^{1/2} = \frac{\alpha}{\sqrt{1-{\cal E}^{\,2}}},
\end{equation}
we obtain
\begin{align}\label{e11.169}
\left(\frac{1}{a_1}-\frac{\alpha}{y}\right)f - \left(\frac{d}{d y}- \frac{1}{a}+\frac{k}{y}\right)g &=0,\\[0.5ex]
\left(\frac{1}{a_2}+\frac{\alpha}{y}\right)g - \left(\frac{d}{d y}- \frac{1}{a}-\frac{k}{y}\right)f &=0.\label{e11.170}
\end{align}
Let us search for power law solutions of the form
\begin{align}\label{e11.171}
f(y)&= \sum_{s} c_s\,y^s,\\[0.5ex]
g(y)&= \sum_s c_s'\,y^s,\label{e11.172}
\end{align}
where successive values of $s$ differ by unity. Substitution of these solutions into Equations~(\ref{e11.169}) and (\ref{e11.170}) leads to the
recursion relations
\begin{align}\label{e11.173}
\frac{c_{s-1}}{a_1} -\alpha\,c_s - (s+k)\,c_s' + \frac{c_{s-1}}{a}&=0,\\[0.5ex]
\frac{c_{s-1}'}{a_2}+\alpha\,c_s' -(s-k)\,c_s + \frac{c_{s-1}}{a}&=0.\label{e11.174}
\end{align}
Multiplying the first of these equations by $a$, and the second by $a_2$, and then subtracting, we eliminate both $c_{s-1}$ and $c_{s-1}'$,
since $a/a_1=a_2/a$. We are left with
\begin{equation}\label{e11.175}
[a\,\alpha-a_2\,(s-k)]\,c_s + [a_2\,\alpha+a\,(s+k)]\,c_{s}' = 0.
\end{equation}

The physical boundary conditions at $y=0$ require that $y\,\psi_a\rightarrow 0$ and $y\,\psi_b\rightarrow 0$ as $y\rightarrow 0$. Thus, it
follows from (\ref{e11.166}) and (\ref{e11.167}) that $f\rightarrow 0$ and $g\rightarrow 0$ as $y\rightarrow 0$. 
Consequently, the series (\ref{e11.171}) and (\ref{e11.172}) must terminate at small positive $s$. If $s_0$ is the minimum value of $s$
for which $c_s$ and $c_s'$ do not both vanish then it follows from (\ref{e11.173}) and (\ref{e11.174}), putting $s=s_0$ and
$c_{s_0-1}=c_{s_0-1}'=0$, that
\begin{align}
\alpha\,c_{s_0}+(s_0+k)\,c_{s_0}'&=0,\\[0.5ex]
\alpha\,c_{s_0}' - (s_0-k)\,c_{s_0} &=0,
\end{align}
which implies that
\begin{equation}
\alpha^2 = - s_0^{\,2} + k^2.
\end{equation}
Since the boundary condition requires that the minimum value of $s_0$ be greater than zero, we must take
\begin{equation}
s_0 = (k^2-\alpha^2)^{1/2}.
\end{equation}

To investigate the convergence of the series (\ref{e11.171}) and (\ref{e11.172}) at large $y$, we shall determine the
ratio $c_s/c_{s-1}$ for large $s$. In the limit of large $s$, Equations~(\ref{e11.174}) and (\ref{e11.175}) yield
\begin{align}
s\,c_s&\simeq \frac{c_{s-1}}{a}+\frac{c_{s-1}'}{a_2},\\[0.5ex]
a_2\,c_s&\simeq a\,c_s',
\end{align}
since $\alpha\simeq 1/137 \ll 1$. Thus,
\begin{equation}
\frac{c_s}{c_{s-1}}\simeq \frac{2}{a\,s}.
\end{equation}
However, this is the ratio of coefficients in the series expansion of $\exp(2\,y/a)$. Hence, we deduce that the series  (\ref{e11.171}) and (\ref{e11.172}) diverge
unphysically at large $y$ unless they terminate at large $s$. 

Suppose that the series (\ref{e11.171}) and (\ref{e11.172})  terminate with the terms $c_s$ and $c_s'$, so that $c_{s+1}=c_{s+1}'=0$. It
follows from (\ref{e11.173}) and (\ref{e11.174}), with $s+1$ substituted for $s$, that
\begin{align}
\frac{c_s}{a_1} + \frac{c_s'}{a}&=0,\\[0.5ex]
\frac{c_s'}{a_2} + \frac{c_s}{a} &=0.
\end{align}
These two expressions are equivalent, because $a^2=a_1\,a_2$. When combined with (\ref{e11.175}) they give
\begin{equation}
a_1\left[a\,\alpha-a_2\,(s-k)\right] = a\left[a_2\,\alpha+a\,(s+k)\right],
\end{equation}
which reduces to
\begin{equation}
2\,a_1\,a_2\,s = a\,(a_1-a_2)\,\alpha,
\end{equation}
or
\begin{equation}
{\cal E} = \left(1+\frac{\alpha^2}{s^2}\right)^{-1/2}.
\end{equation}
Here,  $s$, which specifies the last term in the series, must be greater than $s_0$ by some non-negative integer $i$. Thus,
\begin{equation}
s = i+ (k^2-\alpha^2)^{1/2}= i+[(j+1/2)^2-\alpha^2]^{1/2}.
\end{equation}
where $j\,(j+1)\,\hbar^2$ is the eigenvalue of $J^{\,2}$. Hence, the energy eigenvalues of the hydrogen atom become
\begin{equation}
\frac{E}{m_e\,c^2} =\left\{1 + \frac{\alpha^2}{\left(i+[(j+1/2)^2-\alpha^2]^{1/2}\right)^2}\right\}^{-1/2}.
\end{equation}
Given that $\alpha\simeq 1/137$, we can expand the above expression in $\alpha^2$ to give
\begin{equation}
\frac{E}{m_e\,c^2} = 1 - \frac{\alpha^2}{2\,n^2}- \frac{\alpha^4}{2\,n^4}\left(\frac{n}{j+1/2}-\frac{3}{4}\right)+{\cal O}(\alpha^6),
\end{equation}
where $n= i + j+1/2$ is a positive integer. Of course, the first term in the above expression corresponds to the electron's
rest mass energy. The second term corresponds to the standard non-relativistic expression for the hydrogen energy levels, with $n$ playing the role
of the radial quantum number (see Section~\ref{s5.6}). 
Finally, the third term corresponds to the fine structure correction to these energy levels (see Exercise~\ref{ex7.2}).
Note that this correction only depends on the quantum numbers $n$ and $j$. Now, we showed in Section~\ref{s7.7x} that the fine structure correction to
the energy levels of the hydrogen atom is a combined effect of spin-orbit coupling and the electron's relativistic mass increase. Hence,
it is evident that both of these effects are automatically taken into account in the Dirac equation. 

\section{Positron Theory}
We have already mentioned that the Dirac equation admits twice as many solutions as it ought to, half of them
belonging to states with negative values for the kinetic energy $c\,p^0 + e\,{\mit\Phi}^{\,0}$. This
difficulty was introduced when we passed from Equation~(\ref{e11.15}) to Equation~(\ref{e11.16}), and is inherent in any relativistic theory.

Let us examine the negative energy solutions of the equation
\begin{equation}\label{e11.192}
\left[\left(p^0+\frac{e}{c}\,{\mit\Phi}^{\,0}\right)-\alpha_1\left(p^1+\frac{e}{c}\,{\mit\Phi}^{\,1}\right)-\alpha_2\left(p^2+\frac{e}{c}\,{\mit\Phi}^{\,2}\right)-\alpha_3\left(p^3+\frac{e}{c}\,{\mit\Phi}^{\,3}\right)
-\beta\,m_e\,c\right]\psi=0
\end{equation}
a little more closely. For this purpose, it is convenient to use a representation of the $\alpha$'s and $\beta$
in which all the elements of the matrices $\alpha_1$, $\alpha_2$, and $\alpha_3$ are real, and
all of those of the matrix representing $\beta$ are imaginary or zero. Such a representation can
be obtained from the standard representation by interchanging the expressions for $\alpha_2$ and $\beta$. 
If Equation~(\ref{e11.192}) is expressed as a matrix equation in this representation, and we then substitute $-{\rm i}$ for
${\rm i}$, we get [remembering the factor ${\rm i}$ in Equation~(\ref{e11.13})]
\begin{equation}\label{e11.193}
\left[\left(p^0-\frac{e}{c}\,{\mit\Phi}^{\,0}\right)-\alpha_1\left(p^1-\frac{e}{c}\,{\mit\Phi}^{\,1}\right)-\alpha_2\left(p^2-\frac{e}{c}\,{\mit\Phi}^{\,2}\right)-\alpha_3\left(p^3-\frac{e}{c}\,{\mit\Phi}^{\,3}\right)
-\beta\,m_e\,c\right]\psi^\ast=0.
\end{equation}
Thus, each solution, $\psi$, of the wave equation (\ref{e11.192}) has for its complex conjugate, $\psi^\ast$, a solution of the
wave equation (\ref{e11.193}). Furthermore, if the solution, $\psi$, of (\ref{e11.192})
belongs to a negative value for $c\,p^0 + e\,{\mit\Phi}^{\,0}$ then the corresponding solution, $\psi^\ast$, of (\ref{e11.193})
will belong to a positive value for $c\,p^0-e\,{\mit\Phi}^{\,0}$. But, the operator in (\ref{e11.193}) is
just what we would get if we substituted $-e$ for $e$ in the operator in (\ref{e11.192}). It follows that each negative
energy solution of (\ref{e11.192}) is the complex conjugate of a positive energy solution of the wave equation
obtained from (\ref{e11.192}) by the substitution of $-e$ for $e$. The latter solution represents an electron
of charge $+e$ (instead of $-e$, as we have had up to now) moving through the given electromagnetic field. 

We conclude that the negative energy solutions of (\ref{e11.192}) refer to the motion of a new type of particle having the
mass of an electron, but the opposite charge. Such particles have been observed experimentally, and are called {\em positrons}. 
Note that we cannot simply assert that the negative energy solutions represent positrons, since this would make the dynamical
relations all wrong. For instance, it is certainly not true that a positron has a negative kinetic energy. Instead, we assume
that nearly all of the negative energy states are occupied, with one electron in each state, in accordance with the Pauli
exclusion principle. An unoccupied negative energy state will now appear as a particle with a positive energy, since to make
it disappear we would have to add an electron with a negative energy to the system. We assume that these unoccupied negative
energy states correspond to positrons. 

The previous assumptions require there to be a distribution of electrons of infinite density everywhere in space. A perfect
vacuum is a region of space in which all states of positive energy are unoccupied, and all of those of negative energy are occupied. 
In such a vacuum, the Maxwell  equation
\begin{equation}
\nabla\cdot {\bf E} = 0
\end{equation}
must be valid. This implies that the infinite distribution of negative energy electrons does not contribute to the
electric field. Thus, only departures from the vacuum distribution contribute to the electric charge density $\rho$ in the 
Maxwell equation
\begin{equation}
\nabla\cdot{\bf E} = \frac{\rho}{\epsilon_0}.
\end{equation}
In other words, there is a contribution $-e$ for each occupied state of positive energy, and a contribution $+e$ for each unoccupied
state of negative energy. 

The exclusion principle ordinarily prevents a positive energy electron from making transitions to states of negative energy.
However, it is still possible for such an electron to drop into an unoccupied state of negative energy. In this case, we would
observe an electron and a positron simultaneously disappearing, their energy being emitted in the form of radiation. The
converse process would consist in the creation of an electron positron pair from electromagnetic radiation. 

\subsection*{Exercises}
\begin{enumerate}[label=\thechapter.\arabic*,leftmargin=*,widest=9.20]
\item Noting that $\alpha_i=-\beta\,\alpha_i\,\beta$, prove that the $\alpha_i$ and $\beta$ matrices all have zero trace. Hence,
deduce that each of these matrices has $n$ eigenvalues $+1$, and $n$ eigenvalues $-1$, where $2\,n$ is the dimension
of the matrices. 
\item Verify that the matrices (\ref{e11.28g}) and (\ref{e11.29g}) satisfy Equations~(\ref{e11.19})--(\ref{e11.21}).
\item Verify that the matrices (\ref{e11.26t}) and (\ref{e11.27t}) satisfy the anti-commutation relations (\ref{e11.26a}).
\item Verify that if \label{ex11.4}
$$
\partial_\mu\,j^{\,\mu} = 0,
$$
where $j^{\,\mu}$ is a 4-vector field, then 
$$
\int d^3 x\,j^{\,0}
$$
is Lorentz invariant, where the integral is over all space, and it is assumed that  $j^{\,\mu}\rightarrow 0$ as $|{\bf x}|\rightarrow\infty$. 
\item Verify that (\ref{e11.70}) is a solution of (\ref{e11.69}). 
\item Verify that the $4\times 4$ matrices $\Sigma_i$, defined in (\ref{e11.92f}), satisfy the standard anti-commutation
relations for Pauli matrices: i.e., 
$$
\{\Sigma_i, \Sigma_j\} = 2\,\delta_{ij}.
$$
\end{enumerate}